

\chapter{介绍}

\section{概要}
这本书提出了一个关于灵长类动物前额叶皮层的基本功能的建议。在本章中,我们将解释为什么我们对这个问题采用了一种比较的方法。我们也解释了为什么我们的建议依赖于理解皮层区域之间的连接差异。因为我们依赖于来自细胞记录、功能成像和大脑损伤的发现,本章解释了这些方法是如何相互关联的,并考虑到它们的优缺点。我们强调几个先决条件一个成功的灵长类理论前额叶皮层:它必须包含广泛的发现,它必须说前额叶皮层的功能不同于其他地区,它必须解释前额叶皮层带来的优势,它必须处理前额叶皮层作为一个整体,它必须是可测试的。

\section{介绍}
在这本书中,我们提出灵长类动物的前额叶(PF)皮层执行一个简单的基本功能:它利用当前行为环境的信息,根据当前的生物需求产生目标,它可以在单一事件的基础上实现。PF皮层执行这一功能以及它如何做到这一点是这本书的两个主要主题。当然,我们很清楚,很多书和文章都涉及这些主题,但这本书的不同之处在于还有另外两个问题:为什么前额叶皮层会做它所做的事情,以及它是如何做到的。
\par 
这些问题的出现是因为生物学需要对两种问题的答案(Mayr 1982)。假设有人会问,为什么在危险的情况下,心率会加速。一个答案是,当大脑检测到危险并产生自主输出时,起作用的生理机制,等等。另一个答案是关于进化史,它使我们的大脑、心脏和循环系统保持原样,并做它们所做的事情。生理学和系统发育学都会导致心跳加速。
\par 
Tinbergen(1951)阐述了这个概念,建议我们应该对任何生物系统的四个问题:它是如何进化的(系统发育)?;它如何促进健康(选择)?;它是如何发展的(个体发生)?;以及它是如何工作的(机制)?关于前额叶皮层的文章通常会处理最后两个问题,但很少会处理解决前两个问题。然而,我们相信他们是理解前额叶皮层的关键。
\par 
当然,神经科学家可以说,他们对进化或健康不感兴趣。但我们认为他们这样做犯了一个战术错误。正如第二章所解释的那样,PF皮层的一些部分首先出现在早期灵长类动物中,其他部分则出现在灵长类动物的进化中。忽视这段历史,神经科学家丧失了一些重要的见解。
\par 
事实上,我们是类人猿的灵长类动物,它们是类人猿、人类和猴子的最后一个共同祖先的后代。我们看到的世界就像任何其他的类人猿一样,通过一个中央凹看到的世界在精致的细节,像大多数类人猿一样,在全彩。其他种类的哺乳动物,甚至其他种类的灵长类动物,都缺乏这些视觉专门化。与其他哺乳动物相比,我们的嗅觉和味觉和听觉能力都不佳。但我们对这个世界的了解与其他哺乳动物不同——而且更有效。PF皮层不仅是理解我们如何做到这样做的关键,而且也是理解这是如何发生的。

\section{目的}
为了实现我们的伟大设想,我们针对性的提出了五个目标,并在这本书中明确地说明了我们想要实现的目标。分别是:
\par 
  1.说明灵长类动物PF皮层是如何进化的以及PF皮层带来了什么优势。 
\par 2.说明它的连接如何定向到PF皮层,而不是其他皮层区域,来执行它的功能。 
\par 3.对灵长类动物PF皮层的基本功能提出一个具体的建议。
\par 4.为了展示所提出的功能如何解释当人们在执行复杂的认知任务时获得的成像结果。
\par  5.告诉读者我们的建议与文献中的不同,以及如何对其进行测试。
\par 
所有这些目标都很重要。考虑一下,当我们忽略第一个问题时会发生什么。为了实现这一点,我们需要了解不同哺乳动物物种的皮质区域之间的同源性。关于PF皮层功能的两种流行观点认为,它的基本功能是工作记忆(Goldman-Rakic 1998)或监测工作记忆中的项目(Petrides 1998)。这些功能被归因于PF皮层的一部分,正如第二章所解释的,这是专门在灵长类动物中进化的,而非灵长类哺乳动物缺乏的。
\par 
有人可能会认为,没有PF皮层这部分同源物的动物将会缺乏工作记忆或监测其内容物的能力。然而,证据却并非如此。例如,大鼠可以学习径向手臂迷宫任务。实验者用食物颗粒代替迷宫的八个手臂,饥饿的老鼠只需访问手臂一次来收集颗粒(Olton et al. 1982)。老鼠可以学习桡臂迷宫任务的事实表明,它们可以记住和监测它们的哪条手臂之前访问过。因此,一个全面的理论必须解释为什么灵长类动物需要它们的PF皮层的某些部分来完成其他哺乳动物在没有这些区域的同源物的情况下可以学习和执行的任务。
\par
我们的第二个和第三个目标特别重要,因为它们处理PF皮层的特定功能,与大脑的其他部分相比。如果灵长类动物的部分PF皮层进化出来了,我们需要了解这些区域可以做什么,而大脑的其他部分不能做什么。没有这样做就破坏了几种理论。例如,工作记忆(巴德利\& Della Sala 1998)、全球工作空间(Dehaene et al. 1998)和多重需求理论(Duncan 2010b)未能区分PF皮层和后顶叶皮层的作用。据说这两个皮质区域都有助于这些功能。所以这三种理论都归因于PF皮层的功能,其他区域也有作用。然而,理解灵长类动物PF皮层的关键必须在于理解其功能与皮层的其他部分的不同。
\par
我们的第四个目标需要理解我们提出的PF皮层的简单功能是如何解释当人们执行复杂的认知任务时发生的激活的。这样的任务如此之多,一个简单的叙述似乎没有希望。人们可能会问,一个简单的功能,无论多么基本,是如何解释复杂认知过程中的激活呢?
\par
我们的第五个也是最后一个目标似乎与其他目标不同,但它也同样重要。文献中的许多建议都是如此的笼统,以致于它们永远无法被驳倒。举个例子,执行功能几乎没有提供什么可验证的假设。是否有任何不涉及执行功能的重要行为?与一些关于PF皮层的理论不同,我们提出的观点应该很容易被反驳:人们只需要证明大脑的其他部分执行了我们提出的功能,或者灵长类动物的PF皮层没有执行这种功能。
\par
我们的五个目标决定了这本书的结构。第二章讨论了第一个目标:探索灵长类动物PF皮层的进化。第3-7章提出了第二个目标,那就是说它的连接允许灵长类动物的PF皮层去做那些大脑的其他部分不能去做的事情。第三个目标是关于PF皮层作为一个整体的功能,因此第8章提出了这本书的建议。第9章通过检查人类脑成像文献来探讨第四个目标。第十章实现了我们的第五个也是最后一个目标,通过比较我们的建议和其他突出的想法和建议如何让我们的测试。
\par
我们希望任何了解基本神经解剖学的神经科学家都能理解这本书,但我们不认为他们对PF皮层或用于研究它的方法有任何专业知识。因此,本章的其余部分提供了一些关于术语的背景材料和一些关键的方法要点。



\section{定义和术语}


\section{指纹}

\section{损伤和激活}

\section{损伤和活动}

\section{活动和激活}




\section{结论}



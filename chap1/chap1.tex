

\chapter{介绍}

\section{概要}
这本书提出了一个关于灵长类动物前额叶皮层的基本功能的建议。在本章中,我们将解释为什么我们对这个问题采用了一种比较的方法。我们也解释了为什么我们的建议依赖于理解皮层区域之间的连接差异。因为我们依赖于来自细胞记录、功能成像和大脑损伤的发现,本章解释了这些方法是如何相互关联的,并考虑到它们的优缺点。我们强调几个先决条件一个成功的灵长类理论前额叶皮层:它必须包含广泛的发现,它必须说前额叶皮层的功能不同于其他地区,它必须解释前额叶皮层带来的优势,它必须处理前额叶皮层作为一个整体,它必须是可测试的。

\section{介绍}
在这本书中,我们提出灵长类动物的前额叶(PF)皮层执行一个简单的基本功能:它利用当前行为环境的信息,根据当前的生物需求产生目标,它可以在单一事件的基础上实现。PF皮层执行这一功能以及它如何做到这一点是这本书的两个主要主题。当然,我们很清楚,很多书和文章都涉及这些主题,但这本书的不同之处在于还有另外两个问题:为什么前额叶皮层会做它所做的事情,以及它是如何做到的。
这些问题的出现是因为生物学需要对两种问题的答案(Mayr 1982)。假设有人会问,为什么在危险的情况下,心率会加速。一个答案是,当大脑检测到危险并产生自主输出时,起作用的生理机制,等等。另一个答案是关于进化史,它使我们的大脑、心脏和循环系统保持原样,并做它们所做的事情。生理学和系统发育学都会导致心跳加速。
Tinbergen(1951)阐述了这个概念,建议我们应该对任何生物系统的四个问题:它是如何进化的(系统发育)?;它如何促进健康(选择)?;它是如何发展的(个体发生)?;以及它是如何工作的(机制)?关于前额叶皮层的文章通常会处理最后两个问题,但很少会处理解决前两个问题。然而,我们相信他们是理解前额叶皮层的关键。
当然,神经科学家可以说,他们对进化或健康不感兴趣。但我们认为他们这样做犯了一个战术错误。正如第二章所解释的那样,PF皮层的一些部分首先出现在早期灵长类动物中,其他部分则出现在灵长类动物的进化中。忽视这段历史,神经科学家丧失了一些重要的见解。
事实上,我们是类人猿的灵长类动物,它们是类人猿、人类和猴子的最后一个共同祖先的后代。我们看到的世界就像任何其他的类人猿一样,通过一个中央凹看到的世界在精致的细节,像大多数类人猿一样,在全彩。其他种类的哺乳动物,甚至其他种类的灵长类动物,都缺乏这些视觉专门化。与其他哺乳动物相比,我们的嗅觉和味觉和听觉能力都不佳。但我们对这个世界的了解与其他哺乳动物不同——而且更有效。PF皮层不仅是理解我们如何做到这样做的关键,而且也是理解这是如何发生的。

\section{目的}

\section{定义和术语}


\section{指纹}

\section{损伤和激活}

\section{损伤和活动}

\section{活动和激活}




\section{结论}



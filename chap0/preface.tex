\label{chap:preface}
\begin{table}[htbp]
	\newcommand{\tabincell}[2]{\begin{tabular}{@{}#1@{}}#2\end{tabular}} %换行指令
	\centering
	\caption{名词列表 \label{tab:0_1}}
	\renewcommand\arraystretch{1.0}	%设置表格内行间距
	\setlength{\tabcolsep}{8mm}{
	\begin{tabular}{llll}
		\toprule 
		 英文(缩略词)   && 中文 \\
		\midrule
		anterior intraparietal cortex(AIP)   &&前顶叶内皮层  \\
		
		\midrule
		Actor-Critic (AC)   && 行动者-评论家  \\
		
		\midrule
		boold oxygen-level dependent [singal](BOLD)     &&血氧水平依赖   \\
		\midrule
		rostral cingulate motor area(CAMr)     &&头侧扣带运动区   \\
		\midrule
		cinggulate motor areas(CAMs)      &&扣带运动区   \\
		\midrule
		diffusion tensor imaging(DTI)       &&扩散张量成像   \\
		\midrule
		electroencephalography(EEG)       &&脑电图   \\
		\midrule
		frontal eye field(FEF)       &&额叶视区   \\
		\midrule
		funtional magnetic resonace imaging(fMRI)       &&功能性核磁共振成像   \\
		\midrule
		antero-dorsal granular area(GrAD)      &&前背颗粒区   \\
        \midrule
        antero-lateral granular area(GrAL)       &&前外侧颗粒区   \\
        \midrule
        dorsal granlar area(GrD)      &&背侧颗粒区   \\
        \midrule
        medial granular area       &&内侧颗粒区  \\
        \midrule
        posterior granular area       &&后颗粒区   \\
        \midrule
        poster-lateral granular area(GrPL)       &&后侧颗粒区   \\
        \midrule
        poster-medial granular area(GrPM)      &&后内侧颗粒区   \\
        \midrule
        ventral granular area(CrV)       &&腹侧颗粒区   \\
        \midrule
        thousand years ago(Ka)      &&千年前   \\
        \midrule
        lateral intraparietal cortex(LIP)       &&顶内沟外侧壁   \\
        \midrule
        million years ago(Ma)       &&百万年前   \\
        \midrule
        mediodorsal nucleus of the thalamus(MD)      &&丘脑(丘脑背侧核)   \\
        \midrule
        medial intraparietal cortex(MIP)      &&内侧顶叶内皮层   \\
        \midrule
        middle superior temporal area(MST)     &&中上颞区   \\
        \midrule
        middle temporal area(MT)       &&中颞区   \\
        \midrule
        orbital frontal cortex(OFC)       &&眶额皮层   \\
        \midrule
        positron emission tomography(PEF)       &&正电子发射断层扫描   \\
        \midrule
        prefrontal cortex(PF)       &&前额叶皮层   \\
        \midrule
        presupplementary motor area(preSMA)       &&前辅助运动区   \\
        \midrule
        receiving operating characteristic(ROC)       &&受试者工作特征   \\
        \midrule
        repetitive transcranial magnetic stimulation(rTMS)      &&重复经颅磁刺激   \\
        \midrule
        primary somatosensory cortex(S1)      &&初级躯体感觉皮层   \\
        \midrule
        second somatosensory cortex(S2)      &&第二躯体感觉皮层   \\
        \midrule
        supplementary eye field(SEF)      &&辅助眼区   \\
        \midrule
        standard error of the mean(SEM)      &&均值的标准误差   \\
        \midrule
        supplementary motor area(SMA)      &&辅助运动区   \\
        \midrule
        superior temporal polysensory area(STP)      &&上颞多感官区   \\
        \midrule
        part of the inferior temporal cortex(TE)     &&颞下皮层的一部分   \\
        \midrule
        caudal part of the inferior temporal cortex(TEO)      &&颞下皮层的尾部   \\
        \midrule
        transcranial magnetic stimulation(TMS)     &&经颅磁刺激   \\
        \midrule
        polysensory superior temporal area(TPO)      &&多感官颞上区   \\
        \midrule
        Wisconsin general testing apparatus(WGTA)      &&威斯康星州通用测试仪器   \\

		\bottomrule  

	\end{tabular}}
\end{table}%





\begin{table}[htbp]
	\newcommand{\tabincell}[2]{\begin{tabular}{@{}#1@{}}#2\end{tabular}} %换行指令
	\centering
	\caption{生物学和生理学的术语表 \label{tab:0_2}}
	\renewcommand\arraystretch{1.0}	%设置表格内行间距
	\setlength{\tabcolsep}{8mm}{
		\begin{tabular}{ll}
			\toprule 
			术语   & 用法和同义词 \\
			\midrule
			先进   & 与祖先状况不同  \\
			\midrule
			类比     & 具有共同功能的两个或多个物种的结构或行为   \\
			\midrule
			类人猿     & \makecell{一组灵长类动物,包括所有现代猴子、猿和人类,\\以及祖先类人猿的已灭绝后代}    \\
			\midrule
			注意力      &对可用信息、感官或助记符的子集的增强处理   \\
			\midrule
			狭鼻类动物       & 一组灵长类动物,包括旧世界的猴子、猿和人类   \\
			\midrule
			选择       &在备选方案中选择目标或行动   \\
			\midrule
			结合       & 代表性元素的组合   \\
			\midrule
			当前上下文       &感官输入和最近发生的事件   \\
			\midrule
			决策      &对世界的看法   \\
			\midrule
			情景记忆       &回忆事件,暗示意识   \\
			\midrule
			事件      &特定时间和地点的情境、目标、行动和结果的一次性结合   \\
			\midrule
			外部指导       &基于外部感官输入的行为  \\
			\midrule
			目标       &作为动作目标的物体或地方   \\
			\midrule
			习惯      &过度训练的结果,在不参考预测结果的情况下对刺激产生响应   \\
			\midrule
			类人猿亚目      &一组灵长类动物,包括眼镜猴和类人猿   \\
			\midrule
			同源性      &由于共同祖先的遗传而出现在两个或多个物种中的特征   \\
			\midrule
			“内部”指导      &当没有感官输入提示行为时   \\
			\midrule
			记忆       &存储信息   \\
			\midrule
			需要      &食物和液体等生物学要求; 同义词:动力、动机   \\
			\midrule
			结果      &刺激或行为产生的好处或伤害   \\
			\midrule
			优势响应、行为      &天生的、习惯性的或条件反射   \\
			\midrule
			原始     &类似于祖先的情况   \\
			\midrule
			前瞻、前瞻记忆、前瞻编码       &短期记忆中目标的表示   \\
			\midrule
			强化       &作为反馈的结果   \\
			\midrule
			再表示       &基于其他低阶表示的神经表示   \\
			\midrule
			响应       &依赖于与刺激或结果的条件关联的行动   \\
			\midrule
			奖励       &有益的结果   \\
			\midrule
			规则      &行为输入输出算法   \\
			\midrule
			符号      &小于整个对象但大于基本感官特征的非空间提示   \\
			\midrule
			策略      &(1) 一个问题的两个或多个解决方案中的一个; (2) 部分解决问题   \\
			\midrule
			值      &成本或收益的程度   \\
			\bottomrule  
			
	\end{tabular}}
\end{table}%


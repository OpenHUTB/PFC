\chapter{前言} \label{chap:preface}
\begin{table}[htbp]
	\newcommand{\tabincell}[2]{\begin{tabular}{@{}#1@{}}#2\end{tabular}} %换行指令
	\centering
	\caption{名词列表}
	\renewcommand\arraystretch{1.0}	%设置表格内行间距
	\begin{tabular}{llll}
		\toprule 
		 名词(缩略词)   && 定义 \\
		\midrule
		anterior intraparietal cortex(AIP)   &&前顶叶内皮层  \\
		\midrule
		boold oxygen-level dependent [singal](BOLD)     &&   \\
		\midrule
		rostral cingulate motor area(CAMr)     &&   \\
		\midrule
		cinggulate motor areas(CAMs)      &&   \\
		\midrule
		diffusion tensor imaging(DTI)       &&   \\
		\midrule
		electroencephalography(EEG)       &&   \\
		\midrule
		frontal eye field(fEF)       &&   \\
		\midrule
		funtional magnetic resonace imaging(fMRI)       &&   \\
		\midrule
		antero-dorsal granular area(GrAD)      &&   \\
        \midrule
        antero-lateral granular area(GrAL)       &&   \\
        \midrule
        dorsal granlar area(GrD)      &&   \\
        \midrule
        medial granular area       &&   \\
        \midrule
        posterior granular area       &&   \\
        \midrule
        poster-lateral granular area(GrPL)       &&   \\
        \midrule
               &&   \\
        \midrule
               &&   \\
        \midrule
               &&   \\
        \midrule
               &&   \\
        \midrule
               &&   \\
        \midrule
               &&   \\
       \midrule
              &&   \\
       \midrule
             &&   \\
       \midrule
       &       &&   \\
       \midrule
       &       &&   \\
		
		%
		%
		\midrule
		 \tabincell{c}{前扣带回 (24)、边缘下 (25) \\边缘前 (32)} & \tabincell{c}{前扣带回 (24)、边缘下 (25) \\边缘前 (32)}\\
		&外侧非颗粒状前额叶 & 尾区13和14和非颗粒状岛叶皮质 & 尾区13和14和非颗粒状岛叶皮质\\
		\bottomrule
	\end{tabular}%
\end{table}%
% !TeX document-id = {04d65a32-516d-4b85-9aaa-0206108e4b3a}
%!TEX program = xelatex 
% !BIB program = biber
\PassOptionsToPackage{table}{xcolor}
\documentclass[cn,10pt,citestyle=gb7714-2015, bibstyle=gb7714-2015]{elegantbook}
% citestyle=gb7714-2015 表示使用上标
% https://www.ctan.org/pkg/biblatex-gb7714-2015?lang=en


% 配色
% third: proposition
% 0, 160, 152   (green)
% 244, 105, 102  (cyan)
% 0, 174, 247    (blue)

% second:
% 175, 153, 8  (cyan)
% 230, 90, 7   (green)
% 255, 134, 24 (blue)



\title{前额叶皮层的神经生物学}
\subtitle{Elegant\LaTeX{} 经典之作}

\institute{OpenHUTB}
\date{\today}
\version{2.0}

\extrainfo{认识你自己。——古希腊德尔斐神庙墙上镌刻的箴言}

\setcounter{tocdepth}{3}

%\logo{logo-blue.png}
\cover{cover.jpg}

% 本文档命令
\usepackage{array}
% \usepackage{ctex}%加载ctex宏包,中文支持
%\usepackage{xr}

\newcommand{\ccr}[1]{\makecell{{\color{#1}\rule{1cm}{1cm}}}}

\definecolor{customcolor}{RGB}{32,178,170}
\colorlet{coverlinecolor}{customcolor}

\bibliography{reference}  % 关联参考文献文件 reference.bib

\begin{document}

\maketitle
\frontmatter

\chapter*{献辞}

\markboth{Introduction}{献辞}


\begin{center}
为了纪念

特里夏·戈德曼·拉基奇(1937–2003)

爱德华·埃瓦茨(1926–1985)

爱德华·琼斯(1939–2011)
\end{center}


\vskip 1.5cm

美国人研究的动物疯狂地奔跑着,表现出令人难以置信的兴奋,最后偶然达到了预期的结果。
德国人观察到的动物静静地坐着思考,最后从它们的内在意识中演化出解决方案。


\vskip 0.5cm


\vskip 1.5cm

\begin{flushright}
伯特兰·罗素  《哲学纲要》\\
1925 年
\end{flushright}

\tableofcontents

\mainmatter


\label{chap:preface}
\begin{table}[htbp]
	\newcommand{\tabincell}[2]{\begin{tabular}{@{}#1@{}}#2\end{tabular}} %换行指令
	\centering
	\caption{名词列表}
	\renewcommand\arraystretch{1.0}	%设置表格内行间距
	\setlength{\tabcolsep}{8mm}{
	\begin{tabular}{llll}
		\toprule 
		 名词(缩略词)   && 定义 \\
		\midrule
		anterior intraparietal cortex(AIP)   &&前顶叶内皮层  \\
		\midrule
		boold oxygen-level dependent [singal](BOLD)     &&血氧水平依赖   \\
		\midrule
		rostral cingulate motor area(CAMr)     &&头侧扣带运动区   \\
		\midrule
		cinggulate motor areas(CAMs)      &&扣带运动区   \\
		\midrule
		diffusion tensor imaging(DTI)       &&扩散张量成像   \\
		\midrule
		electroencephalography(EEG)       &&脑电图   \\
		\midrule
		frontal eye field(FEF)       &&额叶眼动区   \\
		\midrule
		funtional magnetic resonace imaging(fMRI)       &&功能性核磁共振成像   \\
		\midrule
		antero-dorsal granular area(GrAD)      &&前背颗粒区   \\
        \midrule
        antero-lateral granular area(GrAL)       &&前外侧颗粒区   \\
        \midrule
        dorsal granlar area(GrD)      &&背侧颗粒区   \\
        \midrule
        medial granular area       &&内侧颗粒区  \\
        \midrule
        posterior granular area       &&后颗粒区   \\
        \midrule
        poster-lateral granular area(GrPL)       &&后侧颗粒区   \\
        \midrule
        poster-medial granular area(GrPM)      &&后内侧颗粒区   \\
        \midrule
        ventral granular area(CrV)       &&腹侧颗粒区   \\
        \midrule
        thousand years ago(Ka)      &&千年前   \\
        \midrule
        lateral intraparietal cortex(LIP)       &&顶内沟外侧壁   \\
        \midrule
        million years ago(Ma)       &&百万年前   \\
        \midrule
        mediodorsal nucleus of the thalamus(MD)      &&丘脑(丘脑背侧核)   \\
        \midrule
        medial intraparietal cortex(MIP)      &&内侧顶叶内皮层   \\
        \midrule
        middle superior temporal area(MST)     &&中上颞区   \\
        \midrule
        middle temporal area(MT)       &&中颞区   \\
        \midrule
        orbital frontal cortex(OFC)       &&眶额皮层   \\
        \midrule
        positron emission tomography(PEF)       &&正电子发射断层扫描   \\
        \midrule
        prefrontal cortex(PF)       &&前额叶皮层   \\
        \midrule
        presupplementary motor area(preSMA)       &&前辅助运动区   \\
        \midrule
        receiving operating characteristic(ROC)       &&受试者工作特征   \\
        \midrule
        repetitive transcranial magnetic stimulation(rTMS)      &&重复经颅磁刺激   \\
        \midrule
        primary somatosensory cortex(S1)      &&初级躯体感觉皮层   \\
        \midrule
        second somatosensory cortex(S2)      &&第二躯体感觉皮层   \\
        \midrule
        supplementary eye field(SEF)      &&辅助眼区   \\
        \midrule
        standard error of the mean(SEM)      &&均值的标准误差   \\
        \midrule
        supplementary motor area(SMA)      &&辅助运动区   \\
        \midrule
        superior temporal polysensory area(STP)      &&上颞多感官区   \\
        \midrule
        part of the inferior temporal cortex(TE)     &&颞下皮层的一部分   \\
        \midrule
        caudal part of the inferior temporal cortex(TEO)      &&颞下皮层的尾部   \\
        \midrule
        transcranial magnetic stimulation(TMS)     &&经颅磁刺激   \\
        \midrule
        polysensory superior temporal area(TPO)      &&多感官颞上区   \\
        \midrule
        Wisconsin general testing apparatus(WGTA)      &&威斯康星州通用测试仪器   \\

		\bottomrule  

	\end{tabular}}
    \end{table}%




\chapter{介绍}

\section{概要}
这本书提出了一个关于灵长类动物前额叶皮层的基本功能的建议。在本章中,我们将解释为什么我们对这个问题采用了一种比较的方法。我们也解释了为什么我们的建议依赖于理解皮层区域之间的连接差异。因为我们依赖于来自细胞记录、功能成像和大脑损伤的发现,本章解释了这些方法是如何相互关联的,并考虑到它们的优缺点。我们强调几个先决条件一个成功的灵长类理论前额叶皮层:它必须包含广泛的发现,它必须说前额叶皮层的功能不同于其他地区,它必须解释前额叶皮层带来的优势,它必须处理前额叶皮层作为一个整体,它必须是可测试的。

\section{介绍}
在这本书中,我们提出灵长类动物的前额叶(PF) 皮层执行一个简单的基本功能:它利用当前行为环境的信息,根据当前的生物需求产生目标,它可以在单一事件的基础上实现。PF皮层执行这一功能以及它如何做到这一点是这本书的两个主要主题。当然,我们很清楚,很多书和文章都涉及这些主题,但这本书的不同之处在于还有另外两个问题:为什么前额叶皮层会做它所做的事情,以及它是如何做到的。
\par 
这些问题的出现是因为生物学需要对两种问题的答案( Mayr 1982)。假设有人会问,为什么在危险的情况下,心率会加速。一个答案是,当大脑检测到危险并产生自主输出时,起作用的生理机制,等等。另一个答案是关于进化史,它使我们的大脑、心脏和循环系统保持原样,并做它们所做的事情。生理学和系统发育学都会导致心跳加速。
\par 
Tinbergen(1951 )阐 述了这个概念,建议我们应该对任何生物系统的四个问题:它是如何进化的(系统发育)?;它如何促进健康(选择)?;它是如何发展的(个体发生)?;以及它是如何工作的(机制)?关于前额叶皮层的文章通常会处理最后两个问题,但很少会处理解决前两个问题。然而,我们相信他们是理解前额叶皮层的关键。
\par 
当然,神经科学家可以说,他们对进化或健康不感兴趣。但我们认为他们这样做犯了一个战术错误。正如第二章所解释的那样,PF皮层的一些部分首先出现在早期灵长类动物中,其他部分则出现在灵长类动物的进化中。忽视这段历史,神经科学家丧失了一些重要的见解。
\par 
事实上,我们是类人猿的灵长类动物,它们是类人猿、人类和猴子的最后一个共同祖先的后代。我们看到的世界就像任何其他的类人猿一样,通过一个中央凹看到的世界在精致的细节,像大多数类人猿一样,在全彩。其他种类的哺乳动物,甚至其他种类的灵长类动物,都缺乏这些视觉专门化。与其他哺乳动物相比,我们的嗅觉和味觉和听觉能力都不佳。但我们对这个世界的了解与其他哺乳动物不同——而且更有效。PF皮层不仅是理解我们如何做到这样做的关键,而且也是理解这是如何发生的。

\section{目的}
为了实现我们的伟大设想,我们针对性的提出了五个目标,并在这本书中明确地说明了我们想要实现的目标。分别是:
\par 
  1.说明灵长类动物PF皮层是如何进化的以及PF皮层带来了什么优势。 
\par 2.说明它的连接如何定向到PF皮层,而不是其他皮层区域,来执行它的功能。 
\par 3.对灵长类动物PF皮层的基本功能提出一个具体的建议。
\par 4.为了展示所提出的功能如何解释当人们在执行复杂的认知任务时获得的成像结果。
\par  5.告诉读者我们的建议与文献中的不同,以及如何对其进行测试。
\par 
所有这些目标都很重要。考虑一下,当我们忽略第一个问题时会发生什么。为了实现这一点,我们需要了解不同哺乳动物物种的皮质区域之间的同源性。关于PF皮层功能的两种流行观点认为,它的基本功能是工作记忆(Goldman-Rakic 1998)或监测工作记忆中的项目(Petrides 1998)。这些功能被归因于PF皮层的一部分,正如第二章所解释的,这是专门在灵长类动物中进化的,而非灵长类哺乳动物缺乏的。
\par 
有人可能会认为,没有PF皮层这部分同源物的动物将会缺乏工作记忆或监测其内容物的能力。然而,证据却并非如此。例如,大鼠可以学习径向手臂迷宫任务。实验者用食物颗粒代替迷宫的八个手臂,饥饿的老鼠只需访问手臂一次来收集颗粒(Olton et al. 1982)。老鼠可以学习桡臂迷宫任务的事实表明,它们可以记住和监测它们的哪条手臂之前访问过。因此,一个全面的理论必须解释为什么灵长类动物需要它们的PF皮层的某些部分来完成其他哺乳动物在没有这些区域的同源物的情况下可以学习和执行的任务。
\par
我们的第二个和第三个目标特别重要,因为它们处理PF皮层的特定功能,与大脑的其他部分相比。如果灵长类动物的部分PF皮层进化出来了,我们需要了解这些区域可以做什么,而大脑的其他部分不能做什么。没有这样做就破坏了几种理论。例如,工作记忆、全球工作空间等等。多重需求理论未能区分PF皮层和后顶叶皮层的作用。据说这两个皮质区域都有助于这些功能。所以这三种理论都归因于PF皮层的功能,其他区域也有作用。然而,理解灵长类动物PF皮层的关键必须在于理解其功能与皮层的其他部分的不同。
\par
我们的第四个目标需要理解我们提出的PF皮层的简单功能是如何解释当人们执行复杂的认知任务时发生的激活的。这样的任务如此之多,一个简单的叙述似乎没有希望。人们可能会问,一个简单的功能,无论多么基本,是如何解释复杂认知过程中的激活呢?
\par
我们的第五个也是最后一个目标似乎与其他目标不同,但它也同样重要。文献中的许多建议都是如此的笼统,以致于它们永远无法被驳倒。举个例子,执行功能几乎没有提供什么可验证的假设。是否有任何不涉及执行功能的重要行为?与一些关于PF皮层的理论不同,我们提出的观点应该很容易被反驳:人们只需要证明大脑的其他部分执行了我们提出的功能,或者灵长类动物的PF皮层没有执行这种功能。
\par
我们的五个目标决定了这本书的结构。第二章讨论了第一个目标:探索灵长类动物PF皮层的进化。第3-7章提出了第二个目标,那就是说它的连接允许灵长类动物的PF皮层去做那些大脑的其他部分不能去做的事情。第三个目标是关于PF皮层作为一个整体的功能,因此第8章提出了这本书的建议。第9章通过检查人类脑成像文献来探讨第四个目标。第十章实现了我们的第五个也是最后一个目标,通过比较我们的建议和其他突出的想法和建议如何让我们的测试。
\par
我们希望任何了解基本神经解剖学的神经科学家都能理解这本书,但我们不认为他们对PF皮层或用于研究它的方法有任何专业知识。因此,本章的其余部分提供了一些关于术语的背景材料和一些关键的方法要点。



\section{定义和术语}
\subsection{PF皮层定义}
甚至连专家有时也会松散地使用额叶这个词。当Teuber(1964)认为“额叶功能之谜”时,他指的是前额叶皮层,而不是额叶皮层整个叶。在阿林和富尔顿(1936年)的一次演讲之后的讨论中,神经学家斯坦利·科布提出了以下抗议:
\par
我可以说一下命名法吗?这是完全混淆了。这种误解主要是由于缺乏对术语的精确使用。当我在这里听作者和讨论者的时候,我听说……他们说额叶被切除了,而他们意味着前三分之二被切除了。我听说运动区过去是三种不同的东西,运动前区域和锥体束同样松散地使用。这种口语化可能很亲切,但不科学!
\par
我们把这些狭窄带到心脏,所以我们区分了PF皮质和额叶的运动区。我们只在指整个额叶或指PF皮层的病变时使用额叶一词,这些病变可能侵犯或削弱了运动前区域。
\par
我们也大大区分了前额叶皮层的颗粒状部分和颗粒状部分。大脑皮层的类型可以根据内部颗粒层,第4层的细胞体的数量和密度进行分类。颗粒状区域有一个明显的层4;颗粒状区域有较少的细胞体浓缩成一层4。当然,这是过于简单化了,但也很有用。灵长类动物PF皮层的最大部分具有颗粒状的细胞结构,因此很多人称之为颗粒状的前额叶皮层 
\par
虽然这本书关注的是颗粒状PF皮层,但我们包括了PF皮层内的某些颗粒区域,正如我们的定义。在关于眼眶和内侧PF皮质的章节中,我们讨论了它们的颗粒状和颗粒状部分。在这样做的时候,我们认为这两个区域的颗粒部分赋予灵长类动物的优势。
\par
表1.1列出了我们包括在PF皮层中的区域,表1.2列出了额叶区域的一组。在书中使用的缩写的列表和一个词汇表分别出现在p. xix和p. xxi上。在这些页面上放置一个书签可能会很有用。
\subsection{PF皮层标记}
为了讨论灵长类动物的PF皮层,我们需要采用一种命名惯例。如图1.1所示,von Bonin和Bailey(1947)将大多数颗粒状PF皮层命名为:FD区。这将不利于我们的目的。然而,我们展示了他们的图谱来显示,颗粒状PF皮质具有足够的同质性,冯·波宁和贝利认为其中大部分是一个单一的区域。
\par
在冯·波宁和贝利的命名法中,一个区域名称的第一个字母指的是一个肺叶:F代表额叶,P代表顶叶,T代表颞叶,O代表枕叶,L代表边缘。最后一个字母表示该肺叶内的一个区域;例如,区域FA指额叶区域a与额叶区域B不同。
\par
布罗德曼(1909)只是从大脑的顶部到底部依次编号他的区域,因为他在一系列水平的大脑切片中遇到了它们。所以第4区就是来自大脑顶部的第四个区域。有时很容易将这两个命名法联系起来;例如,区域FA与区域4相当接近。但有时冯·波宁和贝利对大脑的看法与布罗德曼截然不同。
\par
如今,将这两个命名系统组合起来很常见,因为它看起来最有用出于某种目的。 并且“布罗德曼区域”不再需要对应于任何布罗德曼自己描述过——甚至是想象过的。
\par
除了 PF 皮层的颗粒和无颗粒区域,Brodmann、von Bonin 和 Bailey 以及许多其他人已经认识到中间类型的皮层。 与缺乏或几乎缺乏第 4 层的颗粒区域和具有明显第 4 层的颗粒区域不同,其他区域具有异常的细胞结构。 这种类型的皮质有一层薄
有时内部颗粒层不连续。 Von Bonin 和 Bailey's area FC 拥有此属性。 一些区域,例如 FCBm(图 1.1),具有更多的中间属性,由字母组合指定。 正如我们之前所说,颗粒状和非颗粒状 PF 皮层之间的区别代表了一种有用的简化,而不是严格的二分法。 Mackey 和 Petrides (2010) 使用定量分析来确认当一个人沿着 PF 皮层的内侧或眼眶表面向嘴侧移动时,第 4 层变得更厚和更显眼。
\par
在本书中,我们使用了不拘一格的名称组合。 例如,我们使用字母TE 和 TEO 用于颞下皮层,由 von Bonin 和 Bailey 介绍。我们使用 Brodmann 使用的前运动皮层标签区域 6。 我们使用一个变体的大杂烩,例如,极地 PF 皮层的区域 10,遵循 Walker的规则。
\subsection{PF皮层分裂}
关于 PF 皮层的细分几乎没有达成共识。 正如刚才提到的,冯Bonin 和 Bailey 区分了相对较少的额叶区域,但其他解剖学家已经认识了更多。 例如,Petrides 和 Pandya (1999, 2007) 生成了一个猕猴大脑的结构图(图1.2)通过识别改变了 Walker 的大脑PF 皮层的更多细分。Petrides 和 Pandya ( 1995 ) 也制作了一张人类大脑皮层图(图 1.3),这与他们的猴子地图非常吻合。 然而,Carmichael 和 Price( 1994 ) 以不同的方式看待大脑。 他们研究了眼眶和内侧表面猴子 PF 皮层,比 Petrides 和 Pandya 识别出更多的细分,和 Öngür 等人。 (2003) 对人脑做了同样的事情。神经解剖学家发表了相互矛盾的 PF 皮层图,因为他们不同意关于他们是否以及在何处检测到边界。 在大脑皮层的其他几个部分,神经生理学提供了地形图来帮助定义一个区域。 在视觉、听觉和体感区域,感受野图通常可以定义一个区域及其界限。 神经生理学不为 PF 皮层提供此类帮助。 同样地,连接帮助定义了皮层区域。 但这样做的权力通常会回归到地形图。
\par
剩下的就是建筑学,即通过选定的结构特征识别区域的艺术。当这些特征依赖于染色的细胞体时,这种做法被称为细胞结构学。 当它们依赖于有髓纤维的模式时,术语 myeloarchitectonics适用。 它们一起被称为建筑学。 至少可以说,这是一门不精确的科学。 在本质上,建筑学是一种非常高维的模式识别技能,需要多年掌握。 出于这个原因,几十年来,一种客观、可靠和更快的标记边界的方法一直是皮质建筑学的圣杯。施莱歇尔等人。 (1999) 探索了一种他们称之为“独立于观察者”的方法。他们的意思是人类观察者无法检测到边界,而计算机可以。细胞密度从最浅的皮层(第 1 层)到最深的部分(第 6 层)变化。光密度测量这种变化,反映细胞类型的差异和不同层中的堆积密度。 通过沿条带进行这些测量皮层,统计方法可以检测划定边界的显着变化在不同的皮质区域之间。 因此,希望有一天我们会有一个完整的基于观察者独立方法的猕猴皮层图。
\par
即使可以精确可靠地确定区域边界,他们也可能不对应功能细分。 例如,在初级运动皮层——Brodmann 的 4 区和vonBonin 和 Bailey 的 FA 区——其内侧和外侧部分细胞结构不同,中间部分有较大的细胞体。 但是这个属性仅仅是由于内侧运动皮层控制腿这一事实,并且它具有大细胞体,因为它们的轴突向脊髓下方延伸得更远。 这里的细胞结构差异不对应于功能区别,除了一个涉及身体的不同部位。 如果解剖学家在PF 皮层,他或她可能会将这两个区域标记为单独的区域。 但这种区别可能很少或根本没有说明功能。
\par
鉴于在确定功能细分领域存在的问题和不一致PF 皮层,我们选择使用一般描述性术语。 图 1.4 显示了这些猕猴大脑的术语; 图 1.5 显示了它们与脑沟和编号的区域。 这种方法的优点是不依赖于任何特定的地图,同时与大多数人保持一致。
\subsection{约定和缩写}
我们的建议主要取决于 PF 皮层的连接解剖结构。 因此,当我们审查成像研究的结果时,我们有时会检查激活峰值的位置,以便将它们与我们对猴子同源区域的了解联系起来。为此,我们使用了程序 MRIcro,它采用报告的激活站点坐标并显示它位于关于脑回和脑沟地标的位置。 结果,我们的账户
大脑活动的定位有时与作者不同。 我们希望他们会原谅我们的冒昧。
\subsection{概括}
鉴于不同 PF 区域的标签种类繁多,我们建议读者参考表 1.1 和 1.2。本章中的表格和数字需要一个警告词。 很容易申请
与猕猴和人类大脑皮层的某些部分同名; 这十分的另一件事是确定这些区域是它们的共同名称所暗示的:从共同祖先继承的同系物。 第 2 章讨论了这个主题。
\par
\par






\section{指纹}
现在我们转向一个关键概念,这是我们第二和第三个目标的核心。一个成功的PF皮层理论不仅必须解释它能做什么,还必须解释为什么它本身就能做到这一点。为了实现这些目标,我们要依靠“指纹”。这个比喻来自齐尔斯和帕洛梅罗-加拉格尔(2001),他们用它来描述一个极性图,显示了大脑皮质区域中各种神经递质受体和转运体的密度。正如在法医学中一样,神经递质“指纹”具有基于许多特征的识别功能。帕辛汉姆等人(2002)称一个区域的整体连接模式称为其连接指纹,就像我们在这里所做的那样。
\subsection{连接指纹图谱}
关于大脑皮层联系的严肃研究始于潘迪亚和库伊珀斯(1969)和琼斯和鲍威尔(1970),并一直持续到今天。当然,随着这些方法变得更加敏感和可靠,研究结果也会发生变化,同时也有其他原因。
\par
例如,基底神经节的输出在不同时间单独针对辅助运动区(SMA);针对初级运动皮层(区域4)和运动前皮层(区域6);或针对其他区域,包括运动前前区(前SMA)、扣带运动区(CMA)和颗粒状PF皮层;麦克法兰。最近的证据表明,基底神经节的输出也延伸到顶叶(Clower et al. 2005)和颞叶(Middleton 。毫无疑问,公认的解剖结构有一天会再次改变,但现在我们要利用我们所拥有的。
\par
一个皮层区域的连接指纹具有重要的后果,因为它的连接限制了其功能。显然,如果一个大脑皮层区域没有接收到视觉输入,它就不能执行视觉功能。Passingham等人(2002)利用来自解剖数据库的数据绘制了PF皮层不同部分之间的连接图。这个数据库被称为椰子(http://www.cocomac.org/),它指的是猕猴的共同关系连接。在最新版本中,它包含了来自413项研究的数据,包括39,748个连接条目。
\par
通过使用这些连接数据,Passingham等人表明,每个PF区域都有一组独特的输入和输出。对于每个区域,他们用极坐标绘制了连接的强度。图1.6显示了两个连接指纹,一个用于背侧PF皮层(区域9),另一个用于眼眶PF皮层内侧(区域14)。周长显示与特征区域相连的区域,半径表示每个连接的主观强度从1到3。
\par
帕辛汉姆等人也使用了多维尺度来研究这些联系。如图1.7 A所示,多维尺度显示没有两个区域具有完全相同的连接模式。最近,Averbeck和Seo(2008)也使用Cocomac数据库绘制了不同PF区域的长期皮质皮质连接。他们证实,每个PF区域都有一个独特的连接模式。
\par
大脑的任何区域是孤立的,所以我们需要在整个PF皮层的背景下理解PF皮层的每个部分。因此,帕辛汉姆等人(2002年)继续使用层次聚类分析表明,在PF皮层内,人们可以识别出具有相似连接的区域簇。Averbeck和Seo(2008)使用了一种不同的技术来定义这样的集群,并得出了类似的结论。
\par
图1.7 B显示了PF皮层内的五个簇(Passingham et al. 2002)。与Averbeck和Seo(2008)的分析一样,它仅依赖于连接。然而,当人们将这些基于连接的集群与PF皮层中每个区域的位置信息结合起来时,就会出现一个略微不同的观点。出于这个原因,我们使用了一个与Price和Drevets(2010)提出的方案非常相似的方案。他们识别了PF皮层的五个部分:内侧、眶侧、尾侧、背侧和腹侧。第3-7章逐章考虑这五个区域,图1.8说明了它们。在大多数情况下,这些界限与PF皮层的传统观点一致。
\par
例如,图1.7 A支持在猕猴的内侧PF皮层中包含极性PF皮层。每个区域构成了一个更大的区域网络的一部分,包括运动前、顶叶、颞叶和海马皮层。
\par
关于广泛的分布式神经网络的发现提出了一个挑战,这与我们的第二个和第三个目标有关。仅仅在一个网络中放置一个PF区域是不够的,我们还需要说明该区域与同一网络中的其他区域有什么不同。例如,包括中外侧PF皮层(46区)以及后顶叶皮层的几个部分的神经网络。这两个网络部分的病变对行为有显著不同的影响。
\par
延迟交替任务的经典版本要求猴子学会选择一个食物在一次试验中——选择左边的井,在下一个试验中选择右边的食物井,以此类推,从而在另一个试验中交替。有中外侧PF皮层(46区)损伤的猴子无法重新学习这项任务,但后顶叶皮层的损伤没有影响(Ettlinger et al. 1966)。解释肯定是,虽然中外侧PF皮层和后顶叶区域有许多共同的连接,特别是彼此之间,但它们并不共享所有的连接。每个区域的连接指纹都指向了重要的差异。
\subsection{生理指纹图谱}
一个区域的连接指纹显示了该区域的约束和功能。通过类推,帕辛厄姆等人(2002)引入了功能指纹的概念。他们用生理数据说明了这些特性,所以我们在这里使用生理指纹这个短语。对于一组数据,他们绘制了细胞活动的五种特性: (1)听觉或视觉反应;(2)本体感觉或皮肤反应;(3)类似肌肉的活动模式;(4)活动与运动的时间相关性;(5)持续的延迟期活动。
\par
图1.9 C显示,SMA和腹侧前运动皮层在不同的细胞类别中出现的相对频率有所不同。例如,SMA中有更高比例的细胞有躯体感觉反应。图中还显示了这两个区域的连接指纹的不同(图1.9 A和B)。
\par
对于第二个数据集,Passingham等人(2002年)绘制了细胞对基于记忆或基于视觉线索的运动序列的偏好( Mushiake等人,1991年)。图1.10中的直方图显示了SMA和运动前皮层的结果,活动分类从1到7。第一类细胞对视觉任务具有完全的特异性;第七类的细胞对记忆引导任务有完全的特异性。
\par
四种活动的分类表明它们在统计上是相等的活动。SMA在记忆引导任务中表现出活动的优势,而外侧运动前皮层则表现出相反的偏差。
\par
我们还以生理指纹的形式绘制了这些数据。图1.10顶部的极坐标图显示了与下面的柱状图相同的数据,单元类围绕周长绘制,每个类沿半径的比例绘制。


\subsection{行为指纹}
到目前为止,我们已经暗示,为了理解PF皮层,我们需要进行比较:连接指纹,比较解剖特性和生理指纹比较细胞活动。对行为的理解也能从比较中学到好处。然而,关于PF皮层的神经心理学文献通常只强调一些行为任务。截至2011年,至少有162篇论文出现在PF皮层损伤的猴子的延迟反应任务上,其中许多论文只处理了这一项任务。
\par
这种重点有其优点。例如,它允许在同一任务上进行行为、生理、成像和药理学实验。但是,对一个或几个任务的狭隘关注可能会产生对PF皮层功能的扭曲看法。在第5章和第6章中,我们解释了为什么延迟反应任务及其近亲,即延迟交替任务,会产生这种扭曲。这些任务的结果得出结论,PF皮层主要在工作记忆,如果不是完全在工作记忆。第10章解释了为什么我们拒绝PF皮层的理论。但是人们不需要知道为什么这样做就能认识到仅仅依赖几个任务的问题。根据两个任务得出结论,PF皮层在工作记忆中的作用,就像在参观了史密斯奶奶的两个小丛苹果后,得出所有成熟的苹果都是绿色的结论。
\par
我们并不质疑灵长类动物PF皮层的损伤会对这些任务的学习和执行造成严重和持久的损害,或者一些成熟的苹果是绿色的。灵长类动物的PF皮层一定有什么东西让它对这些任务的表现。但它需要进行任务之间的比较才能理解它是什么。因此,除了前面讨论的连接指纹和生理指纹外,我们还需要行为指纹,其中包括对不同任务之间的损伤效应的比较。我们需要了解患有PF损伤的猴子表现出损伤的广泛任务。

\subsection{总结}
一个区域的连接指纹限制了它能做什么。生理和行为指纹提供了对这种功能的见解。太多的关于PF皮层的理论是从一个或几个任务的发现中发展起来的。在第10章中,我们将我们所谓的通用性测试应用到PF皮层的理论中,即检验它们是否能够解释激发它们的任务之外的数据。为了确定灵长类动物PF皮层的基本功能,我们需要依赖于广泛的发现,以连接、生理和行为指纹为特征。

\section{损伤和激活}

\section{损伤和活动}

\section{活动和激活}




\section{结论}



%\documentclass[UTF8]{ctexart}
\documentclass{article}
\usepackage{ctex}
\usepackage{graphicx}
\usepackage{float}
\title{灵长类动物前额叶皮层的进化}
\author{吴佳妮}
\date{}

\begin{document}
\maketitle

\section{概述}
前额皮层的进化经历了不同的阶段。早期哺乳动物经历了一次进化,产生了所有哺乳动物共有的颗粒状前额(后文简称PF)区域。这些区域的产生能根据预测结果改善行动(第3章)和对象(第4章)之间的觅食选择。
早期灵长类动物经历了另一个进化,形成了第一个颗粒状的PF区域。
原本这些动物的仅在夜间生活于细小的树枝上,它们在那里寻找、选择和获取食物,用一种需要头部和一只手协调运作的方法进食。
他们进化后的PF区域有助于根据当前的生物需求和特定的习惯(第4章)来选择食物,以及在杂乱的环境中保持对食物的注意力(第5章)。
后来,在类人猿灵长类动物的进化过程中,随着这些物种及其大脑体积的增加,出现了额外的颗粒状 PF 区域。
它们依靠最新进化的灵长类中央凹和改进的色觉在白天觅食。 因此,可以比它们的祖先更好地处理时空事件的顺序(第6章)并对资源的迹象进行检测(第7章)。
因为丰富的资源分散在类人猿的园区范围内,它们面临着严峻的资源波动、捕食和竞争问题。
他们的新 PF 区域使他们能够通过使用单一事件来选择觅食目标(第8章)来减少风险和非生产性觅食选择的数量。

\section{介绍}
本章探讨了颗粒状前额皮层在早期灵长类动物中首次出现的结果,以及仅灵长类动物拥有这种皮层的事实(Preuss 2007a)。

由于其名称,一些神经科学家认为关于颗粒状前额皮层的进化历史仅取决于生物的细胞结构。考虑到这一主张的重要性,有人可能会争辩说这是一个薄弱的支撑。幸运的是,许多其他特征支持了“颗粒状前额皮层在灵长类动物中的进化”这一观点。接下来,我们将列出其中的四个特征:皮层区域之间的空间布局、颗粒状前额皮层向纹状体的投射模式、感觉输入的分布、以及通过电刺激皮层引起的自主神经反应。

图2.1展示了我们对人类、猕猴和小白鼠这三个物种颗粒状前额皮层的同源性的看法,这主要归功于Preuss和Goldman-Rakic(1991a)的开创性工作。同源性指的是由于共同祖先的遗传而在相关物种中出现的类似区域。该图以浅灰色显示了仅在灵长类动物中进化出来的颗粒状前额皮层。这些颗粒状区域同样出现在人类和猕猴的大脑中,但不出现在老鼠的大脑中。老鼠只有无颗粒状前额皮层区域,该图为三个物种均以深灰色表示。我们选择这三个物种,是因为我们对前额皮层的大部分知识都基于对它们的大脑的研究。

\begin{figure}[H]
	\centering
	\includegraphics[scale=0.5]{1.png}
\end{figure}

图2.1(A)人类前额皮层的内侧(上)和眶上区(下)(Ongur等人,2003)。 (B)猕猴前额皮层的内侧(上)和眶上区(下)(Carmichael&Price,1994)。 (C)老鼠前额皮层的内侧(上)和侧面(下)(Palomero-Gallagher&Zilles,2004)。在所有图中,向左为前端。上行:所有图中背面向上。下行:(A)和(B)中,侧面向上;在(C)中,背面向上。不按比例。缩写:AC,前扣带皮层;AON,前嗅“核”;cc,胼胝体;Fr2,第二额区;la,不含颗粒的岛叶皮层;ig,灰脑层;IL,下极叶皮层;LO,外侧眶上皮层;MO,内侧眶上皮层;OB,嗅球;Pir,锥体(嗅觉)皮层;PL,前扣带区;tt,盖带;VO,腹侧眶上皮层。区域细分标记为尾部(c);下();侧面(),内侧(m);眶上(o),后部或极端(p),前端(r),或按任意标记(a,b)。 (A)改编自Ongur D. Ferry AT,Price JL。《人类眶上和内侧前额皮层的建筑分区》,《比较神经解剖学杂志》460:425-49,©2003,经John Wiley和Sons许可使用。 (B)改编自Carmichael ST,Price JL.《猕猴颅内和眶上前额皮层的建筑分区》,《比较神经解剖学杂志》346:366-402,©1994,经John Wiley和Sons许可使用。 (C)改编自Palomero-Gallagher N,Zilles K.《大鼠神经系统》中的异皮层.ed.G Paxinos,pp.729-57.圣迭戈,加利福尼亚州:爱尔斯维尔学术出版社。

非颗粒性前额叶皮层区域包括下肢内侧皮层、前扣带皮层、无颗粒的岛叶皮层、无颗粒的眶上皮层和前扣带回。在不同物种中,这些区域往往有不同的名称。例如,啮齿动物的下肢内侧皮层与灵长类动物的25区大致相对应,von Bonin和Bailey将其称为FL区(图1.1)。

众所周知,许多神经科学家认为老鼠拥有与灵长类动物相同的前额叶皮层。并且他们坚持认为,老鼠具有模拟灵长类动物前额叶皮层的微型复制品,或者可以将其所有属性混合在他们的小型无颗粒区域中(Kolb 2007;Seamans等人2008;Schoenbaum等人2009)。虽然我们对此持有不同的观点,但有一个命题应该得到普遍接受:在进化历史的某个时刻,我们的某些祖先缺乏颗粒性前额叶皮层。然而,现在我们不再缺失它。鉴于这个历史事实,询问颗粒性前额叶皮层带来了什么优势似乎是合理的。

尽管不是每个人都同意图2.1所描绘的同源性,但没有人严肃地质疑现代啮齿动物的大脑缺乏颗粒状前额叶皮层这一事实。对于其他哺乳动物,有一点存在争议:有人认为狗(Rajkowska&Kosmal 1988)和猫(Rose&Woolsey 1948)具有颗粒状前额叶皮层区域。但当我们亲自检查组织学材料中,狗和猫所谓的颗粒状区域时,它们看起来很像猴子和啮齿动物的无颗粒区域。

正如第1章所述,这种争议可能是由于观察者缺乏成体系的知识方法所致。当Mackey和Petrides(2010年)在猕猴和人脑中观察到这个问题时,他们发现一些传统上被归类为无颗粒额叶区域的区域实际上在第4层细胞体密度上,与最尾端的区域相比有略微的增加。也就是说,这些区域具有较弱的非颗粒质细胞结构,而不是完全的无颗粒结构。在食肉动物和其他非灵长类哺乳动物中发现颗粒状PF皮层的报告,反映了这种属性。神经解剖学家们都认为,细胞从额叶轨道和内侧表面向头部移动时,第4层的厚度会持续增加。因此,无颗粒皮质是否在第4层完全消失并不重要。我们可以将第4层密度低于给定阈值的区域视为足够无颗粒这一标准,用于我们之后的研究中(如图2.2所示)。

尽管大鼠缺乏细颗粒前额皮层,但一些神经科学家仍认为,大鼠前额皮质的中部与灵长类动物的中侧颗粒前额皮层(区域46)同源(Kolb 2007; Seamans等,2008),尽管后者的是一个颗粒区域(也称为背外侧或周主前额皮层)。同样,还有一些人认为,大鼠前额皮质的侧部与灵长类动物的整个眶前颞皮质同源,包括其颗粒部分(Kolb 2007; Schoenbaum等,2009)。该论点基于解剖学、生理学和神经化学的相似性以及基于声称在老鼠和猕猴中的损伤效应的相似性。

\begin{figure}[H]
	\centering
	\includegraphics[scale=0.5]{2.png}
\end{figure}

图2.2 显示了大脑额叶区域从尾向头发展的过程中,第4细胞层的归一化密度在猕猴(黑线)和人类(灰线)中的变化。误差条:SEM。该图由 Mackey S、Petrides M. 在《欧洲神经科学杂志》2010年32期(1940-1950页)中发表,经John Wiley and Sons出版社许可后再版。该图表明人类和猕猴大脑的腹内侧和侧壁眶前额皮质中具有可比较的成系统的区域。

然而,人们不能仅仅依据常被引用的相似性就推断其同源性。正如 Preuss(1995)所解释的那样,人们需要对特征进行诊断,即区分一组皮层区域和其他区域的特征区别。例如,老鼠的非颗粒状前额皮质与猕猴的颗粒状前额皮质区域具有许多相似之处,例如编码估值的细胞。但是,这三个区域——老鼠的非颗粒状前额皮质以及猕猴的非颗粒状和颗粒状前额皮质——都具有上述同样的特性,其他皮层区域也是如此。因此,它们无法帮助我们理解前额皮层皮质的进化或建立其区域之间的同源性。例如,老鼠的非颗粒状区域的特性与灵长类动物的颗粒状前额皮质相似,但它们也与灵长类动物的非颗粒状前额皮质相似,那么这些特性就与动物的同源性无关。

一些人声称与丘脑中背内侧核(MD)的连接是颗粒状前额皮质的诊断特征(Rose&Woolsey 1948;Akert 1964;Uylings等,2003)。但是,MD核向几乎所有的额叶投射,包括颗粒状和非颗粒状区域。因此,是否与MD核的连接不能作为诊断特征,故对同源性问题几乎没有影响。

曾经有一段时间,人们认为多巴胺能输入是颗粒状前额皮质的特征(Divac等,1978;Porrino&Goldman-Rakic,1982)。但是这些输入也终止于PF皮质的非颗粒状部分和运动前区,以及额叶以外大部分皮质。事实上,在灵长类动物中,多巴胺输入到运动前皮质和后枕叶皮质的强度比大多数颗粒状前额皮质都要强(Gaspar等,1992;Williams&Goldman-Rakic,1998)。因此,多巴胺输入不能帮助我们跨物种识别PF皮质。

\end{document}

\chapter{内侧前额叶皮层:基于输出结果选择动作} \label{chap:chap3}

这本书提出了关于灵长类动物前额叶皮层基本功能的方案。



\section{概述}
内侧前额叶皮层有助于根据此类行为之前的行为结果来评估和选择行为,其连接指纹解释了它是如何做到这一点的。
海马连接提供有关导航和其他涉及行动事件的信息,杏仁核提供基于当前生物需求的预测结果的最新评估,并且与内侧前运动区域的连接提供了行动的路线。
这些涉及行为和动机的“内部”信号与外部信号(如感官输入)形成鲜明的对比。
内侧前额叶皮层根据此类内部因素来偏向行动选择,包括努力成本、更新估值、预测结果对觅食选择的影响,以及使用内在与外在坐标系来指导行动。
在灵长类动物中,内侧前额叶皮层的颗粒部分通过在反馈时间评估自我产生的选择、平衡竞争性任务规则以及根据单个先前事件做出选择来阐述这些“内部”影响。\par



\section{介绍}

第~\ref{chap:chap1}~章解释了连接限制了前额叶皮层的功能。
由于其连接因区域而异,本章开始对前额叶皮层进行区域探索。
我们从内侧前额叶皮层开始,部分原因是它包括前额叶皮层的一些较老的部分(第~\ref{chap:chap2}~章)。\par


第~\ref{chap:chap2}~章区分了内侧前额叶皮层的颗粒状部分和无颗粒状部分,后者是所有哺乳动物共有的。
因此,我们想比较老鼠和猴子的无颗粒状前额叶皮层。
不幸的是,人们对猴子的这些区域知之甚少。
因此,我们被迫依赖啮齿动物(主要是老鼠)的数据。
我们认识到这种方法的危险——啮齿动物和灵长类动物的最后共同的祖先生活在大约 7 千万年-9千万年,从那以后这两个谱系就分开进化了。
这一事实意味着,随着两组动物的进化,两组动物的无颗粒状前额叶区域的连接和功能都将发生变化。
未来的研究将表明这种差异的程度,但现在我们必须利用我们先有的资源来研究。\par


在灵长类动物中,内侧前额叶皮层位于内侧前运动区域的嘴侧,包括前SMA,SMA和CMA(见缩写列表)。
Passingham等\cite{passingham2010medial}认为,所有这些中间领域都显示出指导和分析基于“内部”信号执行动作的专业性。
“内部”一词,在这里使用的意义上,是指传达内部状态和记忆的信号,与视觉、听觉、嗅觉、味觉和触觉等外部信号形成鲜明对比。



\section{区域}

图~\ref{fig:3_1}~显示了猴子和人类中被指定为内侧前额叶皮层的区域,图~\ref{fig:fig_2_1}~显示了其在猴子、人类和大鼠中的细分视图。
在灵长类动物中,内侧前额叶皮层的颗粒部分由前扣带皮层(区域24),边缘前皮层(主要是区域32)和边缘下皮层(区域25)组成。
正如第~\ref{chap:chap2}~章所解释的,所有哺乳动物(包括啮齿动物和灵长类动物)都有这 3 个区域的同源基因。\par
内侧前额叶皮层的颗粒部分包括区域9的内侧部分和区域10的全部,只有灵长类动物有这些区域。
第~\ref{chap:chap1}~章证明将前额叶皮层纳入内侧区域组的合理性(区域10),部分基于连接。
注意我们将猴子的内侧前额叶皮层皮层中的所有\textit{额极皮层}(区域 10)包括在内,但仅将人类的内侧部分包括在内。

\begin{figure}[!htb]
	\centering
	\includegraphics{chap3/3_1}
	\caption{猕猴(左)和人类(右)的内侧前额叶皮层,用阴影表示。
		格式如图~\ref{fig:1_2}~所示。}
	\label{fig:3_1}
\end{figure}


前扣带皮层这个术语在文献中有很多含义。
在本书中,我们将前边缘皮层和边缘下皮层排除在我们称为前扣带皮层的区域之外(图~\ref{fig:fig_2_1})。
我们还排除了扣带运动区域,这些区域我们认为是前运动皮层的一部分。
因此,读者应该意识到当我们使用短语前扣带皮层,我们仅指区域 24 的一部分,而不是到运动前区域或内侧前额叶皮层的其他颗粒状部分。\par


在我们从前扣带皮层中排除的区域中,前边缘皮层和下边缘皮层占据了前边缘皮层和膝下皮层的大部分。
术语膝下皮层是指胼胝体膝腹侧的皮层,术语膝前皮层是指位于膝侧的无核区域。
前生殖皮层不包括位于嘴侧的颗粒区域,例如额极皮层的内侧部分(区域10)。
最后,我们称之为腹内侧前额叶皮层的14区的状态仍不确定。
一些专家将其纳入内侧前额叶皮层,另一些专家则将其视为眶额皮层的最内侧部分。
我们不需要在这些分类之间做出决策,但在大多数情况下,我们保留了第~\ref{chap:chap4}~章关于眶额皮层的区域14的考虑。\par



\section{连接}

图~\ref{fig:3_2}~显示了猕猴内侧前额叶皮层的主要连接。
这个情节和第~\ref{chap:chap4}-\ref{chap:chap7}~章中的类似情节旨在传达神经解剖学文献中出现的最重要概念点。
我们不打算提供一个全面的总结,也不关心指出哪些神经解剖学家首先描述了一个特定的途径。
这些图充当连接指纹,第~\ref{chap:chap1}~章对此进行了解释。
连接指纹强调将前额叶皮层区域相互区分以及与其他皮层区域区分开来的特征,就像人类指纹区分人与人一样。\par


1.海马体和下托与内侧前额叶皮层的边缘下区域和边缘前区域有密集的相互连接\cite{insausti2001cortical}。
海马体和内侧前额叶皮层之间的间接连接包括前扣带皮层(区域24)和内侧颗粒区域9和区域10,其中一些通过脾后皮层运行\cite{kobayashi2003macaque}。
内嗅皮层和海马旁皮层也与内侧前额叶皮层有联系\cite{kondo2003differential,munoz2005cortical}。
海马体和内侧前额叶皮层之间的皮层下通路包括通过乳头体和丘脑的中继。
我们认为与海马体的连接对内侧前额叶皮层的功能特别重要。
其他前额叶皮层区域(如外侧眶额皮层)要么缺乏与海马体的连接,要么连接弱\cite{carmichael1995limbic}。
随后,我们解释了我们的观点,即海马体为前额叶皮层提供了有关导航和其他涉及动作的过去事件的信息。


2.内侧前额叶皮层与杏仁核有着紧密的联系,图~\ref{fig:3_3}~显示,这些投射中密度最大的涉及前额叶皮层的无颗粒部分\par


\begin{figure}[!htb]
	\centering
	\includegraphics{chap3/3_2}
	\caption{猕猴内侧前额叶皮层的选定连接。
		图~\ref{fig:1_4}~和~\ref{fig:1_5}~给出了脑沟和区域的名称。
		线连接着一些有直接轴突连接的区域,除非另有说明,否则假设是相互的。}
	\label{fig:3_2}
\end{figure}


颗粒区域\cite{prather2001increased,morecraft2007amygdala}(如13m区域)也接受图~\ref{fig:3_3}~没有说明的杏仁核输入\cite{saleem2008complementary}。\par
杏仁核通常被视为在情绪、动机和奖励中发挥作用,包括恐惧调节和对社会刺激(如人脸)的情绪响应。
但它在奖励中的作用不一般。
双侧杏仁核损伤后,条件性视运动学习完全正常进行,尽管这取决于学习与奖励的关系\cite{murray1996role}。
因此,奖赏处理本身不能作为杏仁核功能的一般或完整描述。
相反,最有力的证据表明,杏仁核和皮层之间的相互作用会根据当前的需求更新结果评估\cite{baxter2002amygdala}。
在整本书中,我们使用结果一词来指代以下反馈。\par


\begin{figure}[!htb]
	\centering
	\includegraphics{chap3/3_3}
	\caption{猴子杏仁核与大脑皮层的连接。
		阴影表示投射的主观密度,重点是从杏仁核到皮层的投射。
		皮层通常也会发送一个返回投射。
		缩写:EC,内嗅皮层;Iai、Iapm和Iam,分别为无颗粒岛状区、下分区、后内侧分区和内侧分区;Ig,颗粒状岛叶皮层;Id,粒状岛叶皮层;中央前顶盖皮质;颞上回;TEa、TEp、TEO、颞下区、前部、后部和枕部;TG,颞极皮层;V1,初级视觉(纹状体)皮层(17区)。
		经麦克米伦出版有限公司(Macmillan Publishers Ltd.)普莱斯·JL(Price JL)、德雷维茨·WC(Drevets WC)许可转载。《情绪障碍的神经回路》,《神经精神药理学》35:192–216,©2009,自然出版集团}
	\label{fig:3_3}
\end{figure}


一种行动,无论是从发生的事情还是在任何特定时间的价值来看。
当然,目前的需求不仅涉及营养和液体,还涉及避免伤害和其他生物成本和收益。
因此,我们可以说杏仁核有助于评估积极和消极的结果。\par


3.内侧前额叶皮层直接或间接投射到运动前区域。
颗粒状内侧前额叶皮层(9区)与内侧前额叶皮层的其他部分有联系,特别是与前扣带皮层有联系\cite{vogt1987cingulate}。
该区域反过来与前扣带运动区(CMAr)相连,CMAr是内侧前运动皮层的一部分\cite{morecraft1998convergence}。
CMAr位于扣带沟\cite{dum2002motor},位于与之相连的术前运动区(preSMA)的腹侧\cite{luppino1993corticocortical}。
SMA本身也与尾扣带运动区相互连接\cite{luppino1993corticocortical}。
扣带运动区或前SMA和SMA的损伤损害了猴子在没有外部(感觉)提示的情况下做出的运动的产生\cite{thaler1995functions}。
因此,我们可以将这些行动称为“内部”指导\par


4.与腹侧前额叶皮层和眶侧前额叶皮层不同,内侧前额叶皮层不接收来自颞下皮层的视觉输入\cite{carmichael1995sensory,kondo2005differential}。
然而,前扣带皮层与嗅周皮层有一些联系,它也接收来自颞上沟TPO区域的输入\cite{kondo2005differential},该区域处理视觉和听觉信息。尽管有这些输入,内侧前额叶皮层和视觉区域之间的联系并不是特别突出。\par


5.内侧前额叶皮层,尤其是额极皮层(区域10),接收来自颞上皮层的投射,其中大部分来自其吻部,包括颞极\cite{barbas1999medial,kondo2003differential}。\par
这些投射中一些更尾部的投影涉及生理学研究所显示的听觉区域\cite{hackett1998subdivisions},但其他投射的功能,如靠近颞极的投射,仍然未知\par


6.与前额叶皮层的许多其他部分不同,内侧前额叶皮层也投射到下丘脑\cite{rempel1998topographic},以及在内脏运动功能中发挥作用的脑干网状核,这些网状核在内脏功能中发挥作用\cite{ongur1998prefrontal}\par


其中一些投射可能会影响自主神经系统,以及大脑控制身体的其他方式。
例如,下丘脑外侧调节自主神经的唤醒,下丘脑的室旁核控制神经内分泌和神经分泌的输出。\par



\subsection{总结}

内侧前额叶皮层的连接指纹表明以下要点:
(1)与前额叶皮层的其他部分相比,内侧前额叶皮层接收的感觉输入较少;
(2) 它与运动前区域相连,该运动前区域在动物蓄能器网络缺乏来自任何外部线索的动作提示时控制动作;
(3)它与海马体和杏仁核有着密切的联系,这表明它既可以获得过去事件的记忆,也可以获得关于当前生物需求的结果的信息。\par


尽管这里列出的连接并不总是涉及内侧前额叶皮层的相同部分,但不同部分相互连接\cite{barbas2000connections},这意味着一个部分的输入可以影响其他部分。
第~\ref{chap:chap8}~章将这一观点扩展到整个前额叶皮层。\par



\section{决策、选择和目标}

尽管“决策”一词可以应用于动物所做的大多数事情,但我们在使用这个词时却有更严格的限制。
我们将决策与这些决策之后可能做出的选择和采取的行动区分开来\cite{schall2001neural}。
决策涉及基于感官输入的感知。
从这个意义上说,一个决策并不直接涉及动物所做的任何事情。
因此,动物会做出感性的决策,而不是感性的选择。
它们做出觅食的选择,而不是觅食的决策。\par


作为其决策的结果,动物可能会选择一个目标,并基于这个目标的选择,它可能会选择行动。
或者,动物可以直接在动作中进行选择。
在整本书中,我们使用目标一词来指代动物选择作为其行动目标的物体或位置。
然后,这一行动产生了一个结果,包括动物所获得的利益或因其行为而产生的成本。
因此,我们将目标与结果区分开来,从不将目标一词用作结果的同义词。\par


我们知道,在文献中,动物在行为实验中获得的奖励通常被称为目标。
事实上,在这些实验中,奖励是动物的最终目标。
但为了获得奖励,动物通常必须选择一个物体或地点作为其行动的目标。
因此,我们为这些对象和地点保留目标,并始终使用结果来奖励和其他形式的行动或事件反馈。
这个术语有助于本书中许多地方的讨论,读者需要记住我们使用这些术语的方式,以及它与文献中其他使用的不同之处。\par


我们还区分了基于外部或“内部”线索的选择\cite{Passingham et al.2010}。
例如,猴子可以使用远处的视觉标志来选择觅食目标,比如远处树上的水果或树叶。
因此,它们根据外部的感官信号产生一个目标(水果或树叶)。
但猴子也可以根据它们的记忆或内部状态的变化(如饥饿)来设定目标。
为了找一个更好的词,我们说在这种情况下,动物是根据“内部”信号行事的。
稍后,我们将讨论内侧前额叶皮层在外部和“内部”信号竞争时的作用。\par



\section{累加器网络}

累加器-赛道模型为这种神经竞争提供了一种机制,因此我们在这里给出了一个非常简短的描述。
尽管累加器网络已经了解这些模型将使读者更容易理解对内侧前额叶皮层的输入如何导致动作,以及来自内侧前额叶皮层的输出如何会使其他类似网络产生偏见。
在本章的后面,我们将看到这些想法在内侧前额叶皮层的功能中的具体应用,这些功能可以做出与觅食类似的选择。\par


猴子的神经生理学实验说明了累加器网络是如何工作的。
当神经网络达到产生输出的阈值时,就会做出决策、选择和行动。
这些网络就像漏积分器,积累有利于其输出的“证据”:网络所代表的决策、选择或行动。
一旦它们达到阈值,网络的输出就会引起一系列影响,这些影响可能最终导致运动命令的执行,要么通过驱动运动模式发生器的活动,要么将其从紧张抑制中释放出来。
各种类型的累加器网络并行运行,因此可以说是相互竞争。\par


图~\ref{fig:3_4}~显示了弹出实验的结果。
在这类实验中,猴子的视野中会出现许多刺激物。
它们大多数都有相同的特征,如颜色和形状,但有一个不同。
在得出图~\ref{fig:3_4}~的实验中,猴子看到了 7 个红色方块和 1 个绿色方块,这 8 个刺激物在以注视点为中心的圆的圆周上等距出现。
为了在每次试验中获得奖励,猴子必须做一个扫视的眼球运动来固定绿色的正方形。
与所有任务一样,从出现 8 种刺激到猴子开始运动的时间(称为响应时间)因试验而异,但通常在175-225毫秒之间。
在第~\ref{chap:chap5}~章中,这个实验涉及眼球运动这一事实变得尤为重要,但就目前而言,运动的种类无关紧要。\par


\begin{figure}[!htb]
	\centering
	\includegraphics{chap3/3_4}
	\caption{额视野中的神经元活动。
		放电速率是刺激开始后时间的函数。
		随着活动的增加,它达到了扫视眼球运动的阈值(阴影水平条)。
		根据响应延迟将试验分为三部分。
		左上角的插图显示了猴子观察到的显示,其中一个刺激因其不同的颜色(黑色方块)而从八个刺激中“弹出”\cite{schall1999neural}。}
	\label{fig:3_4}
\end{figure}


累加器模型假设一些累积的证据为绿色刺激。
可以说,这些累加器体现了一个“假设”,即刺激是绿色的,其输入作为支持和反对这一假设的证据。
在单细胞水平上,信息的积累表现为上升的活动,直到达到阈值,有时称为攀缘激活。
图~\ref{fig:3_4}~显示了这种攀缘激活,分为 3 组试验,根据猴子的响应时间进行排序。
在最短的响应时间内,细胞上升到由水平灰色条指定的给定水平,然后扫视很快开始。
没有直接的证据表明,整个神经元网络在那一刻达到了阈值,但我们认为是这样。
如图所示,活性增加的速率越慢,响应时间越长。\par


竞争的出现是因为这些神经网络的架构及其相互作用。
首先达到阈值的网络“赢得”比赛,它控制着它所代表的决策、选择或行动。
总的来说,这些回路以赢者通吃的方式工作,这反映了一个事实,即动物不能同时向两个相反的方向移动,同样,在大多数情况下,也不会同时做出相互矛盾的决策和选择。\par


猴子必须辨别相干运动方向的实验说明了累加器网络的竞争方式。
在这项任务中,许多光点以相同的速度向同一方向移动,而其他光点则随机移动。
随着观看时间的延长和更多的光点向同一方向移动,决策的准确性随着观看时间的延长以及当更多的点在同一方向上移动时而提高\cite{schall2001neural}。
Gold和Shadlen已经审查了这些神经元机制的证据\cite{gold2007neural},根据他们的分析,几个区域的细胞增加了它们的活动,直到它们达到阈值:MT和MST区域的网络代表决策,LIP区域的网络表示选择,而额叶视区的网络则代表行动\cite{kim1999neural}。\par


图~\ref{fig:3_5}~显示了自上而下的有偏见的竞争是如何运作的。
在这种情况下,实验者在MT区域使用皮层内微刺激来模拟自上而下的信号。
以一个整合了向上运动证据的网络为例。
通过刺激网络中的神经元,它比没有微刺激的情况下更快地达到阈值。
这种人为的输入使“向上”的累加器更有可能赢得“比赛”,并做出圆点向上移动的决策。\par


激活上升到网络阈值的速度,从而产生决策、选择或行动的速度取决于证据的强度和整合证据的可用时间。
它还取决于竞争累加器的活动,这些累加器为替代决策、选择和行动收集证据。\par


\begin{figure}[!htb]
	\centering
	\includegraphics{chap3/3_5}
	\caption{自上而下注意力的神经机制。
		(A) 猴子观察斑点,并在左右扫视目标之间进行选择,以报告斑点运动的方向。
		虚线在固定点处汇合。
		(B) 一些累加器网络整合了点向上移动的证据,另一些则整合了点向下移动的证据。
		当点向上移动时,由于皮层内微刺激,网络中的活动更大会导致其更快地达到“向上”决策的阈值,而“向下”决策的速度较慢(虚线)。
		缩写:MT UP,位于颞中区的细胞编码向上的斑点运动;MT DOWN,编码向下运动的细胞。+,兴奋性突触输入,抑制性输入。虚线显示皮质内微刺激期间的活动率;实线显示了在这种刺激之前和之后的活动。
		(A) 转载自《神经科学杂志》\cite{roitman2002response}。
		(B) 转载自《自然神经科学》\cite{ditterich2003microstimulation}}
	\label{fig:3_5}
\end{figure}


如图~\ref{fig:3_5}~所示,作为“竞赛”的一个重要方面,积累矛盾证据的细胞可以相互抑制。
其他因素也会影响每个“竞赛”,包括每个网络的阈值、其兴奋性水平,以及取消或否决决策、选择或行动的信号。
总之,这些网络中的许多网络之间的相互作用允许层次更高的网络对低阶网络产生的结果产生偏见。
对于内侧前额叶皮层,这涉及到与海马体、杏仁核、运动前皮层、下丘脑、中脑导水管周围灰质和其他结构的相互作用\par


在后面的章节中,我们将讨论前额叶皮层的其他区域也会对低阶区域发生的竞争产生偏见。
例如,第~\ref{chap:chap5}~章提供的证据表明,尾侧前额叶皮层可以对视觉区域施加自上而下的偏见,以增强对运动或形状的处理,这取决于它们与手头任务的相关性。
同样地,第~\ref{chap:chap8}~章认为,当任务需要注意控制行为时,前额叶皮层作为一个整体会产生自上而下的影响。\par



\subsection{总结}

像单细胞一样,累加器网络集成输入,当它们达到阈值时,就会产生输出。
这些模型在证据和表征方面为决策、选择和行动提供了一种看似合理的神经元机制,而不仅仅是“祖母细胞”概念所体现的输入和激发。\par


总结本章到此的内容,内侧前额叶皮层有一组独特的连接,其特征是缺乏感觉输入,与海马体、杏仁核和内侧前运动皮层的相互作用密切。
其中一些连接驱动内侧前额叶皮层中的网络,一些连接将内侧前额叶皮层的输出传递到大脑的其他部分,在那里它们提供自上而下的偏置。
在下一节中,我们将提出一些证据,证明内侧前额叶皮层的无颗粒部分偏向于竞争控制行为的低阶系统之间的竞争。\par



\section{啮齿动物的颗粒状皮层}

Murray等人\cite{murray2011can}提出,通过偏向控制行为的大脑系统之间的竞争,内侧前额叶皮层可以在不直接产生运动命令的情况下影响动物的动作。
这个想法解释了哺乳动物的祖先在没有内侧前额叶皮层的情况下是如何相处得如此融洽的。
如果没有这些区域,竞争系统中最强的关联就会占上风。 这种力量平衡可以改变,但改变速度很慢。
当它在早期哺乳动物中进化时,无颗粒前额叶皮层提供了一种自上而下的偏见,促进了更快的变化,而不是一种全新的变化能力。
后来,特别是在第~\ref{chap:chap8}~章和第~\ref{chap:chap9}~章中,我们提出这种进步——更快的变化,更少的错误——在前额叶皮层的进化过程中反复发生,第~\ref{chap:chap5}~章对自上而下的偏见竞争进行了更普遍的处理。\par


脊椎动物的许多行为都依赖于系统发育上古老的强化学习机制。
其中包括经典(巴甫洛夫)和工具(操作)条件反射\cite{dickinson1980contemporary}。
在传统的动物学习理论中,奖励会加强存在时活跃的关联,这一过程被称为强化。
通过这种强化学习机制,刺激、相应和结果的表征之间会产生关联。
每个表示可以链接到其他表示,并且它们可以分别缩写为S、R和O。
在经典条件反射中,刺激S与结果O相关联,可以称之为S–O映射。
在仪器条件作用中,响应R与结果O相关联,称为R–O映射。
在后一种情况下,这种关联可能发生在特定的刺激情境S中,从而产生S–R–O映射。
“响应”和“行动”这两个术语的含义差别不大。
前者通常意味着存在一种启动刺激,而后者则不需要。
我们不时使用S–R、R–O等的紧凑表示法,而不区分动作和响应。\par


在动物获得了特定的S–R–O和R–O关联的丰富经验后,它们的行为会失去了对结果的依赖,并成为一种习惯。
程序性记忆一词有时用来指习惯,与陈述性记忆形成对比。我们避免使用这些术语,因为陈述性记忆的概念通常意味着意识,而我们不能谈论非人类动物的类人意识。\par


Balleine\cite{balleine2003effect}使用了一个简单的程序来区分结果导向动作和习惯动作。
实验人员可以操作性地调节动物执行特定的动作,从而产生一种奖励结果。
他们还可以让同一只动物进行第二次动作,从而产生第二种奖励结果。
一般来说,受试者发现这两种奖励都是可取的,尽管他们通常有偏好。
接下来,受试者有机会消费两种奖励中的一种,通常是饱腹感。
消费这种奖励会使其相对于其他选择贬值。
在之后的测试中,被称为测试阶段,受试者可以在这两个操作之间做出选择。
除非他们养成了某种习惯,否则受试者将把大部分时间花在工作上,以获得最有价值的回报,这是根据他们当前的需求来评估的。
这意味着他们会避免产生最近消耗的奖励的行为。
因此,对其中一个奖励的满足会影响他们的行动选择。\par


大多数心理学家将这种性质的行为称为目标导向行为。
但正如我们已经提到的,我们保留目标一词来代指行动的目标。
因此,我们将这些行为称为结果导向行为。\par


这些实验的一个重要特征是,动物在测试阶段接受任何奖励结果之前选择自己的行动。
也就是说,他们不需要在目前的饱腹状态下体验贬值的食物。
相反,受试者预测他们的行为将产生的奖励价值,相应地做出选择。\par


当动物在此类任务上获得了如此多的经验并形成了一种习惯后,对其中一种奖励的满足感就不再影响它的行动。
动物继续选择在最近的一段时间内导致特定结果的行为,通常是首选食物,尽管这种结果已经贬值。
习惯也被称为S–R关联,因为动物选择行动时不考虑预测结果。养成一个习惯所需的训练量被称为过度训练。\par


请注意,一些专家以不太严格的方式使用“习惯”一词来指代动物经常做的事情,或者人们不需要思考就能做的事情。
所以读者需要知道,我们并不是用这种宽泛的方式来使用习惯。\par


前面,我们解释了累加器-跑道模型的基本原理。
现在我们可以把这些原则应用到过度训练的动物习惯的养成上。根据累加器-跑道模型,一旦S-R关联足够强,它们的累加器网络总是在其他竞争网络到达阈值之前达到阈值,习惯就会占上风。
例如,假设S-R网络与S-R-O网络竞争。
我们认为,随着S-R网络连接的加强,它们将更快地达到阈值。
当这个过程达到没有其他网络可能“赢得”竞争的地步时,一个习惯就已经形成了。
在啮齿类动物77关联中,当一种粒状皮质支配其他粒状皮质时,可称为“优势”,而“优势行为”、“优势响应”、“优势行动”等短语都指的是这种概念。
这些术语既适用于先天行为,也适用于通过丰富的经验灌输的行为。优势行为在相对稳定的环境中提供了优势。\par


然而,对优势行为的依赖有代价也有好处。
它们的缺点是动物只能相对缓慢地适应新条件。
在一个经典的例子中,绿蜥蜴(Lacerta)天生倾向于接近绿色,因为这能引导它们接近提供伪装和捕捉猎物机会的树叶。
Wagner\cite{wagner1932farbensinn}试图教这些蜥蜴在与理想食物相关的红色刺激和与掺假食物相关的绿色刺激之间做出选择。
为了训练蜥蜴选择红色刺激,科学家进行了数百次试验,尽管有些蜥蜴能轻易地区分红色和绿色,但它们还是做不到。
蜥蜴的优势行为在它们通常的生态位中表现良好,但结果是它们无法灵活地适应不稳定的环境。\par


然而哺乳动物可以相对较快地学会这项任务。\cite{murray2011can}提供了非哺乳动物脊椎动物行为不灵活的其他例子,与哺乳动物的灵活性形成对比。
例如,老鼠可以学习位置匹配任务。
在这项任务中,老鼠必须学会回到它们刚刚得到食物的地方\cite{marighetto1998effects}。
为了成功地完成这项任务,老鼠必须克服一种天生的倾向,即探索它们最近没有利用过的觅食地点,并避开那些它们已经利用过的地点。
完好无缺的老鼠可以在15-20次训练中学习这项任务,但在内侧前额叶皮层的边缘前和边缘下区域受损后,老鼠的学习速度要慢得多\cite{dias2000effects}。
因此,与缺乏这些区域的动物相比,内侧前额叶皮层似乎具有更快地从优势行为转变的能力,并且错误更少。\par


另一项观察结果强化了这种观点,即病变老鼠无法轻易克服其天生的倾向:同样的病变老鼠可以以大约正常的速度学习不匹配位置的任务\cite{dias2000effects}。
在这项任务中,动物必须避开刚收到食物的地方,选择迷宫的另一个分支。
因此,老鼠不必克服它们避开最近被开发的觅食地点的优势倾向。
这些想法解释了为什么患有边缘前和边缘下皮层病变的老鼠可以以正常的速度学习非匹配定位任务,但以异常缓慢的速度学习高度相似的匹配位置任务。\par


在野外,不同觅食地点的趋势在许多情况下都提供了优势。
耗尽的食物来源几乎没有什么好处。
然而,在某些情况下,例如当资源补充异常迅速时,动物就具有优势,因为动物可以了解这种更新可能发生的背景,并可以抑制到其他地方寻找食物的优势倾向。
通过这种方式,皮层下大脑系统的上下偏置可以增强觅食选择的灵活性,而内侧前额叶皮层似乎在哺乳动物中提供了这种能力。
表3.1列出了动物在选择动作时面临的一些问题,以及内侧前额叶皮层可能带来的一些优势。\par


\begin{table}[htbp]
	\newcommand{\tabincell}[2]{\begin{tabular}{@{}#1@{}}#2\end{tabular}} %换行指令
	\centering
	\caption{选择行动的基本问题}
	\renewcommand\arraystretch{1.5}	%设置表格内行间距
	\begin{tabular}{c c }	 % 双竖线
		\hline	% 表格横线
		问题 & 解决方案 \\	
		\hline  % 横线
		不同的行动在回报和努力方面产生不同的结果 & 基于习得动作-结果关联的偏向觅食选择 \\
		\hline
		动作可以基于外在坐标也可以基于内在坐标 & 偏向于那些基于外在或内在规则的觅食选择 \\
		\hline
		动作可以发生在稳定或不稳定的环境中 & 分别偏向那些基于习惯或基于预测结果的觅食选择 \\
		\hline
	\end{tabular}%
\end{table}%



\subsection{边缘前皮层和结果导向行为}

在老鼠中,内侧前额叶皮层包括三个区域,即前扣带皮层、边缘前皮层和边缘下皮层(见图\ref{fig:fig_2_1})。
正如我们已经提到的,位置匹配和不匹配任务的结果有助于阐明边缘前皮层和边缘下皮层的作用,但不能区分它们。
然而,一些研究已经尝试这样做。\par


当结果值发生变化时,边缘前皮层的损伤会损害老鼠改变其行为的能力。
上一节描述的贬值测试揭示了动物根据当前需求调整选择的能力。
边缘前皮层损伤的老鼠继续做出响应,而这些响应会产生高度贬值的奖励。
因此,边缘前皮层的损伤导致习惯性行为占主导地位,而非外向性行为\cite{balleine1998goal,corbit2003role}。
因此,我们可以得出结论,在正常老鼠中,边缘前皮层会根据结果导向的行为而不是习惯产生对觅食选择的偏见。\par


学习后发生的边缘前皮层损伤没有这种影响\cite{ostlund2005lesions},这也与该区域产生对结果导向行为的偏见的观点一致。
S–R联想(习惯)与S–R–O联想同时发展。经过丰富的经验后,习惯可以控制行为,因此边缘前皮层的影响变得不那么重要了。\par


边缘下皮层似乎提供了相反的偏见。
Killcross\cite{killcross2003coordination}发现,边缘前皮层和边缘下皮层的作用之间存在双重分离。
他们证实了刚刚提到的这一发现,即边缘前皮层的损伤使大鼠对实验操作的当前奖励值不敏感。
边缘下皮层的损伤没有这种影响。在这些损伤之后,即使经过了通常会产生习惯性行为的漫长过度训练期,老鼠的选择仍然会受到当前奖励值的影响。
因此,我们可以得出结论,在正常老鼠中,边缘下皮层提供了对习惯的偏见。
如果没有这种偏见,即使在预期习惯的情况下,行为仍然是以结果为导向的。\par


这些发现表明了这些区域在正常老鼠中所起的作用:边缘前皮层基于S–R–O关联(结果导向的行为)使行为偏向于觅食选择,而边缘下皮层则对同时发生的行为产生偏见学会了S–R联想(习惯)。
更广泛地说,这些领域似乎影响了S–R–O和S–R协会之间对行动选择控制权的竞争。\par


如果是这样,为什么内侧前额叶皮层需要两个区域来产生这种影响?
一个领域就足够了,因为习惯性行动越多,结果导向的行动就越少,反之亦然。
如果两个相互竞争的领域都能学习到强调自己喜欢的行为的背景,那么这两个领域的存在就有意义了。
一项实验结果支持了这一观点。过度训练后,即使已经形成习惯,边缘下皮层的暂时失活也会恢复结果依赖性行为\cite{coutureau2003inactivation}。
尽管对这一结果的其他解释是可能的,但就好像老鼠没有意识到养成习惯的背景一样。\par


从生态学的角度来看,这两个地区之间的竞争提供了在不同资源波动条件下学习觅食环境并在它们之间快速切换的能力。
在波动性较低的情况下,习惯应该占上风,因为在日常情况下快速做出响应是值得的。
当波动性增加到足以使日常行为失败的程度,而不是偶尔失败,但又不会导致结果变得完全不可预测时,转向根据预测结果做出觅食选择是值得的。\par



\subsection{边缘前皮层与规则之间的竞争}

结果导向和习惯性表现之间的区别也有助于解释规则转换实验的结果。
在这些研究中,边缘前皮层和边缘下皮层失活的老鼠根据两种不同的规则在觅食选择之间切换。\par


一个经典的范例是在四臂迷宫(称为十字迷宫)上训练老鼠(图~\ref{fig:3_6})。
在每次试验中,老鼠从一只手臂的末端开始,当它到达所有四只手臂的交界处时,它必须在左臂和右臂之间做出选择。老鼠可以用两条规则来做出这个选择。
第一条规则使用内在坐标:向特定方向转弯,例如向右转弯(图~\ref{fig:3_6},右上角)。
第二条规则使用外在坐标:例如,转向东方(图~\ref{fig:3_6},左上角)。
第二条规则取决于迷宫外部的线索,通常被称为迷宫外线索。
实验人员已经将许多名称应用于这两项任务,第~\ref{chap:chap5}~章对任务名称提出了警告。
术语响应规则一词被应用于内在指导任务,位置规则被用于描述外在指导任务。
因为位置规则也需要响应,所以我们更喜欢其他名称。
因此,就目前的目的而言,我们使用了“内在规则”和“外在规则”这两个术语,而不是“响应规则”和“位置规则”。\par


一些神经科学家将内在坐标的使用等同于习惯,但这是一种误解(表3.1)。
我们之前指出,动物可以选择物体或地点作为它们的目标,也可以直接选择行动。
当使用外部坐标时,动物会选择一个地方作为目标;当使用内在坐标时,它直接选择一个动作。
两者都可以是习惯性的,也可以是结果导向的。
当然,一旦动物非常熟悉一种行为情况,例如某个迷宫,内在坐标就会占主导地位,但这并不意味着使用内在坐标就等同于习惯。\par


\begin{figure}[!htb]
	\centering
	\includegraphics{chap3/3_6}
	\caption{基于外在规则与内在规则的选择累加器网络模型。
		顶部:老鼠在加号迷宫的A点或B点开始每次试验。
		对于外在规则,他们需要选择一个与视觉线索相对的目标,在这个例子中是东方。
		对于内在规则,老鼠需要在选择点右转。
		底部:累加器网络如何导致两个规则(灰色背景)的右转(黑色背景)的概念描述。
		例如,当编码外在规则(左下角)的累加器网络达到阈值时,它们的输出抑制编码内在规则(左上角)的网络,并促进编码提示向左的感觉证据的网络,假设老鼠在A点开始试验。
		这些网络反过来为在这种情况下编码右转的网络提供证据。
		相反,对于内在规则,不同的网络(右上角)首先达到阈值,并促进右转,同时抑制左转。
		关键字:带圆圈的字母表示可能的起点。}
	\label{fig:3_6}
\end{figure}


Ragozzino等人\cite{ragozzino1999involvement}的一项实验使用了带有气味线索的十字迷宫。
在训练老鼠使用一条规则后,Ragozzino等人训练它们使用另一条规则。
这种新的学习涉及抑制或超越已经成为习惯的规则。
边缘前皮层和边缘下皮层的失活并没有损害老鼠学习初始规则的能力,无论它们首先学习的是哪一个。但病变确实削弱了转换到第二条规则的能力。
确认所涉及的关键因素在两种规则之间切换,而不是一般的切换响应,Ragozzino\cite{ragozzino2007contribution}测试了老鼠在两种气味的选择之间的切换,结果老鼠表现正常\par


Rich\cite{rich2007prelimbic}使用视觉迷宫线索扩展了这些结果。
边缘前皮层和边缘下皮层的失活导致内在规则和外在规则之间的切换受损,但不影响老鼠在任一规则内进行选择逆转的能力。
重要的是,在规则切换当天,失活并没有影响性能。
相反,在第二天进行的测试中,这种损伤导致了旧规则的错误使用增加。\par


Rich\cite{rich2009rat}还记录了规则转换过程中边缘前皮层和边缘下皮层的神经元活动。
他们发现,边缘前皮层的活动变化发生得比边缘下皮层早。
当觅食选择在规则改变后有所改善时,边缘下皮层的活动就会发生变化。
这一发现可能反映了边缘前皮层提供的对结果导向行为的偏见,而边缘下皮层提供的是对习惯行为的偏见。
一般来说,这些区域中的细胞编码内在规则和外在规则之间的切换,但不编码规则内的切换。\par


这些结果表明,边缘前皮层和边缘下皮层根据竞争坐标系引导觅食规则的变化。
当老鼠试图学习第二条规则时,它们必须使用结果导向的行为来做到这一点,并且它们需要它们的边缘前皮层来产生适当的偏见。
如果没有这种偏见,第一条规则中的习惯会干扰第二条规则的学习。
如果这第二条觅食规则在很长一段时间内保持有效,老鼠最终会将这条规则作为一种新的习惯。
图\ref{fig:3_6}显示了累加器网络如何实现这些规则,第 \ref{chap:chap6}- \ref{chap:chap8} 章更详细地介绍了前额叶皮层在学习和应用规则中的作用。\par


到目前为止,讨论的重点是内侧前额叶皮层的前边缘和下边缘部分。
下一节研究其剩余的无颗粒成分,即前扣带皮层的作用。\par



\subsection{利益与成本之间的竞争}

当动物面临两种行动之间的选择时,不仅要考虑到它们的相对利益,还必须考虑它们的相对成本。
在实验室里,实验者可以让动物在付出高成本的大回报和付出小成本的小回报之间做出选择。
作为成本的一个例子,实验者可以要求老鼠爬过障碍物以获得奖励\cite{salamone1997behavioral},或者在获得奖励之前等待\cite{cardinal2001impulsive}。
第一种操作会产生能源成本;第二种情况会产生延迟成本。\par


Walton等人\cite{walton2003functional}使用这些操作中的第一种对具有内侧前额叶皮层损伤的老鼠进行了测试。
在T型迷宫中,老鼠必须在两只手臂之间做出选择。
在一只手臂的末端,老鼠可以获得巨大的奖励,但它们只有爬过一道困难的屏障才能达到;
在另一只手臂的末端,他们可以获得少量奖励,而无需付出克服障碍所需的努力。
正常老鼠选择了两种奖励中较大的一种,尽管它们必须爬上相当大的障碍才能到达在这些条件下,前扣带皮层损伤的选择了较小的奖励(图\ref{fig:3_7})。
需要大幅增加较大奖励的数量才能诱导老鼠爬上障碍\cite{walton2002role}。\par


\begin{figure}[!htb]
	\centering
	\includegraphics{chap3/3_7}
	\caption{努力成本对老鼠目标选择的影响。
		三组老鼠在迷宫的两臂之间进行选择。
		一只手臂有少量奖励,没有障碍物;另一只获得了巨大的奖励,并且有一个30厘米高的障碍,老鼠需要克服这个障碍。
		纵坐标显示了老鼠在10次试验中选择高强度手臂的平均次数。术后,前扣带皮层损伤(未填充三角形)的大鼠选择高强度杆的频率明显低于其他两组:假损伤(填充圆形)的大白鼠和边缘前皮层加边缘下皮层损伤(填充三角形)。
		误差条:SEM。横坐标上的1、2和3显示了3天的测试数据,每个测试10次。转载自Walton ME、Bannerman DM、Alterescu K、Rushworth MF。前扣带内侧额叶皮层内评估努力相关决策的功能专门化,《神经科学杂志》23:6475-9©神经科学学会,2003年,经许可。}
	\label{fig:3_7}
\end{figure}


Rudebeck等人\cite{rudebeck2006separate}比较了操纵成本在努力或延迟方面的影响。
老鼠前扣带皮层的损伤扰乱了基于努力的选择,而\textit{眶额皮层}的损伤则导致了基于延迟获得奖励的选择受损。
因为这项研究涉及老鼠,我们知道损伤涉及颗粒缺失区域。
在下一章中,我们将研究眶额皮层这些部分的结果,但目前我们可以得出结论,无颗粒前额叶皮层的内侧部分和眼眶部分都考虑了有关成本和收益的证据,并对所涉及的成本类型进行了一些专门化。
我们不知道同样的结论是否适用于猴子,但我们认为它们适用。\par


当行为不涉及觅食选择时,前扣带损伤不会损害依赖于努力成本的行为。
需要Schweimer\cite{schweimer2005involvement}老鼠越来越频繁地按下压条来获取食物,而前扣带回损伤并没有影响这种行为。
如果损伤只是让老鼠变得懒惰或冷漠,那么当它们需要进行多次压条来生产食物时,它们就会停止压条。
Walton等人\cite{walton2002role}和Schweimer\cite{schweimer2005involvement}的实验不同之处在于,在前一种情况下,老鼠必须在两种动作之间做出选择,但在后一种情况中,它们没有。
沃尔顿等人的实验结果表明,当受损老鼠不得不根据预测结果做出觅食选择时,它们高估了努力成本或低估了回报收益。\par



\subsection{总结}

综上所述,我们利用啮齿类动物的证据证明,内侧前额叶皮层的无颗粒部分的功能如下:\par


1.它们偏向于系统发育上较老的行为控制系统之间的竞争,以加快觅食选择的适应性。\par


2.它们将觅食选择偏向于适合稳定资源环境的习惯,或偏向于适合中等资源波动条件的结果导向行为。\par


3.它们在外在规则和内在规则之间进行导航选择,以指导觅食选择。\par


4.当动物必须根据预测结果的价值(包括努力成本)在行动之间做出选择时,它们在成本效益分析中发挥着至关重要的作用。\par


在这篇选择性综述中,我们强调了结果在奖励方面的当前生物学价值。
当然,对结果的评估更广泛地涵盖了其他类型的成本,如捕食或其他形式的伤害的威胁,以及其他类型的利益,如社会利益。\par


内侧前额叶皮层的连接决定了它如何执行这些功能。
投射到海马体、杏仁核、基底神经节和下丘脑和中脑导水管周围灰质的自主神经控制核可能传达了其自上而下的偏向。
来自海马体的输入提供了关于外在坐标系中导航的信息,而来自杏仁核的输入则提供了关于预测结果的当前值的信息。\par



\section{灵长类动物的无核皮层}

在上一节中,证据完全来自作为代表性啮齿类动物的老鼠,也许更有争议的是,作为代表性哺乳动物的老鼠。
正如第 \ref{chap:chap1} 章和第 \ref{chap:chap2} 章所解释的,老鼠的所有内侧前额叶皮层都具有颗粒缺失的细胞结构,就像其他哺乳动物一样。
在猕猴中,前扣带回皮层(24区)和边缘下皮层(25区)是无颗粒的,而边缘前皮层(32区)的范围从无颗粒到无颗粒\cite{Vogt&Derbyshire,2009;Mackey&Petrides,2010}。
猴子25区的亚属位置、细胞结构及其连接\cite{freedman2000subcortical}支持其被指定为老鼠边缘下皮层的同源物,以及类似的证据支持这样一个结论,即啮齿类动物和灵长类动物的前扣带皮层、边缘下皮层和边缘前皮层是同源的(第 \ref{chap:chap2} 章)。\par



\subsection{动作反转}

一项重要的实验评估了患有内侧前额叶皮层病变的猴子在动作之间切换的能力\cite{kennerley2006optimal}。
在动作反转任务中,猴子要在动作中做出选择。首先,他们学会执行一个动作,例如,在整个试验过程中持续地举起手柄。
之后,他们必须学会执行另一个动作,例如,在整个试验过程中持续地转动手柄。
接下来是一系列进一步的反转。
对于每一次反转,猴子都需要改变自己的动作选择,以产生奖励。
偶然性一词通常用来指一项行动与其结果之间的关系。
在动作反转任务中,猴子对同一个对象(手柄)执行两个不同的动作。
因此,没有任何外部线索促使做出适当的选择。
尽管如此,我们注意到猴子在一个物体上执行动作。\par
Kennerley等人在前扣带皮层(24区)造成损伤,包括头侧扣带运动区(CMAr)。
他们发现,扣带沟皮层受损的猴子在这两种动作之间的切换速度比正常猴子慢(图\ref{fig:3_8})。\par


\begin{figure}[!htb]
	\centering
 	\includegraphics{chap3/3_8}
	\caption{行动选择的反转损伤。
		猴子的术前(实心条)和术后(阴影条)数据。
		对于正常(对照)猴子(白色)和受损猴子(灰色),正确选择后(左侧四条)和错误选择后(右侧四条)的正确百分比。
		各个猴子个体的结果用符号表示,并用线条连接。
		内侧前额叶皮层损伤涉及前扣带沟的皮层,从沟的嘴测到中央前沟的嘴测水平。
		请注意,受损伤的猴子在正确试验后做出的选择上有更大、更一致的损伤。
		经麦克米伦出版社有限公司Kennerley SW、Walton ME、Behrens TEJ、Buckley MJ、Matthew F、Rushworth S.的许可转载。最佳决策和前扣带皮层。《自然神经科学》9:940–7,©2006自然出版集团。}
	\label{fig:3_8}
\end{figure}


重要的是,这种损伤并不是由于坚持之前的行动造成的。
Kennerley 等人在方法论上取得了重大进展。
他们逐次分析猴子的行为,而不是像心理学家传统上所做的那样对多次试验进行平均。
这一过程使他们能够观察到,动物在通过奖励提供的正反馈和缺乏奖励提供的负反馈来指导他们的行动方面表现出低效。
对赤字的持续解释预测,使用负反馈的问题应该比使用正反馈的问题发挥更大的作用。
结果并没有证实这一预测。
事实上,这种损伤可以被描述为严重缺乏持久性:
与持久性相反。
与对照动物相比,那些有前扣带损伤的动物在正确的选择下坚持的时间更少,即使在五次以上的奖励后也是如此。
因此,猴子的前扣带皮层似乎促进了基于正反馈的有益选择的持续性,尤其是在行动发生变化后——结果偶然性促使猴子改变行动选择。\par


Behrens等人\cite{behrens2007learning}认为,这类任务的最佳策略取决于逆转的频率,这与自然觅食中资源可用性的“波动性”有关。
我们之前说过,老鼠的前边缘皮层会以在适度波动的情况下有利的方式偏向行为
例如,如果给定的条件在多次试验中占上风,则奖励的概率很高,但由于一次失败而立即切换到新策略是不值得的。\par


Behrens等人当人类受试者在蓝色或绿色方块的选择之间切换时,对他们进行扫描,这两者都与特定的概率和奖励幅度有关。
因此,这个实验涉及到在刺激之间的选择,而不是在动作之间的选择。
尽管如此,它还是导致了内侧前额叶皮层的激活,这是我们稍后重新讨论的问题。
Behrens等人寻找一种与受试者评估资源波动性的计算模型相结合的激活。
该模型是贝叶斯模型,这意味着它使用关于概率的先验知识来评估关于世界的假设。
当给定某个事件时,如果假设为真,则该事件很可能发生,但如果假设为假,则不太可能发生,贝叶斯推理支持该假设。
在Behrens等人的实验中,由于奖励的波动性,受试者的假设变得错误。
Behrens等人在前扣带皮层发现了与选择相关的激活。
峰值激活发生在扣带沟和上覆的前SMA。
然而,与监测结果波动性相关的激活发生在内侧前额叶皮层,特别是在前膝边缘前皮层(32区)。\par


Behrens等人研究对象之间的选择,而Kennerley等人研究行为之间的选择。
Rudebeck等人\cite{rudebeck2008frontal}特别比较了前额叶皮层损伤对学习对象-结果和行动-结果关联的影响。
他们在\textit{眶额皮层}和内侧前额叶皮层(更具体地说,扣带回两侧的前扣带回皮层)都有损伤。
第~\ref{chap:chap4}~章介绍了眶额皮层,我们比较了\textit{眶额皮层}和\textit{内侧前额叶皮层}。
现在我们关注的是内侧前额叶皮层。
Rudebeck等人发现,前扣带皮层的损伤导致了基于行动的学习的缺陷,而不是基于对象的学习的缺陷。
因此,他们的数据支持这样一个结论,即猴子的内侧前额叶皮层在基于行动-结果关联选择行动方面发挥着重要作用。\par
Glascher等人\cite{glascher2009determining}使用成像技术来支持人类受试者的相同结论。
受试者被要求以两种方式之一移动电脑鼠标。
当受试者在两个动作之间切换时,前扣带皮层发生激活。\par
本章是关于内侧前额叶皮层的,但我们知道它在真空中是不起作用的。在灵长类动物中,它似乎与位于其尾部的内侧前运动区密切合作,如前SMA和SMA。
与前扣带皮层的损伤一样\cite{kennerley2006optimal},前SMA和SMA的病变会导致动作逆转任务的缺陷\cite{chen1995functions}。
在一项涉及行动和结果之间关联的不同任务中,Thaler等人\cite{thaler1995functions}表明,患有前扣带皮层或前SMA和SMA损伤的猴子(一起)在向上伸展以破坏不可见的红外光束方面表现出损伤。
在这个简单的任务中,结果的内部表示(一颗花生)作为动作的提示,猴子在完全黑暗的情况下进行了动作。
因此,前扣带和前运动皮层的损伤都会导致“内部”引导动作受损。
这种相似性表明,内侧前额叶皮层的颗粒缺失部分通过与内侧前运动区的连接影响动作的选择。\par



\subsection{价值的体现}
猴子的神经生理学研究表明,内侧前额叶皮层中的细胞编码各种结果变量。 Kennerley 等人\cite{kennerley2009evaluating}记录了内侧和眶额皮层细胞的活动。
在他们的任务中,猴子学习了类似物体的视觉刺激与奖励概率、奖励幅度或获得奖励所需的按键次数之间的映射。
然后猴子在两种刺激之间做出选择,它们几乎总是选择与更高收益或更低成本相关的刺激。
作者对内侧前额叶皮层和眶额皮层以及其他区域的细胞活动进行了采样,以寻找活动率与三个估值变量中的一个或多个的相关性。\par
根据前面提到的损伤结果,Kennerley等人预计眶额皮层将最显著地编码刺激-结果关联,而内侧前额叶皮层由于其与动作-结果关联的关系,对努力参数的编码更为稳健。
正如预期的那样,眶额皮层中的细胞编码刺激-结果估值(第\ref{chap:chap4}章)。
但内侧前额叶皮层的细胞也表现出这种活动,包括与所有三个决策变量相关的活动:奖励的大小、奖励的概率和获得奖励所需的按键次数。
对前扣带皮层(24区)的其他研究也发现了类似的结果相关信号\cite{seo2007temporal,hayden2010neurons}。
回想一下,在Behrens等人\cite{behrens2007learning}的成像实验中,内侧前额叶皮层的激活反映了刺激之间的选择。\par
我们可以从几个方面来解释这些发现。从一个角度来看,他们同意与内侧前额叶皮层的行动-结果关联的归因。正如预期的那样,与分析努力成本相关的活动发生在内侧前额叶皮层。正如前面所述,人们也可以从决策、选择和行动的角度来解释这些结果。回想一下,从这个意义上说,决策反映了感官世界,而不是原始87种选择或行动中的AGRANULAL CORTEX\cite{schall1991neuronal}。
使用这个术语,我们可以建议内侧前额叶皮层将决策直接映射到行动,而眶额皮层通过物体之间的选择将决策间接映射到行动。
我们在第\ref{chap:chap4}章中再次讨论这个问题。\par
为了更全面地理解神经生理学的发现,我们想知道这些数据是来自内侧前额叶皮层的颗粒部分还是无颗粒部分。
Kennerley等人\cite{kennerley2006optimal}的研究中,损伤从扣带沟的吻端延伸到几乎尾侧的4/6区边界(见图\ref{fig:1_2})。
因此,我们不知道该损伤的哪一部分导致了行动-结果学习的损伤。
如果没有这些知识,我们就无法从细胞结构的角度确定关键区域。
神经生理学数据可能会有所帮助,但不幸的是,无论是Kennerley等人\cite{kennerley2009evaluating}还是其他报告了类似结果的研究人员\cite{seo2007temporal,hayden2011neuronal}都没有描述他们的记录位点的细胞结构。\par


然而,Kennerley等人确实显示,在他们记录区的尾部,即胼胝体膝尾的扣带沟背岸,数值编码突然下降。
这种价值编码的下降对应于行动编码的增加。
Hayden等人还发现了对胼胝体膝部的吻侧编码的值。
人们很容易认为,它们记录区的前半部分对应于9区(颗粒皮层),尤其是扣带沟的背侧。
但Carmichael和Price的地图(图\ref{fig:fig_2_1}B)将该区域指定为猴子24区的一部分,因此我们在这里将其视为无颗粒皮层。
在人类受试者中,扣带皮层的价值激活似乎从头端延伸到不同的区域,即32ac区\cite{Öngür et al. 2003}、32´区(Vogt 2009)或3簇\cite{Beckmann et al 2009}的区域。\par


除了编码结果的价值外,前扣带皮层的细胞还编码预期结果和实际发生的结果之间的差异。
在第\ref{chap:chap8}章中,我们讨论了奖励预测误差信号,它可能导致行为的强化和消失,这取决于差异的符号。
相比之下,前扣带皮层的误差信号在低估和高估结果方面没有差异\cite{Hayden et al,2011a}。
这个无符号错误信号只表明结果没有如预期的那样发生,并且需要对行为进行一些调整。
这几乎没有表明这种调整应该是什么。
然而,这种神经信号反映了在监测结果中的作用,它可能有助于在坚持给定的行动方案和转向某种替代方案之间做出选择。\par


Hayden等人\cite{hayden2011neuronal}的一项细胞记录研究为这一想法提供了支持。
我们已经提到了这项开创性的研究。
它包含了本章的两个核心因素:灵长类动物在野外面临的觅食决策和前面解释的累加器网络。
海登等人发现了编码从递减资源转换到新资源的价值的细胞。
每当猴子不得不在利用当前正在减少的资源和转向新的资源之间做出选择时,前扣带皮层的细胞就会增加它们的活动。
猴子还收到了一个视觉信号,告诉它们开始开发新资源需要多长时间。
例如,比作从一块食物到另一块食物所需的时间,正如累加器-跑道模型预测的那样,每个延迟间隔,活动都向一个固定的阈值攀升,该阈值与切换到新资源的选择相关。
较长的延迟增加了这一阈值,从而实现了在许多动物中观察到的延迟折扣功能。\par


正如Kennerley等人和其他人的研究一样\cite{matsumoto2003neuronal,seo2007temporal},Hayden等人在前扣带皮层中观察到许多编码奖励价值的细胞\cite{hayden2009fictive,hayden2010neurons}。
这一结果似乎与实施轮班选择的角色不一致。
为了调和这两个发现,Hayden等人\cite{hayden2011neuronal}提出,在所有这些研究中,扣带皮层细胞编码了猴子利用有关资源的新信息做出选择的可能性。
在所有情况下,这种选择都涉及从持续开发资源到探索新资源的转变。\par


同样,Quilodran等人\cite{quilodran2008behavioral}表明,前扣带皮层的细胞编码在探索和开发阶段之间切换所需的反馈。
这种活动标志着探索期结束时的第一次奖励,而这些细胞在开发期停止编码奖励。在新的探索期开始时,活动又恢复了。\par


一项针对人类受试者的成像研究支持这样一种观点,即前扣带皮层监测结果,以选择有利的动作。
Walton等人\cite{walton2004interactions}表明,当受试者做出自己的动作选择时,背侧前扣带皮层的激活会增加,但当实验者选择动作时,同一区域的激活会减少。
因此,前扣带皮层似乎在监测自我产生的行为中发挥着作用,这一想法我们稍后在讨论\textit{额极皮层}时会回到这个想法。
同样的研究人员还表明,当结果提供反馈信息时,无论结果是阳性还是阴性,前扣带回皮层都会发生激活。
在之前的研究中,虽然提出了错误监控的作用,但只有错误才能提供反馈\cite{yeung2004neural}。
Walton等人的研究结果表明,内侧前额叶皮层的功能比错误检测或冲突解决更通用。\par



\subsection{啮齿类动物和灵长类动物的比较}

第\ref{chap:chap2}章解释了老鼠缺乏颗粒状前额叶皮层的同源物。
因此,我们看到灵长类动物的内侧前额叶皮层的无颗粒部分与啮齿类动物的整个内侧前额叶皮层相对应。
因此,我们在前两个主要部分中讨论了啮齿类和灵长类动物的无颗粒前额叶皮层。
然而,其他人对此事的看法不同。当他们的观点与内侧前额叶皮层直接相关时,我们在这里简要地提到他们。
稍后,第\ref{chap:chap10}章更全面地讨论了物种比较的问题。\par


Uylings等人\cite{uylings2003rats}以及Seamans\cite{seamans2008comparing}提出,老鼠的内侧前额叶皮层与中外侧前额叶皮层(区域46)同源,尽管前者具有无颗粒细胞结构,而后者具有颗粒结构。
为了支持他们的结论,他们认为老鼠的内侧前额叶皮层与猴子的颗粒状前额叶皮层有许多相同的特性,其中包括空间记忆任务和细胞活动的某些特性受损后的损伤。\par


但这些特性并没有说明同源性,因为它们广泛应用于额叶皮层,包括颗粒和无颗粒区域。
第\ref{chap:chap2}章解释了诊断特征将一个区域与其他区域区分开来,这些特征越多,结论就越有力。
所有相关领域的共同特征没有提供关于同源性的指导。
因此,无论是损伤效应还是其他特性都不支持啮齿类动物内侧前额叶皮层与灵长类动物颗粒状前额叶皮层的任何部分的同源性。\par


我们知道,内侧前额叶皮层的损伤会对老鼠的延迟响应任务造成损害\cite{kolb1974double}。
但猴子的前扣带皮层和边缘前皮层的损伤也是如此\cite{meunier1997effects},类似的损伤会导致延迟交替任务的损伤\cite{rushworth2003effect}。
鉴于这些发现,老鼠的损伤没有提供诊断证据表明老鼠的内侧前额叶皮层与猴子的中外侧前额叶皮层同源。\par


老鼠和猴子的损伤可能对类比有一些影响,而不是同源性\cite{brown2002rodent}。
然而,空间信息处理包含了广泛的功能,包括导航、到达、地点识别、距离识别等。
人们不得不说,一项任务是对其中一种功能的纯粹测试,以便为老鼠的内侧前额叶皮层和猴子的中外侧前额叶皮层之间的类比提供令人信服的理由。
目前还没有人提出有效的论据。\par



\subsection{总结}

在老鼠和猴子中,无颗粒的内侧前额叶皮层似乎在基于预测结果的行动选择中发挥着作用。
基于行动和结果之间的联系,动物可以选择特定的行动。
他们还可以选择继续做他们一直在做的事情,或者换做其他事情。
内侧前额叶皮层对这两种选择都有贡献。
因此,它不仅仅是将行动与结果联系起来,以影响采取特定行动的频率、付出多少努力,或者根据对奖励幅度或概率的预期采取何种行动。
它执行这些功能,但除此之外,它还有助于选择抽象操作(利用或探索)或特定操作。
它在内部或外部坐标中这样做。
后者发生在动物选择一个地方作为行动目标时; 前者发生在他们直接选择一个动作时。
当结果可靠地满足预期时,内侧前额叶皮层的某些部分会使行为偏向于不考虑预测结果(习惯)的行为;
当结果不太符合预期时,内侧前额叶皮层的其他部分会将行为偏向基于预测结果的行为。\par



\section{颗粒皮质}

如果我们接受这些结论,我们仍然需要说明猴子内侧前额叶皮层的颗粒部分与其无颗粒部分有何不同。
区域10对应于额极皮层并且区域9的内侧部分对应于背内侧前额叶皮层。
两者都有颗粒状的细胞结构,第\ref{chap:chap2}章回顾了它们在类人猿灵长类动物中进化的证据。
请注意,我们将前扣带皮层一词保留为无颗粒区域,因此到目前为止,我们所说的前扣带皮质都不是指这两个颗粒区域中的任何一个。\par


首先,我们承认数据匮乏。
除了在人类受试者中的成像发现外,很少有实验涉及灵长类动物9区或10区内侧部分的功能。
查阅老鼠文献并没有提供任何指导,因为正如第\ref{chap:chap2}章所解释的,老鼠缺乏这些颗粒状前额叶皮层的任何同源物。
因此,本章的其余部分所依赖的数据比我们希望的要少。\par


和往常一样,我们从连接解剖学中寻找线索。
在此基础上,我们认为内侧颗粒区阐述了内侧前额叶皮层颗粒缺失部分的功能。
正如前面关于连接的部分所解释的,所有内侧区域主要接受“内部”输入,而外部输入很少。
我们认为,当可用的外部线索没有表明要采取什么行动时,它们都会产生并偏向行动的选择或有关行动的规则。
在这种情况下,行动的选择取决于在没有外部提示的情况下预测结果的值。
我们认为内侧前额叶皮层的颗粒部分扩展了这些功能。\par



\subsection{任务规则}

Desmet等人\cite{desmet2011errors}的一项研究有助于阐明内侧前额叶皮层的嘴测和尾侧部分之间的区别。
当奖励反馈告知人类受试者应该做出什么动作时,前扣带皮层的内侧前额叶皮层的尾侧就会发生激活。
当奖励反馈告知受试者要执行哪项任务,而不是要采取什么具体行动时,内侧前额叶皮层的更靠近嘴侧的部分就会被激活。
Venkatraman等人\cite{venkatraman2009resolving}获得了类似的结果。
这些发现表明,内侧前额叶皮层的嘴侧部分有助于在两项任务之间做出选择,并且它们通过内侧前额叶皮层更多的尾侧部分影响特定的动作。\par


在猴子身上,当猴子需要解决两种规则之间的冲突时,额极皮层(10区)的损伤会导致任务表现的改善(M.J.Buckley,个人交流)。
此外,受损的猴子,而不是正常的猴子,在参与第二项任务或在试验之间提供意外奖励后,成功地回到了正在进行的主要任务。
就像刚才提到的成像结果一样,这些发现指出了最前测前额叶皮层在影响两种任务规则之间的偏差方面的作用。
第\ref{chap:chap9}章针对人类受试者展开讨论。\par



\subsection{自我生成的目标}

猴子内侧 9 的唯一神经生理学发现来自Tsujimoto等人\cite{tsujimoto2006direct}的一项研究。
这些研究人员测量了中间区域9和32的$\theta$振荡。
$ \theta $波是频率为4–7赫兹的电势。
Tsujimoto等人报告了三个发现:
(1)猴子做出自发运动之前发生的$ \theta $振荡的强度增加;
(2) 在 $\theta$的功率也有类似的增长奖励时间;
以及(3)两个区域之间的振荡相干性。
$ \theta $振荡可以组织皮层中表示的信息,以及来自额极皮层(区域 10)的关于奖励预示的自我生成目标的发现,这些结果预示着接下来的解释。\par


\begin{figure}[!htb]
	\centering
	\includegraphics{chap3/3_9}
	\caption{额极皮层(区域10)中反馈时间的细胞活动编码选择。
		A) 奖励时编码向右目标选择的单元格中的活动(右)。
		左侧选择的活动在右侧图中以灰色重复。
		(B) 每个细胞的首选(黑色)和备选(反首选)(灰色)的群体活动。缩写:acq,目标的获取。着色:扫描电镜。(C)选择信号表示为(B)中曲线之间的差异。
		数据来自Tsujimoto S,Genovesio A,Wise SP。
		评估猴子额极皮层中自我生成的决策,《自然神经科学》13:120–126©2010。}
	\label{fig:3_9}
\end{figure}


另一位名叫 Tsujimoto 的神经科学家研究了猴子执行视觉提示策略任务时额极皮层(区域 10)的神经元活动,而猴子则进行视觉提示策略任务\cite{tsujimoto2010evaluating}。
在每次试验中,一个提示指示猴子向与前一次试验相同的方向扫视(“跳跃”提示)或向相反的方向扫视。
如图\ref{fig:3_9}所示,Tsujimoto等人发现,额极皮层的神经元编码了猴子在每次正确执行的试验中选择的目标,并且它们只在反馈(结果)时这样做。\par


注意,提示并没有直接指定扫视的方向或目标。
提示只是告诉动物坚持最近的成功目标,或者转向另一个目标。
猴子必须用它对上一个目标的记忆来选择下一个目标。
换句话说,尽管停留和转移的线索是外部的,但猴子必须在内部信号的基础上产生目标。\par


在同一个实验中,Tsujimoto等人包括了一种条件,在这种条件下,空间线索指示动物在每次试验中选择哪个目标:动眼器延迟响应任务。
他们将正常时间发生反馈时的活动与晚于正常时间发生反馈的活动进行了比较。
在后一种情况下,神经元信号直到反馈到达才持续。
相比之下,在提示策略任务中,在所有情况下,在反馈之后,细胞在反馈后数百毫秒内继续编码所选的空间目标。
Tsujimoto等人将这一发现解释为,与动眼神经延迟响应任务中纯粹由外部指示的目标相比,额极皮层在评估自我产生的目标方面发挥着重要作用。
所有目标编码都发生在结果出来的时候,这一发现表明,在正在进行的试验中,额极皮层在未来目标的选择中发挥作用,而不是在目标的选择上。\par


Tsujimoto等人还比较了错误试验和正确试验期间的活动。
他们假设,在正确的试验中,猴子根据提示策略(“停留”或“转移”)产生了一个目标,而在错误试验中,猴子则在其他基础上产生了目标。
与正确试验中的稳健目标编码相反,相同的神经元在错误试验中几乎没有编码目标,如果他们真的这样做的话。
这一发现表明,额极皮层编码基于提示策略产生的目标,这需要外部提示和对最近先前目标的记忆的结合。
当猴子在其他基础上选择目标时,额极皮层即使有编码,也会对目标进行弱编码。\par


在下一章中,我们建议颗粒状前额叶皮层允许猴子将单个事件与结果联系起来,并将这种学习与许多试验中对关联的缓慢调整进行对比。
所选目标在结果发生时的表现可能有助于学习这些联系。\par



\subsection{单个事件}

其他证据也表明了单一反馈事件的重要性。
Gaffan\cite{gaffan1992amnesia}设计的一项任务,称为物体原位场景任务,允许对基于事件的学习进行研究。猴子看到一系列彩色、复杂的背景场景,每个场景都包含两个彩色形状(字母),它们总是出现在特定场景的同一位置。
猴子需要学会选择哪种颜色的形状才能获得奖励。\par


猴子学习这项任务的速度非常快,即使它们在任何场景重复之前都会看到一系列20个场景。
这种快速学习可能取决于猴子将触摸特定颜色形状的记忆嵌入到特定环境中的能力,该环境由特定的背景场景组成。
目标(彩色形状)、动作(触摸它)、背景(背景场景)和结果(奖励)的结合构成了一个事件。\par


Piekema等人(2009年)将这项任务交给了患有\textit{额极皮层}皮层损伤(10区)的猴子。
这些损伤在每个场景的第二次呈现时都会造成严重的损伤,但此后猴子恢复了相当正常的学习速度(图\ref{fig:3_10})。
当然,对照猴子在第二次演示后的改进空间比受损猴子小。
因此,我们只能说,在第二次演示后,学习率似乎大致相当。
关键是第二次演示对每个场景的记忆评估了猴子记忆单个事件的能力,这些事件包括它们对20个场景中的每个场景的初始选择,并从长期记忆中回忆它们。
第\ref{chap:chap8}章认为这一发现对解释前额叶皮层有很大帮助。\par


\begin{figure}[!htb]
	\centering
	\includegraphics{chap3/3_10}
	\caption{额极皮层(区域10)损伤后,第二次试验的物体原位场景任务损伤。每个条形图显示了正常(对照)猴子(白色)和病变猴子(黑色)连续试验块的正确选择百分比的差异。
		每个选项在20个选项的“块”中出现一次。
		缩写:v,相对于。数据来自Piekema等人(2009年),由Mark J.Buckley博士提供。}
	\label{fig:3_10}
\end{figure}



\subsection{事件存储器}

在认知心理学中,情景记忆是指对单个事件的回忆。
对于猴子的研究,我们避免使用这个术语,因为它意味着对事件的认识。
然而,在人类身上,评估意识往往很容易。
实验人员可以简单地要求受试者回忆他们生活中的事件,称为自传体记忆。
因此,例如,Hassabis等人\cite{hassabis2007using}要求受试者记住他们做过的事情,比如在电影院买票。
这些事件包括在特定时间和地点的行动、目标、背景和结果的结合。
Hassabis等人然后将受试者对这类行为的记忆与对物体的记忆进行了对比。
在内侧前额叶皮层(9区)、脾后皮层和海马体中,两种记忆(事件记忆和对象记忆)发生了不同的激活。
类似地,Summerfield等人\cite{summerfield2009decision}扫描受试者,同时从情景记忆中检索不同类型的事件。
他们在额极皮层的内侧部分(10区)、脾后皮层和海马体中发现了激活。\par



\subsection{总结}

额极皮层损伤(区域10)的猴子在物体原位场景任务的第二次试验中有损伤;
额极皮层中的细胞在反馈时编码自我生成的目标;
并且当人类受试者检索单个自传体事件的记忆时以及当他们在任务规则中进行选择时,在额极皮层(中间区域10)和中间区域9中具有激活。
对象就地场景任务可能反映出未能从长期记忆中检索相关事件,或者未能使用该记忆来影响当前的目标选择。
背景场景提供了当前的背景,第二次实验的正确选择取决于从单个自传事件中进行的一次试验学习。\par


早些时候,我们认为内侧前额叶皮层的无核部分会在动作或基于动作的规则之间产生偏差。
这些偏见取决于“内部”信号,这些信号涉及动机状态和对先前事件的记忆,包括行动和结果。
在本节中,我们提出内侧前额叶皮层的颗粒部分阐述了颗粒缺失部分的功能。两者都主要使用“内部”信号来指导选择。
有时,这些选择涉及对对象执行的动作,有时也涉及动作本身。
细粒度区域通过“内部”生成目标或规则,并通过实施一次尝试学习的机制,即基于单个事件的学习,来阐述非细粒度区域的功能。
其机制的一部分涉及在反馈时对所选目标的表示,以及随后从长期记忆中检索该事件。\par



\section{结论}

\subsection{内侧前额叶皮层如何发挥作用}

内侧前额叶皮层与海马体、杏仁核和内侧前运动区的连接解释了其对整个前额叶皮层的贡献:\par

1.内侧前额叶皮层通过脾后皮层、内嗅皮层和丘脑与海马系统有着强烈的相互联系,既有直接联系,也有间接联系。
由于海马体在外部(非中心)坐标系中引导导航的作用,当海马体控制行为时,应该出现对外部指导规则的偏见。
有证据表明,当基于内在指导的规则占主导地位时,基底神经节的一部分会控制行为\cite{packard1996inactivation}。
内侧前额叶皮层,特别是其前边缘和下边缘区域,提供了一种自上而下的偏向。
更普遍地说,这种自上而下的偏见提供了一种机制,通过这种机制,内侧前额叶皮层可以发挥注意力控制,在第\ref{chap:chap8}章中,我们认为前额叶皮层作为一个整体发挥这种控制。\par


与海马系统的连接也向内侧前额叶皮层提供了有关导航和其他动作相关事件的信息,在第\ref{chap:chap9}章中,关于人类前额叶皮层,我们提出导航和事件信息有助于将自己的动作嵌入事件的表现中。\par


2.内侧前额叶区域与杏仁核的联系解释了这些区域如何将行动与结果的当前值联系起来。
当动物消耗液体或营养物质时,它们的需求会发生变化,杏仁核和前额叶皮层之间的相互作用会更新对涉及这些资源的行为结果的评估。
类似的更新可能发生在负面结果上,如努力成本、行动的有害结果和引发恐惧的刺激。这些评估会影响与这些结果相关的行动选择。\par


当期望值未能实现时,动物需要改变它们的行动,而内侧前额叶皮层促进有效利用奖励反馈来执行这种行动逆转,包括在利用日益减少的资源和让它去探索新的资源之间进行切换。\par


3.与内侧运动前区的连接允许内侧前额叶皮层影响运动命令。\par

尽管这三组连接往往集中在内侧前额叶皮层的尾部无颗粒部分,但无颗粒区和颗粒区之间的牢固连接使内侧前额叶皮层作为一个整体能够使用它们。
第\ref{chap:chap9}章讨论了人类内侧前额叶皮层的层次结构(见图9.7)。\par



\subsection{目的}

在第\ref{chap:chap8}章中,我们提出了一个关于前额叶皮层整体基本功能的建议。
每个区域都有助于实现这一功能,因此我们以一份关于内侧前额叶皮层的声明开始制定我们的提案,首先是最简短的形式,然后是扩展版本。\par


简而言之:\par
内侧前额叶皮层有助于根据当前需求与结果的关联来评估和选择行动。\par
扩展:\par
内侧前额叶皮层通过偏向动作和基于动作的规则之间的选择,对整个前额叶皮层的功能做出贡献。
它通过根据当前需求评估预期结果来做到这一点。
当结果不能满足当前需求时,内侧前额叶皮层会促进这种反馈的有效利用,在利用日益减少的资源和探索新的资源之间切换,在替代行动之间切换、在竞争行动规则之间切换,以及在竞争行为控制系统之间切换。
在灵长类动物中,它可以基于单个事件来做到这一点。\par



\subsection{为什么其他区域不能像内侧前额叶皮层那样}

其他区域不能做内侧前额叶皮层所做的事情,因为它们缺乏一个或多个关键连接。
前额叶皮层的许多其他部分缺乏海马连接。
后顶叶皮层与皮层的运动前区域有直接联系,如内侧前额叶皮层,但至少在任何明显的程度上缺乏杏仁核的联系(图\ref{fig:3_3})。
同样的限制也适用于大脑皮层的许多其他部分。
上颞叶皮层和下颞叶皮层与杏仁核有联系——视觉区域更是如此——但与内侧前额叶皮层的运动前区域缺乏直接联系。\par



\subsection{对觅食选择的贡献}

内侧前额叶皮层帮助动物在竞争动作中做出选择。
在许多情况下,在选择时没有外部提示提示动物。
动物可以看到(或其他感觉)许多刺激,但它们都没有提供选择一种行动而不是另一种行动的基础。
在这种情况下,有将行动与其结果联系起来,动物才能在行动中进行选择。
在觅食过程中,动物通常需要在没有任何外部提示的情况下在潜在动作中进行选择。
当然,自然环境中有很多线索,但它们往往无法为该做什么提供指导。
在这种情况下,行动-结果投射提供了必要的指导。\par


内侧前额叶皮层的细胞根据数量、发生概率和努力成本对结果进行编码,所有这些因素都会进入觅食选择。
因此,无颗粒内侧前额叶皮层会使竞争动作、竞争坐标系和不同类型的联想之间的竞争产生偏差,所有这些都可能在某些情况下导致成功。
例如,当由于资源波动性增加,习惯性觅食选择变得不那么有效时,内侧前额叶皮层的一部分会对结果导向行为产生偏见。
另一部分产生了相反的偏见,在稳定的觅食环境中促进快速、自动的行为。\par


内侧前额叶皮层的颗粒部分对类人猿的觅食选择做出了额外的贡献。
关于它们的功能,我们只有几条线索:
当人们回忆自传事件或产生任务规则时,激活发生在9区的中部或额极皮层(10区);
猴子额极皮层中的细胞在反馈时间编码自我生成的目标;
额极皮层的损伤损害了猴子利用单个事件选择未来目标的能力。
这些发现指出了为目标分配结果的作用,以改善未来的选择\cite{tsujimoto2011frontal}。\par
事件的概念对这些结论很重要。我们所说的事件是指特定时间和地点的背景、目标选择、行动和结果的结合。
正如类人灵长类动物所面临的那样,根据先前的单一事件做出觅食选择可以为寻找高度波动的资源提供优势(第\ref{chap:chap2}章)。
第~\ref{chap:chap4}~章至第~\ref{chap:chap7}~章认为,单个事件在颗粒状前额叶皮层的功能中发挥着重要作用,第~\ref{chap:chap8}~章认为,利用单个事件选择目标的能力是前额叶皮层基本功能的核心。\par
本章回顾了内侧前额叶皮层根据其与价值结果的关联对选择行动做出贡献的证据。
动物可以通过在物体之间进行选择来直接或间接地选择动作。
我们认为,内侧前额叶皮层的连接使其能够在动作选择中发挥作用,尤其是当这些选择发生在没有告诉动物该做什么的外部感觉信号的情况下时。
下一章将讨论眶额皮层,它在物体选择中的作用,以及为什么内侧前额叶皮层和眶额皮层必须相互作用。\par




\chapter{眶额皮层:基于输出结果选择目的}
这本书提出了关于灵长类动物前额叶皮层基本功能的方案。

\section{概述}
眶额皮层有助于评估和选择对象未来行动的目标及其联系解释了其独特的功能。 它与嗅觉、内脏、味觉、体感和皮层的视觉区域,这些信号的结合产生了丰富的,特定结果的高维表示,一个由想象。 眶额皮层与杏仁核更新的相互作用根据当前的生物学需求评估这些成果。由于这些联系,看到有营养的食物或视觉标志与他们相关联唤起他们的味道和他们当前的动机价值。 虽然眶额皮层的很大一部分将物体链接到特定的结果,另一部分对一般结果这样做,在“共同货币”。 通过将行为结果分配给特定的觅食事件,而不是此类事件的平均值,眶额皮层灵长类动物提供了关键的选择优势。
\section{介绍}
第 3 章论证了内侧皮层的连接允许它做出选择在行动中,特别是当这些选择发生在没有外部感官的情况下时告诉动物该做什么的提示。 本章介绍眶额皮层。 它认为眶额皮层的连接允许它为外部提示做选择内侧皮层对“内部”提示的作用。\par
与内侧皮层一样,眶额皮层既有颗粒状部分,也有颗粒状部分。第 2 章论证了早期哺乳动物进化出的颗粒状部分,而颗粒状部分部分在早期灵长类动物中进化。 与第 3 章一样,我们需要利用老鼠的证据深入了解颗粒状的。 原因是一样的:我们对灵长类动物的无颗粒皮层。 也像第 3 章一样,我们需要说什么是粒度部分眶额皮层可以做到其颗粒部分不能做到的。 我们采纳这个想法灵长类动物的眼眶额皮层使用单个事件将特定结果与导致这些结果的选择:一种刺激-结果关联的形式。\par
显然,大多数动物都可以学习刺激-结果关联。巴甫洛夫条件反射依赖于它,就像仪器条件反射依赖于学习行动—结果关联一样。 除了无柄形式,我们假设所有动物物种都可以是仪器和经典条件。 所以我们面临一个挑战:我们需要说什么哺乳动物眶额皮层的颗粒部分以及颗粒部分是什么灵长类动物的眶 额皮层。\par
在本章中,我们区分了刺激-结果关联的两个方面。 首先涉及刺激和结果之间的预测关系:可能性给定刺激就会发生结果。 第二个涉及评估结果在当前生物学需求方面的价值。 这两个方面,我们分别称为关联映射和动机评估,都会影响觅食选择,并且两者需要随着情况的变化而更新。\par
\section{区域}
图 4.1 说明了人类和猴子的眶额皮层,图 2.1 说明了其细分的一个视图。 在猕猴中,第 11 区位于它的喙部,区域 13 和 14 位于更靠后的位置 (Walker 1940)。 在灵长类动物中,大部分皮质具有颗粒状细胞结构,但区域 13 和 14 的尾部是颗粒状的,邻近的前岛叶皮质也是如此。 继 Carmichael 和 Price(1994)之后,我们将所有这些无颗粒区域包括在眶额皮层中。 为了清楚起见,我们有时使用眶额皮层的缩写 OFC。 我们尤其在讨论时这样做OFC的分支机构。\par
我们将区域 12/47 的部分从此处解释的眶额皮层延伸到半球的轨道表面,以及极地皮层(区域 10)。
\begin{figure}[!htb]
	\centering
	\includegraphics{image_pfc/Fig_4_1}
	\caption{猴子(左)和人类(右)的眶额皮层。 格式如图 1.2 所示。}
\end{figure}


\section{连接}
图 4.2 说明了眶额皮层的选定连接,这表明以下结论:\par
1. 与杏仁核最密集的连接涉及眶额皮层的颗粒部分,优先与杏仁核的基底外侧核连接。然而,13 区的颗粒状部分也与杏仁核有关,其他粒度细分(见图 3.3)(Carmichael \& Price 1995a)。\par
\begin{figure}[!htb]
	\centering
	\includegraphics{image_pfc/Fig_4_2}
	\caption{眶额皮层的选定连接。 图 1.4 和 1.5 给出了脑沟和区域的名称。 线连接一些与轨道有直接轴突连接的区域,除非另有说明,否则假设是相连的。}
\end{figure}
2. 无颗粒岛叶皮层接收来自味觉和梨状(嗅觉)皮层(Carmichael \& Price 1995a)。 它还接收来自脑干和丘脑的内脏信号 (Ray \& Price 1992)。 后者包括传达反映动物新陈代谢状态的信号的感觉,例如,缺氧或低血糖,以及来自肺、心脏、压力感受器和消化道 (Craig 2002)。 这些发现导致了这样一种观点,即颗粒状岛叶皮质在内感受中发挥作用。\par
3. 嗅觉、味觉和内脏信息从其到达颗粒状 OFC颗粒部分(Carmichael \& Price 1994)。该信息可以在颗粒皮层中与来自颞下皮层的视觉输入相结合,目标区域 13和 11,以及从投射到区域 13 的鼻周皮质(Saleem 等人,2008 年)。后一种投影提供了关于物体的多模态信息 (Murray et al.
2007 年)。\par
4. S1区和S2区的体感信息也会传到13区,尤其是那些代表喙部(Pritchard et al. 1986)、唇和舌的侧面部分(Carmichael \& Price 1994)。这些联系包括不明确的体感诸如顶叶盖和岛叶颗粒异常部分的区域皮层 (Saleem et al. 2008),以及定义明确的体感区域,例如区域3b 和 S2 本身。 与 S2 的某些联系可能涉及手和口面部表征 (Carmichael \& Price 1994)。\par
5. 与内侧 PF 皮层和腹侧PF皮层(第 3 章和第 7 章)不同,眼眶PF 皮质只有有限的听觉输入 (Saleem et al. 2008)。14区的部分地区其中 13 个与可能提供听觉输入的颞叶区域有联系 (Petrides \& Pandya 1988)。 但是萨利姆等人。(2008) 重新诠释这些连接是根据他们识别的两个连接网络来表示的:轨道和中间网络。他们认为与听觉相关的区域要么是媒体网络的一部分,要么是两个网络的一部分。根据这种观点,眼眶PF皮层的“纯眼眶”部分缺乏听觉输入。原因大概是轨道网络处理有关物体(例如食物)的信息,这些物体具有很少有听觉特征。\par
这些点构成了眼眶 PF 皮层的连接指纹,这表明它是视觉信息与味觉信息汇聚的最早场所,嗅觉和内脏输入。 我们认为大多数视觉输入到达很重要。在灵长类动物特有的颗粒状区域。 因此它不是眼眶 PF 皮层,一般来说,但特别是区域 13 和 11 的颗粒状部分似乎是视觉、嗅觉、味觉和内脏输入之间会聚的最听觉位置。\par
由于这种融合,特定食物的视线,例如特定成熟的阶段的水果,可以唤起它的味道和气味,这构成了它的味道,连同摄入后的内脏感觉。 此外,眼眶 PF 皮层接收来自初级体感皮层的口、唇和舌表征的输入(S1):与食物和液体摄入最相关的身体部位(Carmichael
\& 价格 1995b)。\par
它的连接解剖结构使眼眶 PF 皮层处于一个独特而有趣的位置。 触觉输入提供有关外部环境附近部分的信号; 视觉和嗅觉输入传递来自外部世界遥远部分的信号; 内脏输入告诉动物有关其内部环境的信息; 味觉和口腔体感输入告诉它有关从外部进入内部世界的事物。\par
图 4.3 显示了这些不同的模态和子模态如何组合起来形成联合表征。 其中一些连接发生在无颗粒区域,它们似乎非常适合早期哺乳动物的谷物和昆虫饮食。 这些是专门从事夜间觅食以避免捕食等因素的小动物。 因此,在眼眶 PF 皮层的颗粒部分中表示的连词主要涉及味觉、嗅觉和内脏感觉。 灵长类动物灵长类动物保留了这种基本的特征连接机制,并增加了更大的来自视觉区域的输入。\par
\begin{figure}[!htb]
	\centering
	\includegraphics{image_pfc/Fig_4_3}
	\caption{颞叶和额叶皮层的特征连接。 VisA …VisD 指定视觉
		对象的特征,可以组合成各种连接表示,例如 VisAB,
		这表示 VisA 和 VisB 的表示。 “STV”表示物体的气味 (S)、味道 (T) 和内脏 (V) 特性的结合。 转载自 Murray EA, Wise SP,
		Rhodes SE,不同的大脑可以用奖励做什么? 在感觉神经生物学和
		奖励,编辑。 JA Gottfried © 2011,Taylor 和 Francis,经许可。}
\end{figure}

\subsection{概括}
第 3 章认为——通过与海马体、杏仁核和内侧运动前区的联系——内侧 PF 皮层会偏向于在动作之间或在动作之间的选择有关行动的规则。 在那里,我们审查了证据表明它是基于预测结果的当前值这样做的,尤其是当这些选择取决于“内部”信号而不是感官信号时。 本章认为眼眶 PF 皮层执行在外部信号之间进行选择的类似功能,尤其是那些来自物体的信号。 它之所以能够这样做,是因为来自体感、味觉、嗅觉、内脏和视觉皮层以及杏仁核的信息会聚和整合。
\section{啮齿动物的无颗粒皮质}
颗粒状 OFC 在啮齿动物研究中引起了相当大的关注,其中大部分表明在以结果为导向的行为中起着重要作用。 回想一下,在整本书中我们都区分了目标和结果,因此在其他人可能会说目标导向的地方使用术语“以结果为导向”。 OFC细胞的活动反映了结果预测,尤其是关于奖励的特定感官方面的预测(Schoenbaum 等人,1998 年),这些区域的损伤会损害基于以下因素的选择
结果预期,如下一节所述。
\subsection{刺激-结果关联}
正如第 3 章提到的,当前的生物学需求会影响结果的评估。例如,当前对食物或液体的需求越大,a 的值就越大映射到该结果的刺激。 将食物消耗到饱腹感会使该食物贬值,并且还有其他方法可以改变结果的价值。\par
在使用这些替代方法之一降低食物奖励价值的实验中,加拉格尔等。 ( 1999 ) 比较了具有颗粒状 OFC 损伤的大鼠的行为与正常大鼠相比。 首先,他们教老鼠光表示食物供应。当灯亮时,老鼠会靠近灯。 然后加拉格尔等人。 用过的锂氯化物诱发胃肠道疾病,这一过程称为条件性味觉厌恶。稍后进行测试时,在所有疾病痕迹消失后,正常老鼠接近光线比他们生病前更少,或者完全停止接近光线。受伤的老鼠表现不同。 他们比过去更频繁地接近光正常老鼠。 正如第 3 章所解释的,正常大鼠的结果称为贬值影响。 因为受损的老鼠接近了与奖励贬值相关的刺激,可以说,他们在使用刺激-结果关联来判断方面存在缺陷指导行动。\par
皮肯斯等人( 2003 , 2005 ) 使用相同的任务和贬值程序,但对杏仁核和颗粒状 OFC 造成损伤。 两个病变都废除了大鼠的贬值效应。 这些老鼠的行为就好像它们没有从它们身上学到任何东西一样疾病。\par
这些贬值程序的结果类似于第 3 章描述的结果对于内侧 PF 皮层的病变。 内侧 PF 皮层损伤会导致使用结果预测在竞争行为中进行选择时出现障碍。 在许多这样的实验中,没有任何外部提示可以帮助动物做出选择。 颗粒状 OFC 病变,但是,不要破坏这种行动-结果关联(Ostlund \& Balleine 2007)。
相反,它们破坏了使用关于结果的预测来在对象中进行选择。在这些实验中,外部提示会提示选择,从而揭示缺陷学习刺激-结果协会。 这些研究表明,颗粒状 OFC将刺激与结果联系起来,尤其是食物的感官方面,例如它的气味
或味道。\par
伯克等人的一项实验。 ( 2008 ) 支持这一结论。 老鼠首先学会了将一种刺激与一种特定的食物联系起来。 后来他们看到了这种刺激与额外刺激的结合。 当老鼠看到这种复合刺激时,它们得知它与另一种食物有关。 这两种食物有更多或不太一样的适口性和可取性,但它们的味道不同。\par
伯克等人发现大鼠将第二种食物的发生归因于复合刺激的较新部分。 作为这一结论的证据,他们表明他们的老鼠会按下一根杆来产生额外的刺激。 至关重要的是,Burke等人表明如果老鼠在第二次吃饱了就不太可能这样做食物。 因此,老鼠不仅将额外的刺激与第二种食物联系起来,而且使用该关联来评估刺激的当前值。具有颗粒状 OFC 损伤的大鼠未能显示出这些效果。 关系
在复合刺激的较新部分和特定感觉特性之间第二种食物不再影响他们的行为。 这些结果表明颗粒状 OFC 介导刺激和结果之间的映射,尤其是
结果的特定感官方面。\par
稍后,我们强调了粒度 OFC 的重要性和视觉上的进步
灵长类动物。 然而,正如刚才回顾的实验所表明的那样,老鼠也使用视觉
评估结果的刺激,他们的 OFC 从皮质的视觉区域接收到这些刺激。\par
\subsection{费用}
第 3 章解释了动物不仅根据预测的食物或液体做出决定,而且还有获得它们的成本。 例如,与正常大鼠相比,前扣带回晚期皮质受损的大鼠选择越过障碍的次数较少,因此似乎高估努力成本。\par
无颗粒眼眶 PF 皮层的损伤也改变了大鼠估计成本的方式。当在立即的小奖赏和延迟的大奖赏之间做出选择时,正常老鼠在做出选择时会考虑延迟的时间长短。 鲁德贝克等( 2006b ) 报道有眼眶 PF 损伤的大鼠更多地选择小的即时奖励经常比正常老鼠。“冲动”一词已被应用于选择小额奖励很快,并且“耐心”一词已被用于放弃立即奖励以获得一个更大的以后。 所以在 Rudebeck 等人的实验中,眼眶 PF 皮质病变可以据说会诱发冲动。\par
然而,只要稍微改变实验设计,就会得到不同的结果。鲁德贝克等使用了 T 型迷宫,但 Winstanley 等人( 2004 ) 让老鼠在两个杠杆之间进行选择,一个导致单个颗粒,另一个在延迟后导致四个颗粒。 眼眶 PF 病变的大鼠比正常大鼠更频繁地选择延迟奖励的杠杆。有人可能会说他们表现出比正常人更多的“耐心”。 和马里亚诺等人。 ( 2009 ) 让老鼠在 T 迷宫上的黑色或白色目标框之间做出选择,一个框内有小奖励,另一个框内有大量奖励,只有在延迟后才能获得。同样,患有眼眶 PF 病变的大鼠比 Rudebeck 等人研究中的大鼠表现出更多的“耐心”。\par
泽布等人 (2010) 表明,眼眶 PF 病变是否会导致“冲动”或“患者”选择取决于两个因素:一个明确的信号,表明个体大鼠之间的延迟和差异。 他们使眼眶 PF 皮层失活并比较了提示的效果和无意识的延误。 他们的结果表明,对于明显有提示的延迟,失活会增加立即奖励的选择(冲动),而对于无提示的延迟,失活会减少立即奖励的选择(耐心),但仅限于有强烈冲动倾向的老鼠个体。所以我们不能简单地说大鼠眼眶 PF 皮层偏向于冲动或耐心觅食的选择。 但是,它显然以某种方式在评估延迟成本方面发挥了作用以及偏向延迟或立即行动的行为。 正如第 3 章所解释的那样,累加器-跑道模型提供了一种实现这种偏差的简单机制,既可以通过改变产生输出的阈值,也可以通过调制速率
支持进行特定运动的“证据”不断积累. \par
冲动觅食和耐心觅食之间的竞争通常是根据延迟或远距离获得的食物和液体的贬值或折扣来讨论的。 然而,正如斯蒂芬斯等人( 2004 ) 指出,这个术语意味着动物错误地评估了时间和空间上遥远的食物和液体的价值。或者,动物可能会准确评估价值,但会考虑寻找遥远可用资源的固有风险。 Hayden 和 Platt (2007) 建立了一个模型,其中“风险”期权的估值取决于较大收益的预期时间和收益减少的风险。 他们表明,该模型的预测解释了猴子在赌博任务中做出的选择。 \par
由于等待更长时间或走得更远所固有的风险,选择开发立即可用的资源并不一定意味着对延迟或更远资源的错误评估。 即使眼前的补丁比别处的“更绿的牧场”价值低,回报更确定。
\subsection{活动和激活}




\subsection{结论}



\chapter{尾侧前额叶皮层:搜索目标} \label{chap:chap5}

尾侧前额叶皮层有助于通过显性注意力(眼球运动)和隐性注意力对食物和食物迹象等物体进行视觉搜索,它的连接解释了它如何执行这些功能。
尾侧前额叶皮层,包括额叶视区,与视觉皮层的背侧流和腹侧流以及脑干动眼神经核都有联系。
明显的注意依赖于它与脑干动眼核的连接,直接或间接地通过上丘和基底神经节。
隐蔽注意力依赖于增强的感觉反应,这种反应是通过与视觉皮层以及其他感觉区域的相互作用来调节的。
在早期灵长类动物中,随着眶额皮层的颗粒状部分,尾侧前额叶皮层也在进化(第~\ref{chap:chap2}~章)。
这两个新区域一起导致了在精细分支生态位的杂乱环境中发现、关注和评估物体的改进。



\section{介绍}

在前一章中,我们认为眶额皮层根据当前的生物需求对物体赋值。
本章提出,尾侧前额叶皮层搜索这些物体,它是通过隐蔽地注意周围目标和将眼睛朝向这些目标来实现的。


本章的大部分内容都是关于视觉和眼球运动的,它们在脊椎动物历史的早期就已经进化出来了。
眼睛和眼外肌肉的证据出现在最古老的脊椎动物和前脊椎动物化石中,有些可以追溯到500多万年前\cite{shu2003head}。
但是灵长类动物在视觉和眼球运动方面有一些重要的创新,比如发展出了中央凹和三色视觉(第~\ref{chap:chap2}~章)。
如果我们的结论——尾侧前额叶皮层首先出现在早期灵长类动物身上——正确,那么它比中央凹和全彩视觉都要早:这是关于其功能的重要线索。


为了理解尾侧前额叶皮层,我们首先需要看看它的连接是如何允许灵长类的前额叶皮层使用眼球运动和隐蔽注意力来搜索食物等物体的。



\section{区域}

在猕猴中,尾侧前额叶皮层指的是位于弓状沟膝侧的皮层。
图~\ref{fig:fig_5_1}~描绘了它的位置。


\begin{figure}
	\centering
	\includegraphics[width=0.7\linewidth]{chap5/Fig_5_1}
	\caption{猕猴(左)和人类(右)的尾侧前额叶皮层。
		格式如图~\ref{fig:1_2}~所示。}
	\label{fig:fig_5_1}
\end{figure}


正如我们所定义的那样,尾侧前额叶皮层总是包括第8区,为了本章的目的,它还包括猕猴主沟的尾侧部分。
我们通过注意到,正如在第8区域\cite{chafee1998matching},主沟尾部的大多数细胞调节其活动与眼球运动相关\cite{tanila1993regional}来证明这一分组是正确的。
Petrides\cite{petrides1999dorsolateral}发现了一个位于主沟尾端附近的区域,他们称之为9/46,他们将该区域与相邻的吻侧中外侧前额叶皮层(46区)和背内侧9区区分开来。
顾名思义,Petrides和Pandya认为9/46区具有与9区和46区相似的细胞结构特性,并且这三个区域都具有颗粒状的细胞结构。
我们称主沟尾侧皮层为后外侧前额叶皮层(图~\ref{fig:1_4}),目前将其包括在前额叶尾侧皮层中。
然而,我们承认,在不违反任何解剖学原理的情况下,可以将其包括在前额叶皮层背侧(第~\ref{chap:chap6}~章)。
表~\ref{tab:1_2}~使用一个查询标记(“?”)来表示这两个选项。
因此,关于后外侧前额叶皮层的许多观点都适用于本章和下一章。


在猕猴中,通过微刺激弓状沟靠近主沟尾端的吻侧岸,可以诱发眼跳运动\cite{bruce1985primate},这一特性定义了额叶视区。
更高的电流可以通过电流传播,从更大的区域唤起跳视\cite{robinson1969eye},但人们普遍认为低阈值区域对应于额叶视区。
因此,区域8包括额叶视区,它从典型的颗粒状细胞结构向非颗粒状细胞结构变化\cite{stanton1989cytoarchitectural}。


Amiez\cite{amiez2009anatomical}回顾了通过电刺激在人脑中定位额叶视区的研究。
来自中央前上沟吻侧的低阈值刺激,以及来自中央前上沟上方的低阈值刺激,可以诱发眼跳。
Amiez等人\cite{amiez2006local}使用成像方法来定位与个体受试者皮层解剖相关的激活峰值。
按照这样的定义,额叶视区始终位于中央前上沟的腹侧分支,这个位置与电刺激所定义的位置大致一致。
Amiez\cite{amiez2009anatomical}提供了猕猴和人类的地图,并表明在这两种情况下,额叶视区都可以与运动前皮层区分开来,电刺激也可以唤起眼球运动。


微刺激还在猕猴的额叶中发现了第二个眼场:\textit{辅助视区}\cite{schlag1987evidence}与额叶视区一样,\textit{辅助视区}中的细胞在扫视前增加活动\cite{hanes1995relationship}。
在猕猴中,\textit{辅助视区}位于背内侧额叶皮层的第6区\cite{schlag1987evidence},它在人脑中有类似的位置\cite{amiez2009anatomical}。



\section{连接}

图~\ref{fig:fig_5_2}~显示了猕猴的尾侧前额叶皮层的皮质连接,包括额叶视区。
这些数据主要来自Petrides\cite{petrides1999dorsolateral},他们将示踪剂注入区域8的细分(区域8B、8Ad或8Av),并描述了它们之间的联系。
这项研究比早期的研究更有优势\cite{petrides1984projections,barbas1988anatomic,barbas1989architecture,cavada1989posterior}进行了小规模和相对选择性的注射。


\begin{figure}
	\centering
	\includegraphics[width=0.7\linewidth]{chap5/Fig_5_2}
	\caption{末梢前额叶皮层的选定连接。
		图~\ref{fig:1_4}和图~\ref{fig:1_5}~给出了沟和区域的名称。
		这些线连接了一些与尾侧前额叶皮质有轴突直接连接的区域,除非另有说明,假设是相互的。}
	\label{fig:fig_5_2}
\end{figure}


根据连接得到以下结论:

\begin{enumerate}
	\item 8Ad区、8B区和后外侧前额叶皮层与执行眼肌运动和视觉空间功能的区域相连。例如,它们与位于顶骨内沟的LIP区有联系\cite{cavada1989posterior,andersen1990corticocortical}, LIP中的细胞编码眼球运动\cite{snyder1997coding}。
	同样的前额叶区域与下顶叶皮层的PG区域有连接\cite{cavada1989posterior},该区域的许多细胞编码眼睛方向\cite{sakata1980spatial}。
	最后,8Ad区与颞区MST区有联系,在MST区,细胞对视觉刺激的运动做出反应\cite{celebrini1995microstimulation}。
	\item 这些视觉区域构成了背侧视觉流的一部分\cite{milner2006visual}(Ungerleider和Mishkin 1982),以区别于腹侧视觉流。
	一般来说,背侧流包括后顶叶区域,处理有关动作空间目标的信息,腹侧流包括颞下区域,处理有关视觉刺激的颜色、形状和纹理的信息。
	\item 额叶视区接收早期(低阶)视觉区域的信息,如枕部视觉区V2和V3\cite{stanton1995topography},但8Ad区和后外侧前额叶皮层不接收信息。
	因此,额叶视区接收到的高度处理视觉信息比前额叶皮层尾端的其他部分和前额叶皮层的其他部分要少。
	\item 8Av区不同于8Ad区、8B区和后外侧前额叶区,分别与腹侧流和背侧流有联系。8Av区与TEO区有联系\cite{webster1994connections},也与颞下皮层的其他部分有联系,如颞上沟下岸的皮层\cite{petrides1999dorsolateral}。
	\item 尾侧前额叶皮层各组成区域之间相互连接紧密。
	8Ad区和8Av区彼此相互投射,并与后外侧前额叶皮层(9/46区)相互投射。
	这些相互联系支持我们将后外侧皮层纳入本章。
	如图~\ref{fig:1_8}~所示,我们所定义的尾侧前额叶皮层与Price\cite{price2010neurocircuitry}所定义的尾侧网络非常相似。
\end{enumerate}



图~\ref{fig:fig_5_2}~和前面的列表涉及皮质连接,但尾侧前额叶皮层的皮层连接也解释了其功能的一些重要内容:


\begin{enumerate}
	\item 额叶视区\cite{kunzle1976projection,huerta1986frontal}、8区其余部分\cite{fries1984cortical}和后外侧前额叶皮层\cite{selemon1988common}都向上丘发送直接投影。
	在所有脊椎动物中,上丘及其同源体都有定位头部感受器的功能。
	因此,这些皮质连接指向了尾侧前额叶皮层在控制眼球运动方面的作用,但大脑皮层的许多其他部分也投射到上丘\cite{leichnetz1981prefrontal,fries1984cortical},所以这种解剖特征并不能将前额叶皮层与其他区域区分开来。
	\item 额叶视区还可以通过向基底神经节的投射影响上丘的活动。
	额叶视区投射到尾状核的内侧\cite{stanton1988frontal},后者又投射到黑质网状部\cite{hedreen1991organization}。
	该核投射到上丘,在那里发挥抑制作用\cite{hikosaka1985modification}。
	\item 最后,额叶视区直接投射到脑干动眼肌核\cite{segraves1992activity,yan2001overlap}。
\end{enumerate}



\subsection{总结}

尾侧前额叶皮层的连接指纹表明,它接受直接的、较低的视觉输入,它具有与背侧视觉流和腹侧视觉流平行的背侧-腹侧差异,并且它通过基底神经节到上丘的投射直接或间接地输出到动眼神经核。



\section{额叶视区是前额叶区域}

尽管有输出表明额叶视区在控制眼球运动中起作用,但我们并不认为额叶视区主要在眼球运动控制中起作用。
我们知道,当药物抑制额叶视区时,视觉引导的扫视在被引导到对侧空间时变得不准确\cite{sommer1997reversible}。
在这个意义上,额叶视区类似于前运动皮层。
例如,腹侧前运动皮层的失活会导致肢体运动不准确\cite{kurata1994differential}。


我们也知道额叶视区的永久性损伤不会消除眼跳运动,就像前运动皮层病变不会消除肢体运动一样。
然而,一个原因是除了其他区域外,\textit{辅助视区}\cite{huerta1990supplementary}和顶叶区域LIP\cite{holloway2002brief}也向上丘发送投射。
因此,为了消除眼跳,需要同时去除上丘和额叶视区\cite{schiller1979effects,schiller1987effect}。
只剩下最小的扫视。


然而,尽管有这些运动功能的证据,我们认为额叶视区是前额叶区域,而不是运动前区域。
我们这样做是因为我们区分了产生、发现和关注目标的机制和实现目标的机制。
记住,我们所说的目标指的是对象或地点,而不是奖励或结果。
实现目标的行动会产生结果。
除了在实验室,视觉注视和注意力永远不会产生这种意义上的结果。
在野外,看着或吃着一种食物并不会产生任何营养或补水的好处。
实现这一预期结果需要其他机制。
我们认为前者是前额叶皮层的功能,后者是前运动皮层的范围。


Shadmehr\cite{shadmehr2004computational}在他们对前运动皮层的治疗中提出,它处理的是视运动转换,将伸手运动的视觉目标转换为关节角度和力的变化,从而驱动手到达这些目标。
第~\ref{chap:chap2}~章提到了这种机制。
详细信息可以在Shadmehr和Wise中找到,但这里有一个简短的总结。


假设有人伸手拿咖啡杯。
基本上,运动系统需要建立两个位置:目标位置和手的初始位置。
Shadmehr和Wise提出后顶叶和前运动皮层的细胞在一个坐标框架中编码这些位置,在其原点,视觉注视点。
然而,在理论上,任何视网膜坐标都可以作为原点。
图~\ref{fig:5_3}~显示了这种机制是如何工作的。
图中显示了两个向量:一个是尾部在注视点,尖端在运动目标处;
另一种是尾部在固定点,尖端在手的当前位置。
两个向量的简单相减就得到了一个向量,它的尾部在手上,顶端在目标上。
这个计算的结果相当于一个“运动计划”,将视觉参考系转换为以手为中心的坐标系。
前运动区和初级运动区,连同皮层下结构,将这个矢量转换成关节角度的变化和使这个运动的力量。


\begin{figure}
	\centering
	\includegraphics[width=0.55\linewidth]{chap5/Fig_5_3}
	\caption{运动平面的矢量表示。
		这个人的目标是把他的指尖穿过咖啡杯的把手。
		他或她制定了一个读书计划。
		“+”标志着他当时的注视点。
		来自注视点的两个向量编码目标的位置和手指在非中心的外部坐标框架中的当前位置:注视中心坐标。
		两个矢量之间的差表示以手为中心坐标的电机平面\cite{shadmehr2004computational}。}
	\label{fig:5_3}
\end{figure}


通过这种机制或类似的机制,前运动区和后顶叶区以一种眼球运动或其他形式的注意力都无法实现的方式实现目标。
在我们的例子中,这个人想要咖啡,所以把他或她的注意力转移到杯子上。
注意力可以是显性的(即中央凹朝向杯子),也可以是隐性的(即只关注杯子而不看它)。
杯子里的东西与人当前的需求有关,因此他或她产生了一个目标(咖啡杯),然后搜索并将注意力转移到杯子上。
我们认为,尾侧前额叶皮层执行搜索和注意功能。
虽然视觉固定和注意力不能达到预期的结果,但伸手到杯子里,把它送到嘴里,从里面喝水就可以了。
我们将后一种功能视为前运动区和初级运动区,以及运动系统的其他部分。


那么,有什么证据表明额叶视区参与了对目标的寻找和关注,而不是实现目标的手段?
我们已经看到,与运动前区域不同,额叶视区接收来自低阶视觉区域的直接输入,特别是枕叶和颞叶皮层称为V2、V3、V4和MST的区域\cite{stanton1995topography}。
这些联系解释了为什么额叶视区中的一些细胞在感觉刺激出现后表现出活动增加,而在运动之前却没有\cite{schall1991neuronal}。
它们似乎明确了视觉目标,而不是注视这些目标所需的眼球运动。
此外,在顶叶区LIP投射到额叶视区的细胞中,78\%有视觉反应,但没有扫视相关活动\cite{ferraina2002comparison}。
这种投影可以提供另一种关于视觉目标的信息来源,而且是独立于运动指令的信息来源。


当然,额叶视区中的许多细胞具有视觉运动活动:它们在视觉目标出现时和眼球运动之前调节自己的活动\cite{schall1991neuronal}。
其中很多都是在刺激出现后不久,即刺激引起注意时,指定刺激的位置。
Sato\cite{sato2003effects}教猴子在一系列干扰物中发现一个视觉弹出刺激,然后对该位置进行扫视(prosaccade试验)或对相反方向的扫视目标进行扫视(反扫视试验)。
如果活动反映了刺激的位置,那么两种试验类型的活动应该是相同的。


Sato和Schall比较了两项任务中额叶视区细胞的活动,当弹出的刺激落入细胞的接受野时。
对于反眼跳任务,注意当弹出刺激这样做时,眼跳目标是在相反的方向。
然而,57\%的任务相关细胞的活动最初反映了刺激的位置,尽管随后86\%的这些细胞后来编码为扫视目标。


这些结果表明,额叶视区中的细胞活动可以反映刺激的位置,独立于运动,并且这种活动反映了隐蔽注意的方向。
与这一观点一致,Armstrong\cite{armstrong2007rapid}表明,额叶视区的皮层内微刺激增强了视觉区V4(腹侧视觉流的中层区域)的细胞响应。
它是专门为视觉空间的一个特定部分做的。
刺激额叶视区的效果可能模仿了猴子在该位置秘密关注物体时所发生的情况。


如果额叶视区真的在隐性注意和显性注意中起作用,该区域的暂时失活应该会导致对中央凹外刺激的注意受损。
因此,Wardak等\cite{wardak2004deficit}教猴子在干扰物中发现目标,而动物则保持对中心光点的固定。
在额叶视区失活后,猴子发现周围目标的速度很慢。
Iba\cite{iba2003involvement}表明,失活也会导致猴子对目标进行扫视的速度变慢。
这些发现表明,额叶视区在对刺激的公开注意和隐蔽注意中都有作用,特别是当这些刺激作为后续行动的目标时。


这一观点与第~\ref{chap:chap2}~章中关于前额叶进化的描述非常吻合。
Strepsirrhine灵长类动物没有中央凹,早期灵长类动物可能也没有。
因为包括额叶视区在内的尾侧前额叶皮层在早期灵长类动物中进化而来,它一开始不可能与中央凹或中央凹有任何关系。
所以,从这个意义上说,这些动物的所有视觉都对应于中央凹外视觉,所有的注意力都是隐蔽的。
顺畅的眼球运动可以让类人猿灵长类动物锁定中央凹,将注意力集中在移动的物体上,并保持高分辨率图像。
但这种能力也可能在缺乏中央凹的灵长类动物中进化而来\cite{shepherd2009neuroethology}。
因此,从中央凹视觉或明显的注意力来解释灵长类动物的前额叶皮层的起源是错误的,尤其是尾侧前额叶皮层的起源。


然而,中央凹最终在后来的灵长类动物中进化出来,现代眼镜猴、猴子、猿和人类(单足纲)通过遗传拥有它。
尽管有很多优点,但集中注意力的能力是有代价的。
那视觉世界的其余部分呢?
这个混乱的世界还包含许多其他突出的项目。
在某种意义上,没有中央凹的视网膜提供了一个更平衡的世界观。
即使在中央凹进化之后,保持隐蔽注意力的好处是,它可以增强对有限数量的中央凹外物体的处理,即使最密集的处理是用于中心凹的物体和地方。
隐蔽注意力和搜索的重要性在于,所有被关注的对象,而不仅仅是被聚焦的对象,都可能成为未来行动的目标。
根据这一观点,包括额叶视区在内的尾侧前额叶皮层进化为隐蔽的搜索和注意力,但后来一旦中央凹出现,就适应了公开的注意力。



\subsection{总结}

许多神经科学家将额叶视区视为眼球运动区域,并将其视为眼球运动的前运动区域。
我们提出一个不同的想法。
我们将额叶视区和尾侧前额叶皮层的其他部分视为前额叶区域,而不是前运动区域,并提出它们在寻找和关注灵长类动物重要目标方面发挥作用。
在许多灵长类动物中,包括猴子和人类,对目标的关注通常意味着眼球运动,使中央凹朝向目标(扫视)或在目标移动时保持视觉固定(平稳跟踪)。
但是隐蔽的注意力也扮演着重要的角色,在缺乏中央凹的早期灵长类动物中,它肯定是这样的。
由于注视是注意力的一种形式,我们将眼球运动视为一种注意力功能,而不是一种运动功能。
我们提出,包括额叶视区在内的尾侧前额叶皮层的功能是将公开或隐蔽的注意力引向一个目标,而不是作为实现目标的机制的一部分。



\section{动眼肌延迟响应任务}

我们将额叶视区与尾侧前额叶皮层的其余部分分开处理,因为额叶视区皮层内微刺激会引起眼球运动。
但是,正如在连接部分所解释的那样,额叶视区与尾侧前额叶皮层的其他部分有紧密的连接。
因此,8区\cite{chafee1998matching}和后外侧前额叶皮层\cite{funahashi1989mnemonic}的许多细胞具有与眼球运动相关的活动。


正如我们前面所说的,后外侧前额叶皮层的连接使我们在本章中将其与包括额叶视区在内的后外侧前额叶皮层一起考虑。
我们在第~\ref{chap:chap2}~章中对皮层进化的讨论为这种方法提供了一些可信度,后外侧前额叶皮层中存在编码眼球运动的细胞。
由于后外侧前额叶皮层(9/46区)处于中间状态,在本节中,我们将讨论在第~\ref{chap:chap6}~章中再次出现的主题,最显著的是被称为“空间记忆任务”的任务中的延迟期活动问题。
我们这样做主要是为了方便,并认为没有什么关键取决于我们是否将后外侧前额叶皮层与尾侧前额叶皮层组或背侧前额叶皮层组进行分类。


动眼力延迟响应任务在包括后外侧前额叶皮层在内的尾侧前额叶皮层的研究中发挥了突出作用。
它不同于经典的延迟响应任务,因为猴子通过扫视而不是到达一个空间目标来选择潜在的目标。


\begin{figure}
	\centering
	\includegraphics[width=0.75\linewidth]{chap5/Fig_5_4}
	\caption{普通版本的动眼器延迟响应任务。
		在试验过程中,每个面板在不同的时间显示屏幕,在其他试验中,未填充的白色方块是潜在的空间目标。
		填满的白色方块表示一个例子试验的线索,填满的白色圆圈表示注视点。
		灰色虚线和灰色箭头表示猴子的注视点。左下方的插入图显示了空间记忆信号,由Lebedev等人\cite{lebedev2004representation}从前额叶皮层细胞群中记录,灰色为首选记忆位置的平均活动,黑色为反首选位置的平均活动。
		因为Lebedev等人控制了注意力,这些数据代表了前额叶皮层空间记忆信号的最有力证据,尽管它们只出现在延迟期间编码位置的少数细胞中。}
	\label{fig:fig_5_4}
\end{figure}


动眼肌延迟反应任务有三个阶段:提示、延迟和选择(图~\ref{fig:fig_5_4})。
在第一阶段,一个视觉空间线索会短暂地出现在受试者视野的某个地方。
这个刺激指示了猴子必须扫视的位置,但直到延迟期结束,猴子必须继续盯着一个中心光点。
在一段从几百毫秒到几秒不等的延迟时间后,一个“开始”信号告诉受试者将扫视移动到最近提示的位置。
如果被试的扫视准确,就会得到奖励。


尽管实验人员可以使用任何提示位置的配置,但常见的版本包括以注视点为中心等距排列的八个显著位置,如图~\ref{fig:fig_5_4}所示。
该图将位置显示为空方格,并表示球杆为填充方格。
然而,重要的是要注意,正如通常所呈现的那样,外围位置没有以任何方式标记,这意味着当猴子响应“go”信号时,它会扫视到屏幕上未标记的位置。


在这项任务中,尾侧和后外侧前额叶的许多细胞表现出延迟期活动\cite{chafee1998matching}。
图~\ref{fig:fig_5_4}~显示了Lebedev等人\cite{lebedev2004representation}在种群水平上从尾侧前额叶皮层记录的记忆信号。


Funahashi和他的同事\cite{funahashi1989mnemonic,takeda2002prefrontal}展示了他们的电极穿透皮层的位置映射,我们假设这些细胞延伸到整个采样区域。
Lebedev等人的研究表明,具有记忆特异性信号的细胞位于主沟尾端9毫米处,主要位于主沟背侧及其附近。
Chafee\cite{chafee1998matching}报告说,延迟期活动主要来自8A区,但他们的图表显示,一些细胞也来自后外侧前额叶皮层。


成像实验在类似领域也得到了结果。
Inoue等\cite{inoue2004functional}比较了眼肌运动延迟反应任务中的眼跳激活与猴子对屏幕上标记的位置进行眼跳的任务中的激活。
差异激活位于后外侧前额叶皮层和8区,包括额叶视区。


鉴于以下事实,这是一个重要的观察,下一章将解释,在经典的、达到版本的延迟响应任务上的表现取决于中外侧前额叶皮层,而不是尾侧前额叶皮层。
因此,动眼肌延迟响应任务不同于经典的延迟响应任务,基于动眼肌版本任务的结论不一定适用于经典的延迟响应或延迟交替任务。
后外侧前额叶损伤对这些任务的经典版本缺乏影响\cite{butters1969retention},这强化了这样一种观点,即该区域在目标的寻找和关注中发挥作用,而不是在目标的实现(前运动皮层功能)或目标的产生(背侧前额叶皮层功能,如第~\ref{chap:chap6}~章所述)。


\begin{figure}
	\centering
	\includegraphics[width=0.75\linewidth]{chap5/Fig_5_5}
	\caption{di Pellegrino\cite{di1993visuospatial}使用的空间匹配样本任务。
		格式如图~\ref{fig:fig_5_4}~所示。
		在底部面板上,箭头表示在两种情况下达到运动的目标。
		在一种情况下,提示的位置是伸手动作的目标;
		在另一种情况下,无论球杆的位置如何,90°的位置(图中向上)始终是到达目标。
		左下方的插图显示了前额叶皮层中一个细胞的活动,该细胞在延迟期间编码最左边的线索(180°),在两种情况下都是一样的。}
	\label{fig:fig_5_5}
\end{figure}


\subsection{解释延迟期活动}

传统的解释是,眼肌运动延迟响应任务中的延迟期活动,就像延迟响应任务的标准版本一样,反映了回溯性空间工作记忆。
根据这种观点,这种活动反映了在短期记忆中线索位置的维持。
这种解释部分来自于将该任务称为“空间记忆任务”,并因此假设在延迟期间发生的主要认知过程是感官信息的维持。
事实上,这种观点的支持者经常把延迟期称为“记忆期”。
但是,第~\ref{chap:chap1}~章警告不要接受任务的名称。
事实上,在延迟期间还会发生其他过程,包括对提示位置的隐蔽注意和运动的准备。


有几种方法可以区分这些可能性,我们将它们进行了编号:


\begin{enumerate}
	\item 我们可以比较动眼肌延迟响应任务的两个版本。因此,Funahashi等人\cite{funahashi1993prefrontal}比较了一个版本,其中猴子必须对目标进行扫视(prosaccade),
	另一个版本中猴子必须对距离线索180°的位置进行扫视(反扫视)。
	在后来的一项研究中,Takeda\cite{takeda2002prefrontal}比较了标准的动眼肌延迟反应任务和猴子必须以90°的角度扫视该位置的条件。
	这些实验有如下逻辑:如果延迟期活动编码了线索在记忆中的位置,那么无论运动的目标是什么,它都应该是相同的。
	在这两项研究中,大多数与任务相关的细胞反映了线索的位置,尽管有相当一部分反映了目标的位置。
	Funahashi等人报告说,51个与任务相关的细胞中有31个(61\%)编码了线索的位置,而这51个细胞中有13个编码了目标的位置。
	尽管细胞样本很小,但这些发现使作者得出结论,延迟期活动反映了对提示位置的记忆。
	\item 第二种方法试图排除这种可能性,即通过使用单一的、重复的手部运动来报告线索的位置,而不是移动到该位置,从而对非感官因素进行编码。
	Sawaguchi\cite{sawaguchi1999properties}使用了这种方法。他们在延迟期后呈现一个空间视觉线索,并要求猴子只有在与线索位置匹配的情况下才释放杠杆。
	读者会认为这是匹配样本任务的空间版本。
	在与任务相关的神经元中,48\%显示出延迟期活动,这些细胞中90\%编码了提示位置。
	这些结果也被用来支持延迟期活动在空间记忆方面的解释。
	\item 正如di Pellegrino\cite{di1993visuospatial}所做的实验那样,上述两种方法可以结合起来。
	实验设计如图~\ref{fig:fig_5_5}~所示。
	在Sawaguchi\cite{sawaguchi1999properties}的实验中,猴子需要报告延迟期结束时的线索是否与延迟前出现的线索的位置相匹配。
	然而,猴子以不同的方式报告匹配的刺激,如图~\ref{fig:fig_5_5}~底部的两个箭头所示。
	这种报告要求从一个试验块到下一个试验块交替进行。
	在一个街区,猴子向提示的位置移动。
	在下一个街区,不管线索和匹配的刺激出现在哪里,猴子都到达一个固定的位置。
	有时,提示方向和报告方向之间的夹角等于180°,如Funahashi等人\cite{funahashi1993prefrontal}的反眼跳任务,有时等于90°,如Takeda\cite{takeda2002prefrontal}的实验。
	
	Di Pellegrino和Wise记录了尾侧和后外侧前额叶皮层,并发现,与Funahashi等人一样,无论运动的目标如何,61\%的细胞具有相同的延迟期活动(图~\ref{fig:fig_5_5})。
	这一结果也可以表明,细胞的活动编码了对提示位置的记忆。
	\item 所有这些研究都基于这样一个假设:只有两个重要因素是线索的位置和移动目标的位置。将延迟期称为“记忆期”,人们很容易忽略另一种可能性,即与记忆一样,延迟期也涉及注意力:有时是显性的,有时是隐性的。
	为了测试这种可能性,Lebedev等人\cite{lebedev2004representation}引入了一种新的实验设计,如图~\ref{fig:fig_5_6}A~所示。
	正如第~\ref{chap:chap1}~章所提到的,他们教猴子记住一个空间位置,并秘密地注意到另一个位置,同时这样做。
	
	Lebedev等人记录了尾侧和后外侧前额叶皮层的细胞活动。
	在对空间位置进行编码的任务相关细胞中,大多数(61\%)选择性地对参与的位置进行编码,而不是对记忆中的位置进行编码。
	少数(16\%)细胞选择性地编码记忆位置,其他细胞具有中间属性(图~\ref{fig:fig_5_6} B)。
	注意信号在每个测量中都超过了记忆信号(图~\ref{fig:fig_5_6} C)。
	\item 如果大多数细胞反映隐蔽注意力,那么当受试者需要参加某个地方,但不需要记住这个位置时,就应该有可能证明激活。
	成像可以用来检验这一预测。
	Astafiev等\cite{astafiev2003functional}提示受试者在保持中心注视的同时注意左边或右边。他们报告了额叶视区的激活,以及在顶骨内沟附近的后顶骨区域。
	
	Curtis\cite{curtis2006selection}进行了类似的成像研究,但有两次延迟。
	他们的实验对象看到了即将到来的扫视的几个潜在目标,随后是第一个延迟期,在此期间,实验对象需要记住这些地方。
	在第一次延迟期间,额叶视区和后顶叶皮层都发生了持续的激活。
	然后出现了一个箭头,指示在该试验中哪个目标应该是眼球运动的目标。
	第二次延迟后,受试者按指示进行扫视。
	在第二次延迟期间,持续的激活只发生在额叶视区,而不是在后顶叶皮层。这一发现与中央凹是一种注意力功能的观点一致。
	包括额叶视区在内的尾侧前额叶皮层,在将注意力导向潜在目标方面发挥作用,这在两次延迟期间都发生过混乱,不仅仅是在记忆位置上,发生在第一次延迟的时候。
\end{enumerate}


\begin{figure}
	\centering
	\includegraphics[width=0.7\linewidth]{chap5/Fig_5_6}
	\caption{前额叶皮层的注意力和记忆编码。
		(A)任务设计。视频监视器上出现一个圆(灰色矩形),随后开始围绕中心注视点(FP)旋转。
		它从四个位置中的一个开始,然后停在相同的四个位置中的一个:从中心向上、向下、向左和向右。
		如果圆圈变暗(上叉),猴子需要在这次试验中扫视圆圈的最终位置;如果它变亮了(底部的叉子),猴子必须在这次试验中向圆圈的初始位置扫视。箭头表示在两种情况下正确的扫视。
		缩写:Rem,记忆位置;丙氨酸,attended-location。
		(B)空间调谐的前额叶皮层神经元编码参与或记忆的位置或两者(混合)的百分比。
		(C)注意信号格式为记忆信号,如图~\ref{fig:fig_5_4}~所示\cite{lebedev2004representation}。}
	\label{fig:fig_5_6}
\end{figure}


我们得出结论,在这些实验中,延迟期活动或激活有几种可能的解释。
它可以反映对提示位置的记忆,对该位置的隐蔽注意力,或将公开注意力引导到该位置的准备。
此外,它还可以反映任务规则的编码,这是我们将在下一章讨论的主题。


因此,动眼肌延迟反应任务延迟期间的活动并没有显示任何关于回顾性空间工作记忆的内容。
首先,这种活动反映的是隐蔽注意力,而不是感官记忆。
第二,工作记忆的概念没有区分位置的回顾编码和前瞻性编码。
正如我们在下一章所解释的那样,前瞻性编码一词指的是对选定目标的短期记忆。
尽管一些实验,如Funahashi等人的反眼跳任务,试图排除目标位置的预期编码,但在每种情况下,都有相当数量的少数细胞编码目标。
将空间注意力与空间记忆进行对比的实验表明,前额叶皮层的大部分活动编码的是隐蔽参与的位置,而不是被记住的位置。
我们将在第~\ref{chap:chap6}~章和第~\ref{chap:chap7}~章回到前瞻性编码的主题。



\subsection{中断延迟期活动}

不管对延迟期活动的正确解释如何,中断它都会造成某种程度的损害。
然而,简单地表明,在尾叶前额叶皮层有损伤的猴子在一项任务中表现得很差,并不能证明它们这样做是因为它们不再记得提示的位置。
这也可能表明他们很难注意到提示的位置或在记忆中保持它的预期代码。


受试者在动眼肌延迟反应任务中可能会犯两种类型的错误。
Frank误差涉及错误的空间目标的选择\cite{keller2008effect},而精度误差涉及更接近正确目标而不是任何其他可能目标的移动,但错过了其确切位置。


不准确的扫视可能反映了低水平的运动障碍,因此需要将缺乏延迟期的控制条件与标准任务进行比较。
在这里,受试者只是对一个可见的目标进行扫视。


猴子的额叶视区失活会引起低视眼跳。
也就是说,对可见位置的扫视往往达不到空间目标\cite{dias1999muscimol}。
低眼跳也发生在额叶视区单侧病变的患者中\cite{gaymard1999frontal,ploner1999errors}。


冷却导致8A区失活也会导致眼球运动延迟反应任务的不准确扫视。他们往往达不到正确的目标位置,或在目标位置的一侧\cite{chafee2000inactivation}。
用肌西酚\cite{sawaguchi2001prefrontal}或多巴胺拮抗剂\cite{sawaguchi1991d1}灭活后外侧前额叶皮层具有同样的效果。
后外侧前额叶皮层的单侧消融也是如此\cite{funahashi1993dorsolateral}(图~\ref{fig:fig_5_7})。


尽管额叶视区、尾侧前额叶皮层的其他部分或后外侧前额叶皮层的失活会导致不准确的扫视,但这些扫视通常朝着正确的方向前进,并在更接近提示位置的地方结束,而不是任何其他目标。
然而,Funahashi等人\cite{funahashi1993dorsolateral}得出结论,这些眼跳的不准确反映了回溯性空间工作记忆的失败。


他们从未解释过,如果猴子不记得目标位置,它们的扫视最终会更接近提示的位置,而不是任何其他潜在的目标位置。


\begin{figure}
	\centering
	\includegraphics[width=0.67\linewidth]{chap5/Fig_5_7}
	\caption{一只猴子在动眼力延迟反应任务上的表现。
		(A)术前(正常)表现正方形表示225°(向下和向左)目标的扫视终点;十字表示45°(向上和向右)的目标。
		(B)左半球后外侧前额叶皮层单侧损伤后的表现。
		(C)右半球同一区域额外病变后的表现,以完成双侧病变。
		底部的延迟持续时间\cite{funahashi1993dorsolateral}。}
	\label{fig:fig_5_7}
\end{figure}


工作记忆解释的支持者可能会争辩说,猴子之所以不会犯明显的错误,是因为这些损伤,无论是永久性的还是暂时性的,都是单方面的。
毕竟,我们知道在标准延迟反应任务中,中外侧前额叶皮层的单侧损伤只会引起轻度损伤\cite{rosen1975effects},而双侧损伤后的损伤是严重的\cite{goldman1971analysis}。
因此,Funahashi等人\cite{funahashi1993dorsolateral}在两只猴子身上进行了双侧损伤,并分两个阶段进行。
图~\ref{fig:fig_5_7}~显示了其中一种动物的数据。
这只猴子在第二次损伤后有更大的损伤,但大多数扫视仍然在目标位置附近结束(图~\ref{fig:fig_5_7})。
病变很少引起明显的错误。
在6秒的较长延迟(未显示)下,眼跳端点比较短延迟时分布得更广,但猴子犯准确错误的频率仍然比坦率错误高得多。


话虽如此,我们并不否认,前额叶皮层受损或失活的猴子有时确实会犯明显的错误。
但是,对猴子在犯下明显错误后的行为的研究表明,这种损伤与回溯性工作记忆几乎没有关系。
Tsujimoto\cite{tsujimoto2012prefrontal}报道了前额叶皮层中外侧皮层(46区)失活的影响,所以我们在下一章中再次讨论。
这些失活在动眼肌延迟反应任务上造成了明显的错误。
在猴子犯了这样的错误后,它们的第一次扫视和下一次扫视都瞄准了适当的目标(图~\ref{fig:fig_5_8})。
这一发现表明,猴子并没有忘记线索的位置或适当目标的位置。


\begin{figure}
	\centering
	\includegraphics[width=0.7\linewidth]{chap5/Fig_5_8}
	\caption{在一个明显的错误后的纠正性扫视。
		(A)在中外侧前额叶皮层失活后,猴子对错误的目标进行了一些扫视。
		对于图的顶部部分,正确的目标是右上方的位置,对于中间部分,正确的位置,对于底部部分,正确的位置是右下方的位置。
		(B)对一次不正确扫视的直接后果的检查表明,猴子在那次试验中对适当的目标进行了矫正扫视\cite{tsujimoto2012prefrontal}。}
	\label{fig:fig_5_8}
\end{figure}


辻本和波斯特尔还分析了坦率错误的模式,发现猴子经常选择在前一次试验中合适的目标并获得奖励。
因此,很明显,猴子在回溯性空间工作记忆方面并没有简单的损伤。
相反,正如我们在下一章中提出的那样,有前额叶皮层损伤的猴子在区分之前的目标和当前的目标方面有缺陷。


因此,干扰延迟期活动的证据并不支持损伤导致回溯性感觉记忆丧失的观点。
相反,它表明病变干扰了对未标记位置的扫视的准确性,以及区分以前目标和当前目标的能力。
尾侧前额叶受损的猴子似乎保留了正确目标的知识,但很难将这种知识转化为准确的扫视。



\subsection{消除延迟期}

延迟期线索的缺乏导致了延迟期活动反映线索记忆的假设:回顾性空间记忆。
如果是这样,并且如果回溯性空间记忆是尾侧前额叶皮层的关键功能,那么我们可能会认为在没有延迟期的任务中活动减弱。
更重要的是,尾侧前额叶皮层的损伤不应该导致对缺乏记忆要求的任务的损害。
实验结果与这两种观点都不相符。


戈尔德\cite{gold2007neural}向猴子展示了一个由移动的圆点组成的线索。
这些点的净运动,向左或向右,指示扫视一个红色或绿色的目标,而不管目标出现在哪里。
没有延迟期;猴子一旦确定了线索移动的净方向,就会做出选择。


正如我们在第~\ref{chap:chap3}~章中介绍的累加器-跑道模型所解释的那样,戈尔德发现额叶视区中的活动在提示开始和扫视时间之间逐渐增加。
他们表明,这种活动编码了目标刺激的颜色,红色或绿色。
因此,额叶视区在缺乏延迟期或空间记忆需求的任务中具有与任务相关的活动,并且它似乎与包含延迟期的任务一样健壮。


然而,有人可能会说,“记忆”细胞只是在这些任务中变得不活跃,而其他类型的细胞变得更加突出。
因此,一个更有力的预测是,只有在任务包含延迟期时,病变才会造成损伤。
但Keller等人\cite{keller2008effect}提出的证据表明,这一预测并不成立。
他们教猴子一项有条件的视觉运动任务,在这项任务中,动物必须根据中心注视点的颜色,将扫视指向不同的位置。
Keller等人随后单方面或双边地使额叶视区失效。
失活导致错误增加,这些错误不仅包括不准确的扫视,还包括明显的错误。
双侧病变造成的影响大约是单侧病变的两倍。
在这里,这项任务不涉及对回顾性空间记忆的要求,但病变产生了损害。
然而,这些任务确实需要对目标进行前瞻性编码和关注。


到目前为止所描述的任务包括扫视。
但是,关于伸展运动的实验也得出了同样的结论。
劳勒\cite{lawler1987role}教猴子在看到黑色线索后向左伸手,在看到白色线索后向右伸手。
然后,他们在前额叶末梢皮层做了双侧损伤,这造成了严重的损伤。
Petrides\cite{petrides1985deficits}使用视觉提示告诉猴子是打开一个亮着的盒子还是不亮的盒子。
同样,在尾部前额叶皮层有病变的猴子表现出损伤。
这两个实验都没有涉及需要记忆的延迟期。


在所有这些任务中,提示都指定了试验的目标,该目标可以是一个地点\cite{keller2008effect}或一个物体\cite{petrides1985deficits}。
目标可以是扫视\cite{keller2008effect}或伸手运动\cite{lawler1987role}的目标。


劳勒和考威\cite{lawler1987role}认为,在尾侧前额叶皮层损伤后,动物“忽视”了部分视觉空间,这意味着它们无法将注意力导向这些地方。
但类似的病变只是增加了寻找外周靶点所需的时间,而不是导致搜索失败\cite{wardak2004deficit}。
因此,我们认为,患有尾侧前额叶皮层病变的猴子可以检测到外周目标,只是比正常情况稍慢一些。
他们的核心缺陷是难以将注意力引导到作为目标的地点和物体上。



\subsection{总结}

本节回顾了来自细胞记录和损伤研究的证据,这些证据已被用于支持前额叶皮层的空间工作记忆理论\cite{goldman1996prefrontal}。
在此,我们对尾侧前额叶皮层(第8区)和后外侧前额叶皮层(第9/46区)否决了这一结论,理由如下:


\begin{enumerate}
	\item 大多数延迟期活动反映了对潜在目标的隐性关注,而不是在记忆中维持感官线索。
	\item 尾侧或后外侧前额叶皮层的损伤很少会造成明显的错误,即猴子会扫视到错误的位置,而不是提示的位置。相反,这些病变主要导致不准确的扫视,比任何其他潜在目标更接近正确的位置。当明显的错误确实发生时,受损的猴子会立即纠正错误,从而表明它们记住了提示的位置。
	\item 这些区域为缺少延迟期的任务编码目标。
	\item 受损的猴子在缺乏延迟期的任务上有缺陷,因此这些区域的功能不依赖于弥合时间差距。
\end{enumerate}
因此,有证据表明,尾侧和后外侧前额叶皮层的功能是寻找和关注目标,而不是工作记忆。
我们在第~\ref{chap:chap6}~章讨论了背侧前额叶皮层,讨论了拒绝工作记忆理论的其他理由。
在第~\ref{chap:chap10}~章中,我们将讨论前额叶皮层作为一个整体的工作记忆理论。



\section{基于学习的注意力}

在前一节讨论的实验中,猴子必须通过使用奖励反馈来影响其在目标中的未来选择,从而学习应该关注哪个目标。
任务的这一特点并不总是得到应有的考虑。
尾侧前额叶皮层功能的一个重要方面涉及目标导向和刺激驱动注意之间的区别。


\subsection{目标导向和刺激驱动的注意力}

目标导向注意和刺激驱动注意之间的区别与学习有关。
在本书中,我们所说的目标是指作为行动目标的物体和地点。
刺激驱动的注意力取决于目标的显著性,例如,它的亮度或突然发生,这吸引了注意力。
这种自下而上的注意力依赖于先天机制,而不是学习。
在以目标为导向的注意中,实验对象必须知道要注意哪些刺激,无论是在猴子身上通过奖励反馈,还是在人类身上通过指令或反馈。


值得注意的是,刺激驱动注意和目标导向注意之间的区别并不对应于弹出式搜索和连接式搜索之间的区别。
在这两种情况下,猴子通过反馈学习目标,或者人通过指令或反馈学习目标。
这两种搜索的不同之处在于,在弹出式搜索中,目标从一组相同的干扰物中脱颖而出,而在联合搜索中,目标与干扰物具有相同的特征,但不同之处在于这些特征的特定组合。
干扰物不需要彼此完全相同。
这两种类型的搜索都需要学习,因此额叶视区的失活会损害这两种搜索的性能\cite{wardak2004deficit}。
此外,当Wardak等人\cite{wardak2010searching}在猴子执行弹出式搜索任务时进行了成像实验,主要的激活发生在尾侧前额叶皮层,包括额叶视区,只有少量的激活在后顶叶皮层。


目标导向的注意力涉及到前额叶皮层,因为它比其他区域更直接地接收有关结果的某些信息。
具体来说,眶额皮层可以提供特定条件和“共同货币”条件下潜在目标的估值。
这些信息可以从眶额皮层传递到与尾侧前额叶皮层相连的其他前额叶区域。
眶额皮层和尾侧前额叶皮层之间的相互作用导致对有价值物体的搜索和关注。



\subsection{增强效果}

当尾侧前额叶皮层将注意力导向一个目标时,一个结果被称为增强效应;
与无人看管的地方相比,在有人看管的地方,细胞对刺激的反应更大。
显性注意和隐性注意都会引起这种效应。


Goldberg\cite{goldberg1972activity}发现了对上丘神经元的增强作用。
这些神经元中有许多对扫视目标表现出增强的活动。
在Goldberg\cite{goldberg1981behavioral}的后续研究中,后顶叶皮层的细胞不仅在扫视(显性注意)之前表现出增强,而且在隐性注意期间也表现出增强。
同样,成像显示参与刺激的激活增强\cite{corbetta2000voluntary},激活增强发生在与参与位置相对应的视网膜位置图区域\cite{brefczynski1999physiological}。


在对额叶视区细胞活动的早期研究中,Goldberg\cite{goldberg1985cerebral}发现对扫视目标(显性注意)有增强效应,但对隐性注意没有。
然而,在后来的一项研究中,Kodaka等人\cite{kodaka1997neuronal}报告说,在一项注意力任务中,当外周刺激变暗时,要求猴子释放一个杠杆,许多细胞显示出视觉反应的增强。
因此,就像在后顶叶皮层一样,额叶视区中的细胞对公开和秘密参与的位置都表现出增强的感觉反应。


Hasegawa等人\cite{hasegawa2000search}记录了8Ad区域,并在一系列干扰物中展示了一个类似物体的目标。
在目标出现的135毫秒内,如果目标出现在细胞的接受野中,这些尾部前额叶皮层细胞的反应就会增强。
如果目标出现在细胞的感受野之外,或者目标以外的刺激出现在感受野,则不会发生增强。
这一发现与尾侧前额叶皮层有助于或反映对学习目标的搜索这一观点一致。



\subsection{自上而下的影响}

鉴于增强效应既发生在前额叶皮层,也发生在更尾部的区域,如后顶叶皮层,我们需要知道是什么驱动了增强。
单从增强效果来看,我们无法区分因果关系。
我们不能简单地假设细胞的反应更多是因为自上而下的注意力的影响。
尽管如此,累加器-跑道网络的特性表明了这种机制是如何工作的(第~\ref{chap:chap3}~章)。
如果前额叶皮层网络代表了未来行动的目标,那么它可能会导致皮层其他地方对该目标的感觉表征更快地达到阈值。
结果是基于学习目标的自上而下的有偏见的竞争\cite{desimone1995neural}。
结果就是“寻找”或保持对目标的注意力。


自上而下的偏差可以解释增强如何发生在低阶区域。
我们已经提到,Armstrong\cite{armstrong2007rapid}将皮质内微刺激应用于额叶视区,并表明它增强了视觉区V4细胞的反应。
它是为视觉空间的一个特定部分做的。
这一发现表明,当猴子注意到空间的这一部分时,额叶视区对V4施加了自上而下的偏向,有利于该位置的表示。


第~\ref{chap:chap3}~章阐述了前额叶皮层可以使低阶行为控制系统之间的竞争产生偏见的想法,Desimone\cite{desimone1995neural}提出了一个对低阶视觉区域自上而下影响的偏见竞争模型。


Desimone和Duncan并没有提供证据证明自上而下的效应起源于前额叶皮层,而不是其他区域。
然而,我们知道,尾侧前额叶皮层的细胞根据当前任务的性质表现出不同的活动。
当线索的位置与行动选择的相关性很小时,只有少数细胞对该位置进行编码\cite{chen2001neuronal}。
当线索的颜色和形状具有高度相关性时,大量的尾端前额叶细胞开始编码这些特征\cite{bichot1996visual}。


相应的变化发生在感觉区域。
当任务要求猴子注意运动时,MT和MST区域的神经元对刺激的特征表现出增强的反应\cite{treue1996attentional}。
当任务要求猴子注意方向时,V4区域的细胞对该特征的反应增强\cite{mcadams1999effects}。


然而,增强效果取决于这样一个事实:
猴子已经学会了哪个维度、动作或形状与任务相关。
动物只有在关注相关维度时才会得到有益的反馈。


尾侧前额叶皮层与吻侧前额叶皮层有广泛的联系,前额叶皮层许多部分的细胞活动反映了当前相关的刺激维度。
例如,Lauwereyns等\cite{lauwereyns2001responses}教猴子一个不去不去的任务,其中相关刺激维度在颜色和运动之间交替。
他们发现,44\%的细胞更喜欢与颜色相关的条件,24\%的细胞更喜欢与运动相关的条件。
前者多位于腹侧前额叶皮层,后者多位于背侧前额叶皮层。
这两个区域的吻侧细胞往往比尾侧前额叶皮层的细胞更常反映相关维度,后者的细胞通常编码适当的反应:去或不去。


Lauwereyns等人将这些位于尾侧的神经元称为“整合细胞”,并认为它们将前额叶皮层的部分输出提供给其他皮层区域。
因此,颗粒状前额叶皮层的吻侧部分会影响尾侧前额叶皮层,从而影响皮层其他部分的感觉信息处理。
他们也可以直接这么做。


如果这种增强是自上而下作用的结果,那么就有可能表明这种增强在前额叶皮层比尾侧感觉区发生得更早。
因此,Zhou\cite{zhou2011feature}在猴子执行视觉搜索任务时,同时从额叶视区和V4区域记录。
当猴子注意到刺激的特定特征时,两个区域都出现了增强的感觉反应,但这些影响的潜伏期表明,额叶视区对V4提供了自上而下的偏向。
由此可见,尾侧前额叶皮层可以调节感觉皮层对刺激特征的加工。


自上而下的注意机制可能涉及两个区域慢波振荡的同步发展。就像通过成像实验测量的信号一样(第~\ref{chap:chap1}~章),电位的缓慢变化可能反映的是突触活动,而不是神经元放电率。
在周和德西蒙的实验中,格雷戈里欧等人\cite{gregoriou2009high}记录了额叶视区和V4,当猴子在细胞的接受野中接受刺激时,他们这样做。
他们观察到这两个区域振荡之间耦合的发展,而额叶视区似乎启动了这种耦合。
同步最显著地发生在伽马带,这被认为是皮层区域之间相互作用的一般机制\cite{womelsdorf2007modulation}。


前额叶皮层的增强比视觉区域更早,这一发现并不能证明前额叶皮层会导致这些区域的增强。
因此,Rossi等人\cite{rossi2009prefrontal}研究了前额叶病变的影响。
他们教猴子辨别铁栅栏上铁条的方向。
中央线索的颜色告诉猴子在有相同颜色的光栅中判断方向。
猴子看到了三个栅格,只有一个有相关的颜色,它可以在不同的试验中改变,也可以在一系列试验中保持不变。
背侧前额叶皮层和腹侧前额叶皮层的单侧病变,加上胼胝体的横断,使得Rossi等人能够比较受损半球和完整半球处理刺激时的表现。
对于受损的半球,猴子表现出严重升高的辨别方向的阈值,相关颜色的频繁变化加剧了这种缺陷。


Rossi等人并没有证明前额叶损伤消除了视觉区域的增强效果,但一些皮层刺激实验提供了证据,证明它可能会。
Morishima等人\cite{morishima2009task}在人类受试者的额叶视区上使用单脉冲经颅磁刺激(TMS)。
他们的实验对象看到了由移动的点组成的脸,并必须对点的运动或脸的性别做出判断。
因此,他们要么注意动作,要么注意面孔。
Morishima等人在刺激额叶视区时,用脑电图记录了MT区域和颞皮质梭状面区域的激活。
当被试准备对运动进行判断时,刺激影响MT区激活,而当被试准备对性别进行判断时,刺激影响梭状回面部区的激活。
因此,刺激部分颗粒状前额叶皮层后视区产生了自上而下的对相关刺激维度的偏向。



\section{总结}

\subsection{尾侧前额叶皮层是如何发挥作用的}

本章展示了尾侧前额叶皮层的连接如何解释它对整个前额叶皮层的贡献。
其中三个联系似乎是最重要的:

\begin{enumerate}
	\item 通过12区和46区,与眶额眶皮层的连接间接地为尾侧前额叶皮层提供了所观看项目的学习和更新价值\cite{barbas1989architecture}。
	\item 通过与背侧和腹侧视觉流的连接,尾侧前额叶皮层搜索并引导注意力到有价值的地方和物体上。
	通过与背侧和腹侧视觉流的连接,尾侧前额叶皮层搜索并引导注意力到有价值的地方和物体。
	这些联系包括对顶叶和颞叶区域的投射,这被认为是导致对地点和物体的感觉反应增强的原因,从而在相互竞争的感觉表征中提供了一种自上而下的偏见。
	\item 通过上丘和基底神经节直接或间接地投射到动眼核,使额叶视区通过眼球运动来引导显性注意力。
\end{enumerate}

这些连接使末梢前额叶皮层能够在混乱的环境中寻找有价值的目标,并将注意力引向这些目标。
它专门针对学习目标——目标导向的注意力——而不是反身性或刺激驱动的注意力。



\subsection{提议}

现在,我们可以用简单而详细的形式提出尾侧前额叶皮层的特定功能:

\begin{enumerate}
	\item 简而言之:尾侧前额叶皮层主要基于视觉,寻找并引导注意力朝向有价值的目标。
	\item 扩展:尾侧前额叶皮层,以及后外侧前额叶皮层,主要基于视觉,根据当前生物需求进行评估,搜索并将注意力导向学习价值的目标。
	它通过隐蔽的注意力或公开的注意力(眼球运动)来做到这一点。
\end{enumerate}



\subsection{为什么其他区域不能做尾侧前额叶皮层所做的事情}

由于它的连接,只有尾侧前额叶皮层可以执行这些功能。
如第~\ref{chap:chap4}~章所示,眶额皮层皮层首先接收到关于食物的视觉、嗅觉、味觉和“口感”的详细信息。
通过与眶额皮层的间接连接,尾侧前额叶皮层可以根据当前的生物学需求接收关于特定目标的可取性的信息。


另一方面,后顶叶皮层无法接收到太多关于特定食物或液体的信息。
它几乎没有(如果有的话)传递特定结果的输入,无论是来自眶额皮层还是来自任何其他皮层区域。
此外,它与杏仁核的联系很少,如果有的话。
颞叶皮层确实接受杏仁核的输入,但它不能像前额叶皮层那样轻易地将这些信号与结果的嗅觉、味觉和内脏特征结合起来。
尾侧前额叶皮层和后顶叶皮层都在注意力中起作用。
但正如我们所强调的,尾侧前额叶皮层负责关注猴子已经学会的有价值的目标,这可能是因为它而不是后顶叶皮层接收有关特定结果的前额叶中介信息。



\subsection{对觅食选择的贡献}

在实验室里,灵长类动物经常以彩色形状或光点的形式去触摸或观察目标。
在它们的自然环境中,它们的目标包括食物、食物的标志和有学习价值的地方。
正如第~\ref{chap:chap2}~章所解释的那样,在早期灵长类动物中进化出的前额叶皮层有两个新的颗粒区域:尾侧前额叶皮层和眶额皮层的颗粒部分。


我们认为,这两个颗粒区域在早期灵长类动物中共同起作用,就像在现代灵长类动物中继续起作用一样。
但要了解它们是如何影响早期灵长类动物的觅食选择的,我们需要了解这些已经灭绝的“视觉动物”是如何看待世界的。


通过我们自己的眼睛去看其他动物的视觉,这是我们人类的一种自负。
但是早期的灵长类动物没有像我们这样的眼睛,很大程度上是因为它们缺少一个后来进化出来的中央凹(第~\ref{chap:chap2}~章)。
根据定义,显性注意力依赖于中央凹,所以当早期的灵长类动物关注和搜索物体时,它们使用隐性注意力来做到这一点。
我们提出,早期灵长类动物的眶额皮层根据当前需求提供了可见物体的学习价值(第~\ref{chap:chap4}~章)。
当与尾侧前额叶皮层提供的搜索和注意力功能相结合时,早期灵长类动物可以在物体和地点中进行选择,在杂乱的环境中找到它们,并(秘密地)关注它们。


在灵长类动物进化的后期,出现了中央凹和三色视觉,并出现了新的前额叶区域(第~\ref{chap:chap2}~章)。
接下来的两章,分别在背侧前额叶皮层和腹侧前额叶皮层,探索这些区域给类人猿灵长类动物带来了什么优势。



\chapter{背侧前额叶皮层:基于最近事件生成目标} \label{chap:chap6}

背侧前额叶皮层有助于根据顺序、时间和空间环境生成目标,它的连接解释了为什么只有它才能做到这一点。
背侧前额叶皮层,包括中外侧前额叶皮层(46区),通过与后顶叶皮层、前运动皮层和前额叶皮层的其他部分连接来发挥作用。
顶叶连接提供了许多用于生成目标的空间和时间背景。
与运动前区域的联系导致这些目标的实现,通常是通过手的运动。
与\textit{眶额皮层}的连接使背侧前额叶皮层能够根据\textit{单个事件}预测\textit{目标选择的具体结果}。
背侧前额叶皮层位于背侧视觉流的末端,因此它可以规划目标序列,它可以具体或抽象地指定这些目标。
在产生目标后,背侧前额叶皮层可以前瞻性地编码它们,直到行动的时候到来。
鉴于背侧前额叶皮层在类人猿灵长类动物中进化(第~\ref{chap:chap2}~章),我们认为它在使用最近视觉事件的顺序、时间和位置来指导觅食选择和生成效率优化的目标序列方面具有优势。



\section{介绍}

第~\ref{chap:chap2}~章解释了灵长类动物的前额叶皮层在类人猿中随着新区域的出现而扩展。
第~\ref{chap:chap3}~章涉及了其中的一些,例如额极皮层(10区),但在本章和下一章,它们是主要主题。
由于类人猿依赖于白天漫长的觅食旅行,它们需要消耗大量的能量,并面临着很高的捕食风险。
这种生活方式非常重视正确的觅食选择。
在影响这种选择的因素中,视觉事件的位置、时间和顺序是突出的,因为类人猿利用了它们在中央凹和色彩视觉方面的进步。
正如第~\ref{chap:chap2}~章所解释的,这些进步包括中央凹提供的精致的视觉敏锐度和三色视觉提供的增强的辨别能力。


这一章解释了觅食的选择部分取决于当前的环境,
这是由选择时可用的刺激以及最近视觉事件的记忆所指定的。
为了理解我们的意思,考虑一个简单的实验室任务:延迟匹配样本。
猴子把一个刺激看作一个样本,然后看到一个或多个刺激需要选择。
这种选择不仅取决于选择时的刺激,还取决于基于样本刺激的记忆。
这两个因素共同构成了当前选择目标的背景。


由于记忆对当前情境起作用,这类实验的被试面临一个问题:最近发生了几件事,并选择了几个目标。
本章的大部分内容探讨了背侧前额叶皮层如何帮助类人猿灵长类动物解决这个问题。
许多文献都依赖于一个任务和一个区域:延迟反应任务,它依赖于中外侧前额叶皮层(46区)。
因此,我们将讨论的重点放在第~\ref{chap:chap1}~章和第~\ref{chap:chap5}~章也提到的这个任务上。
我们认为,因为受试者在这个任务中经历了一系列的视觉事件,并实现了一系列的目标,他们需要挑选出决定当前目标的事件。
然后,我们回顾一系列其他任务,这些任务也要求主体在当前环境的基础上生成目标。



\section{区域}

在猕猴中,中外侧前额叶皮层位于主枕沟的嘴侧三分之二(图~\ref{fig:6_1})。
然而,它也延伸到这个沟的背侧和腹侧的凸面皮层。
中外侧前额叶皮层(46区)有很多名字,有些很明确,有些则不太明确。


\begin{figure}
	\centering
	\includegraphics[width=0.7\linewidth]{chap6/6_1}
	\caption{猕猴(左)和人类(右)的背侧前额叶皮层。
		格式如图~\ref{fig:1_2}~所示。}
	\label{fig:6_1}
\end{figure}


\textit{沃克}将沿主沟整个长度的皮层称为46区\cite{walker1940cytoarchitectural},但最近\textit{佩特里迪斯}将9/46区区分为该沟的尾端周围(见图~\ref{fig:1_2})\cite{petrides1999dorsolateral}。
为了避免对46这个术语的不同用法的混淆,我们称沃克46区嘴侧三分之二为中外侧前额叶皮层,尾侧三分之一为后外侧前额叶皮层。
在人类中,这些分区位于额上回和额下沟之间。


尽管出现了这些新术语,但混淆的空间仍然很大。
术语背侧前额叶皮层最初是指猴子的整个侧前额叶皮层\cite{pribram1952effects},但后来仅指主沟内和背侧的皮层\cite{mishkin1969re}。
在影像学文献中,指背外侧前额叶皮层已变得很常见,但该术语的使用通常非常松散,很少注意解剖标志。
因此,我们在本书中避免使用这个术语。
表~\ref{tab:1_2}~给出了我们所采用的术语。
我们将背侧前额叶皮层包括主沟两岸的皮层和其背侧的凸面皮层(区域9),但不包括区域9的内侧部分。
当然,这些划分并不是最终的结果,但它们在一定程度上反映了联系。



\section{连接}

图~\ref{fig:6_2}~展示了\textit{背侧前额叶皮层}的一些皮层连接。
正如第~\ref{chap:chap1}~章所解释的,这些连接构成了解剖指纹。
\par


1.中外侧前额叶皮层(46区)与后顶叶皮层,特别是尾顶叶区有很强的联系\cite{petrides1984projections}。
正如前一章所提到的,后顶叶皮层的许多神经元编码视觉空间信息。
然而,顶叶细胞也编码时间间隔\cite{leon2003representation},当人类受试者对近期事件做出判断时,左侧顶叶内沟会出现激活\cite{dudukovic2007goal}。
\par


2.中外侧前额叶皮层连同外侧9区,与位于上颞沟上排的多感觉\textit{颞顶枕区}(也称为\textit{上颞多感觉区})相连\cite{seltzer1996overlapping}。
该区域的细胞对体感、听觉和视觉刺激有反应\cite{bruce1981visual}。
\par


3.中外侧前额叶皮层也接受来自周围皮层的输入\cite{petrides1999dorsolateral},其功能是识别物体\cite{murray2007orbitofrontal}。
这种联系表明,中外侧前额叶皮层接收到有关物体的直接输入,而不仅仅是腹侧前额叶皮层的间接输入(第~\ref{chap:chap5}~章和第~\ref{chap:chap8}~章)。
\par


4.中外侧前额叶皮层接收来自\textit{次级躯体感觉皮层}\cite{petrides2002comparative}和下顶叶嘴侧PFG区\cite{rozzi2006cortical}的输入。
这些区域的细胞对体感刺激有反应\cite{hyva1981regional},中外侧前额叶皮层的细胞也是如此\cite{tanila1993regional}。
这一特征将中外侧前额叶皮层与尾侧和后外侧前额叶皮层区分开来,后者的细胞主要具有视觉和注意力特性(第~\ref{chap:chap5}~章)。
\par


5.中外侧前额叶皮层连接背侧和腹侧前运动皮层\cite{wang2002spatial},以及\textit{前辅助运动区}\cite{wang2005prefrontal}和位于\textit{扣带沟}的嘴侧\textit{扣带运动区}\cite{dum1993cingulate}。
这些投影主要涉及代表手和手臂的运动前区域,而不是脚和腿。
前肢表征的专门化表现在背侧前前皮层的嘴侧部分\cite{tachibana2004input}、腹侧前运动皮层\cite{he1993topographic}、\textit{前辅助运动区}\cite{luppino1991multiple}和\textit{扣带运动区}\cite{he1995topographic}。
因此,相对于运动的移动,中外侧前额叶皮层在伸手、操作和进食运动中具有优先的作用(第~\ref{chap:chap2}~章)。
\par


6.中外侧前额叶皮层与前扣带皮层有很强的联系\cite{petrides1999dorsolateral}。
第~\ref{chap:chap3}~章解释了前扣带皮层在动作评估和基于这些评估动作之间的切换中发挥作用\cite{walton2007adaptive}。
\par


7.中外侧前额叶皮层与\textit{压后皮层}相连\cite{morris1999fiber}。
反过来,从\textit{压后皮层}投射到海马旁回和下丘前\cite{kobayashi2007macaque}。
我们认为,这些联系可能在有关事件的记忆检索中发挥作用,而这种检索依赖于时间或空间上下文\cite{vann2009does}。
\par


8.区域9的横向部分连接受到的关注相对较少,部分原因是很少有功能数据引起人们的兴趣。
这部分背侧前额叶皮层与\textit{背侧前运动皮层}的嘴侧部分\cite{petrides1999dorsolateral}和\textit{扣带运动区}\cite{morecraft1993frontal}有联系。
就像第9区域的内侧部分一样,它也与上颞皮层有联系,这可能传达听觉信息\cite{petrides1984projections,saleem2008complementary}。
最后,第9区外侧部分与后顶叶皮层下部(PG区)\cite{cavada1989posterior}和\textit{压后皮层}\cite{kobayashi2003macaque}相连。


\begin{figure}
	\centering
	\includegraphics[width=0.7\linewidth]{chap6/6_2}
	\caption{背侧前额叶皮层的选定连接。
		图~\ref{fig:1_4}~和~\ref{fig:1_5}~给出了沟和区域的名称。
		除非另有说明,一些轴突与背侧前额叶皮层有直接连接的区域被认为是相互的。}
	\label{fig:6_2}
\end{figure}



\subsection{总结}

中外侧前额叶皮层与后顶叶皮层、前运动皮层有很强的联系,并间接与海马系统有联系。
它也与前额叶皮层的其他部分相互连接,如眶额皮层。
其他皮层区域没有这种连接模式。
因此,它很好地整合了由\textit{眶额皮层}、\textit{背侧视觉流}和\textit{海马体}处理的信息,并向\textit{前运动皮层}提供信息。



\section{延迟响应任务}

中外侧前额叶皮层在接收后顶叶皮层的视觉空间信息方面与尾侧前额叶皮层相似。
因此,任何一个区域的损伤都会导致猴子在动眼延迟反应任务和经典版本的延迟反应任务上的表现中断,这并不奇怪。
在这两项任务中,空间线索指导目标选择,猴子必须在可能的空间目标中进行选择。
当然,根据定义,动眼肌延迟反应任务需要对目标进行扫视,而经典的延迟反应任务需要达到目标的运动。
第~\ref{chap:chap5}~章解释了尾侧前额叶皮层和后外侧前额叶皮层的损伤会导致动眼力延迟反应任务的准确性误差,根据它们与背侧视觉流的连接,这是有意义的。
这些损伤不会引起很多明显的错误,当它们发生时,猴子会迅速纠正这些错误\cite{tsujimoto2012prefrontal}。


相反,中外侧前额叶皮层(46区)的永久性损伤对经典的延迟反应任务造成毁灭性的损害。
当猴子在手术前学习延迟反应任务时,它们在中外侧前额叶皮层损伤后执行任务的效果并不比偶然水平好\cite{goldman1978prenatal}。
也就是说,受损的猴子犯的错误和正确的选择一样多。
当他们在中外侧前额叶皮层持续受损后第一次尝试学习任务时,即使有短暂的(1秒)延迟期,他们也无法完成任务\cite{battig1960comparison}。
而且它们永远无法恢复,至少在任何人测试过的时间范围内都无法恢复。


一个重要的方法差异可以解释这种差异的部分原因。
在经典的延迟反应任务中,猴子伸手去拿盖在食物井上的盖子。
这意味着在这个版本的任务中,不像动眼肌延迟反应任务,受试者只能犯明显的错误。
如果发生,精度错误将不会被记录。


另一个区别可能也很重要。
猴子可以通过在延迟期间偷偷地关注目标位置来解决动眼肌延迟反应任务。
但是实验者执行经典延迟反应任务的方式使得隐蔽注意力很难集中到目标上。
传统的测试方法包括\textit{威斯康辛通用测验仪}。
在该仪器中,一个不透明的屏幕在延迟期间下降,使猴子在延迟期间无法看到相关位置(图~\ref{fig:6_3})。
在神眼运动版本的任务中,猴子可以在周边视觉中看到目标位置,尽管在扫视时它不再被标记出来。


原则上,执行经典延迟反应任务的猴子仍然可以准确地注意到\textit{威斯康辛通用测验仪}的目标一侧,或者以某种方式将姿势定向到那一侧。
然而,降低屏幕往往会导致猴子在测试室内移动。
Passingham\cite{passingham1971behavioural}在延迟期间监测了正常猴子的位置,发现即使它们已经解决了问题,在40\%的试验中,它们也会从房间的一边穿过到另一边。
因此,猕猴不需要通过身体定位来解决延迟反应任务所带来的问题,似乎测试方法很可能排除了使用隐蔽注意力来解决这个问题。
因此,动眼肌延迟反应相对缺乏坦率的错误可能反映了隐蔽地关注目标位置的持续能力,而延迟反应任务的经典版本的严重损害可能是由于破坏了这一策略。


当然,延迟响应任务可以在没有不透明屏幕的情况下呈现。
在延迟期间,食物井可能在受试者够不到的地方。
在这种情况下,猴子似乎更有可能采取某种姿势或注意力的方法来弥合延迟期,或者使用其他策略来做到这一点。
Wilson\cite{wilson1963effect}发现。正常的猴子在延迟期开始时坐在测试室的正确一侧,在延迟期结束时仍保持在该一侧。
然后,当他们有机会这样做时,他们就会到达离目标最近的距离。
在延迟开始时,有前额叶皮层损伤的猴子也坐在正确的一侧,但在延迟结束时,它们到达相反(错误的)目标的次数与到达正确目标的距离较短的次数一样多。
这一发现表明,对于有前额叶损伤的猴子来说,姿势或注意力的延迟期桥接不足以正确地形成延迟反应任务。


目前还不清楚为什么猴子在整个延迟期间不采取保持在\textit{威斯康辛通用测验仪}正确一侧的策略。
这个策略可以解决问题。
也许他们没有意识到,他们在延迟期开始时看到的东西表明了他们在延迟期结束时应该达到的目标。
换句话说,受损的猴子可能无法识别延迟间隔之前的视觉事件与它们即将做出的选择有任何关联。
正常的猴子确实认识到这种关系,并且不需要采取姿势定向策略或注意力策略来执行任务。


在后面的部分中,我们提出了这样的想法,即为了解决延迟响应任务所带来的问题,猴子需要知道延迟期之前的视觉事件为它们之后选择目标提供了关键。
我们认为,中外侧前额叶皮层受损的猴子要么无法学习这一规则,要么无法记住和应用它。


\begin{figure}
	\centering
	\includegraphics[width=0.32\linewidth]{chap6/6_3}
	\caption{在\textit{威斯康辛通用测验仪}中延迟响应任务的测试程序。
		(A)实验员在两个食物井(+)中的一个上饵,一只猴子从它的测试笼中观察。这个动作作为视觉提示事件。
		(B)实验者在延迟期间放下一个不透明的屏幕。
		相同的物体可以很好地覆盖食物。
		(C)实验者举起屏幕后,猴子在两个食物井中进行选择,将其中一个物体移开,如果正确(+)则获得奖励,如果错误(-)则得不到奖励\cite{murray1991contributions}。}
	\label{fig:6_3}
\end{figure}



\subsection{延迟期的重要性}

第~\ref{chap:chap5}~章提到,尽管尾侧前额叶皮层病变的猴子在动眼力延迟反应任务上只有轻微的损伤,但它们在没有延迟期的条状视动任务上也有损伤。
相比之下,在中外侧前额叶皮层有损伤的猴子只在包括延迟期的任务中有损伤。


Passingham\cite{passingham1985memory}设计了一个没有延迟的条件视觉运动任务。
显示器中央的两个面板,一个在另一个上面,提供线索。
猴子们知道,如果光出现在顶部的面板上,那么它应该选择一侧的目标;
如果一个光出现在底部面板,那么它应该选择另一边的目标。
所以这个任务包括空间线索和空间目标就像延迟反应任务一样,但不像延迟反应任务它不包括延迟期。
中外侧和后外侧前额叶皮层病变的猴子可以正常学习这项任务。


Stamm\cite{stamm1969electrical}的结果也表明,延迟是产生损伤的关键因素。
在一个试验中,他在不同的时间点灭活了中外侧或后外侧前额叶皮层。
在延迟期的早期,中外侧前额叶皮层(46区)的破坏估计导致延迟反应任务的表现下降到机会水平。
刺激后外侧皮层效果较小。
这一发现与Butters\cite{butters1969retention}的一项研究结果一致,他们在该研究中表明,主沟中央三分之一的病变导致延迟交替任务的严重损害,而后三分之一的病变的影响要小得多。


综合来看,这些结果支持两个结论。
首先,延迟响应任务的损害确实是由于延迟期的强加造成的。
其次,中外侧前额叶皮层在这一任务中起着必要的作用。
在密切相关的延迟交替任务中也是如此。



\subsection{延迟周期活动}

关于延迟反应障碍,最常被引用的说法是猴子无法记住线索的位置,因此这种缺陷可以被描述为回溯性空间工作记忆的缺陷之一。
人们可以记录下延迟期间的活动,而且乍一看,这种活动似乎是对线索位置的记忆编码,这一事实被解释为支持这一结论的证据。
然而,这种活动发生在前额叶皮层的许多部分,以及其他区域。
第~\ref{chap:chap5}~章解释了延迟期活动并不总是编码空间记忆。
当研究人员在充分的比较条件下研究细胞活动时,他们可以看到大部分所谓的记忆活动实际上编码了参与的位置。
然而,一些延迟期活动确实编码了记忆的位置,因此,中外侧前额叶皮层的延迟期活动仍然有可能介导回溯性工作记忆。


Kojima\cite{kojima1984functional}从中外侧前额叶皮层(46区)进行记录,试图验证空间工作记忆理论。
他们使用了两种条件。
在一种情况下,线索在延迟期间消失,而在另一种情况下,线索在整个试验过程中仍然可见。
在延迟期间编码位置的62个细胞中,当刺激仍然可见时,44个细胞表现出相同或更大的延迟期活动。
当刺激消失时,只有12个细胞更活跃。
Tsujimoto\cite{tsujimoto2004properties}在类似条件下使用了动眼肌延迟反应任务,并证实了这一结果模式。


对这一发现最有可能的解释有四种:
\par


1.即使线索仍然可见,也没有什么能阻止猴子“记住”它的位置。
因此,在这两种情况下,它们可能会“记住”提示的位置。
\par


2.在这两种情况下,猴子可能会视觉注意住目标。
Kojima\cite{kojima1984functional}没有记录眼睛的位置,所以我们不能排除他们的结果是由于明显的注意力。
\par


3.在这两种情况下,猴子可能会偷偷地关注目标。
\par


4.在这两种情况下,猴子脑内可能会对空间目标的位置进行编码。
这种表述可能包括一个计划好的行动,也可能独立于实现目标所需的行动而指定目标。


在这 4 种解释中,只有第一种与Kojima\cite{kojima1984functional}的解释是一致的。
他们得出的结论是,延迟期的活动反映了线索在回顾记忆中的位置,这与空间工作记忆理论一致。
然而,猴子似乎不太可能“记住”它们能看到的刺激。
第二种说法,就公开关注而言,可以解释island和gleman-lajiki的结果,但不能解benziko的结果。
在后一项研究中,猴子必须在整个延迟期间固定在一个中心位置。
第三种解释,即隐性注意,也可以解释两项研究的结果,但它与回溯性工作记忆方面的解释不相容。


这就剩下了第四种解释,它将延迟期活动解释为反映了空间目标的位置,这是Fuster\cite{fuster1973unit}首次提出的。
换句话说,这表明延迟期反映的是前瞻性记忆,而不是回顾性记忆。
猴子每天进行多次试验,这意味着先前试验的线索和目标位置在记忆中相互干扰。
在任何特定的试验中,只要看到线索,猴子就能通过编码目标位置来克服这种干扰。


第~\ref{chap:chap5}~章回顾了一些似乎反对目标的前瞻性编码的证据。
大多数细胞编码反眼跳任务中的线索位置,只有少数编码目标位置。
在另一项研究中,细胞对提示位置进行编码,即使这些位置不是未来的目标。
然而,这些研究都没有明确地关注中外侧前额叶皮层的细胞,而不是后外侧或尾侧前额叶皮层。


在本章稍后将详细介绍的一项研究中,Genovesio等人\cite{genovesio2006representation}确实研究了中外侧前额叶皮层中的细胞,以及其他神经元种群,他们的实验设计允许他们区分位置的回顾性编码和前瞻性编码。
他们发现,中外侧前额叶皮层的细胞尽可能快地编码当前的目标位置,并且回溯编码在这个时候消失。
我们还在后面的一节中提供数据,表明这种预期的活跃度可以防止记忆干扰\cite{sakai2002active}。



\subsection{总结}

中外侧前额叶皮层损伤的猴子在机会水平上执行延迟反应任务,并且它们没有恢复。
在延迟期间的破坏性电刺激会导致损伤,在没有延迟期的任务中,猴子表现正常。延迟期活动发生在中外侧前额叶皮层,它编码一个位置。
这种活动已经被解释为回顾性空间记忆,但证据表明,这种活动也编码了当前目标的位置(预期编码)和参与的位置。



\section{干扰的作用}

在延迟反应和延迟交替任务上,每天都有一系列的试验展开,这对记忆造成了严重的干扰。
在延迟反应任务中,随着试验次数的增加,猴子会从两个或多个位置获得线索;
在延迟反应和延迟交替任务中,猴子在一系列试验中选择了相同的地点。



\subsection{干扰效应的证据}

Diamond\cite{diamond1989comparison}的一项实验表明,干扰可能是导致延迟反应障碍的重要因素。
他们在修改后的延迟反应任务中测试了背侧前额叶损伤的猴子,并分析了猴子在两次正确选择同一侧后所犯的错误。
当这次测试的正确位置碰巧与前两次测试的正确位置相匹配时,猴子的正确率为85\%。
当当前试验中的正确位置与前两次试验不匹配时,猴子的表现符合概率水平(50\%正确)。


有人可能会试图解释这一发现,认为猴子坚持不懈。
但是,坚持不懈会导致低于机会的表现,相反,猴子表现在机会水平(50\%正确)。
事实上,受损的猴子在机会水平上的表现表明,它们意识到线索与前两次试验不同。
然而,他们不知道该选择哪个位置,所以他们的选择基于这样一个事实:
两种可能的选择产生的奖励频率与最近几次试验的平均值大致相同。


正常的猴子可以学会根据单一事件做出选择。
在经典的延迟反应任务中,这个事件包括看到一颗花生被放在其中一个食物井里。
在其他版本的延迟响应任务中,相关事件由空间某处闪烁的视觉提示组成。


在第~\ref{chap:chap4}~章中,我们指出,患有眶前额叶损伤的猴子不会坚持,而是根据过去的事件历史(经过多次试验的平均值)来选择对象。
相比之下,完整的猴子可以将结果分配给似乎导致结果的单一、具体的选择。
我们认为中外侧前额叶皮层也有类似的情况。
在延迟响应任务中,在延迟期间之前的可视事件导致基于该单个事件生成适当的目标。
如果没有这种影响,我们的选择将取决于许多先前事件的平均值。


Diamond和Goldman-Rakic使用的延迟反应任务是对Piaget\cite{piaget1955construction}设计的任务的修改,该任务用于评估人类婴儿理解物体持久性的能力。
它被称为“A-not-B”任务。实验者在a位置藏了一个玩具,婴儿拿了出来。
然后他们把玩具藏在位置b。即使婴儿看到这个视觉事件,他们倾向于回到位置A\cite{harris1989object}。


A-not-B任务通常要求婴儿伸手去找他们看不到的目标。
值得注意的是,他们有时甚至可以在透明覆盖物下看到位于B位置的物体时到达A\cite{butterworth1977object}。
因此,婴儿似乎记得他们在最近的过去选择了A位置来获得玩具,所以他们再次这样做,而不是根据最近的视觉事件来选择玩具。
换句话说,他们似乎重复一个最近强化的选择,而不是根据玩具隐藏在B位置的视觉事件做出选择。
当儿童更成熟时,他们很容易学会使用最近的事件,即玩具隐藏在B位置,来选择那个位置作为他们的目标。


以同样的方式来看,延迟响应任务的损害可以说是解决试验间干扰的困难,而不是追溯记忆的失败。
从这个角度来看,试验对试验的干扰是由先前在A-非B任务中选择位置A或先前在延迟响应任务中选择替代位置引起的。


为了支持这一想法,有影像学证据表明,当记忆受到干扰时,中外侧前额叶皮层的作用。
Owen等人\cite{owen1999redefining}给人类受试者两个空间任务。
在其中一个位置中,他们介绍了 5 个位置,并随后立即测试了这些物品的记忆。
另一个任务是n-back任务。
在两背任务中,实验者在一系列位置呈现刺激事件,受试者必须指向两个事件之前出现刺激的位置。
因此,n-back任务类似于延迟响应任务,因为出现了一系列项目,并且主题必须将相关项目与无关项目区分开。
空间刺激的顺序决定了相关的位置。
在第一个任务中,尾部前额叶皮层 (区域8)发生了激活,其顺序无关紧要,但中外侧前额叶皮层未发生激活。
但是,在与顺序相关的n-back任务中,激活也发生在外侧前额叶皮层中。


Gray等人\cite{gray2003neural}直接表明n-back任务涉及干扰。
他们对faces使用了三背任务,并比较了高干扰和低干扰条件。
高干扰条件在所有刺激演示中都使用了相同的面孔。
低干扰条件使用了新颖的面孔两次或四次试验作为干扰因素。
与分散注意力的面孔是新颖的相比,当分散注意力的面孔与目标面孔来自同一组时,受试者的表现要差得多。
在同一项研究中,Gray等人发现,在高干扰条件下表现更好的受试者中,前额叶皮层中部和其他区域的激活更多。



\subsection{间隔试验}

与Gray等人\cite{gray2003neural}的成像研究一样,人们可以通过直接操纵干扰来研究干扰。
例如,每天只能在一次试验中测试一下猴子。
Wilson等人\cite{wilson1963effect}做了该实验,发现即使没有试验到试验的干扰,具有较大前额叶皮层损伤的猴子仍然在偶然水平上执行延迟响应任务。


这个结果有两种可能的解释。
首先,猴子可能没有学会这样一个规则,即他们在延迟期之前观察到的结果决定了他们在延迟后应该选择什么目标。
其次,Wilson等人的病变。
包括腹侧前额叶皮层和眶额皮层。
随后的实验表明具有腹侧和眶前额叶皮层合并病变的猴子\cite{passingham1971behavioural},或单独的眶额皮层病变\cite{meunier1997effects},对延迟反应任务有损伤。
如果,正如第~\ref{chap:chap4}~章所讨论的,眶额皮层学习选择和结果之间的关联,这些关联的丧失可以解释威尔逊等人看到的部分损害。
在这个观点上,仍然可能需要中外侧前额叶皮层来解决从试验到试验的干扰,因为Wilson等人报告的损害可能是由其他因素引起的。
根据这一说法,面对先前试验的干扰,中外侧前额叶皮层对于执行延迟响应任务是必要的。
当没有这种干扰时,如在Wilson等人的实验中,根据预测的结果执行任务时,眶额皮层是必需的。



\subsection{对象的使用}

除了简单地减少干扰,还可以通过在每次试验中使用不同的项目来消除干扰。
类似延迟响应的任务只能涉及有限数量的位置,但是类似的任务可以使用无限数量的对象或图片。
Levy\cite{levy1999association}在每次试验中提出了三个新颖的对象,并给了猴子一个选择一个对象的机会。
随后出现了延迟,延迟之后,猴子不得不选择其他物体之一。


\textit{佩特里迪斯}\cite{petrides1995impairments}将此称为 “自我排序” 任务,因为猴子可以选择选择对象的顺序。
因为我们知道对象是命令对象还是实验者命令对象并不重要,所以我们更喜欢将其称为有序对象任务。
Levy和Goldman-Rakic向猴子传授了这项任务,然后形成了病变,包括中外侧和后外侧前额叶皮层,一起或前额叶皮层的背凸度。
两组中的猴子都正常执行有序的对象任务。


然而\textit{佩特里迪斯}\cite{petrides1995impairments}发现,具有背侧前额叶皮层损伤的猴子在有序的对象任务上显示出严重的损伤 (图~\ref{fig:6_4})。
与其的实验一样,由于相同的原因,无法用坚持不懈的方式来解释此结果。


在后来的研究中,\textit{佩特里迪斯}\cite{petrides2000dissociable}显示,当图片用作刺激材料时,具有中外侧前额叶皮层(区域46)或背凸度(区域9)损伤的猴子也显示出此任务的损伤。
在一组更大的物体(从三到五个不等)的情况下,猴子表现出更大的损伤。


对Levy和Goldman-Rakic的研究与\textit{佩特里迪斯}的研究在关键方面有所不同。
\textit{佩特里迪斯}在试验中重复使用了相同的物品,因此猴子必须根据最近接触过的物体或图片来选择。
在此任务中,\textit{佩特里迪斯}产生了高度的审判间干扰。
相比之下,Levy和Goldman使用了新颖的物体,因此避免了干扰效应。
因此,病变在高干扰条件下引起损伤,但在低干扰条件下不会引起损伤。
这些发现支持两个相关的想法: 从试验到试验的干扰会导致中外侧前额叶皮层病变后的缺陷,而该区域会减轻正常猴子的干扰。


\begin{figure}
	\centering
	\includegraphics[width=0.44\linewidth]{chap6/6_4}
	\caption{来自有序对象任务的结果,也称为自有序任务。
		正常(对照)猴子(白条)和有病的猴子(阴影线条)的正确率。
		填充的圆圈表示每个主题的表现,用垂直线表示范围\cite{petrides1995impairments}。}
	\label{fig:6_4}
\end{figure}



\subsection{总结}

在涉及对象的任务中,每个试验使用不同的项目时,具有中外侧前额叶皮层病变的猴子表现正常; 
但是,当实验者从试验到试验使用相同的刺激集时,病变的猴子的表现接近偶然水平。
当然,在延迟响应和延迟交替任务中,实验者使用从试验到试验的同一组地点,而受损的猴子也具有接近偶然的水平。
因此,未能解决思考间的干扰可能会导致延迟响应和延迟交替任务的损害。


然而,结果没有区分对干扰易感性的三种可能解释:
\par


1.猴子在判断事件的时间顺序时可能会出错。
\par

2.他们可能不知道指导任务执行的规则。
\par

3.他们可能无法通过对目标进行前瞻性编码来补偿干扰。

我们将在接下来的三个部分中依次讨论这些可能性。



\section{时间顺序}

第一种解释表明,具有中外侧前额叶病变的猴子在区分两种刺激事件中的哪一种方面存在障碍,例如最近出现的左提示与右提示。
随着一系列试验的展开,它们按时间顺序发生,只有一个是最近的。
可以通过要求猴子根据发生的顺序在两个项目之间进行选择来测试一下第一个解释。
在每次试验中使用不同的物品消除了干扰影响。


\textit{佩特里迪斯}\cite{petrides1991functional}做了这个实验,要求猴子选择更早而不是更晚出现的物体。
他以给定的顺序提出了四个对象,后来又同时提出了两个对象作为选择。
他在每次审判中都使用了新颖的物品。
受害的猴子可以在序列中的第一项与任何替代项之间正确选择,或者在最后一项与任何替代项之间正确选择。
但是这些选择并没有告诉我们什么,因为选择第一个刺激总是会带来回报,就像避免系列中的最后一个刺激一样。


在关键测试一下中,猴子必须在系列中间发生的项目之间进行选择。
在四项系列中,第二项和第三项占据了这个中间地带。
\textit{佩特里迪斯}发现,中外侧前额叶皮层的背侧部分(区域46)的损伤在基于顺序区分项目方面造成了损害,特别是对于第二和第三项目(图~\ref{fig:6_5})。
他在五项系列中获得了类似的结果。


\begin{figure}
	\centering
	\includegraphics[width=0.7\linewidth]{chap6/6_5}
	\caption{串行时序任务的结果。
		在图~\ref{fig:6_4}~的格式中,除了纵坐标绘制了掌握问题的天数。
		根据它们在一系列项目中的排名,每对条将正常(对照)猴子(白条)和损伤猴子(阴影条)进行比较,以在两个项目之间进行选择 [1... 4]。
		垂直线虚线将涉及端点项(项目1或4)的选项与不包括端点的选项分开。
		对于后者,具有中外侧前额叶皮层病变的猴子在测试天数(失败) 的限制内未掌握问题。
		重新绘制自\textit{佩特里迪斯}背外侧额叶皮层内的功能专门化,用于序列顺序记忆\cite{petrides1991functional}。}
	\label{fig:6_5}
\end{figure}


猴子可以学习这项任务,但只有困难。
人类受试者发现使用一系列新颖的图片很容易学习任务。
Milner等人\cite{milner1985frontal}报道,额叶切除术的患者在选择最近发生的图片方面表现不佳。
他们不太可能失败,因为他们不知道规则,因为在每次试验中,实验者都要求他们选择最近的照片。


当然,这种损害可能是由这些大病变的任何部分引起的。
因此,Amiez\cite{amiez2007selective}对人们进行了成像实验,因为他们判断两个刺激中的哪一个在序列中较早出现。
激活发生在前额叶中外侧皮层。


中外侧前额叶皮层中单细胞编码顺序的发现支持了病变和激活研究。
Ninokura等人\cite{ninokura2003representation}依次呈现了三种彩色图案,猴子学会了按此顺序触摸它们。
在中外侧前额叶皮层中,43\% 具有延迟周期活性的细胞编码了出现图片的序列。
Funahashi等人\cite{funahashi1997delay}发现了空间刺激的相似结果。
他们的猴子必须按照它们出现的顺序选择两个目标,以及表现出延迟期活性的中外侧前额叶皮层细胞,62\% 编码的刺激顺序。


Warden\cite{warden2007representation}的一项研究提出了一种可能的顺序编码机制。
他们记录在中外侧前额叶皮层,而猴子执行一项任务,他们必须判断最近发生的两张照片中的哪一张。
第二张图片的活动超过了第一张图片的活动,该属性可以提供判断顺序的机制。



\subsection{总结}

由于无法判断事件的顺序,可能会导致延迟响应任务的损害。
在一系列试验中,视觉事件指导着每个可能目标的选择,只有最后一个与当前试验相关。
本节回顾了具有中外侧前额叶病变的猴子在区分事件顺序方面存在障碍的证据。
同样,中外侧前额叶皮层中的细胞编码事件顺序。
回想一下,Tsujimoto\cite{tsujimoto2012prefrontal}表明,当猴子在延迟响应任务的动眼运动版本中犯了坦率的错误时,他们通常会选择先前试验中合适的目标(第~\ref{chap:chap5}~章)。



\section{规则}

我们对损害的第二种解释表明,受损的猴子无法学习或应用任务规则。
要执行延迟响应任务,猴子必须以某种形式应用以下规则: 最近的视觉事件的位置来确定当前目标。


猴子可以学习许多这样的规则,我们知道,当猴子学习一个规则时,中外侧前额叶皮层中的细胞会像在前额叶皮层的其他部分一样编码这些规则。
第~\ref{chap:chap7}~章针对腹侧前额叶皮层详细介绍了此问题。
例如,前额叶皮层中的细胞对条件视觉运动规则和空间规则\cite{wise1999role}具有不同的活性。
单元格的编码匹配或不匹配规则\cite{wallis2001single}以及匹配规则是否涉及颜色或形状\cite{mansouri2006prefrontal}。
此外,前额叶神经元前瞻性地编码这些规则,即在其实现之前\cite{wallis2001single}。


Buckley等人\cite{buckley2009dissociable}的一项研究支持了中外侧前额叶皮层在执行任务规则中的作用。
他们测试了已经学习了两个版本的匹配样本任务的猴子。
在一个版本中,猴子根据形状进行匹配; 
在另一个版本中,它们根据颜色进行匹配。
在每次试验中,他们选择了三种刺激: 
一种是按颜色匹配样本,
另一种是按形状匹配样本,
另一种是不匹配特征。
一旦动物达到一条规则的标准,实验者就会通过仅包含奖励或非奖励的反馈来提示切换到另一条规则。


猴子在手术前学习了这些规则,并且具有中外侧和后外侧前额叶皮层病变的猴子可以重新学习并在它们之间切换。
但是,如果在他们重新学习了其中一条规则之后,他们遇到了比通常更长的审判间隔,则他们的表现50\% 正确\cite{buckley2009dissociable}(图~\ref{fig:6_6})。
此性能水平与机会水平相对应,前提是人们忽略了不遵循任何规则的选择。
这一发现向Buckley等人暗示,中外侧前额叶皮层在维持记忆中的当前任务规则中起作用。


\begin{figure}
	\centering
	\includegraphics[width=0.63\linewidth]{chap6/6_6}
	\caption{猴子在规则任务上的表现类似于威斯康星卡片排序任务。
		猴子执行与样本匹配的任务,按颜色或形状进行匹配,在试验块中交替进行。
		延迟期后,正常(对照)猴子(白条)和中外侧前额叶皮层(阴影条)病变的猴子的正确性能百分比从 6 秒增加到 11 秒。
		虚线显示了整个任务(正确 33\%)和遵循两个规则之一的选择(正确 50\%)的性能的机会水平。
		误差条: SEM\cite{buckley2009dissociable}。}
	\label{fig:6_6}
\end{figure}


其他证据表明,背侧前额叶皮层(包括外侧区域9,中外侧和后外侧前额叶)在学习规则中起作用。
关于延迟交替任务的规则之一是,如果猴子犯了错误,则奖励将在下一次审判中位于同一侧。
此规则之所以适用,是因为标准的延迟交替任务使用了纠正程序,这有助于动物掌握任务。
这意味着猴子可以采取一种称为 “迷失” 的策略,以帮助他们解决问题。
根据此规则,当一个目标的选择未能得到回报时,该目标应在下一次判断中被拒绝。


Passingham\cite{passingham1975delayed}分析了背侧前额叶皮层病变的猴子在延迟交替任务上的错误。
在测试的三只动物中,一只进行了420试验来学习 “迷失”策略,一只进行了900试验,一只在1000试验中没有学习。
相比之下,两只正常的猴子花了240和60来学习相同的策略,而一只猴子不需要训练就可以应用该规则。
因此,具有背侧前额叶皮层病变的猴子在获得 “迷失转移” 策略方面存在严重障碍。


一旦猴子学会了纠正试验的 “失速策略”,他们就需要学习延迟交替任务的常规试验规则。
此规则定义了任务: 每次试验的目标应该是在先前试验中没有奖励的地方。
换句话说,他们必须学习 “双赢” 规则。
与诸如 “双赢” 或 “亏损” 之类的规则相比,“双赢” 规则具有任意性质。
强化使动物更有可能选择一个有回报的目标,而 “双赢” 规则支持这一原则。
同样相反的是 “迷失转移”。
然而,“双赢” 规则反对最近的加固历史。
在Passingham\cite{passingham1975delayed}的研究中,具有背侧前额叶皮层病变的猴子在1000试验中未能学习 “双赢” 规则。


\subsection{总结}

延迟响应和延迟交替任务的损害可能反映出未能应用适当的任务规则。
延迟响应任务使用以下规则: 最近的视觉事件确定猴子应选择的目标: 
其选择应与该事件的位置匹配。
延迟交替任务使用的规则是,应该拒绝最近的目标选择,而选择替代方案。


这种可能性与第~\ref{chap:chap4}~章讨论的关于选择-结果关联的信用分配的想法有间接的关系。
具有眶额皮层病变的猴子在单个事件的基础上在学习选择结果关联方面存在障碍。
可能是,具有中外侧前额叶皮层病变的猴子有一个相关的问题: 
也许他们无法实现使用单个事件(最近提示的位置)来生成当前目标的规则。



\section{前瞻性编码}

早些时候,我们提出了第三种解释损伤对延迟反应任务的影响。
也许具有中外侧前额叶皮层损伤的猴子可以学习这一规则,但不能通过前瞻性地编码记忆中的目标来补偿记忆中的干扰。
根据这种观点,对当前目标的短期记忆可以避免一审接一审的干扰。


\begin{figure}
	\centering
	\includegraphics[width=0.7\linewidth]{chap6/6_7}
	\caption{独立的编码先前的目标和现在的目标。
		(A)大多数前额叶神经元编码当前目标或者上一个,只有几个编码(混合)。
		这些数据表明,前额叶皮层细胞编码回顾性记忆和前瞻性记忆。
		(B)编码先前的细胞位置和或现在的细胞位置。
		空圆圈表示没有这种编码的位置;
		浅灰色圆圈与深灰色圆圈分别表示具有这些特性的神经元小于15\%或多于15\%的部位。
		虚线显示第二个猴子的沟,叠加在第一只猴子的脑沟上(实线脑线沟)。
		(C)针对群体层面的先前和现在的目标信号。
		虚线显示平均为回顾性编码信号;
		黑色的线显示了潜在的意思是信号编码\cite{genovesio2006representation}。
		表示未来和以前的空间目标的单独的前额叶皮层的神经种群。}
	\label{fig:6_7}
\end{figure}


在一项研究中已经提到,Genovesio\cite{genovesio2006representation}记录的神经元活动的两个部分背前额叶皮层(面积46和9)(图~\ref{fig:6_7}B)。
在每个试验中,这只猴子已经选择在三个空间目标:左,右,注视点。
也在这一实验中,这只猴子已经使用回顾性记忆最近的目标选择的当前目标的选择(前瞻记忆)(图~\ref{fig:6_7}C)。


Genovesio 等人在背侧前额叶皮层发现了 2 个独立的细胞群。
一个编码目标在前一个试验中的位置,另一个编码目标在当前试验中的位置(图~\ref{fig:6_7}A)。
只有少数细胞编码位置更一般(杂交细胞)。
病变的影响可能因此使动物无法区分过去和现在的目标,因为在前瞻性编码失败。
如果无法单独区分以前目标和现在目标的位置,就会造成延迟响应任务的不足。
动作中外侧前额叶背内侧前额叶 B 先前目标未来目标 PS 目标信号(脉冲数/秒)目标选择细胞先前目标未来目标杂交。


\begin{figure}
	\centering
	\includegraphics[width=0.58\linewidth]{chap6/6_8}
	\caption{在成像实验中使用的匹配和回忆任务。
		在展示了一系列的位置(左图,弯曲的箭头)和一个延迟时间(中间)之后,受试者需要执行两个任务中的一项(右图)。
		在顶部的分叉中,受试者看到一系列的空间线索,并必须报告它是否与最初呈现的线索相匹配,称为匹配条件。
		在底部的分叉中,受试者必须复制最初观察到的序列,称为回忆条件。
		只有在回忆条件下,受试者才会在延迟期间计划出一系列目标的动作。
		背外侧前额叶皮层在即将到来的行动准备中的作用:功能磁共振成像研究\cite{pochon2001role}。}
	\label{fig:6_8}
\end{figure}


对人类研究对象的研究也为中外侧前额叶皮层的前瞻性编码提供了证据。
Pochon等人\cite{pochon2001role}研究了受试者执行两项任务来测试记忆序列时的激活。
视觉刺激出现在多达5个位置的序列中,在延迟6秒后,实验者通过两种方式之一来评估记忆(图~\ref{fig:6_8})。
首先,他们在延迟后呈现另一个序列,被试必须按下一个按钮来报告它是否与原始序列相匹配。
第二,在延迟期之后,受试者必须按照他们出现的顺序指向每个位置。
他们称为第一个条件匹配和第二个条件回忆。


在匹配条件下,受试者在延迟期间不知道是否会按下按钮。
因此,这种情况要求他们记住感觉线索(回顾性记忆) ,但排除了未来行动的计划(前瞻性编码)。
相比之下,在回忆条件下,受试者可以在延迟期间准备他们的行动(前瞻性编码)。


Pochon等人发现,在匹配条件下,尽管需要记住一系列线索位置,但在中外侧前额叶皮层中没有显著的延迟期激活。
相反,在回忆条件下,他们发现该区域的延迟期激活非常显著(图~\ref{fig:6_9})。
匹配和回忆条件都需要回顾性空间记忆。
条件的不同之处在于,回忆条件要求计划未来的一系列目标或行动:前瞻性编码。


\begin{figure}
	\centering
	\includegraphics[width=0.62\linewidth]{chap6/6_9}
	\caption{在图~\ref{fig:6_8}~中所示的任务中的成像激活。
		填充方块:召回条件;未填充方块:召回条件控制;
		填充圆:匹配条件,未填充圆:匹配条件控制。
		误差条: S.E.M.最大的延迟期激活,在11秒达到峰值,当受试者计划一系列空间目标(回忆任务)\cite{pochon2001role}。}
	\label{fig:6_9}
\end{figure}


Pochon等人还发现,在尾侧前额叶皮层和后顶叶皮层中存在延迟期激活。
然而,在这些区域,激活发生在匹配任务和回忆任务,任务之间没有差异。
因此,中外侧前额叶皮层与尾侧前额叶和后顶叶皮层在回顾性编码(匹配过程中)缺乏明显的显著激活,这表明前额叶皮层激活的特殊之处涉及前瞻性编码。


同样的研究人员追踪他们的实验设计任务有两个延迟时间\cite{volle2005specific}。
受试者再次看到一系列位置,他们必须记住。
只有第二延迟周期允许未来的目标的潜在编码。
显著的延迟周期中外侧前额叶皮层的激活发生在第二次延迟周期而不是第一个。
在一个控制任务中,一个新的序列连续第二次延迟周期期间出现在屏幕上,和受试者准备执行序列代替原来的。
如果延迟周期激活反映了计划为未来的目标序列,而不是前一顺序的记忆线索,然后每个人都应该看到相同的激活控制任务的主要任务,而这正是Volle等人发现的。


大家可以反对,明显缺乏延迟期激活在第一延迟期是由于不敏感的\textit{血氧水平依赖}信号。
而事实上,戈德曼-拉基奇\cite{goldman2002functional}提出了这个反对当罗\cite{rowe2000prefrontal}报道缺乏延迟期激活空间记忆任务。
在任务被罗等人受试者可以记住刺激延迟期间,但他们不能前瞻性编码的目标。


有可能测试是否缺乏延迟周期激活是否是由于方法的不敏感性。
Sridal\cite{srimal2008persistent}使用特殊的线圈来提高敏感性,他们可以在空间匹配任务中检测到背侧前额叶皮层(额上沟)的嘴侧的延迟期激活。
这一发现与猴子的细胞记录的结果一致。
如前所述,Genovesio等人\cite{genovesio2006representation}在该区域发现了一个编码回顾性记忆的细胞亚群。
因此,不能用成像结果来证明这种活动的缺失。
相反,他们认为在中外侧前额叶皮层(第46区)的前瞻性激活优于回顾性激活,这可能是基于更大程度的突触活动(第~\ref{chap:chap1}~章)。



鉴于在中外侧前额叶皮层中回顾性编码位置的细胞,我们的观点的批评者仍然可以声称这种活动支持回顾性工作记忆。
在Pochon等人后来使用的任务上测试了背侧额叶皮层损伤的患者。
他们发现患者在匹配条件下没有损伤,不涉及目标前瞻性编码的情况。
原因可能是后顶叶皮层的细胞可以支持位置记忆。
然而,正如成像结果所预期的那样,相同的患者在回忆条件下有障碍,涉及对目标进行前瞻性编码的条件\cite{ferreira1998spatio}。等人(2006)也报道了背额叶损伤的患者在回忆来自一系列位置的序列时具有损伤。
因此,本节中的所有研究都指出了前瞻性编码在中外侧前额叶皮层功能中的重要性。



\subsection{预期的编码和干扰}

我们建议未来编码补偿干扰在记忆中。
累加器网络的概念,如第~\ref{chap:chap3}~章介绍,说明了这是如何工作的。
第~\ref{chap:chap5}~章用这个概念来解释自顶向下竞争之间的偏见表示关注。
根据这种观点,前额叶的影响皮层类似一个“规模拇指”为一种感官的代表提供一个优势与另一个集成“证据”对一个阈值。
延迟反应任务,最近有关视觉事件作为至关重要的“证据”代表目标适合一个累加器网络事件。
如果最近的线索或食物出现,例如,这个事件提供了足够的“证据”的生成一个目标。
自顶向下的注意第~\ref{chap:chap5}~章的区别和自顶向下的注意延迟反应任务是,前者涉及感官处理,后者涉及目标的一代。
成功的表现,最近的事件的记忆,右边的提示在这个例子中,必须在竞赛中获胜的记忆更少的最近的事件,如早期的线索。


在一项针对人类研究对象的研究中,Sakai等人\cite{sakai2002active}通过故意在任务中引入干扰物来测试这一建议。
受试者看到了五个空间位置的序列,在一半的试验中,在一个可变的延迟期之后,又出现了5个干扰物的位置(图~\ref{fig:6_10}A)。
在呈现干扰物后,实验者测试被试的记忆:首先是干扰物项目的顺序,然后是原始项目的顺序。


\begin{figure}
	\centering
	\includegraphics[width=0.7\linewidth]{chap6/6_10}
	\caption{(A)空间记忆任务在成像实验中的应用。
		填充的黑色方块表示空间线索的原始序列; 填充的灰色圆圈表示一系列干扰物。
		有些试验缺乏干扰因素。
		灰色的星星代表一个探测器的干扰项目。
		受试者必须说出这个位置是否是干扰器系列中的一个。
		在这个例子中,它是(从顶部开始的第三个)。
		右下方面板中的箭头显示了从一个位置到另一个位置的可能转换。
		受试者必须说明这种转变是否发生在原始序列中。
		在示例中,它做到了,正如左栏第三和第四面板中的箭头所指出的那样(受试者没有看到)。
		(B)性能准确性作为中外侧前额叶皮层激活(\textit{血氧水平依赖}信号)在延迟期间的功能。填充圆圈: 有干扰物的试验; 未填充圆圈: 没有干扰物的试验。
		受试者事先不知道是否会出现干扰物\cite{sakai2002active}。}
	\label{fig:6_10}
\end{figure}


尽管匹配过程测试记忆,使用长延迟迫使科目演练项目。
Tremblay\cite{tremblay2006rehearsal}表明,当允许这样做的时候,主题往往通过他们的眼睛从一个位置移动到另一个地方。
当然,人们可以记住位置没有采用这个策略,但是排练地点以这种方式帮助他们保留信息,如图所示,如果对象移动他们的眼睛在延迟无关紧要的位置,他们也不记得项目\cite{guerard2009processing}。


Sakai\cite{sakai2002active}使用长延迟和发现显著延迟周期激活中外侧前额叶皮层,即使使用了一个匹配的过程。
这支持他们所担负的序列。
Leung\cite{leung2002sustained}也使用一个匹配的过程长延迟(18或24秒),和他们也发现延迟时间越长,分心的可能性就越大,因此熟练的动机就越大。


在Sakai等人\cite{sakai2002active}的研究中,延迟在8-16秒之间变化,直到5秒后才进行了对该位置的记忆测试。
此外,受试者不知道在任何给定的试验中是否会出现干扰物。
图~\ref{fig:6_10}~显示,在使用干扰物的试验中,延迟期激活的程度与受试者反应的准确性密切相关(图~\ref{fig:6_10}B)。
这些结果表明,延迟期激活可以保护记忆不受干扰。


这项研究刚刚描述了所使用的位置作为刺激物。
Sakai\cite{sakai2004prefrontal}随后使用字母作为刺激物,以操纵干扰的程度。
他们通过使用与记忆项目相同或不同的干扰物来做到这一点。
和之前的研究一样,实验者展示了五个要记住的字母的序列,然后是一个延迟。
然后,他们以另外五个字母或五个数字作为干扰物,然后测试原始字母。
显然,带有干扰物的字母比数字产生了更多的干扰。


在早期的研究中,Sakai等人\cite{sakai2002parahippocampal}发现中侧前额叶皮层的延迟期激活。
在记忆测试期间,在检索原始字母序列时也发生了激活。
重要的是,高干扰条件(字母干扰器)比低干扰条件(数字干扰器)产生显著更大的激活。
这一发现支持这样的观点,即前瞻性激活中外侧前额叶皮层可以帮助保护记忆免受干扰。



\subsection{总结}

我们对受干扰敏感性的第三个可能的解释援引了勘探的概念。
本节提出的证据表明,预期编码,以细胞活动和区域激活的形式,发生在中外侧前额叶皮层。
我们不否认这一区域的一部分细胞反映了延迟期间的回顾性编码,但至少在成像数据中,前瞻性编码似乎占主导地位。


我们提出,中外侧前额叶皮层的前瞻性编码可以补偿记忆中的干扰。
随着延迟期激活的增加,面对干扰的记忆变得更加准确。
而随着干扰的增加这些发现支持了这样一种观点,即中外侧前额叶皮层受损的猴子在回忆时会出现激活增加。
外侧前额叶皮层屈服于干扰,因为它们不能前瞻性地编码当前目标。


综合最后三个主要部分,我们得出结论: 中外侧前额叶皮层损伤导致延迟反应和延迟交替任务的损害,其原因有三个相关原因中的一个或多个。
首先,他们可能已经失去了区分最近相关事件和早期事件的能力。
其次,他们可能无法学习或实现任务规则,这些规则要求使用最近的事件来选择当前目标。
第三,他们可能无法通过对当前目标进行前瞻性编码来弥补近期试验的干扰。
我们将这三种可能性分别称为顺序编码、规则编码和预期编码解释。



\section{目标产生}

我们不知道顺序编码、规则编码或前瞻性编码这三种可能性中的哪一种是导致中外侧前部皮层(46区)损伤后延迟反应损伤的原因,也许它们都是导致延迟反应损伤的原因。
不管怎样,所有这些任务的一个关键因素是猴子必须在不同的试验中改变他们的目标选择,人体实验也有相同的要求。


我们早就知道,如果受试者产生一系列动作,尽可能随机地改变它们,中外侧前部皮层就会被激活。
在Deiber等人的研究中\cite{deiber1991cortical},受试者在四个方向中的一个移动操纵杆,在Frith等人的研究中\cite{frith1991willed}他们移动了两个手指中的一个。
在这两项研究中,受试者决定在每种情况下做什么动作。
这导致人们认为这些任务涉及“自由选择”\cite{playford1992impaired}或“意志行动”\cite{frith1991willed}。


自这些研究以来,出现了更多的研究,并且都发现了相同的结果\cite{frith2000role}。
此外,正如Rowe等人\cite{rowe2005prefrontal}所表明的那样,当受试者产生手指运动时,不仅中外侧前额叶皮层被激活,而且当箭头方向指示运动时,这种激活不会发生。


考虑到受试者在不同的试验中动作不同,他们需要注意最后几个动作。
因此,在延迟反应和延迟交替任务中,每次试验的选择取决于先前的选择。


Barraclough等人\cite{barraclough2004prefrontal}的一项神经生理学研究同样要求猴子在左目标和右目标之间做出一系列选择。
一个计算机程序决定一个选择是否会得到奖励,它会随着时间的推移而改变,以鼓励猴子切换目标,很像第四章回顾的对象逆转任务。
作者记录了中外侧前部皮层,发现了相当大比例的细胞在之前的试验中编码了选择,以及其他信息。
例如,只有当前一个目标在左边时,一个单元格才可能编码一个在右边的目标。
许多细胞也会对之前的选择是否产生了奖励进行编码。
这些细胞整合了关于先前选择及其结果的信息。
Barraclough等人得出结论,他们更新了一个决定当前选择的价值函数,就像第~\ref{chap:chap4}~章讨论的眶额皮层中的细胞一样。


Tsujimoto\cite{tsujimoto2004neuronal,tsujimoto2005neuronal}发现有证据表明,中外侧前额叶的细胞编码目标和结果的组合,Tsujimoto等人\cite{tsujimoto2011comparison}也得到了类似的结果。
后一项研究的任务包括一个视觉线索,提供一个从之前的空间目标转移或停留的指令。
如图~\ref{fig:6_11}~所示,最上面的单元格编码了之前的目标,下一个单元格编码了未来的目标,下一个单元格编码了策略和目标的结合,最下面的单元格在得到提示后立即编码了策略,然后又改变编码了当前的目标。


因此,细胞记录结果与成像结果一样,支持了中外侧前部皮层在产生目标中起作用的建议。
这些结果也指向一系列事件的重要性,包括在连续试验中出现的目标和线索。
通常,一次试验的目标取决于之前试验的结果。


\begin{figure}
	\centering
	\includegraphics[width=0.5\linewidth]{chap6/6_11}
	\caption{提示策略任务中中外侧前额皮层活动模式。
		四个单元格中的每一个都显示不同的编码属性。
		细线显示的是停留提示下的平均细胞活动;
		粗线表示移位提示的活动。
		灰线表示导致选择当前目标的试验活动;
		黑线在右边表示当前目标的活动。
		线索类型和先前目标的组合决定了每次试验的当前目标。
		例如,对于顶部的单元格,细灰线显示了带有停留提示的试验活动,左边是先前的目标,因此左边是当前的目标。
		顶部的单元格也更喜欢带有移动提示的试验,将之前的目标移到左边,因此将当前目标移到右边(移动)。
		细胞活动的共同特征是,它会对左边的前一个目标进行编码。
		下一个单元格倾向于左侧的当前目标(灰线);
		下一个更倾向于将停留策略和当前目标结合在一起(细黑线);
		底部的单元格在提示期间编码移动策略(粗线),但在试验后期编码向右当前目标的选择(黑线)\cite{tsujimoto2011comparison}。}
	\label{fig:6_11}
\end{figure}


如果每次试验的顺序不同是至关重要的,那么一个简单的预测就出现了:
在这些任务的第一次试验中,活动和激活应该是缺失的。
首先,第一次试验与前一次试验没有任何不同。
第二,第一次试验的选择不依赖于前一次试验中的任何事件。
在第一次审判中,没有最近发生的事件提供做出选择的背景。


因此,Rowe等人\cite{rowe2010action}在一系列手指运动的第一次试验中检查了激活。
正如预期的那样,在这项任务中,激活发生在中外侧前部皮层,这在许多试验中是平均的。
但至关重要的是,在第一次试验中,该区域没有出现明显的激活。
这一结果并不反映该方法缺乏统计能力或灵敏度;
当作者在这个系列的中间选择一个单一的试验时,他们可以检测到中外侧前部皮层的显著激活。


如果这种激活在不同系列的产生中起关键作用,那么前额叶皮层大损伤的患者在“自由选择”任务中应该受损。
Johns\cite{johns1996effects}收集了20个这样的病人,让他们用一个可以向四个方向移动的操纵杆做一系列的动作。
与对照组相比,患者产生了更多的刻板印象和更少的随机动作序列。


Johns等人的研究涉及较大的病变,但成像和\textit{重复经颅磁刺激}可以更准确地定位关键区域。
Jahanshahi等人\cite{jahanshahi2000role}观察到,当受试者随机生成一系列数字时,中外侧前部皮层被激活,随机序列越多,激活程度越高。
在随后的一项研究中\cite{jahanshahi1998left},同样的研究人员将\textit{重复经颅磁刺激}应用于该区域以破坏其活动,这种损伤导致该系列变得更加刻板化。
因此,中外侧前部皮层似乎在不同试验中做出不同的选择时发挥了必要的作用。



\subsection{总结}

本节与前三节一样,回顾了中外侧前部皮层根据事件顺序产生目标并对目标进行前瞻性编码的证据。
首先,当人们产生一系列目标时,中外侧前额叶皮层会被激活。
第二,激活反应了产生一系列不同目标的需要。
第三,它似乎在一个系列的第一次审判中缺席。



\section{规划一个序列}

在前一节描述的任务中,受试者在试验中产生一系列目标。
然而,我们并不总是知道他们是否只是在每次试验中产生下一个目标,还是他们事先计划了一系列目标。


为了找到答案,Averbeck等人\cite{averbeck2006activity}教猴子三次眼球运动的序列,它们通过反复试验来学习。
猴子可以在几次试验中学会这么短的序列。
作者记录了前额叶皮层的后外侧,额叶视区的嘴侧,并发现了编码特定序列的神经元的猴子在做出动作之前就计划好了。
通过一种简单的解码算法,他们可以预测猴子什么时候会犯错误,以及错误是什么。
这些结果表明猴子提前计划了一系列的动作。


在一项相关研究中,Shima\cite{shima2007categorization}教授猴子一系列动作。
在他们的实验中,猴子做的是伸手的动作,而不是眼球的动作,他们操纵一个可以转动、推或拉的把手。
每个序列由四个动作组成。
每天,猴子通过视觉线索学习一个序列,然后根据记忆重复它。
当猴子准备产生各种序列时,Shima等人从后侧前额皮层进行了记录。
通常,在执行序列之前会有一段延迟期。


在与任务相关的细胞中,超过40\%的细胞表现出延迟期活动。
Shima等人\cite{shima2007categorization}总共训练了11个序列,它们可以比较具有相似模式的序列。
在这个意义上,模式指的是序列中的转换。
例如,一个模式由双重复组成,AABB,其中a和B代表序列中的不同元素。
另一种模式由交替组成:ABAB。
这两个序列都由相同的元素组成,比例相同,但它们在过渡次数上有所不同。
当然,A和B在不同的序列中是不同的,除此之外还有第三个刺激。


在不同序列表现出不同活性的细胞中,超过一半的细胞在相似模式的序列中表现出相似的活性。
例如,作者举例说明了一个神经元,它编码了双重重复序列“转,转,推,推”和“拉,拉,转,转”,但没有编码重复序列:“转,转,转,转”或“拉,拉,拉,拉”。
该细胞更倾向于AABB模式而不是AAAA模式。
Shima等人将这一结果解释为序列类别的编码,如双重复(AABB)、交替(ABAB)或重复(AAAA)。
在具有延迟期活动的细胞中,大约一半的细胞表现出特定类别的活动(图~\ref{fig:6_12})。
Shima等人发现这些细胞在主沟背侧比在主沟腹侧多。


在目前所考虑的序列中,实验者指定了序列中元素的顺序。
但是我们可以修改任务,让猴子从一个目标开始,并且必须找出实现这个目标所需的一系列元素。
例如,Mushiake等人\cite{mushiake2001visually}向猴子展示了一个视觉迷宫(图~\ref{fig:6_13}),任务要求它们使用手柄将光标移动到迷宫的最终目标。


一旦猴子学会了一系列成功的动作,实验人员就会引入一个新的视觉障碍,这样猴子就必须规划一条新的路线。
Mushiake等人记录了中外侧前部皮层的活动,包括主沟的背侧和腹侧,他们的分析集中在运动前的延迟期。
在延迟期间,细胞对当前目标进行编码。如果实现最终目标需要三个从属目标的序列,则不同的细胞亚群为每个目标编码\cite{mushiake2006activity}。
通过操纵手柄和光标运动之间的空间变换,Mushiake等人可以测试一个特定的细胞是编码一个从属目标还是一个特定的肢体运动。
例如,手柄向左移动可以根据当前生效的空间变换,使光标向左或向右移动。
而且,正如预期的那样,大多数细胞都为目标而不是运动编码。


\begin{figure}
	\centering
	\includegraphics[width=0.5\linewidth]{chap6/6_12}
	\caption{背侧前部皮层细胞序列编码。
		实线表示这个神经元的偏好序列,所有这些序列都遵循一个交替的模式:ABAB,不同的运动组成了这个序列。
		虚线表示不遵循这种交替模式的序列\cite{shima2007categorization}。}
	\label{fig:6_12}
\end{figure}


\begin{figure}
	\centering
	\includegraphics[width=0.4\linewidth]{chap6/6_13}
	\caption{视觉迷宫任务。
		猴子需要通过手部动作将光标从当前位置(灰色方块)移动到最终目标(黑色方块)。
		实验者可以改变手的运动来产生给定的光标运动\cite{gaffan1996associative}。}
	\label{fig:6_13}
\end{figure}


为了解决迷宫任务所带来的问题,猴子必须计算从属目标和最终目标。
中外侧前额叶皮层的细胞负责其中一种、另一种或两者的编码\cite{saito2005representation}。
这使得Sakamoto等人\cite{sakamoto2008discharge}能够分离成对的细胞,其中一个编码从属目标,另一个编码最终目标。
当一个神经元编码从属目标和另一个神经元编码最终目标之间发生转换时,放电的同步性达到峰值。
同样,在要求使用先前目标选择当前目标的任务中,Tsujimoto等人\cite{tsujimoto2008transient}发现编码当前目标的细胞与编码先前目标的细胞之间存在活动相关性。
当猴子从之前的目标转移到之前的目标时,这些相关性会增强。


如果中外侧前额叶皮层在计划一系列目标中起着必要的作用,那么该区域受损的猴子应该在这些任务中表现出损伤。
Passingham\cite{passingham1985memory}在Collin等人\cite{collin1982role}设计的空间搜索任务上测试了猴子。
猴子坐在25扇门前,每扇门后面都放着一颗花生。
猴子只需要伸手去打开一扇又一扇的门,取出花生。
我们称之为25扇门搜索任务。


因为门是不透明的,所以这项任务要求猴子去够它看不见的目标。
每扇门打开一次,而且只打开一次,因为实验者只有在经过很长一段时间后才会更换花生。
正常的猴子很快就学会了这项任务。
相比之下,前额皮层中外侧和后外侧受损的猴子更经常回到它们已经打开的门。
他们的搜索模式显示出严重的混乱。


但这种混乱并不一定反映出计划的缺陷。
这可能是由于没有记住他们已经打开了哪些门。
Owen等人在两项损伤效应研究中排除了这一解释。
他们用电脑版的空间搜索任务测试病人。
额叶切除术患者受损;和猴子一样,他们倾向于回到已经试过的盒子。
但是Owen等人也测量了他们搜索模式的一致性。
他们推断,有规律地、有序地触摸这些盒子可以减少记忆负荷。
前额叶受损的患者在选择开始搜索的盒子中表现出较少的规律性。
由于这是第一个选择,这种不规则性不能归因于忘记了一系列的前一个选择,因此可以排除对结果的记忆解释。


定期搜索的一个好处是它减少了对工作记忆的需要。
Taffe\cite{taffe2011rhesus}为猕猴执行空间搜索任务开发了一个“策略得分”。
猴子掌握了最小化动作之间距离的策略,如果没有使用这一策略,它们就会犯错。
实验还没有完成,但根据25门搜索任务的结果,我们预计有中外侧前部皮层损伤的猴子的策略得分会很低。


我们在第~\ref{chap:chap2}~章和第~\ref{chap:chap5}~章中谈到了这种伸展运动的一个关键特征。
我们解释说,在中央凹进化之后,灵长类动物发展了一种新的伸手方式,一种在注视中心坐标系中计算当前伸手目标和手的当前位置的方法(见图~\ref{fig:5_3})。
我们提出,在类人猿进化出一种以注视为中心的坐标到达目标的机制后,这个系统成为以有序顺序到达目标的必要条件,特别是对于看不见的目标。
为了达到可见的目标,运动前和顶叶机制就足够了,所以前部皮层的损伤不会影响这种行为。
当我们稍后考虑为什么中外侧前部皮层在类人猿而不是其他哺乳动物的延迟反应任务中起必要作用时,这些观点变得重要起来(第~\ref{chap:chap10}~章)。



\subsection{总结}

前面的章节表明:(1)中外侧前部皮层根据事件的顺序、时间和位置提供的上下文产生目标,
(2)它可以通过顺序编码、规则编码或预期编码来实现目标。
本节将这个想法扩展到一系列目标。除了具体的目标,比如地点,中外侧前部皮层的活动反映了一个序列的抽象结构。
因此,中外侧前部皮层既代表具体的目标,也代表抽象的目标,它既代表单个目标,因为它们在不同的试验中有所不同,也代表一系列目标,因为它们在结构化的序列中发展。
中外侧前部皮层受损的人和猴子在组织序列方面表现出效率低下,这不能归因于记忆损伤。


随后,在第~\ref{chap:chap8}~章中,我们提出前额皮层可以编码抽象表征,因为它位于目标层次的顶端。
在该层次结构的连续阶段,高阶细胞可以将来自低阶细胞的信息组合在一起。
通过这种方式,前额叶皮层可以对目标的抽象表征进行编码。
下一节认为,前额叶皮层也编码当前情境的表征。



\section{连词和当前语境}

在前面的章节中,我们已经讨论了当前上下文可以通过一系列事件的顺序和它们的空间位置来指定。
但其他信息有助于当前的环境,背侧前额叶皮层也编码这些方面的视觉事件。


\subsection{时间间隔}

时间间隔的持续时间,就像事件在时间上的顺序一样,在建立行为环境中起着重要作用。
后顶叶区\textit{侧顶叶}的细胞编码时间间隔的长度\cite{leon2003representation},该区域投射到背侧前额叶皮层。


因此,人们会期望在前皮层中发现反映时间间隔的活动。
在最近的一项研究中,Yumoto\cite{yumoto2011neural}从外侧区9的皮层记录。
一盏灯在指定的时间内出现,猴子被训练在相同的时间间隔过后按下按钮来报告这个时间间隔。
他们发现了两组神经元。
在指定的时间过去后,一群细胞变得活跃。
Genovesio等人\cite{genovesio2006neuronal}在猴子没有进食时发现了类似的结果报告时间间隔。
Yumoto等人还发现,当猴子在没有外部提示的情况下复制时间间隔时,细胞变得活跃。


因此,我们可以预期,在繁殖的间隔时间内,有外侧9区损伤的猴子会受到损害。
Passingham\cite{passingham1978information}训练猴子报告它们按了一个键1次还是5次,Manning\cite{manning1978dorsolateral}训练猴子报告它们按了杠杆32次还是64次。
背侧前部皮层的损伤导致了这两项任务的损伤。
这些缺陷可能反映了计数方面的困难,但似乎更有可能是计时方面的缺陷。


在Passingham\cite{passingham1978information}的研究中,只有前额叶皮层中外侧和后外侧受损的猴子表现正常。
这一发现表明,背凸皮层(侧区9)的切除导致了损伤。
而且,Yumoto等人\cite{yumoto2011neural}发现,侧区失活9会导致猴子在报告时间间隔时出现错误。


Yumoto等人在研究中报告的细胞活动仅反映了时间间隔。
但我们也可以找到一些活动,它们反映了物体的身份和它们被呈现的时间之间的联系。
Genovesio等人\cite{genovesio2009feature}训练猴子区分相对持续时间,并从中外侧前部皮层的细胞中进行记录。
一个蓝色的圆圈和一个红色的正方形出现在猴子的注视点,它们持续的时间不同。
延迟一段时间后,两个刺激同时出现,一个在左边,一个在右边。
为了获得奖励,猴子需要选择持续时间较长的那个。
Genovesio等人发现,前部皮层背侧的细胞编码刺激特征和相对持续时间之间的连接。
例如,许多细胞认为蓝色刺激持续的时间更长。


\subsection{距离}

我们提到后顶叶皮层的细胞编码时间间隔\cite{leon2003representation}。
那里的一些细胞编码条的长度\cite{tudusciuc2007neuronal}。
因此,Genovesio等人\cite{genovesio2011prefrontal}记录了猴子在执行距离识别任务时中外侧前额皮层的活动。
他们使用了和刚才提到的持续时间辨别任务相同的刺激物:一个蓝色的圆圈和一个红色的正方形。
一种刺激出现在参考点上方的一段距离,另一种刺激出现在参考点下方的不同距离。
猴子需要报告哪个刺激出现得离参照点更远。
一些细胞编码绝对距离,但更多的细胞编码相对距离。
而且,在持续时间辨别任务中,许多细胞编码的刺激特征与相对距离的连接。


如果背侧前部皮层在辨别距离方面起着关键作用,那么该区域受损的猴子在必须报告距离的任务中应该受到损害。
Mishkin等人\cite{mishkin1977kinesthetic}教猴子将杠杆移动两个不同的距离,然后通过在两个刺激中选择来报告距离。
背侧前额皮层受损的猴子在这项任务中表现不佳。
这个结果可能反映了对移动距离的错误判断,但也可能是由于判断移动持续时间的缺陷,或者两者兼而有之。



\subsection{连接}

我们已经说过,前部皮层背侧的细胞代表持续时间和距离,在前一节中,我们提到了一项研究,据报道,许多前部皮层细胞编码顺序\cite{ninokura2003representation,genovesio2009feature}。
考虑到背侧前部皮层接收来自后顶叶皮层的投影,并且那里的细胞编码相同的特征,这些结果不应该令人惊讶\cite{tudusciuc2007neuronal,bueti2009parietal}。


然而,前部皮层的细胞与后顶叶皮层的细胞在一个重要的方面有所不同。
正如刚才提到的,前部皮层背侧的许多细胞编码特征连接。
例如,后顶叶皮层的细胞编码持续时间,但中外侧前部皮层的细胞还编码刺激颜色和持续时间的结合。
在Genovesio\cite{genovesio2009feature}的研究中,猴子需要报告颜色和持续时间的关联,但它们不需要报告刺激顺序,比如第一个还是第二个刺激持续的时间更长。
然而,中外侧前部皮层的许多神经元编码了相对持续时间和刺激顺序的关联。


同样,在同一作者\cite{genovesio2011prefrontal}的后续研究中,猴子必须报告刺激颜色和相对距离的关联,但不需要报告顺序信息。
然而,许多细胞同时编码这两种连接。
较少数量的细胞编码刺激是高于还是低于参考点更远。
这些发现表明,时间顺序在中外侧前额叶皮层的细胞活动中起着特别重要的作用。


图~\ref{fig:6_14}~显示了两个任务中基于顺序的连接词相对于目标的信号编码情况。
时序表明基于顺序的连词在目标的生成中起中介作用。
之后,许多细胞编码到达运动:达到目标的动作。


顺序-持续时间、顺序-距离和位置-距离连词并不是在中外侧前额皮层中观察到的唯一类型的整合。
表6.1列出了在其他任务中发生在其神经元中的一些连接表征。
例如,Hoshi\cite{hoshi2004area}在一个序列中向猴子展示了两个线索,一个告诉他们到达左边或右边的目标,另一个告诉他们使用左臂或右臂。
在中外侧前部皮层腹侧记录的细胞活动倾向于反映提示的空间方面,而在背侧记录的细胞倾向于反映手臂-目标连接。


因此,背侧前部皮层似乎是导致目标产生的多种信号的集合点。
它发挥这种功能是因为它的连接:来自后顶叶皮层的输入传递空间、时间和顺序上下文,来自眶前部皮层的输入提供结果信息,前部皮层的内在连接提供其他类型的信号。


\begin{figure}
	\centering
	\includegraphics[width=0.65\linewidth]{chap6/6_14}
	\caption{背侧前额皮层编码连词的活动。
		(A)编码顺序和相对持续时间结合的细胞的种群活动(灰线)。
		这些细胞编码了在给定的试验中,两种刺激中的第一种还是第二种持续时间更长。
		材质:SEM。
		纵坐标表示\textit{受试者工作特征}在单个细胞上的平均值,它表示每个细胞编码给定顺序-持续时间连接的平均能力。
		箭头表示机会水平的\textit{受试者工作特征}值。
		(B)相同皮层区域的顺序距离连接(灰线)。
		如图(A)所示。
		两幅图中的黑线显示了编码目标的活动,红色或蓝色刺激\cite{genovesio2011prefrontal}。}
	\label{fig:6_14}
\end{figure}



\begin{table}[htbp]
	\newcommand{\tabincell}[2]{\begin{tabular}{@{}#1@{}}#2\end{tabular}} %换行指令
	\centering
	\caption{背侧前额皮层的连词编码}
	\renewcommand\arraystretch{1.5}	%设置表格内行间距
	\begin{tabular}{llll}
		\toprule 
		连接   &  来源  \\
		\midrule
		\tabincell{c}{刺激的特征和作用\\}&Kim and Shadlen (1999)\\
		\midrule
		\tabincell{c}{刺激同一性和定位\\}&Rao et al. (1997)\\
		\midrule
		\tabincell{c}{刺激特征的连词\\}&Roy et al (2010)\\
		\midrule
		\tabincell{c}{规则和行动\\}&Wallis and Miller (2003a)\\
		\midrule
		\tabincell{c}{行动和结果(未来奖励量)\\}&Wallis and Miller (2003b)\\
		\midrule
		\tabincell{c}{行动和结果(奖励或无奖励)\\}&Tsujimoto et al (2009)\\
		\midrule
		\tabincell{c}{刺激的特征、策略和目标\\}&Genovesio et al (2005)\\
		\midrule
		\tabincell{c}{结果(奖励或不奖励)和时间\\}&Tsujimoto and Sawaguchi (2005)\\
		\midrule
		\tabincell{c}{行动和指导(记忆或视觉)\\}&Tsujimoto and Sawaguchi (2004a)\\
		\midrule
		\tabincell{c}{相对持续时间和顺序\\}&Genovesio et al (2009)\\
		\midrule
		\tabincell{c}{相对距离和顺序\\}&Genovesio et al (2011)\\
		\midrule
		\tabincell{c}{相对距离和位置}&Genovesio et al (2011)\\
		\bottomrule
	\end{tabular}%
	\label{tab:tab_6_1}
\end{table}%


\begin{figure}
	\centering
	\includegraphics[width=0.7\linewidth]{chap6/6_15}
	\caption{中外侧和眶额皮层策略信号的时序。
		(A)来自眶额皮层和中外侧前额叶皮层细胞群的策略编码信号,以及来自中外侧前额皮层细胞群的目标信号。
		(B)中外侧前额叶皮层正确与错误试验的活动。
		实线:正确试验的平均活动;虚线:错误试验的平均活动。
		黑线:首选策略;灰线,替代(反首选)策略。用虚线描述的活动在首选策略和反首选策略之间没有显著差异。
		(C)眶额皮层正确与错误试验的活动,见(B)。
		眶额皮层在正确和错误试验中有相同的策略信号。
		基于Tsujimoto S, genvesio A, Wise SP.背外侧和眶侧前额皮层策略信号的比较\cite{tsujimoto2011comparison}。}
	\label{fig:6_15}
\end{figure}


图~\ref{fig:6_15}~提供了这类交互的一个示例。
Tsujimoto等\cite{tsujimoto2011comparison}研究了编码抽象策略的神经信号与编码抽象策略的神经信号之间的关系对基于这些策略选择的目标进行编码。
第~\ref{chap:chap3}~章解释了这个任务,我们在本章的前面也提到过:一个有方向的条或一个彩色的方块提示猴子要么保持之前的目标,要么离开它。
注视点左右的白色方块作为潜在目标。


Tsujimoto等人发现,眶额皮层的细胞比中外侧前额叶皮层的细胞更早地编码该策略(图~\ref{fig:6_15}A),这与眶额皮层与下颞叶皮层的连接一致。
当策略信号在中外侧前额皮层发育时,目标信号也出现在那里,这是眶额皮层没有的特性。


中外侧和眶额皮层在另一个方面也有所不同。
在错误试验中,中外侧前部皮层的策略编码细胞显示出微弱的信号,如果有的话(图~\ref{fig:6_15}B)。
但是眶额皮层中的细胞在错误试验中确实编码了正确的策略(图~\ref{fig:6_15}C)。
综上所述,这些结果表明,眶额皮层向背侧前额叶皮层提供策略信号,在正确的实验中,后者根据该策略和记忆产生当前目标之前的目标。
之后,不管选择是否正确,眶额皮层的其他细胞都会在反馈时对选择的目标进行编码。


除了表6.1中来自前部皮层背侧的连词编码的例子外,表8.2中还出现了其他连词编码的例子,我们将前部皮层视为一个整体。



\subsection{总结}

由于其与后顶叶皮层的连接,文献倾向于强调中外侧前部皮层在处理空间位置信息中的作用。
然而,后顶叶细胞也编码时间间隔、时间和空间顺序、数量、大小、速度、长度和距离等信息。
因此,它与后顶叶皮层的连接使中外侧前部皮层除了对事件的位置进行编码外,还能对事件的许多特征进行编码。
中外侧前部皮层通过其顶叶和其他连接,对一系列先前选择的连词及其结果、顺序连词和相对持续时间、顺序连词和相对距离进行编码。



\section{结论}

\subsection{背侧前额叶皮层是如何工作的}

本章解释了背侧前额叶皮层的功能,一般来说,特别是中外侧前部皮层的功能,是如何依赖于它们的连接的:
\par

1.中外侧前额叶皮层与后顶叶皮层的连接不仅提供了视觉事件的位置信息,还提供了它们的持续时间和顺序,以及其他指标,如数量和相对距离。


这些联系解释了为什么延迟反应任务,以及其他类似的任务,主要依赖于中外侧前部皮层。
给定一系列的试验,合适的目标取决于最近的视觉事件的位置,这决定了合适的当前目标。
这种解释也解释了n-back任务的结果。
例如,在“三回”任务中,正确的选择取决于三个事件之前出现在一系列事件中的项目。
因此,我们认为后顶叶皮层向中外侧前部皮层提供的顺序信息在延迟反应任务等任务中起着关键作用。


由于其后顶叶连接,中外侧前部皮层位于背侧视觉流的末端,这一系统在行为的选择和控制中起着关键作用。
持续的活动发生在整个系统的延迟期,例如在运动前皮层\cite{wise1985primate}, \textit{前辅助运动区}和\textit{辅助运动区}\cite{shima2000neuronal},以及后顶叶皮层\cite{kalaska1995deciding}。
延迟期活动和激活也发生在中外侧前部皮层。
虽然我们知道中外侧前部皮层的许多细胞编码回溯性空间记忆\cite{genovesio2006neuronal},并且它们在反扫视实验中编码提示位置\cite{funahashi1993prefrontal},但成像结果显示,该区域的前瞻性编码激活比回溯性编码激活更占优势。
\par


2.中外侧前额叶皮层与眶额皮层有紧密的连接\cite{barbas1989architecture},因此能够获取与当前需求相关的结果信息。
\par


3.通过与运动前区的连接,背侧前额叶皮层可以促进目标的实现。
这些连接进入了专门控制手和手臂运动的前运动区域,而不是脚和腿的运动。
这种专门化表明,背侧前额叶皮层产生的目标主要是为了达到运动和操纵。
\par


4.我们还提出,背侧前额叶皮层产生一系列有序的目标,这是计划一个序列所必需的,并且它与运动前区域的连接使顺序目标的实现成为可能。
为了支持这一观点,\textit{前辅助运动区}中的一些细胞在一个序列中编码动作的等级顺序\cite{shima2000neuronal},而其他细胞在猴子等待启动第一个动作时编码第二个动作\cite{nakajima2009covert}。
\par


5.中外侧前部皮层(46区)和背内侧前部皮层(9区)的许多细胞编码顺序、持续时间、颜色、形状和结果等特征之间的连接,并且来自后顶叶皮层、眶和腹侧前部皮层、\textit{前辅助运动区}、周围皮层和其他区域的输入提供了大部分这些信息。
我们在第~\ref{chap:chap8}~章中认为,这种高水平的整合源于前额叶皮层位于三个处理层次的顶端,一个处理背景,另一个处理目标,另一个处理结果。


由于延迟反应任务在文献中的重要性,本章强调延迟反应任务。
早在75年前,Jacobsen就发现,有前脑皮层损伤的猴子和黑猩猩\cite{jacobsen1935functions}在这类任务中受到严重损害。
20世纪80年代,Goldman-Rakic\cite{ps1987circuitry}提出了前额叶皮层的工作记忆理论,这在很大程度上是基于延迟反应任务和相关任务的结果。
她认为中外侧前额叶皮层(46区)在回溯性空间工作记忆中起着必要和特殊的作用\cite{goldman1996prefrontal}。
第~\ref{chap:chap10}~章对这一理论进行了一般性的讨论,在这里我们集中讨论它的联系方面。


与当时流行的关于背侧流和腹侧流功能的观点一致\cite{ungerleider1982two}, Goldman-Rakic强调了后顶叶皮层向前皮层提供的空间输入\cite{wilson1993dissociation}。
我们也强调这些连接的重要性,但对于它们为前皮层提供了什么,我们的解释不同。
我们认为后顶叶皮层在视觉引导行为的选择和控制中起作用\cite{shadmehr2004computational,milner2006visual}。
这一观点改变了我们对后顶叶皮层向中外侧前部皮层提供什么的理解\cite{rushworth2000anatomical}。
例如,当人们记住字母或形状时,那里就会发生成像激活\cite{rushworth1998functional}。
人们可能会试图将它们纳入工作记忆理论,将中外侧前部皮层的作用扩展到非空间和空间工作记忆中。
% 要求被试者将刚刚出现过的刺激与前面第n个刺激相比较
但更重要的因素是,这些激活发生在n-back任务中,这需要受试者记住项目出现的顺序\cite{nystrom2000working}。
所以这些观察确实表明,中外侧前部皮层的功能超越了空间分析,但它们可能反映了编码顺序的需要,而不是与工作记忆本身有关。



\subsection{提议}

我们现在可以提出一个关于前部皮层背侧功能的建议,附带条件是它主要集中在前部皮层中外侧(46区)。
它以一种简短而又有些扩展的形式出现:

简而言之:

背侧前额叶皮层生成适合当前情境和期望结果的目标,其中最近发生的事件指定了该情境,并且它对这些目标进行前瞻性编码,直到可以尝试实现它们。


展开的:

背侧前部皮层根据当前环境生成目标,这些环境包括最近发生的事件,尤其是视觉事件的位置、持续时间、距离、数量、大小、速度和顺序。
它利用这些特征,如空间和时间顺序,来产生具体和抽象的目标,以及这些目标的序列。
必要时,它会前瞻性地对这些目标进行编码,直到可以尝试实现它们,并以这种方式击败来自不相关事件的干扰。



\subsection{为什么其他区域不能完成背侧前额叶皮层的功能}

我们认为后顶叶皮层为中外侧前额叶皮层提供了关键的输入。
但如果这就是问题的全部,那么后顶叶皮层的损伤应该和中外侧前部皮层的损伤有同样的效果。
很长一段时间以来,人们都知道它们不会。
正如第~\ref{chap:chap1}~章所提到的,后顶叶皮层的损伤不会导致延迟反应\cite{alexander1973effects}或延迟交替\cite{ettlinger1966tactile}任务的损害。
作为这些发现的一个结果,我们可以得出结论,后顶叶皮层的细胞具有延迟期活动\cite{kalaska1995deciding,snyder2000intention}在这些任务中没有发挥必要的作用。


考虑到细胞活动提供的指导很少,我们转向解剖学来理解为什么中外侧前部皮层在延迟反应任务和相关任务中发挥必要的作用,而后顶叶皮层则没有。
我们认为它们之间联系的差异可以解释这些发现。
例如,后顶叶皮层缺乏中外侧前部皮层从眶前部皮层接收到的特定结果信息。
第~\ref{chap:chap4}~章解释了关于特定结果的信息首先到达颗粒眶额皮层,以颜色,形状和纹理视觉为主导。
通过前皮层的内在联系,它可以很快到达中外侧前额叶皮层,在那里它与有关目标的信息结合在一起。
因此,中外侧前部皮层接收有关事件顺序的证据,以及与潜在目标及其当前价值相关的结果。
后顶叶皮层有但只有中外侧前额叶皮层拥有基于当前情境和预测结果产生目标所需的所有信息。



\subsection{对觅食选择的贡献}

第~\ref{chap:chap2}~章解释了在类人猿灵长类动物的进化过程中出现了背侧前部皮层(参见\cite{preuss2011human})。
如果是这样,那么我们需要了解它带来了什么好处。
关键可能在于基于视觉的觅食策略。
早些时候,我们回顾了25门搜索任务的证据,该任务表明,中外侧和后外侧前部皮层受损的猴子以一种无序的方式搜索看不见的花生\cite{passingham1985memory}。
正常的猴子会按照有序的顺序搜索它们是否能看到目标\cite{desrochers2010optimal,taffe2011rhesus}。
因此,背部前部皮层受损的猴子觅食效率似乎低于正常猴子。
25扇门搜索任务需要“双赢”策略,延迟交替任务也是如此。
一个优化的序列允许正常的猴子有效地实施这一策略。


当然,所有的动物都需要优化觅食,当第一个斑块耗尽时,从一个斑块移动到另一个斑块\cite{plank2008optimal}。
但这项25扇门的搜索任务并不需要通过移动从一个地方移动到另一个地方。
类人猿优化觅食的方式强调达到目标,特别是在视觉事件的基础上这样做。
高效的动作序列的优势可能与个体在群体觅食时面临的竞争有关。
本章涉及中外侧前部皮层在使用视觉事件选择目标和生成有效的目标序列中的作用。
这似乎使类人猿灵长类动物能够知道最近发生的视觉事件,包括它们的顺序、时间和位置,告诉它们下一步该做什么,之后该做什么。
当然,同样的能力使它们能够在新事件发生时改变觅食计划。


第~\ref{chap:chap2}~章还提出了类人猿灵长类动物在面对严重的资源波动问题时进化出背侧前部皮层的观点。
在野外,类人猿根据同一树种的个体之间的同步性、经过的时间和天气事件(如持续的高温)来预测食物的可用性。
这些预测依赖于物体表征(例如树或水果)与事件的顺序和时间的结合。
背侧前部皮层代表这些类型的连词,它在识别先前事件的顺序与当前目标的选择直接相关方面发挥着必要的作用,无论是通过顺序编码还是规则编码。
事件的持续时间和顺序,以及事件之间的间隔、数量、位置和相对距离,也会影响目标序列的规划。


本章提出,类人猿前部皮层的一个新进化部分,即背侧前部皮层,在很大程度上由其与后顶叶皮层的连接提供的当前环境中产生目标。
我们认为它的功能反映了一种对日间觅食的适应:通过使用顺序、时间和距离信息有效地规划当前和未来的目标。
另一部分颗粒状前部皮层也同时出现:腹侧前额叶皮层。
下一章探讨了它用来产生目标的各种环境,以及允许它这样做的联系。


\chapter{腹侧前额叶皮层:基于视听内容生成目标}
这本书提出了关于灵长类动物前额叶皮层基本功能的方案。

\section{介绍}
正如第 2 章所指出的,人们经常将灵长类动物描述为“视觉动物”。原因是中央凹在早期的单鼻科动物中进化,而三色视觉在类人猿中进化。 这些进步使这些动物及其后代能够辨别位置、颜色、形状、视觉纹理、光泽度和半透明度的微小差异。本章回顾了类人猿使用这些视觉特征来提供觅食机会线索的证据,我们称之为标志。正如第2章所解释的那样,我们所说的符号是指用作提示但不一定对应于整个对象的非空间景象和声音。
进化已经设计出许多方法来获得觅食优势。一些哺乳动物通过精心制作身体部位来开发资源,从而利用了它们的生态位。大象的长鼻子使它们能够以其他哺乳动物无法做到的方式觅食。长颈鹿的长脖子同样提供了独特的觅食机会。我们认为,类人猿反而精心设计了某些大脑结构,包括腹侧PF皮层。
前一章解释了背侧PF皮层生成适合当前上下文的目标,如最近事件指定的,尤其是视觉事件。它解释了视觉线索的顺序、位置和时间的重要性,以及其他特征,强调与后顶叶皮层的联系。腹侧PF皮层与下颞叶皮层和上颞叶皮层相连。因此,可见或可听的标志也指定了当前的上下文,类人猿可以单独使用或结合由顺序、位置和时间确定的上下文使用。
\section{目的}

\section{定义和术语}


\section{指纹}

\subsection{损伤和激活}

\subsection{损伤和活动}

\subsection{活动和激活}




\subsection{结论}



\chapter{作为一个整体的前额叶皮层:从当前环境和事件中产生目标} \label{chap:chap8}

% 晓来不戴催征鼓,红日方升已奋楫
\section{概述}

颗粒状前额叶皮层位于三个信息处理层次的顶部:
一个用于当前上下文,
另一个用于目标,
第三个用于行为结果。
作为上下文层次的顶点,前额叶皮层整合了空间和非空间视觉、触觉和内脏感觉、听觉、味觉和嗅觉的皮层输入,以及来自海马体的输入。
作为目标层次的顶点,前额叶皮层代表行动的目标,包括具体和抽象目标的序列、集合和类别,它可以通过与运动前皮层的联系来影响他们的活跃度。
作为结果层次的顶点,它代表了特定食物和液体的所有感官维度,并通过与杏仁核的联系,根据当前的生物需求更新了它们的动机评估。
通过整合这三个层次,类人猿前额叶皮层可以产生适合当前环境和当前需求的目标。
作为对类人猿进化过程中面临的觅食问题的一种特殊适应,它们可以学会在单一事件的基础上产生目标,从而减少危险或无效的觅食选择。
类人猿灵长类动物比它们的祖先更快地解决新的觅食问题,因为它们新进化的前额叶皮层实现了一个快速、通用的学习系统,这增强了动物历史早期进化的祖先强化学习机制。



\section{介绍}
\par

前面的五章把前额叶皮层拆开了,现在是时候把它重新组装起来了。
第~\ref{chap:chap3}~章和第~\ref{chap:chap4}~章分别讨论了内侧前额叶皮层和眶额皮层。
例如,我们认为,关于当前生物需求的信息通过与杏仁核的连接到达眶额皮层,并且海马体向内皮层提供关于导航和其他涉及动作的事件的信息。
第~\ref{chap:chap5}~章从搜索和注意力的角度解释了尾侧前额叶皮层的功能,通过与背侧和腹侧视觉流的连接来实现。
第~\ref{chap:chap6}~章提出,关于空间、时间和序列的信息从后顶叶皮层到达背侧前额叶皮层,并有助于基于这些内容的上下文进行选择。
第~\ref{chap:chap7}~章说,关于视觉和听觉信号的信息从颞叶皮层到达腹侧前额叶皮层,并在其他情境的基础上做出选择。
\par


这种讨论前额叶皮层的方式可能会造成这样一种印象,即它是作为五个独立的区域运作的。
但前额叶皮层是一个整体,它能这样做是因为它的内在联系允许它整合通过不同通路到达的信息。
因此,灵长类动物的前额叶皮层可以根据对结果和整体环境的总体预测来生成目标。
前额叶皮层的贡献超出了这种深远的整合,但以这种方式将信息聚集在一起的能力是其基本功能的基础之一。
\par


由于前额叶皮层作为一个整体发挥作用,我们需要一个全面的理论,而第~\ref{chap:chap1}~章提出了这样一个理论的两个要求:
展示前额叶皮层能做大脑其他部分不能做的事情,并解释为什么它的连接使它能够以这种方式发挥作用。
像以前一样,我们从联系开始。



\section{联系}
\par
第~\ref{chap:chap3}-\ref{chap:chap7}~章每个章节都有一个部分强调了前额叶皮层的一个部分的连接。
我们在这里总结总体模式。



\subsection{皮层和杏仁核}

\begin{enumerate}
\item 颗粒状前额叶皮层接收来自视觉、听觉、体感、嗅觉、味觉、嗅觉和内脏皮层的信息。
因此,灵长类动物的前额叶皮层有相对直接的来自距离感受器的输入,如视觉和听觉感受器,以及传递特定行为结果的输入,如食物和液体的味觉、嗅觉、视觉和感觉。
因此,灵长类动物的前额叶皮层对特定的行为结果有一个强大的、高维的表征,特别是包括它们的视觉属性。
它还具有复杂的视觉和听觉表征,可以指导觅食或社会选择。
视觉标志反映了灵长类动物的一些进化进步,如中央凹和三色视觉。
其他哺乳动物的前额叶皮层和灵长类动物新皮层的其他部分至少缺乏这些特性中的一些。

\item 前额叶皮层还接收来自海马体和海马复合体其他部分的直接和间接输入。
海马体、下托和内嗅皮层都与内侧前额叶皮层有直接联系。
海马体在导航和事件记忆中发挥作用,尤其是对于嵌入其空间和时间背景中的事件和对象。
与前额叶皮层一样,海马复合体接收来自所有感觉模式的输入,并与杏仁核紧密相连,传达行为的某些方面结果。
但是,尽管前额叶皮层通过运动前区域有直接输出,但海马体对这些区域的直接访问较少,缺乏前额叶皮层所具有的那种内在联系,也没有前额叶皮层所拥有的那种直接、特定的结果信息(详情见第~\ref{chap:chap4}~章)。


\item 杏仁核与前额叶皮层的许多部分有着紧密的联系(见图~\ref{fig:3_3})。后顶叶皮层等区域几乎没有这种联系。
一些运动前区与杏仁核有联系,但它们很稀疏\cite{avendan1983evidence}。
前额叶皮层和杏仁核的连接在根据动物的当前状态更新行为结果的动机评估中发挥作用。
因此,前额叶皮层能够以后顶叶和运动前皮层所没有的方式代表更新的评估值。

\item 前额叶皮层直接或间接投射到内侧和外侧前运动区域,从而可以为这些区域提供运动目标。
前运动皮层的嘴侧部分与前额叶皮层有着广泛的联系,但尾侧部分也从前额叶皮层获得信息,尽管不那么直接。
这些连接在很大程度上排除了腿部和脚部的运动表示。
与后肢表现相反,前肢的特化表现在背侧前运动皮层的喙部\cite{tachibana2004input}、腹侧前运动皮层\cite{he1993topographic}、前扣带回皮层\cite{luppino1991multiple}和头侧扣带运动区\cite{he1995topographic}。
正如第~\ref{chap:chap2}~章所解释的,这种前肢偏向反映了灵长类动物以后肢为主的运动形式,这使手可以自由发挥其他功能。
通过与运动前区域的连接,前额叶皮层在伸手、抓握和操纵方面发挥着优先作用,而不是运动能力。
这些联系促进了目标的实现,例如抓住物体或到达某个地方。


\item 前额叶皮层也有控制注意力和搜索功能的连接,包括对应于显性注意力的眼球运动(详情第~\ref{chap:chap5}~章)。
例如,尾侧前额叶皮层既有到脑干动眼神经核的直接投射,也有通过上丘和基底神经节的间接投射。
第~\ref{chap:chap2}~章指出,灵长类动物的大多数颗粒状前额叶皮层具有强烈的皮层顶盖投射\cite{leichnetz1981prefrontal}。
前额叶皮层还向顶叶和颞叶发送投射,介导对背侧和腹侧视觉流以及其他感觉模式的自上而下的注意力。


\item 灵长类动物前额叶皮层的各个部分之间有着广泛的联系。这些预测已在其他地方详细记录\cite{barbas1988anatomic,carmichael1995sensory,barbas1999medial,petrides1999dorsolateral,petrides2002association,price1999delineating}。
来自前额叶皮层外部的任何输入都可以在两个突触内到达前额叶皮层的任何部分\cite{averbeck2008statistical}。
前额叶皮层不仅接收大量输入,而且无论信息最初到达哪里,它都可以快速组合这些输入。
\end{enumerate}
\par


除了与皮层的其他部分和杏仁核的连接外,灵长类动物的前额叶皮层还与屏状核、基底神经节、丘脑、中脑中的多巴胺能神经元和小脑有连接。
接下来的五节依次介绍这些内容。



\subsection{屏状核}
\par

屏状核与前额叶皮层有相互联系\cite{tanne2002projections},也通过丘脑的\textit{背内侧核}投射到它\cite{erickson2004subcortical}。
它与皮层的其余部分也有类似的联系。
来自几个皮层区域的投影汇聚在一个特定的屏状核区,每个屏状核区都连接到额叶的几个部分,包括前额叶皮层\cite{tanne2002projections}。
这种连接模式表明屏状核可能达到了一定程度的整合。
然而,它似乎缺乏前额叶皮层特有的广泛的内在联系。



\subsection{基底神经节}
\par

与大多数大脑皮层一样,前额叶皮层向基底神经节发送一个沉重的投射,靶向其输入结构纹状体。
尽管大部分(如果不是全部的话)大脑皮层都有大量的输入,但基底神经节的输出似乎集中在额叶,尽管一些输出也流向了后顶叶\cite{clower2005basal}和颞叶皮层\cite{middleton1996temporal}。
这种组织表明,前额叶皮层和基底神经节之间的联系也可能有助于其在整合信息方面的作用。
\par


然而,它对这种整合的贡献方式仍然存在争议。
基底神经节缺乏能够整合其各个部分信息的远距离内在联系。主流观点强调皮层-基底神经节环的平行组装,重叠最小\cite{alexander1991basal,nakano2000neural}。
图~\ref{fig:8_1}~描述了其中一些环,包括一个前运动区(辅助运动区)和几个前额叶区。
Middleton\cite{middleton2000basal}得出结论,总的来说,涉及前额叶皮层的环在解剖学上与涉及运动前区域的环不同。
如果这是真的,基底神经节组织的这一特征表明大多数整合发生在皮层水平,尽管纹状体和黑质纹状体投射的某些方面可以提供一些整合能力。
他们这样做可能是因为一种叫做向上螺旋的组织特性\cite{haber2000striatonigrostriatal}。
黑质纹状体突起不仅回到为黑质的特定部分提供纹状体的环,而且还回到相邻的环。



\subsection{丘脑}
\par

前额叶皮层和所有其他皮层区域一样,与丘脑有相互联系。
它的主要丘脑连接是与\textit{背内侧核}。
例如,\textit{背内侧核}的多形部分接收来自上丘的输入\cite{russchen1987afferent,erickson2004subcortical},并投射到尾侧前额叶皮层,而后者又投射到上丘脑\cite{fries1984cortical}。
这些联系有助于引起明显和隐蔽的注意(第~\ref{chap:chap5}~章)。
\par


同样,内侧大细胞\textit{背内侧核}投射到眶额皮层\cite{ray1993organization},并接收来自杏仁核的输入\cite{russchen1987afferent}。
所以研究人员一点过也不惊讶于\textit{背内侧}大细胞分裂的病变具有与眶额皮层或杏仁核病变相似的影响\cite{mitchell2007neurotoxic}(见第~\ref{chap:chap4}~章)。
Izquierdo\cite{izquierdo2010functional}已经证明了在强化剂退化任务中,大细胞\textit{背内侧核}和眶额皮层的功能相互作用。


\begin{figure} 
	\centering
	\includegraphics[width=0.72\linewidth]{chap8/fig_8_1}
	\caption{(A)选定的皮层-基底神经节环\cite{alexander1986parallel}。
	皮层区和皮层下核的缩写:AC,前扣带皮层; Caud,尾状核; 耳鼻喉科,内嗅皮层; FEF,额叶视区; GPi,苍白球内段; 海马,海马; IT,下颞叶皮层; M1,初级运动皮层; MD,丘脑内侧核; 中背侧 PF,中外侧 PF 皮层; 中壳核,不包括喙壳核和尾壳核的壳核; PM,前运动皮层; PP,后顶叶皮层; S1,初级体感皮层;SNr,黑质网状核; ST,颞上皮层; VA,丘脑腹前核; VL,丘脑腹外侧核; VP,腹侧苍白球; VS,腹侧纹状体。 细分缩写:a,前; cdm,尾内侧; cl,尾外侧; dl,背外侧; 纬度,横向; ldm,后内侧; m,内侧; mc,大细胞; mdm,中背内侧; 个人计算机,细细胞; pl,平行板层; pm,后内侧; rd,吻背侧; rl,头外侧; rm,吻内侧; vl,腹外侧。
	(B)皮层-基底神经节环路的总体连接方案。 
	(C)皮层-小脑环路的总体连接方案\cite{shadmehr2004computational}。\label{fig:8_1}}
\end{figure}





\subsection{多巴胺能中脑}
\par

前额叶皮层的一些输入来自中脑的多巴胺能神经元。
正如第~\ref{chap:chap3}~章所提到的,这些细胞提供了奖励预测误差信号\cite{schultz1998predictive}。
关于这种信号的大多数理论工作都集中在它作为纹状体教学信号的作用上,但其中一些细胞也直接投射到前额叶皮层和其他区域\cite{gaspar1992topography},它们可以在皮层和纹状体水平上促进学习\cite{miller2007rules}。



\subsection{小脑}
\par

就像皮层和基底神经节之间存在的回路一样,皮层和小脑在解剖学回路中相互连接。
这些环涉及脑桥基底核和丘脑\cite{houk1995distributed}。
在前额叶区域中,尾侧前额叶皮层和背侧前额叶皮层向桥基底核发送最大的投射,只有少量的桥皮层投射来自腹侧或眶额皮层\cite{schmahmann1997anatomic,glickstein2008cerebellum}。
Middleton\cite{middleton1998cerebellar,middleton2001cerebellar}已经表明,皮层和小脑之间的环路包括中外侧前额叶皮层(区域46)和背内侧前额叶皮层(区域9)。



\subsection{摘要}
\par

前额叶皮层连接的简要草图从第~\ref{chap:chap3}-\ref{chap:chap7}~章中提取了一些关键点,这些关键点涉及每个区域的连接。
我们强调前额叶皮层中信息的汇聚和整合,就像我们之前的许多其他结构一样。
但第~\ref{chap:chap3}~章至第~\ref{chap:chap7}~章也具体说明了灵长类动物前额叶皮层如何做大脑其他部分无法做的事情。
这些其他结构中的一些缺乏足够的内在连接,另一些缺乏关于特定结果更新估值的输入,而其他的则无法直接进入运动前皮层。
\par


前额叶皮层接收各种各样的输入,其内在联系可以快速整合,并对动作产生直接影响。
第~\ref{chap:chap3}~章至第~\ref{chap:chap7}~章提供了证据,证明其细胞负责编码多种信息之间的连词。
下一节将这些特性放在信息处理层次结构的上下文中。



\section{层次结构}
\par

Fuster\cite{fuster2000executive}长期以来一直认为,我们应该从层次结构的角度来看待颗粒前额叶皮层。
在他看来,颗粒前额叶皮层形成了从感知到行动的最高联系:用他的术语来说,就是感知-行动循环。
然而,要理解颗粒前额叶皮层,我们需要的不仅仅是层次概念以及感觉输入和运动输出之间的联系。
我们需要知道当这些联系涉及颗粒前额叶皮层时,它们之间的区别。
Fuster指出,它们在层次等级上有所不同。
我们同意这种观点,但要将这一想法转化为前额叶皮层的成功理论则需要更加精确。
\par


为了达到这一精度,我们需要解释我们所说的层次制度是什么意思。
从某种意义上说,与较低的成分相比,较高的成分位于离感觉输入或运动输出更远的层次中。
从另一个意义上说,更高的级别更抽象地表示信息。
从第三种意义上来说,层次结构的更高层次表示具有更高维度的信息,即具有更高程度的特征连接。
\par


我们认识到,第二种意义和第三种意义在某些方面存在冲突。
正如这个词所暗示的那样,抽象涉及提取不同表征的共同特征,而更高维度则需要通过添加特征来区分表征。
尽管如此,我们认为抽象和高维表示在这里讨论的层次结构中都很高。
例如,“猫”的抽象概念包括豹猫、美洲虎和老虎,但并不具备任何特定猫种的所有特征,当然也不具备邻居的虎斑猫。
从这个意义上说,抽象在层次上是很高的,尽管它们需要更少的特性。
但同样真实的是,将许多特征合成为一个逐渐更具体的表示,反映了沿着等级制度的上升,正如通常所解释的那样。
基于皮层细胞及其末端的层流分布的前馈和反馈连接模式有时反映了这种层次结构\cite{rockland2000feedback}。



\subsection{上下文层次结构}
\par
前额叶皮层比其他区域离初级感觉区域更远,它接收来自所有初级感觉区域的汇总输入。
Jones首次认识到皮层组织的这一特征,他们的结论是基于连接解剖学的研究\cite{jones1970anatomical}。
例如,他们指出了腹侧视觉流,它处理有关颜色、形状和视觉纹理的信息。
它在颞叶的喙部达到顶峰,包括TE区和嗅周皮层区域。
由于TE区\cite{webster1994connections}和嗅周皮层\cite{petrides2002association}都向颗粒状前额叶皮层发送投影,所以前额叶皮层可以被视为位于视觉处理层次的顶点\cite{young1992objective}。
\par


类似的结论适用于听觉输入\cite{romanski2009primate}和触觉输入\cite{jones1970anatomical}。
味觉和内脏输入通常平行于体感通路,并遵循类似的组织原则。
嗅觉系统是一种特殊情况,但这些输入也从初级嗅觉区域梨状皮层通过无颗粒前额叶皮层传递到颗粒前额叶皮层(第~\ref{chap:chap4}~章)。
因此,颗粒前额叶皮层位于信息处理层次结构的顶部,该层次结构集成了所有感觉模式来表示当前的行为背景。
前额叶皮层还可以整合来自海马体和杏仁核的信息,以及上一节提到的其他结构,如基底神经节和小脑。
\par


上下文层次结构可以从其连接中识别出来,但其他方法也揭示了同样的东西。在猕猴中,基底树突树的大小和树突棘的数量从V1到V2、V4、TEO和TE等区域一直在增加,并且这种趋势持续到前额叶皮层\cite{elston2007specialization}。
这种层次结构在额极皮层达到顶峰。
树突发育的增加可能支持神经整合能力的增强。
\par


正如第~\ref{chap:chap4}~章所指出的,前额叶皮层中的上下文层次扩展了腹侧视觉流开始的特征连接\cite{murray2007visual}。
在视觉层次结构的越来越高的层次上,更多的特征进入这些连接。
在颗粒前额叶皮层中,来自无颗粒前额叶皮层的输入与来自颞叶和顶叶皮层的输入汇合,以建立更高阶的多模式连接,包括视觉、听觉、触觉、味觉、嗅觉和内脏特征(见图~\ref{fig:fig_4_3})。
\par


第~\ref{chap:chap2}~章指出,颞下区TE与颗粒状前额叶皮层一样,是在灵长类动物中进化而来的。
这个想法有一个重要的含义。
灵长类动物的进化既不是嗅周皮层的高级特征连接,也不是纹状体皮层(V1)和其他枕叶区域的低级元素表示。
V1和嗅周皮层出现在所有哺乳动物中,因此很可能在早期哺乳动物中进化。
灵长类动物的颞下皮层(TE和TEO区域)进化较晚,这些区域提供中等水平的视觉处理。
它的细胞既不代表整个物体,也不代表低级的基本特征。
颞下皮层和嗅周皮层都向除了尾侧前额叶皮层以外的颗粒前额叶皮层发送输入,但枕叶区域不发送输入(第~\ref{chap:chap5}~章)。
因此,大多数颗粒状前额叶皮层接收来自视觉层次的中间和更高层次的输入,而不是来自较低层次的输入。
\par


第~\ref{chap:chap2}~章和第~\ref{chap:chap7}~章指出了资源视觉标志的重要性。
这些线索也介于物体和简单的视觉特征之间。
此外,颞下皮层处理有关面部的信息\cite{baylis1987functional,tanaka1991coding},这些信息也提供“迹象”或“线索”。
声学呼叫也起到信号的作用,颞上皮层的细胞编码特定的呼叫,比如打招呼\cite{rauschecker1995processing}。
通过与视觉对象的类比,颞叶皮层的声学表征有时被解释为“对象”,但通过与中级视觉信号的类比,将其视为信号可能更有用。
因此,灵长类动物的大脑已经进化为将听觉和视觉线索都视为信号,而且似乎是通过在颞叶中创建一个介于物体和基本特征之间的中间水平的特征连接来做到这一点的。
\par


第~\ref{chap:chap7}~章解释了颗粒状前额叶皮层利用手势作为目标生成的背景,前一段提到了颞下皮层的面部响应细胞。
考虑到颞下皮层与腹侧前额叶皮层和眶额皮层之间的联系,在这些额叶区域出现类似的响应并不奇怪。
ÖScalaidhe等人\cite{o1999face}研究了腹侧前额叶皮层(12/47区)、背侧前额叶皮层(9区和46区)和尾侧前额叶皮层(8区)的细胞活性。
他们发现,在接受颞下皮层输入的区域,即腹侧前额叶皮层,会有选择性地对人脸做出响应,但在其他区域没有。
面部细胞与对物体和彩色图案有响应的细胞混合在一起。
重要的是,这些特性都不依赖于实验室任务的事先训练。
\par


类似的响应发生在眶额皮层。
Rolls等人\cite{rolls2006face}描述了那里对人脸有选择性响应的细胞。
其中一些细胞对人脸识别有响应,另一些细胞对面部表情有响应,还有一些细胞对移动的头部有响应,但对静止的头部没有响应。
其中一些细胞对不同的视角表现出选择性,但另一些细胞的响应是不变的。
\par


所有这些细胞都可以在社交互动中发挥重要作用,它们都有助于上下文层次。
面部表情和面部特征为社会选择提供了重要标志,就像其他视觉标志为觅食选择提供的一样。
声学信号在社交和觅食选择中也起着至关重要的作用。
与觅食选择一样,社会选择涉及基于当前环境和当前生物需求产生目标。
当然,动机往往各不相同。
\par


一些发现指出了整合视觉和听觉信号的重要性,颗粒状前额叶皮层处于执行这一功能的良好位置。
颞上沟下岸的细胞对面部做出响应,并且在面部和叫声的同时出现的情况下表现出增强的活跃度\cite{barraclough2005integration}。
这些区域与颗粒状前额叶皮层有联系,因此腹侧前额叶皮层中的细胞表现出类似的作用也就不足为奇了\cite{sugihara2006integration}。
\par


视觉和听觉整合的另一个例子涉及条件学习,如配对联想学习任务(第~ \ref{chap:chap7}~章)。
Fuster等人\cite{fuster2000executive}训练猴子在一个音调下选择红色刺激,在另一个音调上选择绿色刺激。
他们从尾侧前额叶皮层(8B区)进行了记录,发现了对其中一种音调以及与该特定音调相关的颜色有选择性响应的细胞。
这个属性对应于第~\ref{chap:chap7}~章提到的视觉-视觉关联的配对编码。
模态内和跨模态的结果都反映了延迟区间的积分,这是我们稍后讨论的主题。
\par


额叶整合的最后一个例子涉及不同类型的视觉信息,如形状和位置,分别是腹侧和背侧视觉流的区域。
前额叶皮层连接到两个流(第~\ref{chap:chap5} - \ref{chap:chap7} ~章),但Wilson等人\cite{wilson1993dissociation}认为,它们的信号可能在颗粒状前额叶皮层中保持分离。
然而,Rao等人\cite{rao1997integration}在中外侧前额叶皮层和腹侧前额叶皮层中发现了编码记忆形状和记忆位置的单细胞。
因此,前额叶皮层的内在连接将首先到达其不同部分的信号汇集在一起,就像来自背侧和腹侧流的输入一样(第~\ref{chap:chap5}~章和第~\ref{chap:chap7}~章)。
\par

这些上下文整合的例子涉及感官信息,但要完全指定当前的上下文,类人灵长类动物还必须考虑特定时间和地点的目标、行动和结果。
前额叶皮层与海马体的连接可能提供了上下文层次结构的一些方面(第~\ref{chap:chap3}~章)。
除了感官环境外,导航历史和事件信息都在觅食选择中发挥着至关重要的作用。
除了位置和对象标识外,顺序和时间在事件处理中也起着重要作用。
例如,第~\ref{chap:chap6}~章解释了中外侧前额叶皮层的细胞编码顺序-画面连接\cite{warden2007representation}。
同样,中外侧前额叶皮层和尾侧前额叶皮层中的细胞编码顺序-持续时间\cite{tsujimoto2009monkey}和顺序-距离连接(Genovecio等人,2011年)。
其他前额叶皮层细胞本身也编码刺激顺序\cite{ninokura2003representation,ninokura2004integration}。
据推测,这些细胞为编码连词的细胞提供了输入。
\par


总之,颗粒状前额叶皮层整合了来自所有感觉模式的感觉输入。
它在模态内部和模态之间都是这样做的,它对当前刺激和最近发生的刺激都是这样,对整个物体和位于基本特征和整个物体之间的特征连词都是如此,我们称之为符号。
它将这些多模式输入与有关事件的信息集成在一起,包括事件的顺序、地点和时间,以作为上下文层次结构的最高级别。其他皮层区具有某些财产,但没有其他皮层区具备所有这些属性,以及直接进入运动前区。



\subsection{目标层级}
\par
正如前额叶皮层离初级感觉区相对较远一样,它也比初级体感皮层、运动前皮层(6区)和顶叶后皮层的嘴侧部分(5区)等区域离初级运动皮层更远。
这些额叶和顶叶区域都向初级运动皮层发出直接投射,但颗粒状前额叶皮层没有\cite{jones1970anatomical,lu1994interconnections}。
它们中的每一个也直接投射到脊髓,而颗粒状前额叶皮层缺乏这种投射\cite{murray1991contributions}。
直接进入初级运动皮层和脊髓的程度定义了一个层次。
因此,我们可以将颗粒状前额叶皮层描述为位于运动层次的顶部。
或者,由于运动系统产生动作,我们可以称之为Fuster\cite{fuster1988prefrontal}所说的动作层次。
相反,我们选择将前额叶皮层组织的这一方面称为目标层次。
我们这样做是因为我们认为颗粒状前额叶皮层是产生目标的,而不是像前几章所解释的那样认为是行动或运动。
然而,如果读者将目标层次结构视为目标-行动层次结构,我们也不会反对。
与后顶叶皮层一起,颗粒状前额叶皮层投射到运动前皮层,这一途径为前额叶皮层提供了对动作控制的最直接影响。
后顶叶皮层最大量地投射到背侧前运动皮层的尾部和\textit{辅助运动区},而颗粒状前额叶皮层最大量的投射到前运动皮层\cite{rizzolatti2001cortical,luppino2003prefrontal}。
运动前皮层的喙前部分反过来又与其尾部相连。
例如,前\textit{辅助运动区}投射到\textit{辅助运动区},\textit{辅助运动区}又将轴突直接发送到初级运动皮层\cite{luppino1993corticocortical}和脊髓\cite{murray1981organization}。
\textit{辅助运动区}前区和喙前运动区投射到前运动皮层的尾部,但不投射到脊髓\cite{he1995topographic}。
因此,额叶的目标层次在其较低水平上可以直接进入初级运动皮层和脊髓,而这种层次在较高水平上继续进入颗粒状前额叶皮层。
为了解释这种层次结构,我们从初级运动皮层开始。
图~\ref{fig:1_10}~显示,初级运动皮层(4区)的细胞在记忆引导的运动和可见线索引导的运动中具有相同的活动。
背侧前运动皮层和\textit{辅助运动区}的细胞通常分别表现出外部和“内部”引导的特异性。
Shima和Tanji(2000)在前\textit{辅助运动区}和\textit{辅助运动区}中进行了记录,并在这两个区域发现了编码特定作用序列的细胞。
然而,编码特定转变的细胞在\textit{辅助运动区}中比在前\textit{辅助运动区}中更常见。
相反,发出运动是序列中的第一个、第二个还是第三个信号的细胞在前\textit{辅助运动区}中比在\textit{辅助运动区}中更常见。
Shima和Tanji利用这些发现构建了一个产生运动序列的层次模型。
Cisek和Kalaska(2002)同样记录了背侧前运动皮层的头侧和尾侧部分。
当出现两个潜在靶点时,嘴侧部分的一些细胞编码两个靶点,但尾侧部分具有这种特性的细胞较少。
当猴子知道要瞄准哪个目标时,这两个部位的活动就指定了那个位置。
因此,从神经生理学和神经解剖学的角度来看,嘴侧前运动皮层和前\textit{辅助运动区}在运动层次上似乎比尾侧前运动皮层和\textit{辅助运动区}更高。
反过来,后一个区域似乎比初级运动皮层更高。
因此,这些区域形成了一个运动层次。
我们提出,颗粒状前额叶皮层将这种运动层次细化为目标层次。
例如,前\textit{辅助运动区}和\textit{辅助运动区}中的细胞编码给定的运动序列(Shima\&Tanji,2000),但颗粒状前额叶皮层中的细胞对序列的抽象结构进行编码(Shima等人,2007)。
正如第~\ref{chap:chap6}~章所解释的(见图~\ref{fig:6_12}),在序列之前具有活性的细胞中,超过一半的细胞编码了抽象结构,例如交替:拉、转、拉、转和推、拉、推、拉。
这种抽象并不是指特定的动作,而是指更高阶的表示。
有证据表明,运动前皮层将这些抽象概念转化为具体的运动计划(Hoshi,2008)。
除了抽象序列外,颗粒状前额叶皮层中的细胞还可以编码抽象的行为规则和策略,如匹配和不匹配规则(见图~\ref{fig:7_9})或“重复停留”和“改变-转移”策略(见图~\ref{fig:7_11})。
正如第~\ref{chap:chap7}~章所解释的,这些规则适用于任何刺激,它们产生了可供选择或避免的目标集或类别。
一组可能目标的抽象表示不同于特定目标的具体表示,例如单个对象或地点。
通过这种方式,颗粒状前额叶皮层中编码的表征有助于目标层次。
当与前额叶皮层中编码独立于实现目标的方法的细胞的发现一起考虑时,这一概念变得尤为重要(第~\ref{chap:chap6}~章和第~\ref{chap:chap7}~章)。
基于对这一点的讨论,读者可以理解目标和上下文层次结构有许多共同的财产。
对于这两种情况,颗粒状前额叶皮层位于解剖层次的顶端,该解剖层次包括其较低水平的其他大脑区域。
目标层次结构提供了几种方法,可以将当前上下文与具体动作的规范(根据电机命令)联系起来。
它通过指定用作动作目标的对象或位置,通过指定动作的一系列的抽象结构,或者通过指定生成要选择或避免的对象或地点的规则或策略来做到这一点。
在一系列层次上产生目标的能力,独立于实现目标的手段,代表了颗粒状前额叶皮层所赋予的一个重要优势。



结果层次结构

目标导致行动,行动导致结果。
与上下文一样,结果是由各种感觉模式产生的输入指定的。
然而,动物并没有进化出专门的奖励或结果受体。
然而,颗粒状前额叶皮层的连接使其能够以一种独特的方式分析结果(第~\ref{chap:chap4}~章)。
除了海马体之外,没有任何其他皮层区域具有如此多样化的输入阵列,也没有任何其他皮层区域能够有效地整合它们,同时相对直接地进入运动前皮层。
颗粒状前额叶皮层的结果层次结构既包括随着层次的上升而更大的特征收敛(见图~\ref{fig:fig_4_3}),也包括更大的抽象。
第~\ref{chap:chap4}~章回顾了特征收敛的证据。
眶额皮层中的细胞对视觉-味觉、视觉-嗅觉和味觉-嗅觉的结合做出响应(Rolls等人,1994年)。
这些组合对觅食选择具有明显的重要性:它们对构成结果的特定食物或液体进行编码。
除了编码食物和液体的物理财产外,前额叶皮层中的细胞还根据当前需求对其生物价值进行编码,正如贬值效应所揭示的那样(Izquierdo等人2004年)。
第 \ref{chap:chap4} 章还解释说,除了根据特定的食物和液体对结果进行编码外,作为结合许多特征的高维表征,眶额皮层的颗粒部分还根据抽象的“共同货币”对结果进行了编码,该货币代表一维的价值。
这两个层次在颗粒状前额叶皮层的功能中都起着重要作用。
当猴子在众多选择中选择追求哪个目标时,它们可以用抽象的估值来比较自己的选择;
当他们需要了解哪种选择导致了特定的结果时,他们可以将结果编码为许多刺激维度。
因此,结果层次结构的两个方面——增强的特异性和抽象性——都有助于颗粒状前额叶皮层的功能。
第~\ref{chap:chap4}~章提供的证据表明,颗粒状前额叶皮层的不同细分服从于结果层次的这两个方面:
横向眶额皮层编码特定的食物和液体,用于相互对比各种结果;
中间眶额皮层以“通用货币”对价值进行编码以进行比较。
当与杏仁核在更新结果估值中的作用相结合时(第~\ref{chap:chap4}~章),灵长类动物的前额叶皮层作为一个整体,提供了一种强大的机制,可以将特定结果的复杂表征与觅食选择相结合,包括那些基于“共同货币”抽象估值的表征。



总结

颗粒状前额叶皮层位于三个信息处理层次的顶端,其中包括处于较低水平的其他大脑结构。
上下文层次结构建立在感觉区的功能;目标层次阐述了运动区域的功能,以及内侧前额叶皮层的无颗粒部分;
并且结果层次建立在眶额皮层的无核部分上。
请注意,我们还没有讨论颗粒状前额叶皮层内的层次结构,下一章将对此进行讨论。
通过整合这三个层次,颗粒状前额叶皮层可以将上下文映射到目标和预测结果。
第~\ref{chap:chap3}~章展示了前额叶皮层是如何对目标层次做出贡献的;
第~\ref{chap:chap4}~章讨论了前额叶皮层对结果层次的贡献。
第~\ref{chap:chap5}~章通过整合对象和地点的目标和结果层次,指出前额叶皮层在寻找和关注目标方面的作用。
关于上下文层次,第~\ref{chap:chap6}~章讨论了从位置、顺序和时间上下文生成目标;
第~\ref{chap:chap7}~章涉及从视觉和听觉信号中产生目标。
所有这些章节都解释了这些功能的解剖学基础,并列出了颗粒状前额叶皮层在其中发挥必要作用的许多任务。
它们还提供了活动和激活通常一致的证据。
因此,接下来的三节探讨了这些层次结构之间的交互,这些交互基于上下文和事件生成目标,并且在上下文变化时运行,因此必须在不同的时间生成不同的目标。



\section{基于上下文生成目标}

我们现在转向从上下文生成目标,这需要整合所有三个层次结构。
条件视觉运动任务为这种集成提供了一个有指导意义的例子。
上下文映射到目标和结果,猴子必须使用这些连词来做出选择。
他们需要这三个层次,因为从长远来看,所有的背景和目标都具有相同的价值。
也就是说,猴子不能根据上下文-结果连词或目标-结果连语来选择目标;
他们需要使用上下文-目标-结果连词。
第~\ref{chap:chap7}~章解释说,颗粒状前额叶皮层受损的猴子在学习条件运动任务中视觉上下文和目标之间的映射方面有严重障碍(Bussey等人,2001年)。
正如本章稍后详细解释的那样,即使在数十次尝试学习新的条件性视运动关联后,患有眶侧和腹侧前额叶皮层病变的猴子也不会有超过偶然水平的改善。
普通猴子只需一次经验就能提高几率。
神经元活动支持颗粒状前额叶皮层在条件视运动学习中的作用。
Asaad等人(1998)使用彩色图片来建立扫视眼球运动的背景。在他们记录的与任务相关的单元格中,许多单元格编码了上下文和目标的特定连接。
回想一下,我们所说的结果不仅仅是指回报。
大脑的许多部分具有奖励、奖励预测和奖励预测误差信号,这得益于中脑多巴胺能神经元的投射以及其他来源。正如前一节所说,作为结果层次的顶点,颗粒状前额叶皮层可用的结果信息包括比多巴胺能神经元传递的更具体的信息。
颗粒状前额叶皮层中的结合物代表特定的视觉、味觉和味觉,在实验室中作为奖励的食物和液体的气味、质地和其他特征。
在野外,这些特征定义了动物觅食的食物和液体。
颗粒状前额叶皮层学习许多这种连词。
表6.1列出了其中一些用于背侧前额叶皮层,颗粒状前额叶皮层作为一个整体也编码其他皮层。
这些连词包括刺激特征和行动(Kim\&Shadlen,1999)、刺激特征和策略(Genovesio等人,2005)、刺激特点和空间目标(Genovecio等人,05)、规则和响应选择(Wallis\&Miller,2003a)、行动和结果(Barraclough等人,2004;Tsujimoto和Sawaguchi,2004a;Hayden\&Platt,2010),行动和未来奖励数量(Wallis\&Miller 2003b)以及行动和奖励价值(Kennerley\&Wallis 2009;Kennerley等人2009)。
有的评论可能会争辩说,其他领域也编码了这样的连词。例如,顶叶后区LIP编码是颜色线索还是空间线索指定扫视的方向(Toth\&Assad 2002)。
MIP区域的神经元活动编码由空间目标指定的扫视方向(Snyder等人,1997)。
但是,与前额叶皮层不同的是,后顶叶皮层不接收有关结果物理财产的直接信息。
它也没有从杏仁核获得实质性的输入(Amaral\&Price 1984),这意味着,与前额叶皮层(第~\ref{chap:chap4}~章)不同,后顶叶皮层不能根据当前的生物需求来评估具体的结果。
因此,正如我们早些时候仅基于连接得出的结论,尽管后顶叶皮层在灵长类动物大脑中很重要,但它不能做颗粒状前额叶皮层能做的事情(Stoet\&Snyder,2009)。
评论也可能会指出,运动前皮层是一个有连接词的地方,就像前额叶皮层中的连接词一样。
与后顶叶皮层不同,运动前皮层确实与杏仁核有联系,尽管联系较弱,但并不包括所有的运动前皮层(Avendaño等人,1983年)。
我们早就知道,背侧前运动皮层的损伤会损害条件视运动任务的学习(Petrides 1987)和保留(Passingham 1985b),这需要整合上下文、目标和结果层次。
然而,运动前皮层缺乏腹侧和眶额皮层所具有的视觉和听觉输入,也缺乏许多前额叶皮层通过与眶额皮层的连接而获得的结果的结膜表示。


总结

灵长类动物的前额叶皮层可以整合有关上下文、目标和结果的信息,并且达到了其他区域无法实现的整合水平。许多任务需要这三个层次,也需要前额叶皮层。
例如,延迟响应任务需要顺序-地点-结果连接,这取决于背侧前额叶皮层(第~\ref{chap:chap6}~章)。
条件视运动任务需要符号-动作-结果的连接,它取决于腹侧前额叶皮层(第~\ref{chap:chap7}~章)。
第~\ref{chap:chap2}~章提出了这两个区域,即背侧前额叶皮层和腹侧前额叶皮层,在类人猿灵长类动物中进化而来的观点。
我们提出,这些区域提高了这些动物利用感官环境产生目标的能力。
视觉方面的进步提高了在远处检测资源迹象的能力,颗粒状前额叶皮层使这些类人猿能够在时间和空间上安排这些感官事件。
然后,所有这些信息都可以与目标和结果层次结构的产物相结合,以指导觅食选项之间的选择。
视觉上的进步也改善了对当前觅食地点的利用,包括伸手和抓握。
检测不同浆果之间的细微视觉差异,例如在光泽方面,会对觅食效率产生重大影响,生成优化目标序列的能力也会产生重大影响。
在这一点上,我们强调了融合和整合的重要性,这可能会给人留下这样的印象,即灵长类动物的前额叶皮层的功能与其他哺乳动物的前额叶皮层非常相似,但只是在某些方面更好。
前额叶皮层的许多理论认为,它有助于基于整合、连词、联想、映射或其他术语中的相同概念来产生目标。
因此,人们很容易想象,灵长类动物的前额叶皮层只是利用灵长类动物深刻的视觉进步,如中央凹和三色视觉,来做前额叶皮层在所有哺乳动物中所做的事情。
但我们的建议不仅仅是对感官环境进行更复杂的表达。
下一节建议类人猿通过新的前额叶区域获得质的进步。



\section{基于事件生成目标}

\subsection{基于事件的学习与强化学习}

第~\ref{chap:chap7}~章解释说,有解决一系列条件视觉运动问题经验的猴子可以在几次试验中学习新的上下文到目标的映射。
腹侧前额叶和眶额皮层合并病变的猴子只有经过数百次试验才能做到这一点(Bussey等人,2001年)。
但他们最终可以学习新的映射。这些发现有两个含义。
首先,颗粒状前额叶皮层的损伤阻碍了上下文、目标和结果之间连接词的快速学习。
其次,大脑的其他部分也有助于学习相同的信息,尽管速度较慢,错误也更多。
本节介绍了颗粒状前额叶皮层的整合功能赋予了基于单个事件生成目标的能力。
我们所说的事件是指在特定时间和地点发生的背景、目标、行动和结果。
这些独特的连词既可以存在于短期记忆中,也可以存在于长期记忆中。
我们建议的关键是,使用单一事件是快速学习的基础,快速学习是减少错误的一种方法。
类人猿灵长类动物也可以学习和应用抽象的规则和策略来减少错误,正如我们稍后解释的那样,这也需要使用单个事件。
例如,第~\ref{chap:chap7}~章解释说,在条件视觉运动任务中,猴子可以通过使用先前事件来通过“改变-转移”策略消除一个可能的选择,从而降低错误率。
因此,我们区分了通过祖先强化学习系统的学习和通过依赖颗粒状前额叶皮层的新机制的学习。
祖先强化学习机制使用反馈事件,通常以基于奖励的生成目标的形式,来加强或削弱刺激、响应和结果之间的关联。
基于反馈事件的加权平均值,它的进化速度相对较慢,而且是累积的,而且它在动物历史的早期就进化了。
在新皮层进化的支持下,早期哺乳动物取得了许多进步。
但这些发展并不等于一个新的学习系统。
正如接下来的章节所解释的,我们提出灵长类动物增强了祖先的强化学习系统。
因此,他们可以根据单个事件选择目标,并可以在一次或几次试验中解决广泛的问题。
由于较新的机制依赖于颗粒状前额叶皮层,并且这些区域是在灵长类动物中进化而来的,非灵长类哺乳动物必须完全依赖于较旧的系统,就像灵长类动物的祖先所做的那样,也就像患有前额叶皮层损伤的灵长类动物所必须的那样。
我们认为,由于灵长类动物的新学习机制,它们避免了旧学习系统中固有的许多错误。


\subsection{辨别学习}

我们首先通过考虑歧视学习来说明我们的观点。
在这项任务中,受试者在每次试验中都要在刺激之间做出选择。
版本很多,但一个典型的例子涉及在两个对象之间进行选择。
如果他们选择正确,受试者将获得奖励。所有哺乳动物和其他动物都可以解决这类问题。
但随着不同物种解决一系列歧视问题,它们之间的差异也会出现(图8.2)。
在每一个新问题的第一次试验中,没有一个受试者的得分能超过机会水平,因此这些试验的选择没有纳入图8.2所示的分析中。
然而,在经历了一次选择刺激后,动物可以在第二次试验中做出更好的选择。
恒河猴非常有效地做到了这一点,当它们做到这一点时,据说它们已经形成了一个强大的辨别学习集,通常被缩短为学习集。
图8.2B显示,恒河猴的表现在各种问题中迅速改善,在解决了几百个问题后,它们在第二次试验中的表现接近90$\%$(Harlow和Warren,1952年)。
然而,即使它们接受了大致相同数量的视觉辨别学习问题的训练,并对每个问题进行了相同数量的试验,与恒河猴相比,大鼠和猫的学习能力相对较弱。
图8.2A显示了一个分支图,其中试验次数达到60$\%$的正确性能绘制为右侧的圆圈。
这种方法避免了可以对物种进行排名的建议。
恒河猴被认为是具有代表性的卡他性猴子,在第二次试验中,它很快就达到了60$\%$的正确选择,因为圆圈很小。
其他哺乳动物在第二次试验中表现不佳,正如按比例较大的圆圈所表明的那样。
多年来,这一结论一直受到质疑,第10章也提出了这些挑战。
我们在那里认为,据说与数字相矛盾的结果要么涉及有缺陷的实验设计,要么没有关注三者的性能。
据我们所知,文献中没有关于猕猴在两个视觉对象之间进行选择时的第二级试验表现的报告,就像猕猴在学习场景测试中所做的那样。
开发一套有区别的学习方法并不是一件简单的事情。
在所涉及的因素中,有一个涉及到对任务结构的学习:某一个刺激具有高价值,而另一个没有价值,受试者必须在两者之间做出选择。
动物在学习这种结构时会更有效地处理新问题。
短语“学会学习”指的是这个因素。
另一个因素涉及使用抽象规则或策略。
在歧视任务的情况下,一种有用的策略被称为“赢-留,输-移”。
通过使用这种策略,受试者的第二选择取决于他们的第一选择及其结果:如果得到奖励(“获胜”),猴子会“坚持”这个选择;如果得不到回报(“损失”),他们就会“转向”另一种选择。
从这个角度来看,学习集的发展有两个方面:学习两个新对象总是有不同的价值(任务的结构)和学习“赢-留-输”策略。
图8.2显示,多种哺乳动物可以做其中一种、另一种或两者兼而有之。
那么,为什么猕猴的进步比其他哺乳动物快得多呢?
Murray和Gaffan(2006)的一项观察提供了关键线索。如果猴子同时面临歧视问题,那么在第二次试验中,它们的正确表现将不再接近90$\%$。
在并行学习实验中,猴子在每次测试中都会看到许多对刺激,有时称为问题。
但他们在每次训练中只看到一对,每个问题都要进行一次试验。
在任务的串行版本中,同样的问题一次又一次地出现。
当刺激对同时出现时,猴子会恢复到在没有学习集的情况下发生的较慢的辨别学习速度。
然而,任务的结构和同样的策略也适用。
因此,为了开发一个强大的学习集,猴子必须在连续的试验中反复看到相同的问题,而不需要涉及其他问题的试验的干预。
Murray和Gaffan(2006)建议,在这一系列版本的任务中,猴子学会根据前一次试验中发生的事件来选择下一个目标,然后在试验间隔期间将这个目标保存在短期记忆中。
换句话说,他们前瞻性地编写下一个目标的代码(见第~\ref{chap:chap6}~章和第~\ref{chap:chap7}~章)。
在传统的学习集任务中,前瞻性编码意味着记住第二次试验中应该选择(或避免)的对象。
猴子需要注意第一次试验中发生的事情,包括背景、选择的目标和发生的结果,并在此基础上选择下一个目标,大概是基于“赢-留,输-移”策略。
我们称这种选择事件为基础,因为它取决于一个单一的事件:第一次试验中发生了什么。
随着猴子在串行格式中获得一些辨别问题的经验,它们会学习策略和任务结构。
从某种意义上说,他们的策略有一部分涉及“胜利-停留,失败-转移”的应用,还有一部分涉及在未来记忆中保持所选目标。
这种策略产生了一个要实现或避免的目标,这个目标需要保存在记忆中,直到猴子做出下一个选择。
然后,当猕猴面临新的问题时,根据第一次试验的事件,它们在第二次试验中的表现可以达到近90$\%$的正确率。
试验二性能的快速提高(图~\ref{fig:8_2})在很大程度上取决于猴子学习任务结构和适当策略的速度和效果。
对于我们的论点至关重要的是,学习集的发展取决于颗粒状前额叶皮层。
Browning等人(2007)教猴子一系列歧视问题,直到它们能够迅速解决这些问题。
图~\ref{fig:8_3}~显示了它们的猴子在学习集任务中的表现(灰色线,未填充的圆圈)。
研究人员没有将猴子训练到其他猴子所拥有的高水平学习,因此第二次试验的表现保持在约70$\%$的正确率,而不是猴子最终能够达到的几乎没有错误的表现。
然后,Browning等人移除了一个半球的颞下皮层和另一个半球上的额叶皮层,这在功能上断开了颗粒状前额叶皮层与相关视觉输入的连接。
结果,猴子失去了他们的学习集所赋予的优势(图~\ref{fig:8_3},黑线,实心圆圈)。
他们恢复了最初学习解决视觉辨别问题的缓慢速度(未图示),并且当刺激对同时出现时,它们解决问题的速度相同(图~\ref{fig:8_3},三角形)。
Browning等人的研究结果表明,如果没有对颗粒状前额叶皮层的正常视觉输入,猴子就无法实现正常猴子所能达到的那种强大的学习能力。
同样,Harlow等人(1970)对患有双侧额叶病变的猴子进行了600个辨别问题的训练。
尽管这些猴子确实在问题上有所改善,但在第二次试验中,它们的正确率仅为77$\%$,而正常猴子的正确率接近90$\%$。
要么哈洛的猴子无法在一次试验的基础上做出正确的选择,要么它们无法在记忆中保持目标。
无论是什么解释,如果没有前额叶皮层与颞下皮层的相互作用,猴子就会恢复到缺乏颗粒状前额叶皮层的哺乳动物的缓慢学习速度(图~\ref{fig:8_2})。

\begin{figure} 
	\centering
	\includegraphics[width=0.7\linewidth]{chap8/fig_8_2}
	\caption{开发针对精选哺乳动物物种的视觉辨别问题的学习集。
	(A) 选定哺乳动物在线性时间尺度上的分支图。
	右侧圆圈的直径显示了在试验二中达到 60\% 正确率所需的问题数量,根据 (B) 中提供的数据进行插值。
	(B) 在新的视觉辨别问题的试验二中正确选择的百分比,作为先前解决的此类问题的数量的函数。
	该图基于仅针对六次试验提出每个问题的研究。
	水平虚线显示 (A) 中所示直径图的参考。
	(B) 改编自 Passingham\cite{kraemer1984human}。\label{fig:8_2}}
\end{figure}


\begin{figure} 
	\centering
	\includegraphics[width=0.7\linewidth]{chap8/fig_8_3}
	\caption{切断下颞皮层 (IT) 与额叶 (FL) 对学习集表现的影响。
		连续和同时呈现的辨别问题中,错误率随试验次数的变化。
		前者(圆圈)表示猴子在连续试验中面临相同的选择。
		后者(三角形)表示在任何给定选择重复之前,会经历多种选择。
		灰线表示正常(对照组)猴子的表现。
		黑线表示交叉断开损伤的猴子的表现。
		只有损伤效应对连续辨别(学习集)具有统计学意义\cite{browning2007frontal}。\label{fig:8_3}}
\end{figure}



\subsection{逆向学习}


就像猴子在一系列视觉辨别问题上有所改善一样,它们在一系列的视觉辨别逆转问题上也有所改善。
正如第~\ref{chap:chap4}~章所解释的,在歧视逆转学习中,被奖励的选择从一个选择刺激突然改变为一对中的另一个成员。
在多次这样的逆转之后,猴子可以在犯下两三个错误后成功地改变他们的选择(Wilson和Gaffan,2008年)。
因此,猴子获得了一个类似于学习集的反转集(图~\ref{fig:fig_4_7})。
与歧视学习集一样,如果猴子必须同时解决问题,它们就不会表现出逆转集(Wilson和Gaffan,2008年)。
例如,在两个刺激之间:A和B。让我们假设奖励偶然性刚刚从刺激A变为刺激B。
在同时进行的测试条件下,猴子在试验间隔期间无法在短期记忆中保持刺激B的表现,即适当的目标,因为下一次试验将涉及不同的一对刺激。
刺激措施A和B将不会在几次试验中再次出现。
因此,就像同时进行辨别学习一样,猴子会恢复到较慢的学习速度,这不取决于基于事件的目标选择。
正如辨别学习集的情况所证明的那样,反转学习集取决于前额叶皮层和颞下皮层之间的相互作用。
Wilson和Gaffan(2008)对猴子进行了许多逆转问题的训练,然后通过前面描述的手术将前额叶皮层与颞下皮层断开。
随着猴子获得更多的逆转经验,它们仍然有所改善,但它们的速度很慢,而且从未达到手术前的快速学习水平。
同样,Izquierdo等人(2005年)表明,患有眶额皮层损伤的猴子在学习反转集方面有障碍(图~\ref{fig:fig_4_7})。
正如Wilson和Gaffan(2008年)的研究一样,猴子恢复了较慢的反转学习,这是他们第一次完成任务时的典型经历。
总之,我们认为来自辨别和反向学习的证据表明,颗粒状前额叶皮层以减少错误的方式增强了祖先的强化学习系统。
我们建议它通过使用单个事件来生成目标来实现这一点,在这些实验中,目标由一个要选择或避免的对象组成。


\subsection{延迟响应和延迟交替}


由于实验者使用诱饵食物井作为线索(见图~\ref{fig:6_3}),经典版本的延迟响应任务需要猴子学习“双赢-停留”策略,而延迟交替任务需要“双赢-转移”策略。
这些任务通常只有两个可能的目标选择,如左和右,这会在一系列试验中对记忆造成干扰。
为了成功地完成任何一项任务,受试者必须根据记忆中相互干扰的一系列事件中的最新事件来进行当前选择。
第~\ref{chap:chap6}~章表明,猴子可以在一定程度上通过前瞻性编码来对抗这种干扰。
在延迟响应任务的经典版本中,这意味着一旦受试者很好地看到左侧或右侧食物的诱饵,他们就可以在试验延迟期内保持对该位置的代表性作为目标。
因此,他们使用单个事件来生成目标,就像他们在学习集实验中所做的那样。
在延迟交替任务中,这意味着猴子一旦完成了之前的试验,就可以在试验之间的延迟期内建立并保持其目标的代表性。
同样,他们也可以通过使用单个事件生成当前目标来实现这一点。
正如第~\ref{chap:chap6}~章所解释的,关于中外侧前额叶皮层损伤引起的损伤,最引人注目的事实是其严重性和持久性。
在1000次试验中,这些动物未能提高偶然性表现,此时测试通常会停止(Butters\&Pandya,1969)目前尚不清楚他们失败是因为他们无法在损伤后重新学习规则,还是因为他们无法通过使用前瞻性编码机制来对抗干扰。
但无论哪种情况,他们都无法使用单个事件来生成适当的目标,从而减少错误。具有背侧前额叶损伤的猴子可以以正常速率(Passingham 1975)或以比正常速率稍慢的速率(Goldman et al.1971)。
在这项任务中,受试者学会在一系列连续的试验中选择正确的空间目标。
因此,累积强化学习系统可以通过慢慢加强一个地方与有益结果的关联来指导行为,而不需要使用任何单一事件。


\subsection{在位对象场景任务}


第3章介绍了物体到位场景任务(Gaffan 1992)。
正如上面所解释的,猴子看到一系列的背景场景,每一个场景都包含两个彩色的形状,它们出现在前景的固定位置。
在一般的辨别任务中,为了获得奖励,猴子必须选择正确的颜色形状。
第3章解释说,与正常猴子不同,额极皮层病变的猴子在试验二中对给定场景和选择的机会水平上进行(Piekema et al. 2009年)。
试验结束后,猴子以相当正常的速度恢复学习(见图~\ref{fig:3_10})。
同样的猴子发展了一个正常的辨别学习集。
第~\ref{chap:chap7}~章还介绍了一些结果,从对象在现场的场景任务。
具有任一腹侧的猴(巴克斯特et al. 2007)或眶额皮层病变(Wilson et al. 2007)在这项任务中表现出轻微的损伤。
相比之下,如图~\ref{fig:8_4}~所示,双侧前额叶皮层皮层病变导致非常严重的功能缺损(布朗宁et al. 2005年)。
额叶与颞叶下皮层交叉分离也会导致较大的损伤,但不如双侧前额叶皮层损伤后那么大。
在双侧前额叶皮层损伤或这些交叉损伤之后,学习速率减慢到正常猴子在没有背景场景的情况下可以达到的水平(图~\ref{fig:8_4})。
图~\ref{fig:8_4}B还比较了这些额叶损伤的结果与额极皮层损伤(第~\ref{chap:chap3}~章)和穹窿横断(Gaffan 1992)后试验2中损伤效应的大小。
这些结果表明,颗粒状前额叶皮层使用背景场景提供的上下文,这导致更快的学习,更少的错误比猴子可以管理没有背景场景或没有他们的颗粒状前额叶皮层。
额极皮层似乎是特别重要的一次试验学习。
在现场的场景中的对象任务类似于并发的歧视任务之前描述的,
许多试验之间的再现一个给定的场景和它的选择刺激干预。
因此,我们可以得出结论,场景背景和颗粒状前额叶皮层所带来的好处并不依赖于短期记忆或前瞻性编码。
相反,猴子必须依赖于对过去事件的长期记忆,因为并行设计阻止它们在短期记忆中保持所有20个上下文选择结果连接,直到同样的选择再次出现。
因此,与区分学习集一样,关键优势来自于使用单个事件来生成下一个目标。
在物体就位场景任务中,背景场景促进了这种认知操作。
不同之处在于,对于就地对象场景任务,这些事件在长期记忆中存储,而在辨别学习集中,它们存储在短期记忆中。


\begin{figure} 
	\centering
	\includegraphics[width=0.7\linewidth]{chap8/fig_8_4}
	\caption{各种病变对物体就位场景任务表现的影响。
		(A)双侧\textit{前额叶皮层}病变(圆圈)和颞下皮质与额叶 (FL × IT) 交叉断开(三角形)的影响。
		每条曲线比较术前(术前,未填充符号)和术后(术后,填充符号)的表现\cite{browning2005role}。
		(B)几种病变组的损伤指数。该指数将病变造成的损伤绘制为最大可能表现缺陷的比例。
		这种标准化方法消除了术前表现变量的影响。虚线显示选择用于比较的选择性病变的影响。
		误差线:SEM\cite{wilson2010functional}。\label{fig:8_4}}
\end{figure}



\subsection{条件性视觉运动学习}

我们将在本章前面和第~\ref{chap:chap7}~章讨论条件性视觉运动任务。
Murray和Wise(1996)表明,猴子可以在几次试验中学习条件性视觉运动映射,其他人已经证实并将该结果推广到几种刺激和运动(Brasted et al. 2005 ; Cromer等人2010年)的报告。
在一次试验后,猴子会有显著的学习(图~\ref{fig:8_5}),此后猴子通常不会犯错。
图~\ref{fig:8_5}~显示了两组正常猴子的显著一次性学习。
两组都解决了三个选择的条件性视觉运动问题,一组通过移动操纵杆,另一组通过点击或保持与触摸屏接触不同的时间间隔(点击-保持)。
关键的试验是第三个。
在第一次试验中,猴子没有特定的刺激经验可以借鉴,因此表现在机会水平上。
可能使用的“重复停留”策略使试验二复杂化。
在试验三中,我们可以消除这种复杂性,并比较两种类型刺激的表现:一个从未见过的,一个在第一次试验中经历过的。校正过程确保了猴子将正确地选择一次,只有一次的第一次审判。
用操纵杆测试的猴子发展了“换档”策略,但那些执行轻拍保持版本任务的猴子没有。
我们不知道这种差异的原因,但它会影响图~\ref{fig:8_5}~所示的分数。
当猴子碰巧在试验三中看到一个新的提示时,使用“改变-转移”策略的猴子的正确率约为50$\%$(左),而没有该策略的猴子的正确率约为33$\%$(右)。
先把这个小差异放在一边,关键是当第一次实验的提示出现在第三次实验中时,两组猴子的表现都明显更好(图~\ref{fig:8_5}~中的白色条)。
也就是说,他们从一个先前的事件中学到了很多东西-背景,目标选择和结果的结合-在不同的试验类型中进行干预。
像辨别学习集和反转学习集一样,它需要一系列问题的经验来开发条件性视觉运动问题的学习集。
除了更快的学习,抽象策略的应用也减少了错误。
第~\ref{chap:chap7}~章阐述了策略。
它还解释了颗粒状前额叶皮层中的细胞编码快速学习和策略(Genovesio et al. 2005年)。
一些细胞编码“重复-停留”策略,而其他细胞编码“改变-移位”策略。
此外,前额叶细胞在错误试验中显示出较少的策略编码活性(如果有的话)(Genovesio et al. 2008),这支持了颗粒状前额叶皮层基于这些策略产生目标的想法。
与辨别和反转任务不同,条件性视觉运动任务允许区分由于学习而导致的错误减少和由于应用策略而导致的错误减少,也称为转移。
这种对比是可能的,因为任务的三选择版本有两种试验:重复试验和改变试验。
在改变试验中,“改变-转移”策略将三选择题简化为二选择题,但并不能完全解决问题(图~\ref{fig:8_5},左)。
通过这种方式,抽象学习允许从以前的问题转移到当前的问题。
因此,猴子可以使用该策略以50$\%$的正确率执行,但仍然必须学习视觉运动映射以进一步减少错误。
如果没有这样的策略,他们的正确率只有33$\%$,如前所述。像物体到位场景任务一样,条件性视觉运动任务需要长期记忆的映射,以及实现转移的策略。
像辨别和逆转集一样,快速和抽象学习都需要根据前一次试验中发生的事件生成目标。
有 4 点值得强调:
1.快速学习取决于事件。
Brasted等人(2005)发现,如果他们的猴子在第一次出现提示时做出了正确的选择,如果在下一次试验中出现同样的提示,它们的正确率超过80$\%$。
如果干预试验有不同的提示,猴子的表现就不那么好了。
这一发现表明,猴子在前一次试验的基础上产生了一个目标,并在试验间隔期间保持它。
如果猴子犯了先前的错误,它们的表现也会较差,我们的解释是,错误选择的记忆会干扰记忆。
2.快速学习依赖于颗粒状前额叶皮层。
布塞等人(2001)研究了改变试验,以消除“重复停留”策略的影响。
他们发现颗粒状前额叶皮层损伤不仅阻止了一次试验的学习,而且在大约50次或更多的试验过程中完全阻止了学习(图~\ref{fig:8_6})。
3.策略依赖于单个事件。
“重复-停留”和“改变-转移”策略的应用需要猴子记住之前试验中的两个事件:
上一个目标和上一个提示。
4.这些策略依赖于颗粒状前额叶皮层。
颗粒状前额叶皮层病变的猴子未能应用“重复-停留”、“改变-转移”或“失去-转移”策略,即使它们在病变前几乎完美地学习和使用了这些策略(图~\ref{fig:8_6}和图~\ref{fig:8_7})。
比较病变的海马或其途径之一,穹窿,提供了一些了解他们的合作功能。
回想一下,海马体与前额叶皮层有着广泛的联系(第 \ref{chap:chap3} 章)。
图~\ref{fig:8_7}B~显示,穹窿或海马损伤导致猴子在解决条件性视觉运动问题之前丧失一次试验学习能力,错误率增加2 - 3倍,所以损伤是轻微的。
并且相同的病变对策略没有影响(图B)(Wise\&Murray 1999 ; Brasted等人,2003)。
总之,这些结果表明,在没有海马体的情况下,颗粒状前额叶皮层仍然能够在约50-100次试验中有助于条件性视觉运动学习,并通过各种策略减少错误(图~\ref{fig:8_6}B~和图~\ref{fig:8_7}B)。
相比之下,如果没有前额叶皮层,海马既不能在约50次试验中减少错误,也不能应用相关策略(图~\ref{fig:8_6}A,B)。
因此,颗粒状前额叶皮层受损后仍然存在的能力类似于缺乏颗粒状前额叶皮层的哺乳动物的缓慢刺激响应学习(第~\ref{chap:chap2}~章)。
无论是受损的猴子还是其他物种,都需要很多天的训练才能学习刺激与响应、动作、目标、运动或目标之间的任意映射(取决于一个人的偏好)。
然而,对海马系统或颗粒状前额叶皮层结构的损伤引起损伤。
因此,它们必须以某种方式协同工作,使猴子能够使用单个事件来应用抽象策略,并在一次或几次试验中学习任意的刺激-目标映射。
因此,猴子犯的错误更少。
事实上,如果运气好的话,他们有时可以完全避免错误。


\begin{figure} 
	\centering
	\includegraphics[width=0.7\linewidth]{chap8/fig_8_5}
	\caption{条件性视觉运动任务中一次试验学习显著。显示了两组数据:
		一组是三项选择任务,其中猴子必须向不同方向移动操纵杆;
		另一组是三项选择任务,其中猴子必须点击触摸屏或将手放在屏幕上一段时间。
		每组三条柱状图显示集合 [A、B、C] 中新刺激的数据。
		在第一次试验中,猴子的表现符合预期,处于随机水平。
		然后,纠正程序迫使它们在“试验”结束前选择正确的响应。
		在第二次试验中,另一种刺激介入,在本例中为刺激 B,表现有时会受到策略效应的影响。
		第三次试验提供了一次试验学习的测试,与策略效应无关。
		黑柱状图显示刺激的第一次呈现,白柱状图显示第二次呈现。与刺激 C 第一次出现相比,猴子在刺激 A 第二次出现时表现出明显更好的表现(星号)。
		对于自发采用换班策略的猴子(左图),正确率有 50\%;对于没有采用换班策略的猴子(右图),正确率有 33\% \cite{murray2002arbitrary,brasted2005conditional}。 \label{fig:8_5}}
\end{figure}



\begin{figure} 
	\centering
	\includegraphics[width=0.7\linewidth]{chap8/fig_8_6}
	\caption{颗粒状前额叶皮层损伤对条件性视运动学习及“重复-停留”和“改变-转移”策略应用的影响。
		(A)表现曲线。
		实线:术前表现;
		虚线:术后表现。
		术前和术后数据均分为重复试验和改变试验,后者发生的频率是重复试验的 2 倍。
		术后表现与随机水平无统计学差异。
		(B)策略实施。
		策略指数给出了实施“重复停留”和“改变-转移”策略所实现的误差减少百分比,该百分比占最大可能减少百分比的比例。
		这种标准化方法可以补偿不同的随机水平。
		左图:在损伤前(术前),一组后来接受了腹侧和眶侧前额叶皮层损伤的猴子实施了这些策略,但在损伤后(术后),这些猴子不再这样做。
		差异非常显著(星号)。
		右图:为了进行比较,来自海马损伤的可比数据显示对策略实施没有影响\cite{bussey2001role}。 \label{fig:8_6}}
\end{figure}


\begin{figure} 
	\centering
	\includegraphics[width=0.7\linewidth]{chap8/fig_8_7}
	\caption{猴子的条件性视运动学习。
		(A)正常猴子在三项选择问题上的快速条件性视运动学习,如在变化试验中所示。
		猴子学会了根据每次试验中出现的三个新颜色形状线索中的哪一个,将操纵杆向三个方向之一移动。
		在重复试验(黑色圆圈)中,几乎不需要学习。
		在变化试验(灰色圆圈)中,猴子学习这三种映射的时间常数为7.8次试验,对于每个情境-目标关联,大约需要2到3次试验\cite{murray1996role}。
		(B)在三项选择任务上的条件性视运动学习:
		海马(hipp)消融术和穹窿横切术的影响。
		对于海马,学习曲线比较了术前(术前,灰色圆圈)和术后(术后,白色圆圈)的表现。
		对于穹窿横切术,曲线比较了病变组(白色三角形)和正常(对照)猴子(灰色三角形)\cite{brasted2005conditional}。\label{fig:8_7}}
\end{figure}


\subsection{学分分配}

使用事件记忆选择当前目标的另一个例子涉及学分分配的概念,如第~\ref{chap:chap4}~章所述。
Walton等人(2010)研究了猴子学习物体选择和结果之间联系的能力。
正常的猴子可以根据记忆做出适当的选择,而记忆中的选择似乎会导致特定的结果。
但是,颗粒状眶额皮层损伤的猴子在试验中学习相对较慢,并且是累积性的。
他们依赖于几个过去事件的时间加权平均值,就像非灵长类哺乳动物一样,而正常的猴子可以根据单个事件学习选择和特定结果之间的映射。



\subsection{时间扩展事件}

布朗宁和\cite{browning2008prefrontal}研究了他们所谓的时间扩展事件。
他们增加了选择和结果出现的时间之间的分离。
在一个条件下,延迟是未被填满的,但在另一个条件下,它被一个提示所填满,提供了一个信号,表明结果即将到来(图8.8 A)。
正常猴子在后一种情况下的学习速度要快得多,因为在这种情况下,暂时延长的事件会被干预性的线索联系在一起。
但是,前额叶皮层和下颞叶皮层分离的猴子并没有从这种方式中获益(图8.8 B)。
这个结果类似于到目前为止讨论的其他任务。
功能完整的颗粒状前额叶皮层在减少错误方面提供了优势,并且干扰其功能消除了该优势。
前额叶皮层可以产生目标的基础上的时间延长的事件,因为它的位置在三个处理层次。
这些事件的表示包括特定时间和地点的背景、目标和结果的结合。
目标的产生需要随着时间的推移而整合,或者用Fuster的术语来说就是“跨时间的偶然事件”(Fuster 2008)。
目标生成还需要跨空间的整合,因为结果来自于与行动的背景或目标不同的地方。
更多的证据支持前额叶皮层以这种方式整合空间和时间上的事件的观点。
Rushworth等人(2005)教导具有腹侧和眶额皮层病变的猴子条件性视觉运动任务,并系统地改变线索位置和空间目标之间的分离。
病变的猴子学习更慢,更大的分离,但正常的猴子没有表现出这种效果。


\begin{figure} 
	\centering
	\includegraphics[width=0.7\linewidth]{chap8/fig_8_8}
	\caption{研究时间延长事件的任务中的事件。
		(A)上图表示猴子看到的屏幕。
		在猴子通过触摸其中一张图片做出选择后,如果猴子选择正确,则在获得奖励之前会有一个延迟期。
		在填充延迟条件下,延迟期间会出现刺激,该刺激与所选刺激特别相关。
		对于未填充延迟条件,延迟期间不会出现任何图片。
		(B)与正常(对照)猴子(白色条)相比,将前额叶皮层与下颞皮层 (PF × IT) 断开的效果(灰色条)。
		条形图显示组平均值,实心圆圈表示单个猴子的表现\cite{wilson2010functional}。 \label{fig:8_8}}
\end{figure}



\subsection{事件记忆与情景记忆}

在本章中,我们有意使用事件这个术语,而不是情景记忆。
首先,在我们提出的账户中,选择目标并不总是需要记忆。
有时候,猴子使用事件来立即选择未来的目标,就像在辨别学习任务中一样。
第二,情景记忆这个术语通常指的是比事件更具体的东西。
最初,情景记忆适用于与时间和空间背景相关的事件的外显回忆(Tulving 1983)。
我们承认,没有任何实验表明,猴子回忆事件的意义上重新体验的事件和他们的参与,因为人类做的(Suddendorf \& Busby 2003)。
然而,我们不需要提出这种主张。
可以说,一个事件是由特定时间和地点发生的背景、目标、行动和结果的结合构成的。
回想一下,在我们的术语中,术语目标指的是一个对象或一个地点,术语结果指的是成本和收益方面的反馈。
因此,一个事件,正如我们所使用的术语,考虑到选择的背景基础,选择作为行动目标的对象或地点,选择和做出的行动,以及该行动的结果。
前额叶皮层似乎通过与海马系统的连接(第~\ref{chap:chap3}~章)获得事件记忆,通过与感觉皮层的连接获得当前事件。
当前和记忆中的事件一起构成了我们所说的当前背景。



\subsection{总结}

我们建议,颗粒状前额叶皮层允许类人猿灵长类动物使用一个单一的事件,以改善选择。
通过依赖于这种能力的快速和抽象的学习,类人猿做出的风险或浪费的选择比其他情况下要少。
当病变或实验操作消除了这些优势时,类人猿灵长类的表现与其他哺乳动物更接近--可能也与它们的祖先相似。



\section{基于注意力控制生成目标}

在前面的两节中,我们已经将基于事件和当前上下文的目标生成与基于强化学习的行为进行了对比。
强化学习系统积累了许多过去经验的平均值,以调整指导行为的关联,例如行动-结果关联。
相比之下,灵长类动物选择目标的方式涉及使用单个事件选择物体和地点作为目标的能力,这种新的学习机制在快速变化的环境中提供了优势。
除了在资源异常波动的情况下,更古老、更慢的学习机制工作得很好,这解释了为什么几乎所有的动物王国都可以在没有颗粒状前额叶皮层的情况下茁壮成长。
但是,在早期灵长类历史的特定时间和地点,以及类人猿历史的特定时期(第~\ref{chap:chap2}~章),颗粒状前额叶皮层新区域的发育改善了它们的选择质量,从而提供了选择优势。
在本节中,我们将把灵长类动物选择目标的方式与注意力的概念联系起来,并将注意力控制行为与自动控制行为进行对比。
在我们的术语中,注意力提供了一种目标生成机制,与自动的目标的选择。
自动行为使动物能够将注意力集中在其他事情上,并且在行为的速度和可靠性方面提供了优势。
我们提出,旧的强化学习系统产生自动行为,而依赖于颗粒状前额叶皮层的新系统产生注意行为:
基于单一事件的行为和第~\ref{chap:chap6}~章和第~\ref{chap:chap7}~章描述的高级当前环境。
自动行为和注意行为之间的区别类似于自动行为和“控制”行为之间的区别,因为这些术语用于过程解离程序。
然而,我们要强调的是,当我们把注意力控制这个短语用在猴子身上时,我们并不意味着暗示任何关于意识的事情。
在实验室中,双重任务范式提供了自动性测试。
该程序测试次要任务的执行干扰主要任务的执行的程度。
例如,Baddeley\cite{shallice1996domain}表明,当人类受试者产生一系列随机数字时,如果同时他们必须执行记忆任务,则该系列变得更加刻板。
当然,不同的任务对注意处理的要求不同,但原则是一样的。
因此,一个主要任务的竞争任务的脆弱性提供了一个操作定义的专心控制的行为。
不知何故,主要任务“失去”控制行为的“比赛”,或者至少在某种程度上处于不利地位。
累加器网络提供了一种思考这种竞争的方法。
在第~\ref{chap:chap3}~章中,我们认为习惯可以战胜结果导向行为,因为与后者相比,前者依赖于更少和更强的关联。
因此,习惯网络可以更快地达到阈值,并“赢得”控制行为的竞争。
相反,注意控制的行为可能比自动行为依赖更多和更弱的关联。
因此,注意力行为网络达到阈值的速度比那些替代品和“失去”的竞争。
通过抑制依赖于较强联想的优势行为或通过增强注意力,较弱的联想可以占上风。
因此,注意力控制的行为与本书讨论的其他注意力形式类似。
第~\ref{chap:chap3}~章表明,前额叶皮层的部分产生的系统发育的老系统之间的偏见,竞争控制行为。
第~\ref{chap:chap5}~章回顾了一些证据,这些证据表明,前额叶皮层的某些部分会在皮层其他部分的相互竞争的感觉表征中产生偏见。
同样,我们认为前额叶皮层的部分区域会产生对目标生成网络的偏好,这些网络依赖于许多弱关联。
影像学研究支持注意控制和自动控制之间的区别。
例如,Toni等人\cite{toni1998time}教导了人类受试者运动序列,并绘制了受试者改善其表现时前额叶皮层激活的程度。
大约45分钟后,激活水平降低到接近基线水平。
Floyer-Lea\cite{floyer2004changing}在类似的任务中使用了双重任务范式,以表明该序列随着练习而自动化。
他们还发现,在这种情况下,前额叶皮层的激活明显减少。
这些发现表明,随着任务变得更加自动化,前额叶皮层变得不那么活跃。
我们并不是说,当一项任务变成自动化的时候,前额叶皮层就没有活动了。
例如,Rainer\cite{rainer2002timecourse}在猴子执行延迟匹配样本任务时,在前额叶皮层中记录。
在延迟的早期,他们发现对熟悉的物体几乎没有响应,但在延迟结束时,这种响应又出现了。
因此,在刚刚引用的成像研究中缺乏显著的激活可能反映了\textit{血氧水平依赖}信号的不敏感性。
尽管如此,成像结果显示了前额叶皮层神经处理的一些重要内容。
随着任务自动化,有些东西会减少。
而且,随着任务变得自动化,同时在几个区域的激活增加。
在运动任务中,这些结构包括壳核和小脑\cite{floyer2004changing,toni1998time}的报告。
在文献中,将自动行为等同于习惯并假设基底神经节有助于习惯已经变得普遍\cite{fernandez2001visual,broadbent2007rats}的报告。
直接证据表明,这种想法在两个方面都是错误的。
第一,自动行为比习惯更重要,这一点在第~\ref{chap:chap10}~章会讲到。
第二,只有一部分基底神经节维持习惯。
请注意,前面引用的成像结果是指壳核,而不是整个纹状体或基底神经节。
只有部分壳核被激活。
对大鼠的研究表明,纹状体中的不同区域在自动行为的不同方面起作用\cite{yin2008reward}。
纹状体的某些部分调节习惯,但其他部分调节结果导向行为,如第~\ref{chap:chap3}~章所定义的。
有些人声称,所有灵长类动物的行为,包括人类的行为,可以归类为习惯或结果导向的行为,他们将这两种行为分配给这两个纹状体区域\cite{balleine2010human}。
但比较证据表明,灵长类动物的纹状体中已经进化出了额外的区域,比如那些从颗粒状前额叶皮层接收输入的区域(第~\ref{chap:chap2}~章)。
下一节提出,基底神经节的这些新部分与颗粒状前额叶皮层一起发挥作用,以单个事件为基础产生目标。



前额叶-基底神经节环

神经生理学数据支持这一建议。
例如,Pasupathy\cite{pasupathy2005different}记录了当猴子解决新的条件性视觉运动问题时颗粒状前额叶皮层和尾状核头部的活动。
回想一下,尾状核的头部是颗粒状前额叶皮层的主要纹状体区域,而纹状体的更腹侧部分则从无颗粒前额叶皮层接收主要输入。
在Pasupathy等人的研究中,颗粒状前额叶皮层和尾状核头部的细胞在试验中随着学习的进展而逐渐提前编码目标。
活动的变化发生在纹状体比在皮层更快,虽然在整个试验的行为改善更接近观察到的前额叶皮层的变化较慢。
这些研究结果表明,颗粒状前额叶皮层和纹状体的领土成为从事在专注学习的视觉运动协会。
在条件性视觉运动学习期间,前运动皮层(区域6)投射到的纹状体区域中的细胞具有与运动前皮层中的细胞平行的与学习相关的活动变化\cite{brasted2004comparison}。
然而,在这些更尾部的地区,学习相关的活动的变化如下性能的改善,而不是之前,发生在颗粒状前额叶皮层的纹状体领土。
此外,活动的变化继续作为科目练习的运动,它的进展自动化。
这一证据表明,颗粒状前额叶皮层和纹状体领土调解注意的行为,而更多的尾侧皮层基底神经节回路调解自动行为。
最近的证据支持这一结论。
Antzoulatos\cite{antzoulatos2011differences}记录了颗粒状前额叶皮层及其纹状体区域中的神经元活动。
他们的猴子学会了将视觉类别映射到左或右目标的扫视眼球运动。
Antzoulatos和米勒首先证实了刚才提到的条件性视觉运动学习的发现。
当猴子能够学会将样本刺激映射到目标时,纹状体活动比皮层活动更早地编码目标。
然而,当猴子学会对每个类别中的许多样本进行分类时,编码类别到目标映射的皮层活动比纹状体活动更早发展。
因此,总的来说,颗粒状前额叶皮层和纹状体的领土功能,以支持注意,而不是自动的行为。
Miyachi等人\cite{miyachi1997differential}提供了进一步的证据来区分用于注意行为的吻侧纹状体系统和用于自动行为的更尾部纹状体系统。
他们教猴子一系列的动作,并对它们进行过度训练,直到这些动作成为自动动作。
然后,他们灭活了头侧纹状体或更多的尾侧部分。
喙失活包括头的尾状核和最喙的部分壳核,从颗粒状前额叶皮层接收输入。
更多的尾部失活包括壳核的部分,从初级运动皮层和运动前皮层接收输入。
Miyachi等人研究表明,失活的吻侧纹状体损害新的学习,而失活更多的尾部受损的自动性能。
此外,随着执行变得自动化,头侧纹状体中的活动减少,而中纹状体中的活动增加\cite{miyachi2002differential,brasted2004comparison}的研究结果一致。
因此,作为一个任务成为自动的转移发生从颗粒状前额叶皮层及其纹状体领土的自动控制由运动前皮层及其纹状体领土的注意控制。
为了测试运动前皮层和纹状体之间的连接是否在运动任务成为自动时必不可少,Nixon等人\cite{nixon2004cortico}教授猴子条件视觉运动任务,并对它们进行了3个月的过度训练。
这种行为就自然而然地发生了。
然后,作者通过在两个半球各做一个损伤,将纹状体与运动前皮层分离(图~\ref{fig:8_9}B)。
在一侧,他们损伤了苍白球,它通过丘脑将纹状体输出传递到运动前皮层。
另一边他们损伤了前运动皮层。这种结合损伤消除了对任务的所有记忆。
动物重新学习的时间和它们第一次学习的时间一样长(图~\ref{fig:8_9}A)。
因此,相对自动的行为控制依赖于运动前皮层和纹状体区域之间的相互作用。


\begin{figure} 
	\centering
	\includegraphics[width=0.7\linewidth]{chap8/fig_8_9}
	\caption{断开运动前皮层与基底神经节的连接对条件性视运动任务表现的影响。
		(A)白色条表示初始学习的错误次数,以及在测试中断后进行术前“重新学习”测试的错误次数。灰色条表示损伤后重新学习任务所需的试验次数(术后测试)。
		(B)损伤和相关连接的图示。
		灰色:每个半球的损伤结构和受影响的连接。
		黑色:每个半球的完整结构和连接。转载自 Nixon PD、McDonald KR、Gough PM、Alexander IH、Passingham RE。
		皮层-基底神经节通路对于回忆已建立的视运动关联至关重要\cite{nixon2004cortico}。 \label{fig:8_9}}
\end{figure}



前额叶-cerebellar循环

就像涉及纹状体和颗粒状前部皮层的回路一样,也有涉及小脑和颗粒状前部皮层的回路,特别是中外侧前部皮层\cite{kelly2003cerebellar}。
其他回路涉及小脑、前运动皮层和初级运动皮层\cite{strick2009cerebellum}。
我们已经提到,当运动序列任务变为自动时,小脑的激活会增加\cite{floyer2004changing}。
其中一些激活发生在齿状核,它通过丘脑投射回皮层。
为了观察小脑损伤是否会导致对前额叶皮层损伤敏感的任务受损,Nixon\cite{nixon1999cerebellum}在齿状核和间质核中进行了损伤。
猴子可以重新学习延迟的交替任务,而颗粒状前部皮层的损伤严重损害了交替任务。
在随后的一项研究中,有相同损伤的猴子可以学习新的动作序列。
然而,在响应时间的测量上,它们从未达到与正常动物相同的自动性水平\cite{nixon2000cerebellum}。
Lu等人\cite{lu1998role}也教猴子一些不同的动作序列,他们在其中的一个子集上对动物进行了过度训练。
在齿状核失活期间,猴子可以正常学习新的序列,但对于过度训练的自动序列,它们的眼手协调能力受损。
综上所述,这些发现支持小脑对自动行为计时的贡献。
为了支持这一观点,Ramnani\cite{ramnani2001changes}报告说,当受试者学习运动序列的时间直到它们成为自动的时候,小脑皮层的激活会增加。
这些变化发生在连接前部皮层的小脑小叶。
Nixon\cite{nixon2001predicting}发现,当目标发生在可预测的时间,而不是不可预测的时间时,齿状核和间质核受损的猴子在响应时间上没有表现出改善。
因此,小脑对自动行为控制的具体贡献可能涉及行动的时机,而不是行动的选择。



接合注意力控制

在资源稳定的环境中,依赖于先前事件的平均值,快速而自动地行动是值得的。
当这种行为不能产生一致的结果时,转向注意力控制是有意义的。
向注意控制模式的转换可能由奖励预测错误信号或第~\ref{chap:chap3}~章提到的其他有符号和无符号错误信号触发,这些信号可能来自几个来源中的任何一个,包括中脑多巴胺能细胞或杏仁核细胞。
Rowe等人\cite{rowe2002attention}利用人体成像技术研究了从自动控制到注意控制的转换。
他们的实验对象用四个手指做简单的动作。
以固定的顺序移动四个手指几乎不需要注意,在这种情况下,颗粒状前额叶皮层没有发生明显的激活\cite{rowe2002attention}。
但当受试者被指示注意他们的行为时,颗粒状前额叶皮层和\textit{前辅助运动区}发生了显著的激活。
Lau等人\cite{lau2004attention}报告说,当他们通过要求受试者报告他们第一次意识到自己要移动的意图的时间来操纵受试者的注意力时,同样的两个区域被激活。
总而言之,当人们注意自己的行为或意图时,激活发生在颗粒状前部皮层和前脑区,从而参与注意力控制。
在第~\ref{chap:chap9}~章中,当我们考虑人类前额叶皮层的额外层次时,我们再次讨论这些发现。



总结

本节强调,当一种行为成为自动行为时,由颗粒状前部皮层及其纹状体区域进行的注意控制将让位给由运动前皮层及其纹状体区域进行的自动控制。
有些人可能会问,颗粒状的前部皮层和纹状体区域在哪里。
从某种意义上说,它们位于涉及颗粒状前额皮层和前运动皮层的回路之间。
第~\ref{chap:chap3}~章回顾了颗粒状前部皮层偏向于调节习惯(S-R关联)和条件结果导向行为(R-O和S-R - o关联)的大脑结构的证据。
我们建议偏向于最适合当前行为背景的那种关联。
这个想法表明,前部皮层的颗粒部分,就像颗粒部分一样,在行为的集中控制中发挥作用。
但颗粒状区域与颗粒状区域的不同之处在于,它们是通过调节自动控制来实现的:这是强化学习系统的产物。
其功能可视为介于颗粒状前额叶 -基底神经节环和运动前基底神经节环之间。



\section{前脑皮层的基本功能}

在本章的这一点上,我们解释了颗粒状前部皮层的连接将其置于一个独特的位置:它位于上下文,目标和结果层次的顶端。
我们回顾了证据,表明它的功能是基于复杂的当前环境和单一事件产生目标。
因此,颗粒状的前部皮层减少了错误,并在专注的过程中加快了学习速度,而不是自动控制行为。
这种能力代表了行为控制的质变,从依赖于缓慢调整的刺激-响应-结果关联的系统发育上的旧机制到基于参与的一次性事件的新机制。
从最广泛的意义上说,颗粒状前部皮层赋予灵长类动物一种新的方式来知道在非常规情况下该做什么\cite{wise2008forward},而这些情况需要集中控制。
考虑到这一切,现在是时候提出颗粒状前部皮层的基本功能了,它比“知道在非常规情况下该做什么”更具特异性和更强的可测试性。
到目前为止,我们所说的取决于对这个问题采取自上而下的办法。
我们想看看我们的想法是否能解释现有的证据。
接下来,我们使用自下而上的方法,看看我们是否可以从观察中建立相同的想法。



行为和生理指纹

表8.1列出了一些有前额叶皮层病变的猴子表现不佳的任务。
正如第~\ref{chap:chap1}~章所解释的那样,它们共同构成了一种行为指纹。
该表还显示了各种任务所需的行为的一些组成部分。
表8.2列出了前额叶皮层神经元的一些特性。
它们共同构成了生理指纹。
为了允许对表进行引用,每个任务(T)和细胞活动类别(C)都有一个字母和数字标识。
例如,T1表示延迟响应任务(表8.1),C1表示回顾性延迟期活动(表8.2)。
我们从这些表格中看到了关于前额叶皮层功能的六个主题:整合、干扰、灵活性、前景、序列和估值。
毫无疑问,其他作者会列出不同的任务和额外的单元格属性。
他们可以强调不同的主题,如分类、抽象、集合、抑制、计划、监控和注意。
其中一些差异代表了真正的分歧,第~\ref{chap:chap10}~章讨论了一些关于前部皮层功能的不同观点。
有些只是代表术语上的差异:注意解决干扰;探矿包括设定和规划;分类是整合的结果。
1.集成。
表8.1中的许多任务依赖于前部皮层的整合功能,我们在前面的连词中讨论过。
条件任务(T6-8)依赖于上下文-目标-结果连词,对象就位场景任务(T19)需要使用背景来进行上下文连接,并且贬值任务(T22)要求将特定食物的外观、味道和当前价值结合起来。
许多种类的细胞活动也反映这样的连词(C5, C6, C28),如第~\ref{chap:chap6}~章所讨论的(见表6.1)。
2.干扰。
表8.1列出了几个任务的干扰,干扰的解决取决于干扰者对相关项目的注意力。
第~\ref{chap:chap6}~章解释了延迟响应(T1)和延迟交替(T2)任务,猴子在记忆中有一系列的位置。
然而,只有最近的位置提供了选择当前目标以获得预期结果的上下文。其他人的作用是分散注意力。
第~\ref{chap:chap6}~章对有序对象任务(通常称为自有序任务(T4))上的对象给出了类似的说明。
与这种需求相一致,前额叶皮层细胞编码顺序(C3)和顺序连词(见表6.1)。
3.灵活性。
表8.1列出了许多任务的灵活性。
条件任务(T6, T7)和策略任务(T9)要求猴子在不同的试验中改变目标。
有时条件变化不那么频繁,在试验块中,如在逆向任务中(T13, T20)。
逆转会引起奖励预测错误信号,而前皮层的细胞活动反映了这种错误(C31)。
细胞在逆转任务(C19)、条件任务(C5、C6)和策略任务(C9、C12)中反映目标的转变。
4. 勘察。
表8.1列出了许多任务的前瞻,前瞻编码,在我们使用这个术语的意义上,涉及作为目标的对象和地点的短期记忆。
如前所述,前瞻编码对辨别学习集和反转学习集有重要贡献。
在这些任务中,猴子使用事件来产生目标,前瞻编码在间隔时间内将这些目标维持在短期记忆中。
在延迟响应任务(T1)中,前瞻编码维持了延迟期间的目标记忆,并保护其不受先前试验选择目标的干扰。
前瞻性编码发生在试验内(C8, C10)和试验间(C10)延迟期。
5. 序列。
许多任务需要产生一系列目标的能力(T17),细胞活动在一系列目标中编码每个目标(C16)。
规划一系列次要目标(T16, T17)的能力涉及到为该系列的不同组成部分(C17)编码的细胞亚群,它们在各种各样的时间框架内发挥作用。
延迟交替任务(T2)也可以看作是涉及序列的任务,因为当前目标依赖于前一个目标。
6. 估值。
猴子通过处理奖励反馈来学习表8.1中的所有任务,因此评估涉及到所有目标选择。
例如,使规则成为当前规则(T15)的事实是,如果遵循它,它会导致有益的结果。
颗粒状前额叶皮层中的许多细胞(C22-30)显示出各种类型的与结果相关的活动,如第~\ref{chap:chap4}~章所述。



建议

我们的自底向上方法产生了主题列表,而我们的自顶向下方法解释了许多任务的结果。
然而,为了弄清灵长类动物前额皮层的基本功能,我们需要综合一切。
为了迎接这一挑战,我们提交了一份提案,首先是简短的形式,然后是稍微详细阐述的版本,然后是对术语和概念的说明。
这一建议代表了第~\ref{chap:chap3}-\ref{chap:chap7}~章所阐述的关于前额叶皮层各主要区域的建议的高潮。
简而言之:颗粒状前额叶皮层生成适合当前上下文和当前需求的目标,并且它可以基于单个事件来实现。
扩展:颗粒状前部皮层的基本功能,作为一个整体,是产生目标-作为行动目标的物体和地点-在当前环境和预期结果中是适当的,根据当前的生物需求进行评估。
与系统发育上较老的学习系统相比,它可以基于单个事件专心地学习。
因此,它提供了一种减少错误的机制,它通过两种方式做到这一点:通过快速学习,以及通过提供一种学习和应用抽象规则和策略的机制。
必要时,前脑皮层在前瞻记忆中保存目标和目标序列,直到可以尝试实现它们。
由于我们使用的许多术语在不同的研究领域具有不同的含义,因此我们添加了以下解释性注释:“当前环境”包括当前感官受体可用的刺激以及与产生当前目标相关的最近事件。
通过“产生目标”,我们排除了习惯性的、条件化的、自动的、显性的或常规的行为。
通过“当前需求”,我们指的是这样一个事实,即在消耗某种特定资源达到一定数量后,对该资源的生物需求、驱动力或动机会减少。
通过“目标”,我们指的是作为行动目标的对象或位置。
在这个意义上,我们把目标和结果区分开来。
通过“事件”,我们指的是在特定时间和地点的背景、目标、行动和结果的结合。
通过“基于单个事件生成目标”,我们将基于单个事件的目标选择与基于多个事件平均强化的缓慢学习区分开来。
通过“抽象规则”,我们指的是以“与样本匹配规则”为例的认知操作。抽象是指一个特定的任务操作适用于任何刺激项目,无论是新的还是熟悉的。
我们所说的“策略”是指对某些行为问题的部分解决方案,或者是对一个问题的两个或多个解决方案中的一个。
从抽象规则的意义上说,策略必然是抽象的。
所谓“前瞻记忆”,我们指的是在短期记忆中对目标的编码和维护。



后果

如果我们的建议是有道理的,那么灵长类动物前额皮层的基本功能一定有很多后果。
表8.3根据前面讨论过的信息层次(上下文、目标和结果)和主题(集成、解决干扰、灵活性、前景、序列和估值)对其中一些进行了总结。
以上下文层次结构和集成主题为例。
这张表表明,灵长类动物可以整合所有感官领域的信息,它们可以整合不同时间的事件,它们可以整合背景、事件、行动,并且输出。跨时间整合导致“跨时间偶然性”;跨事件的整合导致序列和“临时扩展事件”;跨感觉域的整合产生多模态特征连词;跨范例的集成导致了分类和抽象。
该表继续列出剩余的层次结构和功能主题。
当我们把这些综合起来看,灵长类动物前额叶皮层的基本功能似乎有一个重要的结果:那些不能产生最有利结果、浪费精力或增加被捕食风险的觅食选择的数量减少。
在实验室测试中,该功能在各种行为测试中导致更少的错误和更快的学习:
1. 如果有足够的经验,猴子可以在一次试验中学习视觉辨别(图~\ref{fig:8_2})、条件视觉运动映射(图~\ref{fig:8_5})和物体就位连词(图~\ref{fig:3_10}~和~\ref{fig:8_4})。
2. 他们可以利用自己在类似问题上的经验来学习和应用抽象的规则和策略。抽象的规则和策略适用于任何刺激材料,因此心理学家有时称之为能力转移\cite{warren1966reversal}。
这种迁移可以减少在学习样本之前的错误\cite{bussey2001role}。
3.他们可以在当前事件的基础上选择未来目标,并将这些目标保存在前瞻性记忆中,就像他们在学习设定任务时所做的那样\cite{murray2006prospective,wilson2008prefrontal}。
4. 他们可以将结果分配给对象中的单个选择\cite{walton2010separable}。
5. 他们可以把这些减少错误的机制结合起来,因为他们的颗粒状前部皮层是一个整体。



自适应的优点

刚才提到的实验室任务似乎与灵长类动物的日常觅食行为相距甚远。
但正如第~\ref{chap:chap2}~章所解释的那样,灵长类动物往往寿命较长,而类人猿往往依赖于不稳定的食物资源。
因此,类人猿可以通过使用单一或不频繁的事件来指导他们的觅食选择。
减少觅食错误的能力具有至关重要的优势。
生活有时是残酷的,成功的机会很少,而危险却很多。
在一个充满竞争对手和捕食者的多变环境中,即使是一个错误的后果也可能是严重的,甚至是致命的。
在第一章中,我们承诺不仅解释灵长类动物的前皮层做什么以及它是如何执行该功能的,而且还解释了其发展背后的选择因素。
为了实现这一目标,我们在觅食选择方面重新提出了我们的建议:
灵长类动物的颗粒状前部皮层进化为一种适应,以减少系统发育上较老的学习机制产生的非生产性、高风险或昂贵的觅食选择的数量。
这些较旧的机制依赖于基于强化的缓慢、累积的关联强化,在许多反馈事件中平均。灵长类动物进化的新机制带来了单一事件——环境、目标、行动和结果的结合——来影响觅食目标的选择。
关于这种新的前额叶机制的信息在很大程度上取决于灵长类动物进化出来的视觉进步。
请注意,我们并没有声称灵长类动物的颗粒状前部皮层赋予了从单一事件中学习或在此基础上选择目标的独特能力。
说灵长类动物能做某事并不意味着其他动物不能。
老鼠和许多其他动物一样,如果吃完后感到恶心,就再也不会吃新的食物,这种现象被称为味觉厌恶效应\cite{rozin1971specific}。
一些物种表现出印记:从一次观看事件中对个体的长期识别\cite{bateson1969development}。
我们认为,许多其他特殊目的的学习机制也已经发展起来,可以在其特定领域调解一次试验学习。
在第~\ref{chap:chap1}~章中,我们讨论了这样一个事实,即大鼠可以使用单一事件来拒绝放射状迷宫的一只手臂作为当前目标(Olton et al 1982):这是一种优势的“赢-转移”策略。
第~\ref{chap:chap3}~章解释了大鼠自发地执行非匹配位置任务,该任务利用了它们对单个先验事件的响应的优势行为。
我们知道,老鼠可以利用情境,比如迷宫地板的纹理,根据一次经历来记住一个可见物体及其位置(Eacott \& Easton 2010)。
我们接受这一切,甚至更多。
然而,我们所主张的是,灵长类动物减少错误的方式比这种专门的系统更普遍地适用于更广泛的觅食问题。
由此带来的优势可能看起来很微妙,但它可能对类人猿灵长类动物的成功至关重要。
早些时候,我们列出了六个主题,似乎出现在大量的文献关于前部皮层:整合,干扰,灵活性,前景,序列和估值。
我们现在可以提出这些能力是如何影响觅食选择的。
1. 集成。
灵长类动物可以整合所有感官领域的觅食信息来代表当前的环境。
第~\ref{chap:chap2}~章解释了现代类人猿的祖先如何适应白天的生活,并发展出一个最终支持三色视觉的中央凹。
同一章解释了灵长类动物的背侧和腹侧视觉流,包括新的后顶叶区和颞叶区,也在进化。
中央凹视觉使类人猿在很远的地方就能注意到树木的许多细微的方面,当距离足够近时,还能看到哪些动物、树叶和果实可能在那里。
前部皮层在整合来自背侧流和腹侧流的信息以及来自其他感觉模式的信息方面起着关键作用。
这些信息使灵长类动物能够更好地选择探索哪块资源,以及在一块区域内获得(或避免)什么物品,以及以什么顺序。
在遥远的脑区之间的选择最终涉及到导航,因此海马体的功能需要与前皮层的功能相结合(第~\ref{chap:chap3}~章)。
选择在一个小块中获取什么,最终涉及到由前运动皮层控制的手部运动,特别是在灵长类动物进化的视觉主导的参照系中,伸手和抓握。
这些行为通常涉及视觉和体感信息的整合。
操作、触诊和触觉的探索在很大程度上依赖于“触觉中央凹”的输入,这代表了灵长类动物的另一种创新(第~\ref{chap:chap2}~章)。
此外,它们的听觉系统检测灵长类动物或鸟类在特定地点觅食时发出的叫声和其他声音(第~\ref{chap:chap7}~章),并将这些信息与视觉相结合。
2. 干扰。
记忆的干扰对觅食的类人猿来说是一个特别的问题。
它们主要以水果、昆虫和树叶为食,它们以许多相似的东西为食。
这些其他物品中的大多数不能满足它们目前的生物需要。自上而下的注意力会使感官输入的处理和记忆偏向于那些与这些需求相关的信息。
正如第~\ref{chap:chap6}~章所说,背侧前部皮层似乎在减轻不相关事件的干扰方面起着关键作用,部分是基于它们的时间顺序。
第~\ref{chap:chap5}~章解释了尾侧前部皮层在调节自上而下对感觉环境相关特征的注意中的作用。
3.的灵活性。
在第~\ref{chap:chap2}~章中,我们指出了类人猿灵长类动物进化的证据,以利用挥发性资源,如水果和嫩叶。
在这种情况下,灵活一点是值得的。
类人猿灵长类动物可以灵活地行动,部分原因是如果它的第一个选择的行动没有达到它的目标并产生预期的结果,它可以在记忆中保持目标并转向另一个行动。
因此,类人猿前部皮层提供的一个关键进展涉及独立于实现该目标所需的行动的目标表征的产生、维持和长期存储。
在很大程度上,正是由于这个原因,将行动层次结构细化为目标层次结构才显得如此重要。
本章解释了颗粒状前部皮层对目标层次的重要性。
4. 勘察。
如果一个目标可以独立于实现的方式来表示,那么它也可以独立于实现的时间来表示。
前瞻编码可以在记忆中维持暂时的远距离目标,第~\ref{chap:chap6}~章和第~\ref{chap:chap7}~章分别讨论了背侧和腹侧前额叶皮层在这一功能中的作用。
5. 序列。
目标可以按顺序和层次结构组合。
第~\ref{chap:chap6}~章解释了当一个目标需要一系列行动或从属目标时,类人猿灵长类动物可以提前准备序列中的每个元素,并规定它们的顺序和时间。
最终目标可以在前瞻记忆中保持即使次级目标可能会改变。
第~\ref{chap:chap6}~章还提出了背侧前额叶皮层在制定有效的顺序目标策略中的关键作用。
由于大多数类人猿在社会群体中觅食,个体在一个斑块内激烈竞争资源,因此获得一组选定的水果或叶子的有效计划可能提供特别重要的优势。
6. 估值。
如果在进食过程中,动物对某种食物接近饱腹感,这将降低这种食物的主观价值和生产这种食物的树木的种类。
当生物需求或优先级以这种方式改变时,觅食目标也会改变。
第~\ref{chap:chap4}~章解释了颗粒眶额皮层在灵长类动物评估物体方面的进展。



总结

根据我们的建议,类人猿灵长类动物的颗粒状前部皮层为它们提供了一种新的能力:利用单一事件产生目标的能力。
由颗粒状前部皮层实现的学习系统因此增强了祖先的强化学习系统,该系统通过缓慢和累积的联想调整来学习。
当面对类人猿灵长类动物在进化过程中遇到的觅食问题时——其特征是依赖于被子植物树的特定产品而导致的一段时间的匮乏和消耗——祖先系统不可避免地产生了大量错误。
错误是危险的,而且代价高昂。
它们的新学习系统和祖先的学习系统都能减少错误并最大化奖励,但灵长类动物的前皮层做得更快,这提供了一个至关重要的适应优势。
尽管我们在新的和古老的学习系统之间进行了对比,但我们的建议并不意味着前脑皮层完全独立于后者运作。
第~\ref{chap:chap3}~章和第~\ref{chap:chap4}~章提出,前部皮层的颗粒部分在影响习惯和其他条件行为方面发挥关键作用。
我们关于觅食选择的建议也不应该被理解为暗示在社会选择中没有类似的作用。



\section{结论}

早期灵长类动物

我们认为,早期灵长类动物的颗粒状前部皮层提高了它们在细枝生态位中发现和评估物体的能力。
他们可以学习,基于单一的经验,在视觉项目中选择产生给定的结果(第~\ref{chap:chap4}~章),他们可以对他们的目标保持公开和隐蔽的注意力(第~\ref{chap:chap5}~章)。
颗粒状前额叶区域的发展伴随着对细枝生态位的一系列适应。
这种以后肢为主导的运动模式的转变为一种新的进食技术解放了双手。
立体视觉的进步开启了灵长类动物成为通常被称为“视觉动物”的过程。
早期灵长类动物也出现了新的后顶叶和下颞叶区域,以及利用顶叶输入来指导视觉坐标框架内的伸手和抓握运动的前运动区域(第~\ref{chap:chap2}~章)。
在这种情况下,尾侧前部皮层在克服细枝生态位中的杂波和干扰方面具有优势(第~\ref{chap:chap5}~章)。
同样,颗粒状前部皮层通过对比细枝生态位中各种物品的当前生物学价值,主要基于视觉和特定选择与特定结果的联系,提高了在物体之间进行选择的能力(第~\ref{chap:chap4}~章)。


类人猿

后来,在类人猿灵长类动物的进化过程中,随着这些动物体型的增加,觅食范围的扩大,并开始依赖丰富但易挥发的资源,如成熟的果实和未成熟的叶子,更多颗粒状的前额叶区域出现了。
它们白天在竞争激烈的环境中觅食,面临着严重的捕食威胁。
这样的生活安置了一个重视做出正确的觅食选择,避免低效和昂贵的选择,并在有限的经验基础上学会快速做出选择。
新的类人猿颗粒区包括背侧和腹侧前额叶皮层,也可能包括极侧前额叶皮层。
这些区域随着视觉的进步而进化,如中央凹视觉和三色视觉(第~\ref{chap:chap2}~章)。
它们的视觉进步使这些灵长类动物在分析视野内物体的位置、颜色、形状、纹理、光泽度和半透明性方面达到了一个新的复杂水平。
他们新的前额叶区域使他们能够基于单一事件产生一个目标或一系列目标。
为此,他们复杂的感知环境,包括使用的地方,时间,和秩序的视觉事件(第~\ref{chap:chap6}~章),视觉场景(第~\ref{chap:chap3}~章),从视觉和声学信号(第~\ref{chap:chap7}~章)。
这些领域允许类人猿进化学习,根据单一的经验,目标的视觉场景选择(图3.10和~\ref{fig:8_4}),这两个对象的选择作为一个目标(图8.2),和目标选择的任意符号(图~\ref{fig:8_5})。
根据这本书中提出的观点,类人猿灵长类动物的前额叶区域是一种特殊目的的适应,它克服了类人猿历史上特定时间和地点的特定问题。
这些区域减少了这些动物在白天长途觅食时可能犯的错误,目的是为了获得反复无常的资源。
然而,类人猿解决特定觅食问题的方式对后来灵长类动物的认知产生了深远的影响。
他们通过进化出一种新的通用学习系统来解决他们的特殊问题——周期性的资源匮乏和人员流失,这种系统超越了祖先系统的局限性。
祖先的通用学习系统缓慢地调整刺激、响应和结果之间的联系,产生了太多的错误。
类人猿的学习方式有三个关键方面。类人猿的前额叶皮层通过快速学习,通常基于单一经验,非常迅速地解决觅食问题。
灵长类动物的前皮层通过学习和应用抽象规则和策略,通过吸收类似情况的经验来解决新的觅食问题。
通过对行为的细心控制,类人猿灵长类动物在表现不佳时可以克服祖先的学习机制,因为它们在觅食危机中不可避免地会这样做。
我们得出结论,在灵长类历史的两个关键时期,颗粒状的前皮层为我们的祖先提供了一种微妙的适应优势,一种特别适合他们的生态位、问题和机会的优势。
这个边缘的微妙之处解释了为什么没有颗粒状前额叶区域的非灵长类哺乳动物可以很好地生存,就像有损伤的灵长类动物一样。
灵长类动物和其他哺乳动物一样,没有颗粒状的前部皮层,而是依靠祖先的学习机制。
本章主要基于对猴子的研究,提出了灵长类动物前额叶皮层的一个简单功能。
如此基本的功能似乎永远无法解释人们在执行复杂的认知任务时前额叶皮层的激活。
因此,下一章将讨论如何解释这些激活。










\chapter{人类前额叶皮层:从指令和想像中产生目标}
这本书提出了关于灵长类动物前额叶皮层基本功能的方案。

\section{概述}
本章的目的是:检验我们关于PF皮层基本功能的建议是否可以解释当人们执行复杂认知任务时在那里发生的成像激活。人类PF皮层中的许多激活似乎反映了对背景、目标和结果层次结构的阐述,从背景中生成目标,使用事件和抽象规则来选择目标,或目标的前瞻性编码。 其中一些激活发生在可能出现在类人猿或人类进化过程中的区域中,我们试图通过重新表示这一概念来解释这些。现有PF区域的扩展以及新区域的可能出现,导致人脑的大小和形状发生变化,连接方式也发生变化。我们提出了一个观点:现代人和猴子的前额叶皮层执行从他们最后的共同祖先继承的共同功能:它以减少错误的方式产生目标。在一个关键的进化过程中,人类前额叶皮层进一步减少了错误,因为人们可以从指令或模仿中学习,并且因为人们可以在采取任何行动之前进行心理试错行为。 得出结论:人类可能可以完全避免错误。
\section{介绍}
第8章提出PF皮层具有简单的基本功能。本章探讨该功能是否可以解释在受试者执行复杂认知任务时在人类PF皮层中看到的成像激活。
\par
读者会体会到我们这项工作的艰巨性。我们在第8章中提出的建议在很大程度上取决于来自猴子的数据。这些数据主要涉及地点和物体之间的选择,如通过伸手或眼球移动来实现的。说这些功能与类比推理或做出道德判断相去甚远,至少可以说是一种轻描淡写的说法。
\par
这个问题类似于O’Keefe和Nadel(1978)首次提出海马体具有导航功能时所面临的问题。当时,导航机制如何支持情景记忆似乎还是个谜,尽管从那时起,这种组合就变得不那么神秘了(Burgessetal.2002)。我们现在可以理解海马体在导航中的祖先作用如何在进化过程中得到详细阐述,以涵盖复杂的认知功能,如情景记忆。同样,本章试图根据为做出更好的觅食选择而进化的祖先机制的阐述来解释复杂认知任务的激活(第8章)。
\par
有足够的时间来详细说明。只需考虑一个事实,即旧大陆猴子和人类的最后一个共同祖先生活在23万年或更久以前(Ma)(Kay等人,2004年;Kumar等人,2005年)。这个事实意味着这两个世系已经分别进化了数千万年,在那段时间里,人类和猴子的大脑肯定都发生了变化。在此期间,人类的大脑变得比同样体重的猴子预期的大4.8倍(MacLeod等人,2003年)。与此同时,人脑发展出了自己的专业化,这在整本书中都有评论(Passingham2008)。
\par
本章首先简要总结了人类大脑在进化过程中大小和形状的变化,重点是颗粒状PF皮层。然后它处理连接和组织,包括新区域的可能出现。本章的其余部分解决了关键问题:第8章为类人前脑皮层提出的简单功能能否解释高级认知期间人类前脑皮层中发生的成像激活?
\section{人类进化中的额叶}
遗传证据表明,类人猿在5-7Ma(Kumar等人,2005年)与人类世系分道扬镳,而最早的化石可能是原始人Sahelanthropustchadensis(Guy等人,2005年),其年代约为7Ma(LeFur等人,2009年)。这种灵长类动物的颅骨容量约为360–370cm3,与小黑猩猩大致相同(Guyetal.2005)。现代人脑平均约为1350–1550cm3(Sowelletal.2007)。当然,人往往比黑猩猩大,但我们的大脑仍然比相同体型的假想类人猿的大脑大3.5倍(MacLeod等人,2003年)。
\par
所以原始人的大脑开始时很小,至少以现代人类的标准来看是这样。这段历史类似于第2章为类人猿大脑描绘的历史:在其早期进化历史中,大脑相对于身体尺寸而言相对较小,随后在其后期进化过程中大脑尺寸出现“等级增加”。
不幸的是,我们无法测量化石原始人额叶的大小。与第2章描述的化石类人猿大脑不同,人科动物硬脑膜的厚度阻止了在其头骨内表面形成清晰的脑沟印迹。所以我们只能评论大脑的整体形状。由此,我们知道在原始人进化过程中,前额叶变得更宽、更圆。
\par
福尔克等人(2000)制作了纤细而健壮的南方古猿的内脑模型,早期原始人的寿命约为1.5–2.5Ma。在“健壮”的南方古猿旁人属中,额叶具有相对尖的形状,类似于现代黑猩猩和大猩猩。在细长的南方古猿A.africanus中,额叶的形状略微更圆。我们很幸运有一个保存完好的南方古猿的头骨,它来自南非,年代约为2Ma(Carlsonetal.2011)。眼眶和极PF皮层的形状表明向更圆的形式过渡似乎在人属的原始人中。
\par
Bruner和Holloway(2010)测量了南方古猿额叶的最大宽度,并将这一结果与直立人和尼安德特人这两种后来进化的原始人的宽度进行了比较。相对于内脑的最大宽度,更晚近的原始人的额叶宽度超过了南方古猿。这一发现表明,在原始人类进化过程中,额叶的相对大小有所增加。
\par
现代人类可能是从与海德堡人(Homoheidelbergensis)有亲缘关系的原始人进化而来,有时也被称为古人类。来自埃塞俄比亚60万年前(Ka)的Bodo头骨和来自赞比亚300Ka的Kabwe头骨(Conroyetal.2000)来自这个物种。布克斯坦等人(1999)测量了这些头骨中大脑内壳前部的斜率,并将其与早期现代人类的两个头骨的斜率进行了比较:来自埃塞俄比亚的OmoI和OmoII头骨,日期为195Ka(1969年)。尽管古代人类头骨与早期现代人的头骨不同,眉骨较大,但内额脑壳的倾斜度并无不同。这一发现表明额叶的形状在古代人类中达到了现代状态,~300-600Ka。当然,形状本身告诉我们关于PF皮层的信息很少,但这些发现表明它的形状在我们最近的进化中相对较早地稳定下来。
\par
研究结果表明,在早期现代人类发生一项关键技术革命之前的某个时间,额叶已经发育良好。根据刀片技术和骨骼工具的证据,McBrearty和Brooks(2000)认为这些发展发生在~100Ka,但它们似乎直到很久以后才得到很好的确立(d'Errico&Stringer2011)。
\par
这些人的亲属在60–80Ka之前从非洲分散开来(Mellars2006),基因证据表明所有的亚洲人和欧洲人都是他们的后裔。这些祖先取代了尼安德特人,创造了更专业的工具包,并适应了世界上几乎所有的环境。
\section{现代灵长类动物的PF皮层大小}
人们只能从化石中获得关于额叶进化的线索,因为形状和大小告诉我们关于PF皮层的组织或功能的信息太少了。当然,颗粒状PF皮层仅构成额叶的一部分。
第2章解释了在类人猿进化过程中大脑尺寸的增加伴随着新的颗粒状PF区域的产生,特别是腹侧、背侧和极PF皮层。类似的事情也可能在人类进化过程中发生,研究现存灵长类动物的颗粒状PF皮层可以对这种可能性产生一些了解。不幸的是,尽管人们对人类大脑的进化给予了广泛关注,但将人类大脑与其他灵长类动物的大脑进行比较的文献并不支持像我们在第2章中回顾的类人灵长类动物那样可靠的结论。尽管如此,它确实支持一些初步的建议。
\par
Brodmann(1912)估计了颗粒状PF皮质的表面积,并将其与整个新皮质的表面积联系起来。正如Brodmann所见,颗粒状PF皮层占猕猴新皮层的11%,黑猩猩约17%,人类约28%(见图2.6)。
\par
这些差异非常大。如果Brodmann的数据是正确的,人类的PF皮层平均为34,770mm2,而黑猩猩为6719mm2。这使得它在人类身上大了约5倍,尽管一个典型的人的体重只比黑猩猩重10-20公斤。当人们认为初级运动皮层(区域4)在人类和黑猩猩中的大小略有不同时,差异就更加显着了(Preuss2011)。
\par
塞门德费里等人(2002)对Brodmann对黑猩猩的估计提出异议。他们估计位于类人猿中央前沟嘴侧的额叶皮层的百分比为26-30%,并发现它与人类相应的年龄百分比(29-33%)仅略有不同。但是,正如他们承认的那样,估计位于中央前沟头端的皮质百分比与估计颗粒状PF皮质的百分比之间存在差异。
\par
Passingham(2008)指出Bailey等人(1950)认为黑猩猩中央前沟的一些头侧皮质,区域FC,是颗粒状的而不是颗粒状的。从Brodmann(1912)的黑猩猩大脑图谱可以清楚地看出,他可能已将这个区域排除在他对颗粒状PF皮层的测量之外。这个因素可能部分地解释了Semendeferi等人给出的估计。
\par
埃尔斯顿等人(2006)使用Brodmann的数据来估计颗粒状PF皮层占据额叶的百分比。他们发现颗粒状PF皮层占人类额叶的80%,黑猩猩额叶的55%,长臂猿的53%,卡他林(旧世界)猴子的45-50%,41-46桔梗(新世界)猴子的%,链霉灵长类动物的41-43%(见图2.6E)。这些发现表明,颗粒状PF皮层在人类进化过程中显着扩大。
\par
最近对基因表达的分析对人类大脑如何在进化过程中变得如此大以及它们在胎儿发育过程中如何变得如此大产生了一些重要的见解。张等人(2011)检查了基因的位置和转录程度,在灵长类动物和啮齿动物分化后在灵长类动物中进化而来。图2.8以分支图的形式说明了这种分裂。他们称这种分裂后进化的基因为新基因,而不是从灵长类动物和啮齿动物的最后共同祖先那里继承的旧基因。与小鼠大脑相比,Zhang等人发现人类大脑发育过程中新基因的转录增强,其中大部分发生在新皮质中。年轻的基因编码许多转录因子,控制发育模式和速度。他们还展示了比旧基因更快地改变他们编码的氨基酸的证据。张等人得出结论,选择大脑功能某些方面的因素有助于年轻基因的起源。出于目前的目的,我们发现他们最感兴趣的发现是,在新皮质区域中,许多人类特异性基因的转录特别发生在PF皮质中。
\par
到目前为止,我们已经讨论了额叶的整体形状和颗粒状PF皮层的大小。但这些数据无法告诉我们特定区域的情况。极地PF皮层可能是最早出现在类人猿中的区域之一(第2章)。它在类人猴中很小,但它已成为人类额叶中最大的细胞构造区域(Öngüretal.2003)。极有可能是原始人进化过程中极地PF皮层的扩张导致了前面提到的形状变化。古人类头侧额骨的圆形和加宽可能意味着极PF皮层的扩张达到其现代状态300-600Ka。
\par
塞门德费里等人(2001)估计了现代类人猿和人类极地PF皮层(区域10)的范围。相对于整个大脑,与黑猩猩相比,极地PF皮层大约是人类大脑的两倍(图9.1)。
\par
图9.1人类极地PF皮层(区域10)的扩展。选定灵长类动物的分支图,左侧(箭头)的近似发散时间为数百万年(Ma)。每个圆圈的直径编码极PF皮层的大小参数,由底部的比例给出,按每个分支图下方显示的灰度代码分类。插图显示了人类极地PF皮层的位置。左侧,内侧视图,嘴侧向右,背侧向上;顶部,右侧,侧视图,嘴侧向左,背侧向上;右下角,腹面观,嘴向左,侧面向上。从TsujimotoS、GenovesioA、WiseSP修改而来。认知科学趋势15:169–76,©2011,经Elsevier许可。
\par
图9.2。在选定的灵长类动物群体中,极PF皮层体积的回归作为脑体积的函数。实线:来自猿类数据的回归;虚线:外推到人类大脑的大小。顶部:线性刻度。底部:对数刻度,单位斜率由虚线表示。缩写:G,长臂猿;A,类人猿;哈,人类。修改自SemendeferiK、ArmstrongE、SchleicherA、ZillesK、VanHoesenGW。人类和类人猿的前额皮质:区域10的比较研究。美国体质人类学杂志114:224–41,©2001,JohnWileyandSons。
\par
同一作者对第13区进行了类似的分析,发现与其他类人猿相比,人类没有这种扩张。这些发现支持了极地PF皮层在人类进化过程中显着扩展的结论。
然而,Holloway(2002)淡化了这些结果。他指出,人脑极PF皮层的大小仅比大脑与人脑一样大的猿类的预测大6%。图9.2显示了极PF皮层大小与类人猿大脑大小的对数对数回归(Semendeferi等人,2001年),它表明人类价值仅位于回归线上方。
\par
但是Holloway(2002)假设如果人类极PF皮层符合对数-对数回归线,或者接近符合,那么它具有相等的容量。这种信念忽略了一个事实,即对数-对数回归线的斜率为1.6,大大超过了单位斜率(1.0)。陡坡意味着较大的大脑比较小的大脑具有更大比例的极PF皮层。所以我们不能假设有类似的容量。
神经科学以外的一个例子解释了这一点(Gould1973)。已灭绝的爱尔兰麋鹿有巨大的鹿角,但当将它们的长度与衡量体型(肩高)进行对比时,它们的长度接近回归线。因为斜率超过了统一,所以它们的鹿角占身体的比例更大。因此,它们的功能更有效,大概是在吸引配偶方面。因此,结构符合回归线的事实并不意味着等效的功能能力。
\par
\textbf{总结}
\par
人脑不仅仅是猕猴和人类最后一个共同祖先的大脑的放大版。大脑的某些部分比其他部分扩展得更多。当考虑作为一个整体的新皮层的一部分时,颗粒状PF皮层在人类中比在猕猴中大~2.5。
\par
然而,大小只能告诉我们一点点,所以我们接下来考虑人类和其他大脑在微观结构和内部连接方面的差异。
\subsection{微观结构和内部连接}
在他们对极PF皮层的研究中,Semendeferi等人(2010)发现,与类人猿相比,人类第3层细胞体之间的这一区域存在大量神经纤维网。换句话说,细胞体的间距更大。与猿类相比,这种解剖学特征可能反映出人类有更多的树突、树突棘和末端。这一特征指向颗粒PF皮层的综合功能的详细说明,一般而言,特别是极PF皮层。
\par
Elston(2001)测量了PF皮层第3层锥体细胞树突上的棘数。他发现人类大脑中的这些细胞比猕猴大脑中的细胞多70%。由于连接终止于脊柱,这一发现意味着与猕猴相比,每个细胞都可以在人类PF皮层中整合更多信息。
\par
Elston(2007)还绘制了各种灵长类动物(包括人类)的棘数与PF皮层大小的关系图。他发现PF皮层越大,树突棘的数量就越多。鉴于人类PF皮质的大尺寸,脊柱的数量与预期非常吻合。这似乎不是细胞大小增加的产物。
\par
内在联系在皮层下白质中运行,有两种类型。申克等人(2005)区分位于白质核心的长联合纤维和连接PF皮层回旋内相邻区域的较短纤维。他们比较了一些灵长类动物的这些值。在人类中,包含长关联纤维的白质体积与人类大脑的预测一致。然而,短纤维比人类预期的更广泛。
\par
\textbf{总结}
\par
在第8章中,我们回顾了证据表明,当一个人提升了针对上下文、目标和结果的各种处理层次时,刺的比例会增加。我们认为,这种解剖学特征允许大脑在越来越抽象的层次上形成表征。人类PF皮层可以将这种发展提升到一个新的水平。此外,人类PF皮层似乎有大量短纤维将一个PF区域连接到另一个。因此,人类PF皮层可能特别适合整合信息。在后面的部分中,我们提供证据表明PF皮层中的一些激活反映了整合来自不同认知领域的信息的能力。
\subsection{外部连接}
上一节考虑了连接数,但没有考虑远程连接的整体模式。最近的研究利用了水沿着轴突在外部和内部扩散的事实(Basser&Ozarlsan2009)。这一特性催生了一种研究人脑连接的方法,称为扩散张量成像(DTI)。施马曼等人(2007)使用这种方法的修改来绘制猴子的连接图,并发现类似于标准轴突纤维追踪方法的结果。然而,DTI有严重的局限性。这些方法永远不会像在猴子和其他动物身上使用的方法那样敏感,这些方法可以揭示微观水平上的联系,并且在足够重要的情况下,还可以揭示电子显微镜水平上的联系。它们不能像猴子身上使用的方法那样揭示投射的确切起源和终止。尽管如此,DTI数据提供了有关人类PF皮层连接的宝贵信息。
\par
克罗克森等人(2005)将PF皮质分为七个部分:背侧PF、腹侧PF、外侧眶PF、中央眶PF、内侧眶PF、前扣带回和扣带回。他们使用Hubbard等人设计的方法测量了种子区域与另一个区域连接的可能性。(2005),并发现人类PF皮层连接的一般模式与猴子相似。例如,后顶叶皮层与背侧PF皮层相连,而下层颞叶皮层与腹侧和眼眶PF皮层相连。在猴子和人类的大脑中,杏仁核都与眼眶PF皮层和前扣带回皮层相连。
同一组研究人员使用DTI将前扣带皮层分为九个区域(Beckmannetal.2009)。人脑的结果与猴子轴突纤维追踪研究的结果相似(Carmichael&Price1996)。例如,在内侧PF皮层中,膝前和膝下区域与两个物种的眼眶PF皮层、杏仁核、下丘脑和腹侧纹状体的联系最强。
\par
DTI还表明,丘脑内侧核(MD)和PF皮质之间的整体连接模式在人类和猴子中是相似的(Kleinetal.2010)。MD的内侧部分与眼眶PF皮质相连,尾侧MD与内侧PF皮质相连,包括前扣带皮层,外侧MD与背侧PF皮质相连。
\par
据我们所知,人类和猕猴PF皮层连接的整体模式似乎相似。这种相似性大概反映了我们最后一个共同祖先的血统。然而,现代猕猴和人类已经分别进化了20到3000万年,这意味着在任何一个谱系中都可能发展出特化。在接下来的部分中,我们将指出人类大脑的三个可能的专业化:布罗卡区、极PF皮层的外侧部分和背侧扣带回皮层。
\subsection{布罗卡区}
布罗卡区,布罗德曼区44和45,位于腹侧运动前皮层(区域6)的前面。从尾端到头端,腹侧前运动皮层是无颗粒的,44区是异常颗粒状的,45区是颗粒状的(Petridesetal.2005)。在拓扑可比较的区域中,可以在猴子(图9.3)中发现相同的进展。
\par
克莱因等人(2007)使用DTI绘制了人脑中44和45区连接的整体模式。他们得出结论,区域44和45以其连接区分,与细胞结构定义的区域44和45非常吻合(图9.3)。DTI数据显示,在人脑中,44区与下顶叶皮层相连,45区与颞叶皮层相连(Freyetal.2008)。和克罗克森等人(2005)在猴子中使用DTI发现了相同的结果模式,Petrides和Pandya(2009)使用标准示踪剂也是如此。
\par
里林等(2008)以这一发现为起点,比较了人类、黑猩猩和猴子的布罗卡区。他们发现45区与中颞叶皮层的联系比44区更强。Rilling等人比较了黑猩猩和人类,发现人类的中颞叶皮层与布罗卡区的联系比黑猩猩的更广泛。最后,在人类中,这条通路在左半球比在右半球更大更广泛,而在黑猩猩中它是对称的。随着儿童的成长,这些连接的强度在左半球增加,但在右半球却没有(Paus等人,1999年)。
\par
图9.3(A)人脑中的布罗卡区。缩写:CS,中央沟;SF,外侧裂;6V,腹侧区域6。(B)猴子第4层的密度相对于相同三个额叶区域的平均值。经MacmillanPublishersLtd.许可转载。PetridesM、CadoretG、MackeyS.布罗卡区猕猴同系物的口面部躯体运动反应。自然435:1235-8,©2005,自然出版集团。
\par
布罗卡区在人类PF皮层左侧进化的想法当然不是什么新鲜事,但成像激活支持了这个想法。人们可以记住发音(Conrad1972)并且可以推理单词(Goel&Dolan2004)。当他们这样做时,成像激活往往发生在左半球。但是当他们处理视觉空间信息时,激活往往发生在右半球(Smithetal.1996)。尽管这里审查的比较证据还远非定论,但它似乎与在人类大脑中观察到的语言专业化大体一致。
\subsection{极地PF皮层}
对布罗卡区的研究表明,与人类和黑猩猩的最后共同祖先相比,人类的连通性发生了变化。此外,文献包含暗示性证据表明人脑中的两个区域在猴子中缺乏同系物:极PF皮层的外侧部分,10区的一部分,以及背侧扣带回皮层,32区的一部分。
\par
塞门德费里等人(2001)表明极PF皮层的外侧部分存在于类人猿和人类中,但不存在于猴子中。如果被接受,这个想法将意味着它是在类人猿和人类的祖先中进化而来的。然而,他们的结论完全取决于拓扑学和细胞构造证据。鉴于后者取决于主观标准,需要更多的证据来接受这个想法,尽管它似乎有道理。
\par
纳尔逊等人(2010)发现了支持证据,其形式为下后顶叶皮层中心的激活与外侧极PF皮层的激活之间存在相关性。火星等(2011)比较了人类和猴子的静息状态相关性。他们在中外侧PF皮层和极PF皮层之间的边界上选择了一个感兴趣区域。在人类中,该区域的静息激活与下顶叶皮层中央部分的激活相关,这与Nelson等人的发现一致。但是火星等人没有在猴子中发现相应的相关性,这一发现与轴突运输研究中两个区域之间缺乏联系一致(Petrides&Pandya2007)。这些发现与外侧极PF皮层(区域10)或与其相邻的区域在猴-猿分裂后进化的建议一致。但是,需要进一步的证据证明这种可能性。
\par
如果极PF皮层的外侧部分在人类和类人猿中进化,那么这个区域可以被视为位于处理层次结构的顶部。Summerfield和Koechlin(2009)为外侧PF皮层提出了一个从尾端到头端的层次结构,它们与指定动作的上下文的复杂性有关。和Koechlin等人(1999)已经建议最延髓的部分,外侧区域10,在处理次要目标时牢记主要目标。
\par
有限数量的证据支持这样一种观点,即极PF皮层的外侧部分(区域10)出现在人类和猿脑中,但不存在于猴脑中。它可能会在颗粒状PF皮层侧面的尾端到头端层次结构中创建一个新级别。




\chapter{结论}
这本书提出了关于灵长类动物前额叶皮层基本功能的方案.

\section{摘要}
本章将我们的建议与文献中的其他建议进行比较,并通过五个测试对每个建议进行评估:(1)该建议是否考虑到了PF皮层的进化史?(2)它是否解释了连接解剖学如何允许PF皮层执行所提议的功能?(3)它是否明确了PF皮层的功能与其他区域的功能有何不同?(4)是否与PF皮层的广泛发现相一致?(5)它的陈述是否精确到足以被测试?最后,我们提出一些测试我们建议的方法。
\section{介绍}
第一章阐述了五个目标,我们现在可以更全面地阐述这些目标:
\par
1.说颗粒状PF皮层让灵长类动物做了他们的祖先和其他哺乳动物做得效率较低的事情,如果有的话,以及说类人猿灵长类动物颗粒状PF皮层的扩张让他们比其他灵长类动物做得更好。
\par
2.说颗粒状PF皮层的轴突连接如何允许的,而不是其他区域,提供这些优势。
\par
3.作为一个整体解释粒状PF皮层的基本功能,并解释它的功能与大脑其他部分的功能有何不同。
\par 
4.在复杂的认知任务中,这个功能是如何解释在人类PF皮层中观察到的激活的。
\par 
5.解释我们的建议与文献中其他建议的不同之处,并告诉读者什么样的观察可以反驳它。
\par 
现在是时候评估一下我们在多大程度上实现了这些目标。第二章讨论第一个问题。它研究了PF皮层的进化,包括一些导致灵长类动物进化过程中特定进展的选择压力。我们认为,颗粒状PF皮层的进化是分阶段发生的,这些阶段伴随着其他进步,如视觉和手功能的发展。早期的前辈们的眼睛是向前看的,他们采用了新的移动、抓取和用手喂嘴的方式。在这些动物中出现了第一个颗粒状的PF区域,即PF尾部皮层和颗粒状的OFC。背侧、腹侧和极侧的PF皮层在类人猿灵长类动物中出现得更晚,因为它们变得更大,不得不与它们喜欢的食物严重短缺作斗争,更不用说竞争和捕食的威胁了。
\par 
第三-七章讨论第二个目标。这些章节解释了灵长类PF皮层各主要区域的连接如何允许它做它所做的事情,以及为什么只有它才能执行它的功能。我们逐个区域检查PF皮层,因为它的连接因区域而异:
\par 
1.第三章指出了内侧PF皮层与海马体、杏仁核和内侧前运动区之间的联系,这使得它能够更好地使用“内部”信号来指导在行动和行动规则之间的觅食选择,包括评估它们与所涉及的努力成本相关的当前价值。
\par 
2.第四章强调,眶侧PF皮层的连接使其能够更好地利用外部信号来指导觅食选择,在物体之间进行选择。眶PF皮层与许多感觉区域有联系,包括视觉、躯体感觉、味觉、嗅觉和内脏皮层。这些连接允许眶PF皮层发展有关特定行为结果的高维信息连接,并强调其视觉特征。我们认为,基于单个事件,细粒度OFC将特定结果分配给似乎导致它的特定选择。与杏仁核的相互联系根据当前需求提供了更新的结果评估。
\par 
3.第五章重点介绍了注意力和搜索功能。尾侧PF皮层从视觉区域接收到的连接传递了来自低阶和高阶视觉的信号,包括背侧和腹侧视觉流。基于与这些区域的皮质皮质连接以及控制脑干动眼肌核的皮质投射,尾侧PF皮层可以将显性和隐性注意力引导到潜在的行动目标上。我们认为,尾侧PF皮层(第8区),包括额叶眼场(FEF),参与了对学习产生的目标的搜索——目标导向的注意力——而不是反射性或刺激驱动的注意力。我们不认为FEF是眼球运动区或运动前区,而是将其视为前额叶皮层的一部分:一个将注意力导向有学习价值的物体和地方的部分,包括隐蔽注意力和显性注意力(以眼球运动的形式)。这样,它加强了中央凹和中央凹外信息的处理。
\par 
4.第六章强调了由背侧PF皮层产生目标,部分基于与后顶叶皮层的连接。这些投影提供了有关视觉事件的顺序、时间和位置的信息,这些信息构成了当前行为环境的重要部分。目标的产生也依赖于眶PF皮层关于与这些目标及其当前值相关的结果的信息。前额叶皮层的背侧提供了一种机制,可以消除以前事件记忆引起的干扰,它似乎是通过前瞻性地编码当前目标来做到这一点的,至少在一定程度上是这样。与前发动机区域的连接为实现这些目标提供了一条途径。
\par 
5.第七章提出,腹侧前额叶皮层根据视觉或听觉环境产生目标。它之所以能发挥这一功能,是因为它与下颞叶和上颞叶皮层、眶前皮层和杏仁核有联系。这些联系为它提供了视觉和听觉信号、预测的结果,以及根据当前生物需求对该结果的估值。符号由基本特征和整体对象之间的特征结合的中间层次组成。腹侧前额叶皮层使用同样的连接来应用抽象的规则和策略,从而将以前的经验转移到新的行为问题上。
\par 
在以这种方式将PF皮层拆开后,第八章将其重新组装起来。它将PF皮层作为一个整体来研究,并最终实现了我们的第三个目标:一个关于灵长类PF皮层基本功能的具体建议,重点是在灵长类中进化的区域。我们提出,颗粒状PF皮层生成的目标适合于当前环境和当前需求,并且它可以在单个事件的基础上做到这一点。因此,类人猿可以根据一个或几个经验解决广泛的问题,并且可以避免祖先强化学习机制中固有的许多错误。我们提出,在灵长类动物历史上的特定时间和地点(第2章和第8章),为了应对特定的适应压力,粒状PF区域进化为实现一种新的通用学习机制,这一机制增强了祖先的通用学习系统。祖先系统通过加强反馈来调整关联的强度,从而控制自动行为;灵长类动物解决问题的方式包括对行为的注意控制,以及更少的错误。
\par 
第9章讨论我们的第四个目标:解释在人类认知过程中观察到的基本PF功能是如何解释大脑激活的。它强调了重新再现的力量。因此,人类的PF皮层可以重新表示知觉状态、他人的意图和精神状态以及关系之间的关系。第9章还表明,人类的PF皮层阐述了PF皮层在其他类人猿中的功能;就像猴子的颗粒状PF皮层允许它们通过快速学习和抽象策略来避免错误一样,人类的PF皮层允许我们通过指令、模仿和心理试错行为来避免错误。
\par 
本章的剩余部分将讨论我们的第五个也是最后一个目标:将我们对灵长类动物PF皮层的描述与其他文献进行比较,并提出一些测试方法。因此,接下来的两个部分解释了其他的建议要么缺乏我们提供的进化视角,只适用于灵长类PF皮层的一部分,没有解释为什么它独特的连接组合解释了它的功能,没有说明灵长类PF皮层执行了哪些其他大脑区域不能执行的功能,未能解释PF皮层贡献的广泛行为,或者未能产生可验证的假设。
\par 
我们将可供选择的建议分为两组,分别在不同的部分进行讨论:一组主要依赖猴子的结果,另一组主要依赖人类的证据。当然,前一组的支持者也试图将其推广到人类,后一组的支持者也提到了他们从猴子身上得到的证据来支持他们的观点。所以我们这样划分只是为了方便,有时,当一个理论同时涉及猴子和人类的研究时,我们会在两个部分讨论。
\par 
我们遵循Wood和Grafman(2003)在表格中列出各种理论,并试图根据一套标准来评估它们。我们使用了五个标准,这表明一个成功的PF皮层理论应该:
\par 
1.结合PF皮层的进化史,特别是在灵长类动物中颗粒状PF皮层的出现和在高级类人猿中新颗粒状区域的出现:历史测试。
\par 
2.解释为什么PF皮层的连接解剖使其有可能执行所提议的功能:解剖测试。
\par 
3.识别PF皮层的特定功能,与大脑的其他部分形成对比:特异性测试。
\par 
4.考虑可用数据的广泛范围:通用性检验。
\par 
5.精确到可以被可行的观察所检验:可证伪性检验。
\section{目的}

\section{定义和术语}


\section{指纹}

\subsection{损伤和激活}

\subsection{损伤和活动}

\subsection{活动和激活}




\subsection{结论}



\chapter{强化学习的心理学} \label{chap:chap11}

在前面的章节中,我们仅基于计算考虑就提出了算法的想法。
在本章中,我们从另一个角度来看其中的一些算法:心理学的角度及其对动物如何学习的研究。
本章的目的是,首先,讨论强化学习的思想和算法与心理学家对动物学习的发现相对应的方式,其次,解释强化学习对动物学习研究的影响。
事实证明,强化学习提供的清晰形式主义将任务,回报和算法系统化,在理解实验数据,提出新的实验种类以及指出可能对操作和测量至关重要的因素方面非常有用。
长期优化回报的想法是强化学习的核心,这有助于我们理解动物学习和行为的其他令人困惑的特征。


强化学习与心理学理论之间的一些对应关系并不令人惊讶,因为强化学习的发展受到了心理学学习理论的启发。
然而,正如本书所述,强化学习从人工智能研究人员或工程师的角度探索理想化的情况,目的是用有效的算法解决计算问题,而不是复制或详细解释动物如何学习。
因此,我们描述的一些通信将各自领域中独立产生的想法联系起来。
我们认为这些接触点特别有意义,因为它们揭示了对学习很重要的计算原理,无论是通过人工还是通过自然系统学习。


在大多数情况下,我们描述了强化学习和学习理论之间的对应关系,这些理论是为了解释大鼠,鸽子和兔子等动物如何在受控实验室实验中学习而开发的。
在整个20世纪进行了数千次这样的实验,其中许多至今仍在进行中。
虽然有时被认为与心理学中更广泛的问题无关,但这些实验探索了动物学习的微妙特性,通常是由精确的理论问题驱动的。
随着心理学将重点转移到行为的更多认知方面,即思维和推理等心理过程,动物学习实验在心理学中的作用比以前少了。
但是,这项实验导致了学习原则的发现,这些原则在整个动物界都是基本的和广泛的,这些原则在设计人工学习系统时不应该被忽视。
此外,正如我们将看到的那样,认知处理的某些方面自然地与强化学习提供的计算视角相关联。


本章的最后一节包括与我们讨论的联系以及我们忽视的联系相关的参考文献。
我们希望本章鼓励读者更深入地探讨所有这些联系。
最后一节还讨论了强化学习中使用的术语与心理学术语的关系。
强化学习中使用的许多术语和短语都是从动物学习理论中借用的,但是这些术语和短语的计算/工程意义并不总是与其在心理学中的意义一致。


\section{预测和控制}


我们在本书中描述的算法分为两大类:预测算法和控制算法。
这些类别自然出现在强化学习问题的解决方法中。
在许多方面,这些类别分别对应于心理学家广泛研究的学习类别:经典或巴甫洛夫式条件作用和工具性或操作性条件作用。
由于心理学对强化学习的影响,这些对应关系并不完全是偶然的,但它们仍然引人注目,因为它们将来自不同目标的想法联系起来。


本书中介绍的预测算法估计的数量取决于代理环境的特征在未来的发展情况。
我们特别关注于估计代理在与环境交互时未来可能获得的回报量。
在这个角色中,预测算法是策略评估算法,它是改进策略算法的组成部分。
但预测算法不仅限于预测未来的回报;他们可以预测环境的任何特征(例如,参见Modayil,White和Sutton,2014)。
预测算法和经典条件反射之间的对应关系取决于它们预测即将到来的刺激的共同特性,无论这些刺激是否有益(或惩罚)。


仪器或操作条件实验的情况是不同的。
在这里,实验装置的设置是为了根据动物的行为给予动物喜欢的东西(奖励)或不喜欢的东西(惩罚)。
动物学会增加产生奖励行为的倾向,减少产生惩罚行为的倾向。
据说强化刺激取决于动物的行为,而在经典条件反射中则不然(尽管在经典条件反射实验中很难消除所有行为偶然性)。
仪器调节实验就像我们在第一章中简要讨论的那些启发桑代克效应定律的实验。
控制是这种学习形式的核心,它对应于强化学习的策略改进算法的操作。


在预测方面思考经典条件反射,在控制方面思考工具条件反射,是将我们的强化学习的计算观点与动物学习联系起来的起点,但实际上,情况比这更复杂。
经典条件作用比预测更多;它还涉及行动,控制模式也是如此,有时被称为巴甫洛夫控制。
此外,经典和工具性条件反射以有趣的方式相互作用,这两种学习都可能在大多数实验情况下进行。
尽管存在这些复杂性,但将经典/工具区别与预测/控制区别相结合是将强化学习与动物学习联系起来的方便的rst近似。


在心理学中,强化一词用于描述经典条件反射和工具条件反射的学习。
最初只指强化一种行为模式,也经常用于弱化一种行为模式。
被认为是行为改变原因的刺激被称为增强剂,它是否取决于动物以前的行为。
在本章的末尾,我们将更详细地讨论这个术语,以及它与机器学习中使用的术语的关系。


\section{经典条件反射}

在研究消化系统的活动时,著名的俄罗斯生理学家伊万·巴甫洛夫(IvanPavlov)发现,动物对某些触发刺激的先天反应可能会被与先天触发因素无关的其他刺激触发。
他的实验对象是经过小手术的狗,以准确测量其唾液re-ex的强度。
在他描述的一个案例中,这只狗在大多数情况下都不会流涎,但在喂食后约5秒钟,它会在接下来的几秒钟内产生约6滴唾液。
在几次重复呈现另一种刺激后,一种与食物无关的刺激,在这种情况下是节拍器的声音,在引入食物之前不久,狗对节拍器的声音做出了唾液分泌的反应,就像它对食物的反应一样。
“因此,唾液腺的活动被声音的冲动所激发,这是一种与食物完全不同的刺激”(巴甫洛夫,1927,第22页)。
巴甫洛夫总结了这一发现的重要性,写道:


很明显,在自然条件下,正常动物不仅必须对自身带来直接好处或伤害的刺激作出反应,而且还必须对其他物理或化学机构(声波,光波等)作出反应,这些物理或化学机构本身只发出这些刺激的接近信号;
虽然猎物的视觉和声音本身对较小的动物有害,但它的牙齿和爪子有害。(巴甫洛夫,1927年,第14页)


以这种方式将新刺激与先天性反应联系起来,现在被称为经典或巴甫洛夫条件反射。
巴甫洛夫(或者更确切地说,他的翻译人员)将先天性反应(例如,上述演示中的流涎)称为“无条件反应”(URs),其自然触发刺激(例如食物)“无条件刺激”(USs),以及由预测刺激触发的新反应(例如,这里还有流涎)“条件反应”(CRs)。
最初是中性的刺激,意味着它通常不会引起强烈的反应(例如节拍器声音),当动物知道它预测美国并因此产生CR以响应CS时,它就会成为“条件刺激”(CS)。
这些术语仍然用于描述经典的条件反射实验(尽管更好的翻译应该是“有条件的”和“无条件的”,而不是有条件的和无条件的)。
美国被称为强化者,因为它强化了对CS产生CR的反应。


右侧显示了两种常见类型的经典调节实验中刺激的排列。
在延迟调节中,CS延伸到整个刺激间隔或ISI,这是CS发作和美国发作之间的时间间隔(当美国以此处显示的常见版本结束时,CS结束)。
在跟踪条件反射中,US在CS结束后开始,CS o集合和US开始之间的时间间隔称为跟踪间隔。


\begin{figure}[!htb]
	\centering
	\includegraphics[width=0.5\linewidth]{chap11/fig_11_0}
	\caption{  \label{fig:11_0}}
\end{figure}


巴甫洛夫的狗对节拍器的声音流涎只是经典条件反射的一个例子,已经在许多动物的许多反应系统中进行了深入研究。
URs通常在某种程度上是准备性的,比如巴甫洛夫的狗流涎,或者在某种程度上是保护性的,比如对刺激眼睛的东西眨眼,或者看到捕食者时冻结。
在一系列试验中经历CS-US预测关系会使动物了解到CS预测美国,因此动物可以通过CR对CS做出反应,为动物做好准备或保护其免受预测的美国的影响。
一些CR类似于UR,但开始得更早,并且以增加其有效性的方式开始。
例如,在一项深入研究的实验中,音调CS可靠地预测了兔子眼睛的空气pu(美国),触发了UR,该UR由称为切口膜的保护性内眼睑闭合组成。
在一次或多次试验后,音调开始触发由膜闭合组成的CR,该膜闭合在空气pu之前开始并最终定时,以便在空气pu可能发生时发生峰值闭合。
这种CR是在预期空气pu和适当时间的情况下启动的,比简单地启动关闭作为对刺激我们的反应更好。通过学习刺激之间的预测关系来预测重要事件的能力是如此有益,它广泛存在于动物界。





\subsection{阻塞和高阶条件反射}

在实验中已经观察到经典条件反射的许多有趣性质。
除了CRs的预期性质之外,在经典调节模型的发展中,两个广泛观察到的特性得到了显着体现:阻塞和高阶调节。
当一个潜在的CS与之前用于调节动物产生CR的另一个CS一起出现时,当动物无法学习CR时,就会发生阻塞。
例如,在涉及兔子切口膜调节的阻断实验的第一阶段,兔子首先用音调CS和空气pu US调节,以产生在预期空气pu的情况下关闭其切口膜的CR。
实验的第二阶段包括额外的试验,其中第二个刺激(例如光)被添加到音调中以形成复合音调/光CS,然后是相同的空气pu US。
在实验的第三阶段,仅第二个刺激(即光)被呈现给兔子,以查看兔子是否已经学会了用CR对其作出反应。
结果表明,兔子对光的反应产生了很少或没有CR:对光的学习已经被先前对音调的学习所阻止。
2这样的阻止结果挑战了这样的观点,即调节只依赖于简单的时间连续性,即必要和足够的条件条件反射是美国经常在时间上紧跟CS。
在下一节中,我们将描述Rescorla{Wagner模型(Rescorla和Wagner,1972),该模型为阻塞提供了全面的解释。
	


当先前条件化的CS充当US来调节另一个最初中性的刺激时,就会发生高阶条件化。
如上所述,巴甫洛夫描述了一个实验,在这个实验中,他的助手首先调节一只狗,让它随着节拍器的声音流涎,节拍器可以预测食物的味道。
在这个调节阶段之后,进行了许多试验,其中将狗最初独立的黑色方块放置在狗的视线中,然后是节拍器的声音,而不是食物。
在仅仅十次试验中,这只狗只在看到黑色方块时才开始流涎,尽管事实上,看到它之后从来没有食物。
节拍器的声音本身就像一个US,将流涎的CR调节为黑色方块CS。
这是二阶条件反射。如果黑色方块被用作美国来建立另一个中性CS的唾液CRs,那么它将是三阶条件反射,等等。高阶条件反射很难证明,特别是在二阶以上,部分原因是高阶增强剂由于在高阶条件反射试验中没有被原美国反复遵循而失去了增强价值。
但是在正确的条件下,例如将一阶试验与高阶试验混合或通过提供一般的激励刺激,可以证明二阶以上的高阶条件反射。
正如我们在下面所描述的,经典条件反射的TD模型使用了自举思想,这是我们的方法的核心,以扩展Rescorla{Wagner模型对阻塞的描述,以包括CRs的预期性质和高阶条件反射。
	
	
高阶仪器调节也会发生。
在这种情况下,持续预测初级强化的刺激本身就变成了强化剂,如果通过进化将其奖励或惩罚的品质建立在动物体内,则强化是主要的。
预测刺激成为二级增强剂,或者更一般地说,是高阶或条件性增强剂|当预测的增强刺激本身是二级或甚至更高阶的增强剂时,后者是更好的术语。
条件性强化剂提供条件性强化:条件性奖励或条件性惩罚。
条件性强化就像初级强化一样,增加动物产生导致条件性奖励的行为的倾向,并减少动物产生导致条件性惩罚的行为的倾向。
(请参阅本章末尾的评论,解释我们的术语有时与心理学中使用的术语有何不同。)



条件性强化是一个关键现象,它解释了为什么我们为条件性强化货币工作,而条件性强化货币的价值完全来自于拥有它所预测的东西。
在第13.5节描述的演员{批评家方法(并在第15.7节和第15.8节的神经科学背景下进行了讨论)中,批评家使用TD方法来评估演员的政策,其价值估计为演员提供了条件性强化,从而使演员能够改进其政策。
这种更高阶的工具性条件作用的类似物有助于解决第1.7节中提到的信贷分配问题,因为当主要奖励信号延迟时,评论家会对演员进行瞬间强化。
我们将在下面的第14节中对此进行更多讨论。四。


\subsection{雷斯科拉-瓦格纳模型}

Rescorla和Wagner创建模型主要是为了解释阻塞。
\textit{雷斯科拉-瓦格纳模型}的核心思想是,动物只有在事件违反其预期时才能学习,换句话说,只有当动物感到惊讶时(尽管不一定意味着任何有意识的期望或情绪)。
我们首先使用Rescorla和Wagner的术语和符号来呈现模型,然后再转向我们用来描述TD模型的术语和符号。


以下是Rescorla和Wagner如何描述他们的模型。
该模型调整化合物CS的每个成分刺激的“关联强度”,这是一个数字,表示该成分预测US的强度或可靠性。
当在经典调节试验中呈现由多个成分刺激组成的化合物CS时,每个成分刺激的关联强度的变化方式取决于与整个刺激化合物相关的关联强度,称为“聚合关联强度”,而不仅仅取决于每个成分本身的关联强度。


Rescorla和Wagner考虑了一种化合物CS AX,由成分刺激a和X组成,其中动物可能已经经历了刺激a,而刺激X可能是动物新的。设VA,VX和VAX分别表示刺激A,X和化合物AX的结合强度。
假设在试验中,化合物CS AX后面跟着一个US,我们将其标记为刺激Y。
然后刺激成分的关联强度根据以下表达式变化:

\begin{equation}
	\Delta V_A = \alpha_A \beta_Y
		(R_Y - V_{AX})
\end{equation}


\begin{equation}
	\Delta V_X = 
		\alpha_X \beta_Y
		(R_Y - V_{AX})
\end{equation}

其中A Y和X Y是步长参数,取决于CS分量和US的恒等式,Y是US Y可以支持的结合强度的渐近水平。
(Rescorla和Wagner在这里使用R代替R,但我们使用R是为了避免与我们使用and混淆,因为我们通常认为这是奖励信号的大小,但需要注意的是,经典条件反射中的美国不一定是奖励或惩罚的。)该模型的一个关键假设是总联想强度VAX等于VA+VX。
由这些s改变的关联强度在下一次试验开始时成为关联强度。


为了完成,该模型需要一种响应生成机制,这是一种将V s值映射到CR的方法。
因为这种映射将取决于实验情况的细节,所以Rescorla和Wagner没有指定映射,而是简单地假设较大的V s会产生更强或更可能的CRs,并且负V s意味着不会有CRs。


Rescorla{Wagner模型以一种解释阻塞的方式解释了CRs的获取。
只要刺激化合物的总结合强度VAX低于美国Y可以支持的结合强度RY的渐近水平,预测误差RY􀀀VAX为正。
这意味着在连续的试验中,成分刺激的关联强度VA和VX增加,直到总关联强度VAX等于RY,此时关联强度停止变化(除非美国改变)。当将新组分添加到动物已经调理过的化合物CS中时,用增强化合物进一步调理会使添加的CS组分的缔合强度几乎没有增加,或者没有增加,因为误差已经减小到零或低值。
美国的发生几乎已经被完美地预测到了,因此新的CS组件几乎没有引入错误或惊喜。先前的学习阻碍了对新组件的学习。


为了从Rescorla和Wagner的模型过渡到经典条件反射的TD模型(我们称之为TD模型),我们根据本书中使用的概念重新构建了他们的模型。
具体来说,我们将用于学习的符号与线性函数近似(第9.4节)相匹配,并且我们认为调节过程是在试验中基于该试验中提出的化合物CS“学习预测美国的大小”的过程之一,其中美国Y的大小是如上所述的Rescorla{Wagner模型的Y。
我们还引入了状态。因为Rescorla{Wagner模型是一个试验级模型,这意味着它处理了从试验到试验的联想强度如何变化,而不考虑试验内部和试验之间发生的任何细节,我们不必考虑在试验期间状态如何变化,直到我们在下一节。
相反,在这里,我们简单地将状态视为根据试验中存在的组件CSs的集合来标记试验的一种方式。


因此,假设试验类型或状态s由特征x(s)=(x1(s)的实值向量描述;x2(s);:::;xd(s))>如果试验中存在化合物CS的第i个组分CSi,则xi(s)=1,否则为0。
然后,如果结合强度的d维向量为w,则试验类型s的总结合强度为

\begin{equation}\label{key}
	v(s, \textbf{w}) = 
		w^T x(s).
\end{equation}
这对应于强化学习中的价值估计,我们认为它是美国的预测。

现在暂时让t表示完整试验的次数,而不是它作为时间步长的通常含义(当我们将其扩展到下面的TD模型时,我们恢复到t的通常含义),并假设St是对应于试验t的状态。
调节试验t将关联强度向量wt更新为wt+1,如下所示:

\begin{equation}\label{key}
	w_{t+1} = w_t + \alpha \delta_t x(S_t),
\end{equation}

步长参数在哪里,因为这里我们描述的是Rescorla{Wagner模型| t是预测误差
	
\begin{equation}\label{key}
	\delta = R_t - v (S_t, w_t).
\end{equation}


Rt是试验t预测的目标,即美国的规模,或者用Rescorla和Wagner的话说,是美国在试验中可以支持的联合强度。
请注意,由于(14.2)中的因子x(St),因此只有试验中存在的CS组件的关联强度才能作为该试验的结果进行调整。
你可以将预测误差视为惊喜的衡量标准,将总联想强度视为动物的期望,当它与美国的目标幅度不匹配时,就会被违反。


从机器学习的角度来看,Rescorla{Wagner模型是一种纠错监督学习规则。它本质上与最小均方(LMS)或Widrow-Ho学习规则(Widrow and Ho,1960)相同,该规则将权重(这里是关联强度),使所有误差的平方平均值尽可能接近零。
它是一种“曲线”或回归算法,广泛用于工程和科学应用(见第9.4节)。3


Rescorla{Wagner模型在动物学习理论的历史上是非常重要的,因为它表明一个“机械”理论可以解释关于阻塞的主要事实,而不需要诉诸更复杂的认知理论,例如,动物明确认识到另一个刺激成分已被添加,然后向后扫描其短期记忆,以重新评估涉及美国的预测关系。
Rescorla{Wagner模型显示了传统的连续性调节理论,即刺激的时间连续性是学习的必要和充分条件,可以通过一种简单的方式来调整,以解释阻塞(Moore和Schmajuk,2002年)2008年)。
	
	
Rescorla{Wagner模型提供了经典条件反射的阻塞和其他一些特征的简单说明,但它不是经典条件反射的完整或完美模型。
不同的想法解释了各种其他观察到的影响,并且在理解经典条件反射的许多微妙之处方面仍在取得进展。
我们接下来描述的TD模型虽然也不是经典条件反射的完整或完美模型,但它扩展了Rescorla{Wagner模型,以解决刺激之间的试验内和试验间的时间关系如何影响学习以及如何产生更高阶的条件反射。
	

\subsection{时间差分}

TD模型是一个实时模型,而不是像Rescorla{Wagner模型这样的试验级模型。在我们上面的Rescorla和Wagner模型的公式中,一个单步t代表了一个完整的条件反射试验。
该模型不适用于关于试验期间发生的事情或试验之间可能发生的事情的细节。
在每个试验中,动物可能会经历各种刺激,这些刺激的发作发生在特定的时间,并且具有特定的持续时间。
这些时间关系强烈影响学习。Rescorla{Wagner模型也不包括高阶条件反射的机制,而对于TD模型,高阶条件反射是TD算法基础上的自举思想。
	
	
	
为了描述TD模型,我们从上面的Rescorla{Wagner模型的公式开始,但t现在标记了试验内或试验之间的时间步长,而不是完整的试验。将t和t+1之间的时间想象为一个小的时间间隔,例如0.01秒,并将试验视为一系列状态,每个时间步长都有一个关联,其中步骤t的状态现在表示刺激在t处如何表示的细节,而不仅仅是试验中CS成分的标签。
事实上,我们可以完全放弃试验的想法。
从动物的角度来看,试验只是其与世界相互作用的持续经验的一部分。
遵循我们通常的观点一个与环境相互作用的主体,想象动物正在经历一系列无尽的状态s,每个状态s都由特征向量x(s)表示。也就是说,将试验称为实验中刺激模式重复的时间片段仍然很方便。


状态特征不限于描述动物经历的外部刺激;它们可以描述外部刺激在动物大脑中产生的神经活动模式,这些模式可能与历史有关,这意味着它们可能是由外部刺激序列产生的持续模式。
当然,我们不确切知道这些神经活动模式是什么,但是像TD模型这样的实时模型允许人们探索学习关于外部刺激的内部表征的不同假设的后果。
由于这些原因,TD模型不适用于任何特定的状态表示。
此外,由于TD模型包括跨越刺激之间时间间隔的折扣和资格痕迹,因此该模型还可以探索折扣和资格痕迹如何与刺激表征相互作用,以预测经典调节实验的结果。


下面我们描述了TD模型中使用的一些状态表示及其含义,但目前我们对表示不可知,只假设每个状态s由特征向量x(s)=(x1(s)表示;x2(s);:::;xn(s))>。
然后(14.1)给出了对应于状态s的总结合强度,与Rescorla-Wgner模型相同,但TD模型不同地更新了结合强度向量w。
t现在标记了一个时间步长而不是一个完整的试验,TD模型根据此更新来管理学习:

\begin{equation}\label{key}
	w_{t+1} = w_t + \alpha \delta_t z_t,
\end{equation}

它将Rescorla{Wagner update(14.2)中的xt(St)替换为合格跟踪向量zt,而不是(14.3)的t,这里t是TD错误:
	
	
\begin{equation}\label{key}
	\delta_t = 
		R_{t+1} + \gamma v (S_{t+1}, w_t) - v(S_t, w_t),
\end{equation}


其中是折扣因子(介于0和1之间),Rt是时间t的预测目标, v(St+1;wt)和 v(St;wt)是t+1和t的总结合强度,如(14.1)所定义的。


资格跟踪向量zt的每个分量i根据特征向量x(St)的分量xi(St)递增或递减,否则会以以下公式确定的速率衰减:

\begin{equation}\label{key}
	z_{t+1} = \gamma \lambda z_t 
		+ x(S_t).
\end{equation}

这是通常的资格跟踪衰减参数。

请注意,如果=0,TD模型将简化为Rescorla{Wagner模型,但以下情况除外:t的含义在每种情况下都是不同的(Rescorla{Wagner模型的试用号和TD模型的时间步长),并且在TD模型中,预测目标R中有一个时间步长领先。
TD模型相当于具有线性函数近似的半梯度TD()算法的后视图(第12章),除了模型中的Rt不必像TD算法用于学习用于政策改进的值函数时那样是奖励信号。




\chapter{强化学习的神经科学} \label{chap:chap12}


神经科学是对神经系统的多学科研究:它们如何调节身体功能;控制行为;随着时间的推移,由于发展、学习和衰老而发生的变化;以及细胞和分子机制如何使这些功能成为可能。
强化学习最令人兴奋的方面之一是来自神经科学的越来越多的证据,证明人类和许多其他动物的神经系统实现的算法与强化学习算法惊人地对应。
本章的主要目的是解释这些相似之处,以及它们对动物奖励相关学习的神经基础的建议。


强化学习和神经科学之间最显著的联系点涉及多巴胺,这是一种深入参与哺乳动物大脑奖励处理的化学物质。
多巴胺似乎会将\textit{时间差分}误差传递给进行学习和决策的大脑结构。
\textit{多巴胺神经元活动的奖赏预测误差假说}表达了这种平行性,该假说是由计算强化学习和神经科学实验结果的汇聚引起的。
在本章中,我们讨论了这一假设,导致这一假设的神经科学发现,以及为什么它对理解大脑奖励系统有重要贡献。
我们还讨论了强化学习和神经科学之间的相似之处,这些相似之处不如\textit{多巴胺}/\textit{时间差分误差}的相似之处引人注目,但为思考动物基于回报的学习提供了有用的概念工具。
强化学习的其他元素有可能影响神经系统的研究,但它们与神经科学的联系仍相对未开发。
我们讨论了其中几个不断发展的联系,我们认为这些联系将随着时间的推移而变得越来越重要。


正如强化学习的早期历史所概述的,强化学习的许多方面都受到了神经科学的影响。
本章的第二个目标是让读者了解对我们贡献的强化学习方法关于大脑功能的想法。
从大脑功能的理论来看,强化学习的一些元素更容易理解。
\textit{资格迹}是强化学习的基本机制之一,它起源于突触的一种推测性质,突触是神经细胞(神经元)相互交流的结构。


在本章中,我们没有深入研究动物基于奖励的学习背后的神经系统的巨大复杂性:本章太短,我们不是神经科学家。
我们没有试图描述,甚至没有命名,许多大脑结构和通路,或任何分子机制,被认为与这些过程有关。
我们也没有公正地对待那些与强化学习非常一致的假设和模型的替代品。
该领域的专家之间存在分歧并不奇怪。
我们只能一窥这个引人入胜、不断发展的故事。
不过,我们希望本章能让你相信,一个非常富有成效的渠道已经出现,它将强化学习及其理论基础与动物基于奖励的学习的神经科学联系起来。


许多优秀的出版物涵盖了强化学习和神经科学之间的联系,其中一些我们在本章的最后一节中引用。
我们的处理与大多数处理不同,因为我们假设熟悉强化学习,但我们不假设了解神经科学。
我们首先简要介绍神经科学的概念,这些概念对于基本理解接下来的内容是必要的。



\section{神经科学基础} \label{sec:neuroscience_basics}

一些关于神经系统的基本信息有助于我们理解本章的内容。
我们后面提到的术语都是斜体字。
如果你已经掌握了神经科学的基本知识,跳过这一节就不会有问题。


\textit{神经元}是神经系统的主要组成部分,是专门利用电信号和化学信号处理和传递信息的细胞。
它们有多种形式,但神经元通常有一个细胞体、\textit{树突}和一个\textit{轴突}。
树突是从细胞体分支以接收来自其他神经元的输入(或者在感觉神经元的情况下也接收外部信号)的结构。
神经元的轴突是将神经元的输出传递给其他神经元(或肌肉或腺体)的纤维。
神经元的输出由一系列称为\textit{动作电位}的电脉冲组成,这些电脉冲沿着轴突传播。
动作电位也被称为\textit{脉冲},据说神经元在产生脉冲时会重新启动。
在神经网络模型中,通常使用实数来表示神经元的\textit{激活率},即某个单位时间内脉冲的平均数量。


神经元的轴突可以广泛分支,从而使神经元的动作电位达到许多目标。
神经元轴突的分支结构称为神经元轴突轴。
因为动作电位的传导是一个活跃的过程,与保险丝的燃烧没有什么不同,当动作电位到达轴突分支点时,它会“点亮”所有传出分支上的动作电位(尽管传播到分支有时会失败)。
因此,具有大轴突轴的神经元的活动可以影响许多靶位点。


\textit{突触}是一种通常位于轴突分支末端的结构,它介导一个神经元与另一个神经元的通信。
突触将信息从突触前神经元的轴突传递到突触后神经元的树突或细胞体。
除了少数例外,突触在突触前神经元的动作电位到达时会释放一种化学\textit{神经递质}。
(神经元之间直接电耦合的情况除外,但这里我们不关心这些。)
从突触突触前侧释放的神经递质分子扩散穿过突触间隙,即突触前末端和突触后神经元之间的非常小的空间,然后与突触后神经元表面的受体结合,以刺激或抑制其刺突生成活动,或以其他方式调节其行为。
一种特定的神经递质可能与几种不同类型的受体结合,每种受体对突触后神经元产生不同的影响。
例如,神经递质多巴胺至少有五种不同的受体类型可以影响突触后神经元。
许多不同的化学物质已被确定为动物神经系统中的神经递质。


当神经元似乎不受与实验者感兴趣的任务相关的突触输入的驱动时,例如,当神经元的活动与作为实验的一部分传递给受试者的刺激不相关时,神经元的背景活动是其活动水平,通常是其环速。
由于来自更宽网络的输入,或者由于神经元或其突触内的噪声,背景活动可能是不规则的。
有时,背景活动是神经元固有的动态过程的结果。
与背景活动相比,神经元的阶段性活动由通常由突触输入引起的脉冲活动爆发组成。
缓慢且经常以分级方式变化的活动,无论是否作为背景活动,都被称为神经元的\textit{强直性}活动。


突触释放的神经递质影响突触后神经元的强度或有效性是突触的有效性。
神经系统可以通过经验改变的一种方式是通过突触前和突触后神经元活动组合引起的突触能力的改变,有时还通过神经调节剂的存在,神经调节剂是一种具有除直接快速兴奋或抑制之外或除直接快速激励或抑制之外的作用的神经递质。


大脑包含几个不同的神经调控系统,由具有广泛分支轴突轴的神经元簇组成,每个系统使用不同的神经递质。
神经调控可以改变神经回路的功能,介导动机、唤醒、注意力、记忆、情绪、情绪、睡眠和体温。
重要的是,神经调节系统可以分布标量信号,如强化信号,以改变对学习至关重要的广泛分布部位突触的操作。


突触效应改变的能力称为\textit{突触可塑性}。
它是负责学习的主要机制之一。
通过学习算法调整的参数或权重对应于突触。
正如我们下面详细介绍的那样,通过神经调节剂多巴胺调节突触可塑性是大脑如何实现本书中描述的许多学习算法的一种合理机制。


\section{奖励信号、强化信号、价值和预测误差} \label{sec:reward_signals}

神经科学和计算强化学习之间的联系始于大脑中的信号与在强化学习理论和算法中发挥重要作用的信号之间的相似之处。
在有限马尔可夫决策过程中,任何学习目标导向行为的问题都可以归结为代表动作、状态和奖励的三个信号。
然而,为了解释神经科学和强化学习之间的联系,我们必须不那么抽象,而是考虑在某些方面与大脑中的信号相对应的其他强化学习信号。
除了奖励信号之外,这些信号还包括强化信号(我们认为这与奖励信号不同)、价值信号和传达预测误差的信号。
当我们以这种方式通过信号的函数来标记信号时,我们是在强化学习理论的背景下进行的,在该理论中,信号对应于方程或算法中的一个项。
另一方面,当我们提到大脑中的信号时,我们指的是一种生理事件,如动作电位的爆发或神经递质的分泌。
用神经信号的功能来标记神经信号,例如将多巴胺神经元的阶段性活动称为强化信号,意味着神经信号的行为类似于相应的理论信号,并被推测为其功能类似。



揭示这些对应关系的证据涉及许多挑战。
与奖励处理相关的神经活动几乎可以在大脑的每一个部位找到,并且很难明确地解释结果,因为不同的奖励相关信号的表示往往彼此高度相关。
实验需要仔细设计,以允许一种类型的奖励相关信号以任何程度的确定性与其他信号或与奖励处理无关的大量其他信号区分开来。
尽管存在这些分歧,但已经进行了许多实验,目的是将强化学习理论和算法的各个方面与神经信号相协调,并建立了一些令人信服的联系。
为了准备研究这些链接,在本节的其余部分中,我们提醒读者根据强化学习理论,各种与奖励相关的信号意味着什么。


在上一章末尾的术语评论中,我们说过 $ R_t $ 就像动物大脑中的奖励信号,而不是动物环境中的物体或事件。
在强化学习中,奖励信号(以及智能体的环境)定义了强化学习智能体试图解决的问题。
从这方面来说,$ R_t $ 就像动物大脑中的信号,将主要奖励分配到整个大脑的各个部位。
但动物大脑中不太可能存在像 $ R_t $ 这样单一的主奖励信号。
最好将 $ R_t $ 视为一个抽象概念,它总结了大脑中许多系统生成的大量神经信号的总体影响,这些系统评估感觉和状态的奖励或惩罚质量。


强化学习中的\textit{强化信号}与奖励信号不同。
强化信号的功能是指导学习算法在智能体的策略、价值估计或环境模型中做出的改变。
例如,对于时间差分方法,时间 $ t $ 时的强化信号是时间差分误差 $ \delta_{t-1} = R_t + \gamma V(S_t) - V(S_{t-1})$。
某些算法的强化信号可能只是奖励信号,但对于大多数算法,我们认为强化信号是通过其他信息调整的奖励信号,例如 时间差分误差中的值估计。


状态值或动作值(即 $ V $ 或 $ Q $)的估计指定从长远来看对智能体来说是好是坏。
它们是对智能体未来预期累积的总奖励的预测。
智能体通过选择导致具有最大估计状态值的状态的动作,或者通过选择具有最大估计动作值的动作来做出好的决策。


预测误差衡量预期信号或感觉与实际信号或感觉之间的差异。
\textit{奖励预测误差}专门衡量预期和收到的奖励信号之间的差异,当奖励信号大于预期时为正,否则为负。 
时间差分误差是特殊类型的\textit{奖励预测误差},它表明当前和早期的长期奖励预期之间存在差异。
当神经科学家提到\textit{奖励预测误差}时,他们通常(尽管并非总是)指的是时间差分 \textit{奖励预测误差},我们在本章中简称为时间差分误差。
同样在本章中,时间差分误差通常是不依赖于动作的误差,这与 Sarsa 和 Q-学习等算法在学习动作值时使用的时间差分误差相反。
这是因为最著名的神经科学链接是用无动作时间差分误差来表述的,但我们并不意味着排除涉及依赖动作的时间差分误差的可能的类似链接。
(时间差分误差对于预测奖励以外的信号也很有用,但我们在这里不关心这种情况\cite{modayil2014prediction}。)


人们可以就神经科学数据和这些理论上定义的信号之间的联系提出许多问题。
观察到的信号是否更像是奖励信号、价值信号、预测误差、强化信号,还是完全不同的东西?
如果它是一个误差信号,它是奖励预测误差、时间差分误差还是像\textit{雷斯科拉-瓦格纳模型}误差这样的更简单的误差?
如果是时间差分误差,它是否取决于 Q-学习或 Sarsa 的 时间差分误差之类的操作?
如上所述,探测大脑来回答此类问题是极其困难的。
但实验证据表明,一种神经递质,特别是神经递质多巴胺,向奖励预测误差发出信号,并且进一步,产生多巴胺的神经元的阶段性活动实际上传达了时间差分误差(有关阶段性活动的定义,请参见第~\ref{sec:neuroscience_basics}~节)。
这一证据导致了\textit{多巴胺神经元活动的奖励预测误差假设},我们接下来将对此进行描述。


\section{奖励预测误差假设}

多巴胺神经元活动的奖励预测误差假说提出,哺乳动物中产生多巴胺的神经元的阶段性活动的功能之一是将预期未来奖励的旧估计和新估计之间的误差传递给整个大脑的目标区域。
\textit{蒙太古}\cite{montague1996framework}首先明确提出了这一假设(尽管没有用这些确切的措辞),他们展示了强化学习中的时间差分误差概念如何解释哺乳动物多巴胺神经元阶段性活动的许多特征。
得出这一假设的实验是在 20 世纪 80 年代和 90 年代初在神经科学家\textit{沃尔夫勒姆$\cdot$舒尔茨}的实验室进行的。
第~\ref{sec:experimental_support}~节描述了这些有影响力的实验,第~\ref{sec:td_dopamine}~节解释了这些实验的结果如何与时间差分误差保持一致,本章末尾的参考文献和历史评论部分包括围绕这一有影响力的假设的发展的文献指南。


\textit{蒙太古}\cite{montague1996framework}等人将经典条件反射的时间差分模型的时间差分误差与经典条件反射实验中产生多巴胺的神经元的相位活动进行了比较。
回想一下~\ref{sec:classical_conditioning}~节,经典调节的时间差分模型基本上是具有线性函数近似的半梯度下降 TD($ \lambda $) 算法。
\textit{蒙太古}等人做出了几个假设来进行这种比较。
首先,由于时间差分误差可以是负的,但神经元不能有负的激活率,他们假设多巴胺神经元活动对应的量是 $ \delta_{t-1} + b_t $,其中$ b_t $是神经元的背景激活率。
负时间差分误差对应于多巴胺神经元的激活率下降到其背景速率以下。



需要第二个假设,关于每个经典条件反射试验中访问的状态以及它们如何表示为学习算法的输入。
这与我们在~\ref{sec:td_simulation}~节中针对时间差分模型讨论的问题相同。 
\textit{蒙太古}等人选择了\textit{全串行复合}表示,如图~\ref{fig:11_1}~左栏所示,但短期内部信号序列持续到\textit{非条件刺激}开始,这里是非零奖励信号的到来。
这种表示允许时间差分误差模拟这样一个事实:多巴胺神经元活动不仅预测未来的奖励,而且对奖励预计到达的预测提示之后的时间也很敏感。
必须有某种方法来跟踪感官提示和奖励到来之间的时间。
如果刺激启动一系列内部信号,并在刺激结束后继续,并且如果刺激后的每个时间步都有不同的信号,则刺激后的每个时间步都由不同的状态表示。
因此,时间差分误差与状态相关,可能对试验中事件的时间安排很敏感。


在使用有关背景激活率和输入表示的这些假设的模拟试验中,时间差分模型的时间差分误差与多巴胺神经元相位活动非常相似。
预览我们在下面第~\ref{sec:experimental_support}~节中对这些相似性的详细描述,时间差分误差与多巴胺神经元活动的以下特征相似:
1)多巴胺神经元的阶段性响应仅在奖励事件不可预测时发生;
2)在学习早期,奖励之前的中性线索不会引起实质性的阶段性多巴胺响应,但随着学习的继续,这些线索获得预测价值并引发阶段性多巴胺响应;
3)如果更早的提示可靠地先于已经获得预测值的提示,则阶段性多巴胺响应会转移到较早的提示,而停止等待较晚的提示;
3)如果在学习之后,预测的奖励事件被忽略,多巴胺神经元的响应在奖励事件的预期时间之后不久就会降低到其基线水平以下。


尽管\textit{舒尔茨}及其同事的实验中监测到的每个多巴胺神经元并非都以所有这些方式表现,但大多数监测神经元的活动与时间差分误差之间的惊人对应关系为奖励预测误差假说提供了强有力的支持。
然而,在某些情况下,基于假设的预测与实验中观察到的结果并不相符。
输入表示的选择对于时间差分误差与多巴胺神经元活动的一些细节的匹配程度至关重要,特别是有关多巴胺神经元响应时间的细节。
关于时间差分学习的输入表示和其他特征,人们提出了不同的想法,其中一些我们在下面讨论,以使时间差分误差更好地处理数据,尽管主要的相似之处与\textit{蒙太古}等人用过的\textit{全串行复合}表示相似。
总体而言,奖励预测误差假说已得到研究基于奖励的学习的神经科学家的广泛接受,并且事实证明,面对神经科学实验不断积累的结果,它具有显著的适应性。


为了准备我们对支持奖励预测误差假说的神经科学实验的描述,并提供一些背景以便可以理解该假说的重要性,我们接下来介绍一些关于多巴胺的已知知识,以及它影响的大脑结构,以及它如何参与基于奖励的学习。


\section{多巴胺} \label{sec:dopamine}

多巴胺作为神经递质由神经元产生,其细胞体主要位于哺乳动物\textit{中脑}的两个神经元簇:\textit{黑质致密部}和\textit{腹侧被盖区}。
多巴胺在哺乳动物大脑的许多过程中发挥着重要作用。
其中最突出的是动机、学习、行为选择、大多数形式的成瘾以及精神分裂症和帕金森病。
多巴胺被称为神经调节剂,因为它除了直接快速兴奋或抑制目标神经元之外还执行许多功能。
尽管关于多巴胺的功能及其细胞效应的细节仍有很多未知之处,但很明显,它对于哺乳动物大脑中的奖励处理至关重要。
多巴胺并不是唯一参与奖励处理的神经调节剂,它在厌恶情况(惩罚)中的作用仍然存在争议。
多巴胺在非哺乳动物中也可以发挥不同的作用。
但没有人怀疑多巴胺对于哺乳动物(包括人类)的奖励相关过程至关重要。


早期的传统观点认为,多巴胺神经元向与学习和动机有关的多个大脑区域广播奖励信号。
这一观点源自\textit{詹姆斯$\cdot$奥尔兹} 1954 年发表的一篇著名论文,该论文描述了电刺激对大鼠大脑某些区域的影响。
他们发现,对特定区域的电刺激在控制老鼠的行为方面起到了非常强大的奖励作用:“……通过这种奖励对动物行为的控制是极端的,可能超过以前在动物实验中使用的任何其他奖励\cite{olds1954positive}。”
后来的研究表明,刺激最有效地产生这种奖赏效应的位点会直接或间接地兴奋多巴胺通路,而这些通路通常会受到自然奖赏刺激的兴奋。
在人类受试者中也观察到了与大鼠相似的效应。
这些观察结果强烈表明多巴胺神经元活动发出奖励信号。


但是,如果奖励预测误差假设是正确的(即使它只解释了多巴胺神经元活动的某些特征),这种多巴胺神经元活动的传统观点并不完全正确:
多巴胺神经元的阶段性响应发出奖励预测误差的信号,而不是奖励本身。
用强化学习的术语来说,多巴胺神经元在时间 $ t $ 的相位响应对应于 $ \delta_{t-1} = R_t + \gamma V(S_t) - V(S_{t-1}) $,而不是 $ R_t $。


强化学习理论和算法有助于协调奖励预测误差观点与多巴胺信号奖励的传统观念。
在我们在本书中讨论的许多算法中,它起着强化信号的作用,这意味着它是学习的主要驱动力。
例如,$ \delta $是经典条件反射时间差分模型中的关键因素,是\textit{行动者-评论家}架构中的价值函数和策略的强化信号(第~\ref{sec:neural_ac}~节)。
动作依赖形式是 Q 学习和 Sarsa 的强化信号。
奖励信号 $ R_t $ 是 $ \delta_{t-1} $ 的关键组成部分,但它并不是其在这些算法中的强化效果的完全决定因素。
附加项 $ \gamma V(S_t) - V(S_{t-1}) $ 是 $ \delta_{t-1} $ 的高阶强化部分,即使奖励发生($ R_t \neq 0 $),如果奖励完全预测,时间差分误差也可以保持沉默(在下面第~\ref{sec:td_dopamine}~节中有完整解释)。


事实上,仔细观察\textit{奥尔兹} 1954 年的论文就会发现,它主要是关于电刺激在\textit{工具性条件反射}任务中的强化作用。
电刺激不仅通过多巴胺对动机的影响激发了老鼠的行为,还使老鼠很快学会了通过按下杠杆来刺激自己,而它们会经常长时间这样做。
电刺激触发的多巴胺神经元的活动增强了大鼠的杠杆按压。


最近使用光遗传学方法的实验证实了多巴胺神经元的相位响应作为强化信号的作用。
这些方法使神经科学家能够在清醒行为动物中以毫秒为单位精确控制所选神经元类型的活动。
光遗传学方法将光敏蛋白引入选定的神经元类型中,以便可以通过激光的灰烬激活或沉默这些神经元。
第一个使用光遗传学方法研究多巴胺神经元的实验表明,在小鼠体内产生多巴胺神经元阶段性激活的光遗传学刺激足以使小鼠更喜欢接受刺激的室的一侧,而不是接受刺激的室的另一侧 无刺激或频率较低的刺激\cite{tsai2009phasic}。
在另一个例子中\cite{steinberg2013causal}使用多巴胺神经元的光遗传学激活,在预期奖励刺激的时间在大鼠中产生多巴胺神经元活动的人工爆发,但在多巴胺神经元活动正常暂停时却忽略了奖励刺激。
当这些停顿被人工爆发所取代时,当响应通常由于缺乏强化而减少时(在灭绝试验中),响应就会持续,而当响应通常由于已经预测到奖励而被阻止时,学习就会被启用(阻塞范式; 第~\ref{sec:blocking_higher_order}~节)。


多巴胺增强功能的额外证据来自于水果的光遗传学实验,但在这些动物中,多巴胺的作用与其在哺乳动物中的作用相反:光触发的多巴胺神经元活动爆发就像电击足部一样增强回避行为 ,至少对于激活的多巴胺神经元群体而言\cite{claridge2009writing}。
尽管这些光遗传学实验都没有表明相位多巴胺神经元活动特别像时间差分误差,但它们令人信服地证明了相位多巴胺神经元活动就像算法中的强化信号一样(或者可能像水果中的负行为) 用于预测(经典条件反射)和控制(\textit{工具性条件反射})。


多巴胺神经元特别适合向大脑的许多区域广播强化信号。
这些神经元具有巨大的轴突乔木,每个神经元释放的多巴胺的突触位点比典型神经元的轴突多出 100 至 1,000 倍。
图~\ref{fig:12_1}~显示了单个多巴胺神经元的轴突,其细胞体位于大鼠大脑的\textit{黑质致密部}中。
\textit{黑质致密部}或\textit{中脑腹侧被盖区}多巴胺神经元的每个轴突在目标大脑区域的神经元树突上产生大约 500,000 个突触接触。


\begin{figure}[!htb]
	\centering
	\includegraphics[width=0.5\linewidth]{chap12/fig_12_1}
	\caption{产生多巴胺作为神经递质的单个神经元的轴突乔木,其细胞体位于大鼠大脑的\textit{黑质致密部}中。
		这些轴突与目标大脑区域中的大量神经元树突进行突触接触。 \label{fig:12_1}}
\end{figure}


如果多巴胺神经元像强化学习那样广播强化信号,那么由于这是一个标量信号,即单个数字,因此\textit{黑质致密部}和\textit{中脑腹侧被盖区}中的所有多巴胺神经元将被期望或多或少相同地激活,以便它们近乎同步地行动,向轴突目标的所有部位发送相同的信号。 
尽管人们普遍认为多巴胺神经元确实像这样一起行动,但现代证据指出了更复杂的情况,即不同的多巴胺神经元亚群对输入的响应不同,具体取决于它们发送信号的结构和不同的结构。
这些信号以不同的方式作用于其目标结构。
多巴胺除了发出\textit{奖励预测误差}信号之外还有其他功能,即使对于发送\textit{奖励预测误差}信号的多巴胺神经元来说,根据这些结构在产生强化行为中所扮演的角色,将不同的\textit{奖励预测误差}发送到不同的结构也是有意义的。
这超出了我们在本书中详细讨论的范围,但从强化学习的角度来看,向量值\textit{奖励预测误差}信号是有意义的,因为决策可以分解为单独的子决策,或者更一般地说,作为解决问题的结构版本的一种方法。
信用分配问题:如何在可能参与制定决策的许多组成结构中分配决策成功的功劳(或失败的责任)?
我们在下面的~\ref{sec:collective_rl}~节中对此进行了更多讨论。


% The axons of most dopamine neurons
大多数多巴胺神经元的轴突与额叶皮层和基底神经节中的神经元进行突触接触,这些区域是大脑中参与\textit{自主运动}、决策、学习和认知功能(例如计划)的区域。
由于大多数将多巴胺与强化学习相关的想法都集中在基底神经节上,并且多巴胺神经元的连接在那里特别密集,因此我们在这里重点关注基底神经节。
基底神经节是位于前脑底部的神经元群或细胞核的集合。 基底神经节的主要输入结构称为纹状体。
基本上所有的大脑皮层以及其他结构都向纹状体提供输入。
皮层神经元的活动传达了有关感觉输入、内部状态和运动活动的大量信息。
皮层神经元的轴突与纹状体主要输入/输出神经元的树突进行突触接触,称为中棘神经元。
纹状体的输出通过其他基底神经节核和丘脑循环回皮层的额叶区域和运动区域,使纹状体能够影响运动、抽象决策过程和奖励处理。
纹状体的两个主要细分对于强化学习很重要:背侧纹状体,主要参与影响动作选择;
腹侧纹状体,被认为对奖励处理的不同方面至关重要,包括将情感价值分配给感觉。



\textit{中型多棘神经元}的树突上覆盖着棘,皮层神经元的轴突在其尖端进行突触接触。
与这些棘进行突触接触的还有多巴胺神经元的轴突(图~\ref{fig:12_2})。
这种排列汇集了皮层神经元的突触前活动、\textit{中型多棘神经元}的突触后活动以及多巴胺神经元的输入。
这些刺上实际发生的事情很复杂并且尚未完全了解。
图~\ref{fig:12_2}~通过显示两种类型的多巴胺受体、谷氨酸受体(皮层输入的神经递质)以及各种信号相互作用的多种方式暗示了复杂性。
但越来越多的证据表明,从皮层到纹状体通路上的突触(神经科学家称之为皮层纹状体突触)的效率变化很大程度上取决于适当定时的多巴胺信号。



\begin{figure}[!htb]
	\centering
	\includegraphics[width=0.7\linewidth]{chap12/fig_12_2}
	\caption{纹状体神经元的脊柱显示来自皮层和多巴胺神经元的输入。
		皮层神经元的轴突通过皮层纹状体突触影响纹状体神经元,在覆盖纹状体神经元树突的棘尖释放神经递质谷氨酸。
		显示\textit{腹侧被盖区}或\textit{黑质致密部}多巴胺神经元的轴突经过脊柱(从右下)。
		该轴突上的“多巴胺静脉曲张”在脊柱干处或附近释放多巴胺,其排列将皮层的突触前输入、纹状体神经元的突触后活动和多巴胺结合在一起,使得几种类型的学习规则控制可塑性成为可能。
		多巴胺神经元的每个轴突与大约 500,000 个棘突的茎部进行突触接触,这里通过其他神经递质途径和多种受体类型(例如多巴胺通过的 D1 和 D2 多巴胺受体)显示了我们讨论中省略的一些复杂性\cite{schultz1998predictive}。
		\label{fig:12_2}}
\end{figure}



\section{奖励预测误差假设的实验支持} \label{sec:experimental_support}

多巴胺神经元对强烈的、新颖的或意想不到的视觉和听觉刺激做出响应,这些刺激会引发眼睛和身体的运动,但它们的活动很少与运动本身相关。
这是令人惊讶的,因为多巴胺神经元的退化是帕金森病的一个原因,帕金森病的症状包括运动障碍,特别是自发运动的缺陷。
由于多巴胺神经元活动与刺激触发的眼睛和身体运动之间的微弱关系,\textit{舒尔茨}\cite{romo1990dopamine,schultz1990dopamine}通过记录猴子移动手臂时多巴胺神经元的活动和肌肉活动,朝着奖励预测误差假说迈出了第一步。


他们训练两只猴子,当猴子看到并听到箱子的门打开时,将手伸入装有一些苹果、一块饼干或葡萄干的箱子。
然后猴子就可以抓住食物并将其送到嘴里。
当猴子擅长这项工作后,它又接受了另外两项任务的训练。
第一项任务的目的是观察当自发运动时多巴胺神经元会做什么。
箱子是开着的,但从上面盖住了,这样猴子就看不到里面的东西,但可以从下面伸手进去。
没有出现任何触发刺激,在猴子伸手去拿食物并吃掉后,实验者通常(尽管并非总是如此)在猴子看不见的情况下,将食物粘在一根坚硬的金属丝上,以取代箱子中的食物。
在这里,\textit{舒尔茨}监测到的多巴胺神经元的活动与猴子的运动无关,但每当猴子第一次接触食物时,这些神经元中的很大一部分都会产生阶段性响应。
当猴子只触摸电线或在没有食物的情况下探索箱子时,这些神经元不会做出响应。
这是神经元对食物而不是任务的其他方面做出响应的有力证据。


\textit{舒尔茨}的第二个任务的目的是看看当刺激触发运动时会发生什么。
这项任务使用了一个带有可移动盖子的不同箱子。
箱子打开的景象和声音引发了向箱子伸出手的动作。
在这种情况下,\textit{舒尔茨}发现,经过一段时间的训练,多巴胺神经元不再对食物的触摸做出响应,而是对食物箱盖打开的视觉和声音做出响应。
这些神经元的阶段性响应已经从奖励本身转变为预测奖励可用性的刺激。
在后续研究中,\textit{舒尔茨}发现,他们监测的大多数多巴胺神经元的活动在行为任务背景之外对箱子打开的景象和声音没有响应。
这些观察结果表明,多巴胺神经元既不响应运动的启动,也不响应刺激的感觉特性,而是发出奖励期望的信号。


\textit{舒尔茨}的小组进行了许多涉及\textit{黑质致密部}和\textit{中脑腹侧被盖区}多巴胺神经元的其他研究。
一系列特定的实验具有重要意义,表明多巴胺神经元的相位响应对应于\textit{时间差分}误差,而不是像\textit{雷斯科拉-瓦格纳模型}(\ref{sec:instrumental_conditioning}) 中的那些更简单的误差。
在第一个实验中\cite{ljungberg1992responses},猴子被训练在灯光亮起后按下杠杆作为“触发提示”,以获得一滴苹果汁。
正如罗莫和舒尔茨之前观察到的那样,许多多巴胺神经元最初对奖励--果汁滴做出响应(图~\ref{fig:12_3},上图)。
但随着训练的继续,许多神经元失去了奖励响应,并产生了对预测奖励的光照射的响应(图~\ref{fig:12_3},中图)。
随着持续的训练,按下杠杆的速度变得更快,而对触发信号做出响应的多巴胺神经元的数量却减少了。


\begin{figure}[!htb]
	\centering
	\includegraphics[width=0.55\linewidth]{chap12/fig_12_3}
	\caption{多巴胺神经元的响应从最初对初级奖励的响应转变为早期的预测刺激。
		这些是受监测的多巴胺神经元在小时间间隔内产生的动作电位数量的图,是所有受监测的多巴胺神经元(这些数据的范围从 23 到 44 个神经元)的平均值。
		上图:多巴胺神经元被意外输送的一滴苹果汁激活。
		中图:通过学习,多巴胺神经元对奖励预测触发线索产生响应,并失去对奖励传递的响应。
		下图:通过在触发提示之前添加 1 秒的指令提示,多巴胺神经元将其响应从触发提示转移到更早的指令提示\cite{schultz1994reward}。
		\label{fig:12_3}}
\end{figure}



在这项研究之后,同样的猴子接受了一项新任务的训练\cite{schultz1993responses}。
在这里,猴子面对着两个杠杆,每个杠杆上方都有一盏灯。
点亮其中一个灯是一个“指令提示”,指示两个杠杆中的哪一个会产生一滴苹果汁。
在此任务中,指令提示在前一个任务的触发提示之前有 1 秒的固定间隔。
猴子学会了在看到触发提示之前不伸手,多巴胺神经元活动增加,但现在受监测的多巴胺神经元的响应几乎完全发生在较早的指令提示上,而不是触发提示上(图~\ref{fig:12_3},底部面板)。
当任务被很好地学习时,对指令线索做出响应的多巴胺神经元的数量再次大大减少。
在学习这些任务的过程中,多巴胺神经元活动从最初对奖励的响应转变为对早期预测刺激的响应,首先发展到触发刺激,然后发展到更早的指令提示。
随着响应时间提前,它会从后来的刺激中消失。
这种对较早奖励预测变量的响应的转变,同时失去对较晚预测变量的响应是\textit{时间差分}学习的一个标志(例如,参见图~\ref{fig:12_4})。


\begin{figure}[!htb]
	\centering
	\includegraphics[width=0.45\linewidth]{chap12/fig_12_4}
	\caption{在预期奖励未能发生后不久,多巴胺神经元的响应就会降至基线以下。
		上图:一滴苹果汁的意外输送会激活多巴胺神经元。
		中:多巴胺神经元对预测奖励的\textit{条件刺激}做出响应,但不对奖励本身做出响应。
		下图:当\textit{条件刺激}预测的奖励未能发生时,多巴胺神经元的活动在预期奖励发生后不久就会降至基线以下。
		每个面板的顶部显示了受监测的多巴胺神经元在指定时间附近的小时间间隔内产生的动作电位的平均数量。
		下面的光栅图显示了受监测的单个多巴胺神经元的活动模式;
		每个点代表一个动作电位\cite{schultz1997neural}。
		\label{fig:12_4}}
\end{figure}


刚刚描述的任务揭示了多巴胺神经元活动与\textit{时间差分}学习共有的另一个特性。
猴子有时会按错键,即按指示以外的键,因此得不到任何奖励。
在这些试验中,许多多巴胺神经元在奖励的通常传递时间后不久表现出其环率急剧下降至基线以下,并且这种情况发生时没有任何外部线索来标记通常的奖励传递时间(图~\ref{fig:12_4}) 。
不知何故,猴子在内部记录了奖励的时间。
(响应时间是需要修改最简单版本的\textit{时间差分}学习的一个领域,以考虑多巴胺神经元响应时间的一些细节。
我们将在下一节中考虑这个问题。)


根据上述研究的观察结果,\textit{舒尔茨}和他的团队得出结论,多巴胺神经元对未预测的奖励和最早的奖励预测因子有响应,如果奖励或奖励预测因子没有在预期时间出现,多巴胺神经元的活动就会下降到基线以下。
熟悉强化学习的研究人员很快认识到,这些结果与\textit{时间差分}算法中\textit{时间差分}误差作为强化信号的表现惊人地相似。
下一节将通过一个具体示例详细探讨这种相似性。


\section{时间差分误差/对应的多巴胺} \label{sec:td_dopamine}

本节解释\textit{时间差分}误差与刚才描述的实验中观察到的多巴胺神经元的相位响应之间的对应关系。
我们研究了在类似上述任务的学习过程中如何变化,其中猴子首先看到指令提示,然后在固定的时间内必须正确响应触发提示才能获得奖励。
我们使用此任务的简单理想化版本,但我们比平常更详细,因为我们想强调\textit{时间差分}误差和多巴胺神经元活动之间并行的理论基础。


第一个简化假设是智能体已经学会了获得奖励所需的操作。
那么它的任务就是学习对其经历的状态序列的未来奖励的准确预测。
这是一个预测任务,或者更技术地说,是一个策略评估任务:学习固定策略的价值函数。
要学习的价值函数为每个状态分配一个值,该值预测如果智能体根据给定策略选择操作,该状态将遵循的回报,其中回报是所有未来奖励的(可能是折扣的)总和。
作为猴子情况的模型,这是不现实的,因为猴子可能会在学习正确行动的同时学习这些预测(就像学习策略和价值函数的强化学习算法一样,例如\textit{行动者}\textit{评论家}算法),但这种场景比同时学习策略和价值函数的场景更容易描述。


现在想象一下,智能体的经验分为多个试验,在每个试验中重复相同的状态序列,并且在试验期间的每个时间步上发生不同的状态。
进一步想象一下,预测的回报仅限于试验的回报,这使得试验类似于我们定义的强化学习事件。
当然,实际上,预测的回报并不局限于单次试验,试验之间的时间间隔是决定动物学到什么的重要因素。
对于时间差分学习来说也是如此,但这里我们假设回报不会通过多次试验累积。
鉴于此,\textit{舒尔茨}及其同事进行的实验相当于强化学习的一个阶段。
(尽管在本次讨论中,我们将使用术语“试验”而不是“情节”,以便更好地与实验联系起来。)


像往常一样,我们还需要对状态如何表示为学习算法的输入做出假设,该假设会影响\textit{时间差分误差}与多巴胺神经元活动的对应程度。
我们稍后讨论这个问题,但目前我们假设\textit{蒙太古}等人使用相同的\textit{全串行复合}表示\cite{montague1996framework}。
其中,在试验的每个时间步骤访问的每个州都有一个单独的内部刺激。
这将流程简化为本书第一部分中介绍的表格案例。
最后,我们假设智能体使用 TD(0) 来学习值函数 $V$ ,该函数存储在所有状态下初始化为零的查找表中。
我们还假设这是一项确定性任务,并且折扣因子$\gamma$ 非常接近 1,因此我们可以忽略它。


图~\ref{fig:12_5}~显示了 $R$、$V$ 和 $\delta$ 在该策略评估任务的几个学习阶段的时间过程。 
时间轴表示在试验中访问一系列状态的时间间隔(为了清楚起见,我们省略了显示各个状态)。
在每次试验中,奖励信号为零,除非智能体达到奖励状态(显示在时间线右端附近),此时奖励信号变为某个正数(例如 $R^{\star}$)。
\textit{时间差分}学习的目标是预测试验中访问的每个状态的返回,在这种未贴现的情况下,并假设预测仅限于单个试验,对于每个状态这简单地说是$R^{\star}$。


\begin{figure}[!htb]
	\centering
	\includegraphics[width=0.5\linewidth]{chap12/fig_12_5}
	\caption{\textit{时间差分}学习过程中\textit{时间差分}误差的行为与多巴胺神经元的阶段性激活特征一致。
		(这里是时间 $t$ 时的\textit{时间差分}误差,即 $\delta_{t-1}$)。
		顶部:状态序列,显示为常规预测变量的区间,后面跟着非零奖励 $R^{\star}$。
		\textit{早期学习}:初始值函数 $V$ 和初始值,最初等于 $R^{\star}$。
		\textit{学习完成}:价值函数准确地预测了未来的奖励,在最早的预测状态下为正,在非零奖励时 $\delta = 0$。
		\textit{$R^{\star}$忽略}:当预测奖励被忽略时,变为负数。
		请参阅文本以获取发生这种情况的完整解释。
		\label{fig:12_5}}
\end{figure}


奖励状态之前是一系列奖励预测状态,最早的奖励预测状态显示在时间线左端附近。
这类似于试验开始附近的状态,例如类似于上述\textit{舒尔茨}等人\cite{schultz1993responses}的猴子实验试验中由指令提示标记的状态。
(当然,实际上,先前试验中访问的状态甚至是更早的奖励预测状态,但是因为我们将预测限制在单个试验中,所以这些状态不符合本次试验奖励的预测因子。
下面我们给出一个更令人满意的,但更多的 摘要,最早的奖励预测状态的描述。)
试验中最新的奖励预测状态是紧接在试验的奖励状态之前的状态。
这是图~\ref{fig:12_5}~中时间线最右端附近的状态。
请注意,试验的奖励状态并不能预测该试验的回报:该状态的值将预测所有以下试验的回报,在这里我们假设在这个情景公式中为零。


图~\ref{fig:12_5}~显示了 $V$ 和 $\delta$ 的首次尝试时间课程,并标记为“早期学习”。 
因为除达到奖励状态外,整个试验过程中奖励信号为零,并且所有 $V$值均为零,因此\textit{时间差分}误差也为零,直到变为 $R^{\star}$ 处于奖励状态。
这是因为 $\delta_{t-1} = R_t + V_t - V_{t+1} = R_t + 0 - 0 = R_t$,当奖励发生时,在等于 $R^{\star}$ 之前它为零。
这里$V_t$和 $V_{t-1}$ 分别是试验中在时间$t$和$t-1$访问的状态的估计值。
此学习阶段的\textit{时间差分}误差类似于多巴胺神经元在训练开始时对不可预测的奖励(例如一滴苹果汁)的响应。



在第一个试验和所有后续试验中,TD(0) 更新发生在每个状态转换处。
这会连续增加奖励预测状态的值,并且增加从奖励状态向后传播,直到值 收敛到正确的回报预测。
在这种情况下(因为我们假设没有折扣)正确的预测等于 $R^{\star}$ 对于所有奖励预测状态。
这可以在图~\ref{fig:12_5}~中看到,作为标记为“学习完成”的 $V$ 的图,其中从最早到最新的奖励预测状态的所有状态的值都等于 $R^\star$。
最早奖励预测状态之前的状态值仍然很低(图~\ref{fig:12_5}~显示为零),因为它们不是可靠的奖励预测器。


当学习完成时,即当 $V$ 达到其正确值时,与任何奖励预测状态的转换相关的\textit{时间差分}误差为零,因为预测现在是准确的。
这是因为对于从奖励预测状态到另一个奖励预测状态的转变,我们有$\delta_{t-1} = R_t + V_t - V_{t-1} = 0 + R^{\star} - R^{\star} = 0$,对于从最新奖励预测状态到奖励状态的转变,我们有 $\delta_{t-1} = R_t + V_t - V_{t-1}$。
另一方面,从任何状态转换到最早的奖励预测状态的\textit{时间差分}误差都是正的,因为该状态的低值与后续奖励预测状态的较大值之间不匹配。
事实上,如果最早奖励预测状态之前的状态值为零,那么在过渡到最早奖励预测状态之后,我们将得到 $\delta_{t-1} = R_t + V_t - V_{t-1} = 0 + R^{\star} - 0 = R^{\star}$。 
图~\ref{fig:12_5}~中的“学习完成”图显示了最早奖励预测状态下的正值,而其他地方为零。


过渡到最早的奖励预测状态时的正\textit{时间差分}误差类似于多巴胺对最早预测奖励的刺激的持续响应。
出于同样的原因,当学习完成时,从最新的奖励预测状态到奖励状态的转换会产生零\textit{时间差分}误差,因为最新的奖励预测状态的值是正确的,会取消奖励。
这与观察结果相似,即与未预测的奖励相比,对完全预测的奖励产生阶段性响应的多巴胺神经元更少。


学习后,如果突然忽略奖励,则在通常的奖励时间,\textit{时间差分}误差会变为负值,因为最新的奖励预测状态的值太高:$\delta_{t-1} = R_t + V_t - V_{t-1} = 0 + 0 - R^{\star} = -R^{\star}$,如图~\ref{fig:12_5}~中“R 省略”图右端所示。
这就像在\textit{舒尔茨}等人的实验中看到的那样\cite{schultz1993responses},当预期奖励被忽略时,多巴胺神经元活动降低到基线以下,如上所述并如图~\ref{fig:12_4}~所示。


最早奖励预测状态的想法值得更多关注。
在上述场景中,由于经验被分为试验,并且我们假设预测仅限于单个试验,因此最早的奖励预测状态始终是试验的第一个状态。
显然这是人为的。
考虑最早的奖励预测状态的更一般方法是,它是不可预测的奖励预测器,并且可以有很多这样的状态。
在动物的一生中,许多不同的状态可能先于最早的奖励预测状态。
然而,由于这些状态后面经常跟随其他不预测奖励的状态,因此它们的奖励预测能力(即它们的值)仍然很低。
\textit{时间差分}算法如果在动物的整个生命周期中运行,也会更新这些状态的值,但更新不会持续累积,因为根据假设,这些状态中没有一个可靠地先于最早的奖励预测状态。
如果他们中的任何一个这样做了,他们也将获得奖励预测状态。
这也许可以解释为什么在过度训练的情况下,即使是试验中最早的奖励预测刺激,多巴胺响应也会减弱。
通过过度训练,人们会期望,即使是以前无法预测的预测状态也会被与早期状态相关的刺激所预测:动物在实验任务内外与其环境的相互作用将变得司空见惯。
然而,当引入一项新任务来打破这一惯例时,人们会看到\textit{时间差分}误差再次出现,正如在多巴胺神经元活动中所观察到的那样。


上面描述的例子解释了为什么当动物在类似于我们例子的理想化任务的任务中学习时,\textit{时间差分}误差与多巴胺神经元的阶段性活动共享关键特征。
但并非多巴胺神经元阶段性活动的每个属性都与 $\delta$ 的属性如此完美地一致。
最令人不安的差异之一是当奖励早于预期出现时会发生什么。
我们已经看到,预期奖励的遗漏会在奖励的预期时间产生负预测误差,这对应于发生这种情况时多巴胺神经元的活动降低到基线以下。
如果奖励到达的时间晚于预期,那么它就是意外奖励并产生正预测误差。
\textit{时间差分}误差和多巴胺神经元响应都会发生这种情况。
但是,当奖励比预期更早到达时,多巴胺神经元不会像\textit{时间差分误差}那样做——至少\textit{蒙太古}等人\cite{montague1996framework}和我们在我们的例子中使用的\textit{全串行复合}表示是这样。
多巴胺神经元确实会对早期奖励做出响应,这与正\textit{时间差分}误差一致,因为预计奖励不会在那时发生。
然而,当奖励被预期但被忽略时,\textit{时间差分}误差为负,而与此预测相反,多巴胺神经元活动不会按照\textit{时间差分}模型预测的方式降至基线以下\cite{hollerman1998dopamine}。
动物大脑中正在发生比简单地使用\textit{全串行复合}表示进行\textit{时间差分}学习更复杂的事情。



\textit{时间差分}误差和多巴胺神经元活动之间的一些不匹配可以通过为\textit{时间差分}算法选择合适的参数值并使用\textit{全串行复合}表示之外的刺激表示来解决。
例如,为了解决刚刚描述的早期奖励不匹配问题,Suri\cite{suri1999neural}提出了一种\textit{全串行复合}表示,其中早期刺激引发的内部信号序列因奖励的出现而被取消。
Daw\cite{daw2006representation}提出的另一个建议是,大脑的\textit{时间差分}系统使用在感觉皮层中进行的统计建模产生的表示,而不是基于原始感觉输入的简单表示。
Ludvig\cite{ludvig2008stimulus}发现,使用微刺激表示的\textit{时间差分}学习(图~\ref{fig:11_1})比使用\textit{全串行复合}表示更好地控制早期奖励和其他情况下多巴胺神经元的活动。
Pan\cite{pan2005dopamine}发现,即使使用\textit{全串行复合}表示,延长资格迹线也会改善多巴胺神经元活动某些方面的\textit{时间差分}误差。
一般来说,\textit{时间差分}误差行为的许多细节取决于\textit{资格迹}、贴现和刺激表示之间的微妙相互作用。
这些发现详细阐述并完善了奖励预测误差假说,但没有反驳其核心主张,即多巴胺神经元的阶段性活动被很好地表征为信号\textit{时间差分}误差。


另一方面,\textit{时间差分}理论和实验数据之间还存在其他差异,这些差异不容易通过选择参数值和刺激表示来解决(我们在本章末尾的参考文献和历史评论部分提到了其中一些差异) ,并且随着神经科学家进行更加精细的实验,可能会发现更多的不匹配。
但奖励预测误差假说一直非常有效地发挥着催化剂的作用,有助于提高我们对大脑奖励系统如何运作的理解。
人们设计了复杂的实验来验证或反驳从该假设得出的预测,而实验结果反过来又导致了\textit{时间差分}误差/多巴胺假设的完善和阐述。


这些发展的一个显著方面是,在完全不了解多巴胺神经元相关特性的情况下,从计算角度开发了与多巴胺系统特性密切相关的强化学习算法和理论-- 请记住,\textit{时间差分}学习及其与最优控制和动态规划的联系早在任何揭示多巴胺神经元活动的\textit{时间差分}性质的实验进行之前许多年就已经开发出来了。
这种计划外的对应关系尽管并不完美,但表明\textit{时间差分}误差/多巴胺平行捕获了有关大脑奖励过程的重要信息。


除了解释多巴胺神经元阶段性活动的许多特征之外,奖励预测误差假说还将神经科学与强化学习的其他方面联系起来,特别是与使用\textit{时间差分}误差作为强化信号的学习算法联系起来。
神经科学还远未完全理解多巴胺神经元的阶段性活动的回路、分子机制和功能,但支持奖励预测误差假说的证据,以及阶段性多巴胺响应是学习的强化信号的证据表明, 大脑可能会实现类似于\textit{行动者-评论家}算法的算法,其中\textit{时间差分}误差起着关键作用。
其他强化学习算法也是可能的候选者,但\textit{行动者-评论家}算法特别适合哺乳动物大脑的解剖学和生理学,正如我们在下面两节中描述的那样。



\section{行动者-评论家的神经实现} \label{sec:neural_ac}

行动者-评论家算法学习策略和价值函数。
“行动者”是学习策略的组件,“\textit{评论家}”是了解行动者当前遵循的任何策略的组件,以便“评价”行动者的行动选择。
\textit{评论家}使用\textit{时间差分}算法来学习行动者当前策略的状态值函数。
价值函数允许\textit{评论家}通过向行动者发送\textit{时间差分}误差来评价行动者的动作选择。
积极意味着该行动是“好的”,因为它导致了一个具有比预期价值更好的状态;
负数意味着该操作是“坏的”,因为它导致了一个比预期值更差的状态。
根据这些评价,该行为者不断更新其政策。


行动者-评论家算法的两个显著特征导致认为大脑可能会实现这样的算法。
首先,\textit{行动者}的两个组成部分-\textit{评论家}算法——\textit{行动者}和\textit{评论家}|表明纹状体的两个部分——背侧和腹侧细分(第~\ref{sec:dopamine}~节),这两个部分对于基于奖励的学习都至关重要——可能分别发挥类似 \textit{行动者}和评论家。
行动者-评论家算法建议大脑实现的第二个属性是,\textit{时间差分}误差具有双重作用,即作为行动者和\textit{评论家}的强化信号,尽管它对每个组件的学习有不同的影响 。
这与神经回路的几个特性相吻合:多巴胺神经元的轴突以纹状体的背侧和腹侧分区为目标;
多巴胺似乎对于调节这两种结构的突触可塑性至关重要;
多巴胺等神经调节剂如何作用于目标结构取决于目标结构的特性而不仅仅是神经调节剂的特性。


textit{行动者-评论家}算法可以呈现为策略梯度方法,但是\textit{巴托}\cite{barto13neuron}的\textit{行动者-评论家}算法更简单,并且被呈现为人工神经网络。
在这里,我们描述了类似于\textit{巴托}等人的人工神经网络实现,并且我们遵循 \textit{高桥}\cite{takahashi2008silencing}给出了如何通过大脑中的真实神经网络实现这种人工神经网络的示意图。
我们将对行动者和\textit{评论家}学习规则的讨论推迟到第~\ref{sec:ac_rules}~节,我们将它们作为策略梯度公式的特例,并讨论它们关于多巴胺如何调节突触可塑性的建议。


图~\ref{fig:12_6}a 显示了作为人工神经网络的行动者/评论家算法的实现,其中组件网络实现了行动者和评论家。
\textit{评论家}由一个类似神经元的单元 $V$ 组成,其输出活动表示状态值,以及一个显示为标记为\textit{时间差分}的菱形的组件,该组件通过将 $V$ 的输出与奖励信号以及先前的状态值相结合来计算\textit{时间差分误差}(如 从 TD 菱形到自身的循环)。
\textit{行动者}网络有一个单层的 $k$ 个\textit{行动者}单元,标记为 $A_i, i=1,...,k$。
每个行动者单元的输出是 $k$ 维动作向量的组成部分。 
另一种方法是有 $k$ 个单独的动作,每个动作由每个行动者单元命令,彼此竞争执行,但这里我们将整个 $A$ 向量视为一个动作。


\begin{figure}[!htb]
	\centering
	\includegraphics[width=0.8\linewidth]{chap12/fig_12_6}
	\caption{\textit{行动者-评论家}人工神经网络和假设的神经实现。
		a) 作为人工神经网络的\textit{行动者-评论家}算法。
			\textit{行动者}根据从\textit{评论家}处收到的\textit{时间差分}误差来调整策略;
			\textit{评论家}使用相同的方法调整状态值参数。
			\textit{评论家}根据奖励信号 $R$ 及其状态值估计的当前变化产生\textit{时间差分}误差。
			\textit{行动者}无法直接访问奖励信号,\textit{评论家}也无法直接访问\textit{动作}。
			b) 假设的\textit{行动者-评论家}算法的神经实现。
			\textit{评论家}和\textit{行动者}的价值学习部分分别位于腹侧纹状体和背侧纹状体。
			\textit{时间差分}误差由位于\textit{中脑腹侧被盖区}和\textit{黑质致密部}的多巴胺神经元传递,以调节从皮层区域到腹侧纹状体和背侧纹状体的输入突触效率的变化\cite{takahashi2008silencing}。
		\label{fig:12_6}}
\end{figure}


评论家和行动者网络都接收由代表智能体环境状态的多个特征组成的输入。
(回想一下第一章,强化学习智能体的环境包括包含该智能体的“有机体”内部和外部的组件。)
该图将这些特征显示为 $x_1,x_2,...,x_n$ 标记的圆圈,显示两次只是为了使图形简单。
代表突触效率的权重与从每个特征 $x_i$ 到评价单元 $V$ 以及到每个动作单元 $A_i$ 的每个连接相关联。
\textit{评论家}网络中的权重参数化\textit{价值函数},而\textit{行动者}网络中的权重参数化\textit{策略}。
当这些权重根据我们在下一节中描述的\textit{评论家}和行动者学习规则变化时,网络会进行学习。


\textit{评论家}中的回路产生的\textit{时间差分}误差是用于改变\textit{评论家}和\textit{行动者}网络中权重的增强信号。
这在图~\ref{fig:12_6}a 中通过标记为“TD error”的线显示,该线延伸到评论家和行动者网络中的所有连接。
网络实现的这一方面,加上奖励预测误差假设以及多巴胺神经元的活动通过这些神经元的广泛轴突乔木广泛分布的事实,表明像这样的\textit{行动者-评论家}网络可能不太适合 作为关于奖励相关学习如何在大脑中发生的假设是牵强的。


图~\ref{fig:12_6}b 非常示意性地表明了图左侧的人工神经网络如何根据\textit{高桥}等人的假设映射到大脑中的结构\cite{takahashi2008silencing}。
该假设将行动者和\textit{评论家}的价值学习部分分别置于纹状体的背侧和腹侧细分中,纹状体是基底神经节的输入结构。
回想一下~\ref{sec:dopamine}~节,背侧纹状体主要参与影响动作选择,而腹侧纹状体被认为对于奖励处理的不同方面至关重要,包括将情感价值分配给感觉。
大脑皮层与其他结构一起向纹状体发送输入,传达有关刺激、内部状态和运动活动的信息。


在这个假设的行动者-\textit{评论家}大脑实现中,腹侧纹状体将价值信息发送到\textit{中脑腹侧被盖区}和\textit{黑质致密部},其中这些核中的多巴胺神经元将其与奖励信息结合起来,以生成与\textit{时间差分}误差相对应的活动(尽管多巴胺能神经元究竟如何计算这些误差是还没明白)。
图~\ref{fig:12_6}~a中的“TD误差”线变成图~\ref{fig:12_6}b中标记为“多巴胺”的线,它代表多巴胺神经元的广泛分支的轴突,其细胞体位于\textit{中脑腹侧被盖区}和\textit{黑质致密部}中。
回顾图~\ref{fig:12_2},这些轴突与\textit{中型多棘神经元}树突上的棘进行突触接触,\textit{中型多棘神经元}是纹状体背侧和腹侧部分的主要输入/输出神经元。
将输入发送到纹状体的皮层神经元的轴突在这些棘的尖端上形成突触接触。
根据这一假设,正是在这些棘上,从皮层区域到地层的突触效率的变化受到学习规则的控制,而学习规则主要取决于多巴胺提供的强化信号。


图~\ref{fig:12_6}b 所示假设的一个重要含义是,多巴胺信号不是强化学习标量 $R_t$ 那样的“主”奖励信号。
事实上,这一假设意味着人们不一定能够探测大脑并记录任何单个神经元活动中的任何信号,如 $R_t$。
许多互连的神经系统生成与奖励相关的信息,根据不同类型的奖励而招募不同的结构。
多巴胺神经元从许多不同的大脑区域接收信息,因此图~\ref{fig:12_6}b 中标记为“奖励”的\textit{黑质致密部}和\textit{中脑腹侧被盖区}的输入应被视为沿着多个输入通道到达这些核中的神经元的奖励相关信息的向量。
那么,理论标量奖励信号 $R_t$ 可能对应的是所有奖励相关信息对多巴胺神经元活动的净贡献。
它是大脑不同区域许多神经元活动模式的结果。


尽管图~\ref{fig:12_6}b 中所示的\textit{行动者-评论家}神经实现在某些方面可能是正确的,但它显然需要细化、扩展和修改,才能成为多巴胺神经元阶段性活动功能的全面模型。
本章末尾的历史和书目评论部分引用了一些出版物,这些出版物更详细地讨论了这一假设的实证支持及其不足之处。
我们现在详细研究行动者和\textit{评论家}学习算法对控制皮层纹状体突触突触效率变化的规则的建议。





\section{行动者和评论家学习规则} \label{sec:ac_rules}

如果大脑确实实现了类似行动者-评论家算法的东西,并假设多巴胺神经元群向背侧纹状体和腹侧纹状体的皮层纹状体突触广播共同的强化信号,如图~\ref{fig:12_6}b~所示(这可能是我们上面提到的过于简单化),那么这个强化信号以不同的方式影响这两个结构的突触。
\textit{评论家}和\textit{行动者}的学习规则使用相同的强化信号,即\textit{时间差分}误差 $\delta$,但其对学习的影响对于这两个组件来说是不同的。
\textit{时间差分}误差(与\textit{资格迹}相结合)告诉行动者如何更新动作概率以达到更高价值的状态。
行动者的学习就像使用效应法则类型的学习规则的工具调节:行动者努力保持尽可能积极的态度。
另一方面,\textit{时间差分}误差(与\textit{资格迹}结合时)告诉\textit{评论家}改变价值函数参数的方向和幅度,以提高其预测准确性。
\textit{评论家}致力于使用像经典条件反射的\textit{时间差分}模型(第~\ref{sec:classical_conditioning}~节)这样的学习规则,将 $\delta$ 的幅度降低到尽可能接近于零。
\textit{评论家}和\textit{行动者}学习规则之间的差异相对简单,但这种差异对学习有深远的影响,并且对于行动者-评论家算法的工作方式至关重要。
差异仅在于每种类型的学习规则使用的\textit{资格迹}。


在像图~\ref{fig:12_6}b~所示的行动者-评论家神经网络中,可以使用多于一组的学习规则,但是,具体来说,这里我们重点关注具有\textit{资格迹}的连续问题的\textit{行动者-评论家}算法。
每次从状态 $S_t$ 转换到状态 $S_{t+1}$ 时,采取动作 $A_t$ 并接收动作 $R_{t+1}$,该算法计算时间差分误差 ($\delta$ ),然后更新\textit{资格迹}向量($z_t^w$ 和 $z_t^{\theta}$ )以及\textit{评论家}和\textit{行动者}($w$ 和 $\theta$),根据

\begin{equation}
	\delta_t = R_{t+1}
		+ \gamma \hat{v} (S_{t+1}, w)
		- \hat{v} (S_t, w),
\end{equation}


\begin{equation}
	z_t ^w = \lambda^w z_{t-1}^w
		+ \Delta_w \hat{v} (S_t, w),
\end{equation}

\begin{equation}
	z_t^{\theta} = \lambda^{\theta} z_{t-1}^{\theta}
		+ \Delta_{\theta} ln(\pi(A_t | S_t, \theta)),
\end{equation}


\begin{equation}
	w \longleftarrow w + \alpha^w \delta_t z_t^w,
\end{equation}

\begin{equation}
	\theta \longleftarrow \theta + \alpha^{\theta} \delta z_t^{\theta},
\end{equation}
其中$\gamma \in $ [0, 1) 是折扣率参数,$\lambda^w \in$ [0, 1] 和 $\lambda ^w a \in$ [0; 1] 分别是\textit{评论家}和\textit{行动者}的自举参数,$\alpha ^w$ > 0 和 $\alpha ^{\theta}$> 0 是类似的步长参数。


将近似值函数 $\hat{v}$ 视为单个线性神经元类单元的输出,称为评价单元并在图~\ref{fig:12_6}a 中标记为 $V$。
那么值函数是状态 $s$ 的特征向量表示的线性函数,$x(s) = (x_1(s), ..., x_n(s))^T$,由权重向量 $w = (w_1, ..., w_n) ^T$:

\begin{equation} \label{eq:15_1}
	\hat{v}(s, w) = w^T x(s).
\end{equation}


每个 $x_i(s)$就像神经元突触的突触前信号,其效能为 $w_i$。
\textit{评论家}的权重根据上述规则增加 $\alpha^w \delta_t z_t^w$,其中强化信号 $\delta_t$ 对应于广播到所有\textit{评论家}单元突触的多巴胺信号。
评价单位的\textit{资格迹}向量 $z_t^w$ 是 $\Delta_w \hat{v} (S_t, w)$ 的轨迹(最近值的平均值)。
因为 $\hat{v}(s, w)$ 的权重是线性的,所以 $\Delta_w \hat{v} (S_t, w) = x(S_t)$。


用神经科学术语来说,这意味着每个突触都有自己的\textit{资格迹},它是向量 $z_t^w$ 的一个组成部分。
突触的合格轨迹根据到达该突触的活动水平(即突触前活动水平)进行累积,此处由到达该突触的特征向量 $x(S_t)$ 的分量表示。
否则,迹线将以分数 $\lambda ^w$ 控制的速率向零衰减。
只要突触的\textit{资格迹}非零,它就有资格进行修改。
突触的有效性实际上如何改变取决于突触合格时到达的强化信号。
我们将像评价单元突触这样的资格痕迹称为非偶然资格痕迹,因为它们仅依赖于突触前活动,而不以任何方式取决于突触后活动。


评价单元突触的非偶然资格痕迹意味着评价单元的学习规则本质上是第~\ref{sec:classical_conditioning}~节中描述的经典条件反射的\textit{时间差分}模型。
根据我们上面给出的\textit{评论家}单元及其学习规则的定义,图~\ref{fig:12_6}a 中的\textit{评论家}与 Barto 等人的神经网络\textit{行动者-评论家}中的\textit{评论家}相同\cite{barto13neuron}。
显然,像这样仅由一个类似线性神经元的单元组成的\textit{评论家}是最简单的起点;
这个评价单元是一个更复杂的神经网络的智能体,能够学习更复杂的价值函数。


图~\ref{fig:12_6}a 中的行动者是一个由 $k$ 个类似神经元的行动者单元组成的一层网络,每个行动者在时间 $t$ 接收与\textit{评论家}单元接收的相同特征向量 $x(S_t)$。
每个行动者单元 $j, j=1, ..., k$ 有自己的权重向量 $\theta_j$ ,但由于行动者单元都是相同的,因此我们仅描述其中一个单元并省略下标。
这些单元遵循上面等式中给出的\textit{行动者-评论家}算法的一种方法是每个单元都是伯努利逻辑单元。
这意味着每个行动单元每次的输出是一个随机变量$A_t$,取值0或1。
将值1视为神经元环,即发出动作电位。
单元输入向量的加权和 $\theta^T x(S_t)$ 通过指数 softmax 分布确定单元的动作概率,对于两个动作来说,它是逻辑函数:

\begin{equation} \label{eq:15_2}
	\pi (1|s, \theta) = 
		1 - \pi(0|s, \theta) 
		= 1 \frac{1}{1 + e^{-\theta^T x(s)}}.
\end{equation}


如上所述,每个行动者单元的权重增加:$\theta \longleftarrow \theta + \alpha^{\theta} \delta_t z_t^{\theta}$,其中 $\delta$ 再次对应于多巴胺信号:发送到所有评价单元突触的相同强化信号。
图~\ref{fig:12_6}a 显示 $\delta_t$ 被广播到所有\textit{行动者}单元的所有突触(这使得该\textit{行动者}网络成为一个强化学习智能体团队,我们将在下面的~\ref{sec:collective_rl}~节中讨论)。
行动者\textit{资格迹}向量 $z_t^{\theta}$ 是 $\Delta_\theta ln \pi (A_t|S_t, \theta)$ 的轨迹(最近值的平均值)。
要了解此\textit{资格迹},它定义了此类单元并要求您为其提供学习规则。
该练习要求您通过计算梯度,用 $a$, $x(s)$ 和 $\pi(a|s, \theta)$(对于任意状态 $s$ 和动作 $a$)来表达 $\Delta _\theta ln \pi (a|s, \theta)$。
对于在时间 $t$ 实际发生的动作和状态,答案是:

\begin{equation} \label{eq:15_3}
	\nabla_{\theta} \pi (A_t | S_t, \theta)
		= (A_t - \pi(A_t|S_t, \theta)) x(S_t).
\end{equation}


与仅累积突触前活动 $x(S_t)$ 的\textit{评论家}突触的非偶然\textit{资格迹}不同,行动者单元突触的\textit{资格迹}还取决于行动者单元本身的活动。
我们称其为偶然资格痕迹,因为它取决于突触后活动。
每个突触处的合格迹线不断衰减,但增加或减少取决于突触前神经元的活动以及突触后神经元是否有响应。
当 $A_t = 1$ 时,(\ref{eq:15_3}) 中的因子 $A_t - \pi(A_t|S_t, \theta)$ 为正,否则为负。
行动者单元的资格痕迹中的突触后偶然性是\textit{评论家}和行动者学习规则之间的唯一区别。
通过保留有关在哪些州采取了哪些行动的信息,或有\textit{资格迹}允许奖励信用(正)或惩罚责备(负),以便在政策参数(行动者单位的有效性)之间进行分配。突触)根据这些参数对单元输出的贡献可能会影响以后的值。
偶然资格痕迹标记了突触,告诉它们应该如何修改以改变单元的未来响应以支持 的正值。


关于皮层纹状体突触的效率如何变化,\textit{评论家}和行动者学习规则表明了什么?
这两种学习规则都与 Donald Hebb 的经典提议相关,即每当突触前信号参与激活突触后神经元时,突触的效率就会增加\cite{hebb2005organization}。
\textit{评论家}和\textit{行动者}学习规则与赫布的提议一致,即突触效率的变化取决于多个因素的相互作用。
在评价学习规则中,相互作用发生在强化信号和仅依赖于突触前信号的\textit{资格迹}之间。
神经科学家将此称为双因素学习规则,因为相互作用是在两个信号或数量之间进行的。
另一方面,\textit{行动者}学习规则是一个三因素学习规则,因为除了取决于 $\delta$ 之外,其\textit{资格迹}还取决于突触前和突触后活动。
然而,与赫布的提议不同的是,这些因素的相对时间对于突触功效如何变化至关重要,其中资格痕迹进行干预,以允许强化信号影响最近活跃的突触。


\textit{行动者}和评论家学习规则的信号时序的一些微妙之处值得密切关注。
在定义类似神经元的\textit{行动者}和\textit{评论家}单元时,我们忽略了突触输入影响真实神经元环所需的少量时间。
当突触前神经元的动作电位到达突触时,神经递质分子被释放,穿过突触间隙到达突触后神经元,在那里它们与突触后神经元表面的受体结合;
这会激活分子机制,导致突触后神经元重新激活(或在抑制性突触输入的情况下抑制其环)。
这个过程可能需要几十毫秒。
然而,根据(\ref{eq:15_1})和(\ref{eq:15_2}),评论家和\textit{行动者}单元的输入立即产生该单元的输出。
像这样忽略激活时间在赫布式可塑性的抽象模型中很常见,其中突触效能根据同时突触前和突触后活动的简单产物而变化。
更现实的模型必须考虑激活时间。


激活时间对于更现实的\textit{行动者}单元尤其重要,因为它影响偶然资格痕迹必须如何工作,以便正确地将强化的功劳分配给适当的突触。
定义上面给出的执行者单元学习规则的条件资格迹的表达式 $(A_t - \pi(A_t|S_t, \theta)) x(S_t)$ 包括突触后因子 $(A_t, \pi(A_t | S_t, \theta))$ 和突触前因子 $x(S_t)$。
这是有效的,因为通过忽略激活时间,突触前活动 $x(S_t)$ 参与引起出现在 $(A_t - \pi(A_t | S_t, \theta))$ 中的突触后活动。
为了正确分配强化的信用,定义\textit{资格迹}的突触前因素必须是也定义轨迹的突触后因素的原因。
更现实的\textit{行动者}单位的偶然\textit{资格迹}必须考虑激活时间。
(激活时间不应与神经元接收受该神经元活动影响的强化信号所需的时间混淆。资格迹线的功能是跨越这个时间间隔,该时间间隔通常比激活时间长得多。
我们讨论这个下一节将进一步介绍。)


神经科学提供了关于这一过程如何在大脑中发挥作用的线索。
神经科学家发现了一种称为\textit{脉冲时序的可塑性}的赫布可塑性形式,它为大脑中类似\textit{行动者}的突触可塑性的存在提供了合理性。
\textit{脉冲时序的可塑性}是赫布式的可塑性,但突触功效的变化取决于突触前和突触后动作电位的相对时间。
这种依赖性可以采取不同的形式,但在研究最多的一种形式中,如果通过突触传入的尖峰在突触后神经元恢复之前不久到达,突触的强度就会增加。
如果时间关系相反,突触前尖峰在突触后神经元释放后不久到达,则突触的强度会降低。
\textit{脉冲时序的可塑性}是赫布可塑性的一种,它考虑了神经元的激活时间,这是类\textit{行动者}学习所需的要素之一。


\textit{脉冲时序的可塑性}的发现促使神经科学家研究\textit{脉冲时序的可塑性}三因素形式的可能性,其中神经调节输入必须遵循适当定时的突触前和突触后峰值。
这种形式的突触可塑性,称为奖励调节\textit{脉冲时序的可塑性},很像这里讨论的\textit{行动者}学习规则。
只有在突触前尖峰紧随其后的突触后尖峰之后的时间窗口记忆在神经调节输入时,常规\textit{脉冲时序的可塑性}才会产生突触变化。
越来越多的证据表明,奖励调节的\textit{脉冲时序的可塑性}发生在背侧纹状体的\textit{中型多棘神经元}的棘上,多巴胺提供了神经调节因子|在图~\ref{fig:12_6}~所示的\textit{行动者-评论家}算法的假设神经实现中,\textit{行动者}学习发生在这些位点。
b. 实验已经证明了奖赏调节\textit{脉冲时序的可塑性},其中只有当神经调节脉冲在突触前尖峰紧随其后紧随突触后尖峰之后持续长达 10 秒的时间窗口内到达时,皮层纹状体突触效能才会发生持久变化\cite{yagishita2014critical}。
尽管证据是间接的,但这些实验表明存在具有较长时间进程的偶然资格痕迹。
产生这些痕迹的分子机制,以及可能构成\textit{脉冲时序的可塑性}的更短痕迹,尚不清楚,但针对时间依赖性和神经调节剂依赖性突触可塑性的研究仍在继续。


我们在这里描述的类似神经元的行动者单元,及其效应法则式的学习规则,以某种更简单的形式出现在 Barto 等人的行动者-评论家网络中\cite{barto13neuron}。
该网络的灵感来自于生理学家\textit{克洛普夫}提出的“享乐神经元”假说\cite{klopf1972brain,klopf1982hedonistic}。
\textit{克洛普夫}假说的所有细节并不都与突触可塑性的知识相一致,但\textit{脉冲时序的可塑性}的发现以及越来越多的证据表明\textit{脉冲时序的可塑性}的奖励调节形式表明\textit{克洛普夫}的想法可能并没有离题。
接下来我们讨论\textit{克洛普夫}的享乐主义神经元假说。



\section{享乐主义神经元} \label{sec:hedonistic_neurons}


\textit{克洛普夫}\cite{klopf1972brain,klopf1982hedonistic}在他的享乐主义神经元假说中猜测,个体神经元试图通过根据自身动作电位的奖励或惩罚结果调整突触的大小,最大限度地扩大被视为奖励的突触输入和被视为惩罚的突触输入之间的差异。
换句话说,单个神经元可以通过响应相关强化进行训练,就像动物可以在工具调节任务中进行训练一样。
他的假设包括这样的想法:奖励和惩罚是通过刺激或抑制神经元尖峰生成活动的相同突触输入传递给神经元的。
(如果\textit{克洛普夫}知道我们今天对神经调节系统的了解,他可能会将强化作用分配给神经调节输入,但他想避免任何集中的训练信息来源。)
过去突触前和突触后活动的突触局部痕迹在\textit{克洛普夫}的假说中起着关键作用,即使突触有资格通过后来的奖励或惩罚进行调整。
他推测这些痕迹是由每个突触局部的分子机制实现的,因此与突触前和突触后神经元的电活动不同。
在本章的参考文献和历史评论部分,我们提请注意其他人提出的一些类似建议。


\textit{克洛普夫}具体推测突触效能以下列方式变化。
当神经元释放动作电位时,其所有积极贡献该动作电位的突触都有资格经历其功效的变化。
如果在适当的时间段内随着动作电位的增加奖励的增加,所有符合条件的突触的效率都会增加。
对称地,如果动作电位在适当的时间段内增加惩罚,则符合条件的突触的效率就会降低。
这是通过在突触那里触发资格迹来实现的,这种触发只在突触前与突触后的活动碰巧一致的时候才会发生(或者更准确地说,是在突触前活动和该突触前活动所参与引发的突触后活动同时出现的时候才会发生)。
这本质上是上一节中描述的行动者单元的三因素学习规则。


\textit{克洛普夫}理论中\textit{资格迹}的形状和时间过程反映了神经元嵌入其中的许多反馈回路的持续时间,其中一些完全位于生物体的大脑和身体内,而另一些则通过运动与感知系统延伸到机体外部的环境中。
他的想法是,突触\textit{资格迹}的形状就像神经元嵌入的反馈循环持续时间的直方图。
然后,\textit{资格迹}的峰值将出现在该神经元参与的最普遍反馈循环的持续时间内。
本书中描述的算法使用的资格迹是\textit{克洛普夫}原始想法的简化版本,是由参数 $\lambda$ 和 $\gamma$ 控制的指数(或几何)递减函数。
这简化了模拟和理论,但我们将这些简单的\textit{资格迹}视为更接近\textit{克洛普夫}原始概念的轨迹的占位符,通过改进学分分配过程,这将在复杂的强化学习系统中具有计算优势。


\textit{克洛普夫}的享乐主义神经元假说并不像乍看起来那样难以置信。
大肠杆菌是一个经过充分研究的单细胞寻求某些刺激并避免其他刺激的例子。
这种单细胞生物的运动受到其环境中化学刺激的影响,这种行为称为趋化性。
它通过旋转附着在其表面的称为 Agella 的毛发状结构,在液体环境中游泳。 (是的,它旋转它们!)细菌环境中的分子与其表面的受体结合。
结合事件调节细菌逆转无细胞旋转的频率。
每次逆转都会导致细菌原地翻滚,然后随机转向新的方向。
当细菌游向其生存所需的较高浓度的分子(引诱剂)时,少量的化学记忆和计算会导致无糖反转的频率降低,而当细菌游向有害的较高浓度的分子(排斥剂)时,无胶反转的频率会增加。
结果是,细菌倾向于持续沿着引诱剂梯度游动,并且倾向于避免沿着排斥剂梯度游动。


刚才描述的趋化行为称为调转运动。
这是一种试错行为,尽管不太可能涉及学习:细菌需要一点短期记忆来检测分子浓度梯度,但它可能无法维持长期记忆。
人工智能先驱\textit{奥利弗$\cdot$塞尔弗里奇}将这种策略称为“奔跑和旋转”,并指出其作为基本自适应策略的实用性:“如果情况变得更好,则继续以同样的方式前进,否则就四处走动”\cite{selfridge1978tracking,selfridge1984some}。
类似地,人们可能会认为神经元在由复杂的反馈回路集合组成的介质中“游泳”(当然不是字面上的意思),神经元嵌入其中,作用是获取一种类型的输入信号并避免其他类型的输入信号。
然而,与细菌不同,神经元的突触强度保留了有关其过去的试错行为的信息,如果神经元(或只是一种类型的神经元)的行为的这种观点是合理的,那么神经元如何相互作用的闭环性质。
与其环境的关系对于理解其行为非常重要,神经元的环境由动物的其余部分以及动物作为一个整体相互作用的环境组成。


\textit{克洛普夫}的享乐主义神经元假说超越了单个神经元是强化学习智能体的想法。
他认为,智能行为的许多方面可以被理解为一群自私的享乐主义神经元在构成动物神经系统的巨大社会或经济系统中相互作用的集体行为的结果。
无论这种神经系统观点是否有用,强化学习智能体的集体行为对神经科学都有影响。
我们接下来讨论这个话题。


\section{集体强化学习} \label{sec:collective_rl}

强化学习智能体群体的行为与社会和经济系统的研究密切相关,如果\textit{克洛普夫}的享乐神经元假说是正确的,那么它也与神经科学密切相关。
上面描述的关于如何在大脑中实现\textit{行动者-评论家}算法的假设仅狭隘地解决了纹状体的背侧和腹侧细分、\textit{行动者}和\textit{评论家}根据假设的各自位置这一事实的含义,每个包含数百万个\textit{中型多棘神经元},其突触经历由多巴胺神经元活动的阶段性爆发调节的变化。


图~\ref{fig:12_6}a 中的\textit{行动者}是一个由 $k$ 个\textit{行动者}单元组成的单层网络。
该网络产生的行为是向量 $(A_1, A_2, ..., A_k) ^T$,被认为驱动动物的行为。
所有这些单元的突触功效的变化取决于强化信号。
因为行动者单位试图尽可能大,所以有效地充当了他们的奖励信号(因此在这种情况下,强化与奖励相同)。
因此,每个参与者单元本身就是一个强化学习主体——如果你愿意的话,就是一个享乐神经元。
现在,为了使情况尽可能简单,假设这些单元中的每个单元在同一时间接收相同的奖励信号(尽管如上所述,假设多巴胺在相同条件下在相同时间在所有皮层纹状体突触释放可能过于简单)。


当强化学习智能体群体的所有成员根据共同的奖励信号进行学习时,强化学习理论可以告诉我们什么?
多智能体强化学习领域考虑了强化学习智能体群体学习的许多方面。
尽管这个领域超出了本书的范围,但我们相信它的一些基本概念和结果与思考大脑的多功能神经调节系统相关。
在多智能体强化学习(以及博弈论)中,所有智能体试图最大化它们同时接收的公共奖励信号的场景被称为合作博弈或团队问题。


团队问题之所以有趣且具有挑战性,是因为发送给每个智能体的共同奖励信号评估了整个群体产生的活动模式,即评估了团队成员的集体行动。
这意味着任何单个智能体影响奖励信号的能力都有限,因为任何单个智能体仅贡献由公共奖励信号评估的集体行动的一个组成部分。
在这种情况下,有效的学习需要解决结构性的信用分配问题:哪些团队成员或团队成员群体应该因有利的奖励信号而受到赞扬,或因不利的奖励信号而受到指责?
这是一个合作博弈,或者说是一个团队问题,因为智能体人团结起来寻求增加相同的奖励信号:智能体之间不存在利益冲突。
如果不同的智能体收到不同的奖励信号,则该场景将是一场竞争性游戏,其中每个奖励信号再次评估群体的集体行动,并且每个智能体的目标是增加自己的奖励信号。
在这种情况下,智能体之间可能存在利益冲突,这意味着对某些智能体有利的行为对其他人不利。
甚至决定什么是最好的集体行动也是博弈论的一个重要方面。
这种竞争环境也可能与神经科学相关(例如,解释多巴胺神经元活动的异质性),但在这里我们只关注合作或团队案例。


团队中的每个强化学习智能体如何才能学会“做正确的事”,从而使团队的集体行动获得高额回报?
一个有趣的结果是,如果每个智能体都能有效地学习,尽管其奖励信号被大量破坏 尽管无法获得完整的状态信息,但总体而言,即使每个智能体无法相互沟通,总体上也会产生集体行动,这些行动会根据共同奖励信号的评估而得到改善。
它自己的强化学习任务,其中它对奖励信号的影响深深地埋藏在其他智能体的影响所产生的噪声中。
事实上,对于任何智能体来说,所有其他智能体都是其环境的一部分,因为它的输入都是其环境的一部分。
传达状态信息的部分和奖励部分取决于所有其他智能体的行为方式。
此外,由于无法访问其他智能体的操作,实际上也无法访问确定其策略的参数,因此每个智能体只能部分地观察状态。其环境。
这使得每个团队成员的学习任务非常困难,但如果每个团队成员都使用强化学习算法,即使在这些困难的条件下也能够增加奖励信号,那么强化学习智能体团队就可以学会产生随着时间的推移而改进的集体行动,如下所示: 通过团队的共同奖励信号进行评估。


如果团队成员是类似神经元的单元,那么每个单元都必须有随着时间的推移增加其收到的奖励量的目标,就像我们在第~\ref{sec:ac_rules}~节中描述的\textit{行动者}单元所做的那样。
每个单元的学习算法必须具有两个基本特征。
首先,它必须使用或有资格痕迹。
回想一下,用神经术语来说,当突触前输入参与导致突触后神经元重新激活时,突触处就会启动(或增加)偶然\textit{资格迹}。
相反,非偶然\textit{资格迹}是由突触前输入启动或增加的,与突触后神经元的行为无关。
正如第~\ref{sec:ac_rules}~节中所解释的,通过保留有关在哪些州采取了哪些行动的信息,或有\textit{资格迹}允许根据这些参数的值在哪些状态中做出的贡献,将奖励信用或惩罚责备分配给智能体的策略参数。
决定智能体的行动。
通过类似的推理,团队成员必须记住其最近的操作,以便可以根据随后收到的奖励信号增加或减少产生该操作的可能性。
或有\textit{资格迹}的操作组件实现此操作记忆。
然而,由于学习任务的复杂性,偶然资格只是学分分配过程中的初步步骤:单个团队成员的行动与团队奖励信号变化之间的关系是一种统计相关性,必须在许多方面进行估计。
试验。
或有资格是这一过程中的一个重要但初步的步骤。


使用非偶然\textit{资格迹}进行学习在团队环境中根本不起作用,因为它无法提供将行为与奖励信号的后续变化相关联的方法。
非偶然\textit{资格迹}对于学习预测来说是足够的,就像行动者-\textit{评论家}算法的\textit{评论家}组件一样,但它们不支持学习控制,而行动者组件必须这样做。
类似\textit{评论家}的智能体群体的成员可能仍然会收到共同的强化信号,但他们都会学习预测相同的数量(在行动者-评论家方法的情况下,这将是当前策略的预期回报) 。
每个群体成员在学习预测预期回报方面的成功程度将取决于其收到的信息,而对于不同的群体成员来说,这些信息可能非常不同。
人们不需要产生差异化的活动模式。
这不是此处定义的团队问题。


团队问题中集体学习的第二个要求是,团队成员的行动必须具有可变性,以便团队探索集体行动的空间。
强化学习智能体团队做到这一点的最简单方法是每个成员通过其输出的持续变化独立探索自己的行动空间。
这将导致团队作为一个整体改变其集体行动。
例如,第~\ref{sec:ac_rules}~节中描述的行动者单元团队探索集体行动的空间,因为每个单元的输出(作为伯努利逻辑单元)在概率上取决于其输入向量分量的加权和。
加权和使概率出现上下偏差,但总是存在可变性。
因为每个单元都使用强化策略梯度算法(第~\ref{chap:chap11}~章),所以每个单元都会调整其权重,以最大化其经历的平均奖励率,同时随机探索自己的动作空间。
正如\textit{威廉姆斯}\cite{williams1992simple}所做的那样,我们可以证明,一组伯努利逻辑强化单元根据团队共同奖励信号的平均率整体实施了一种策略梯度算法,其中的行动是团队的集体行动 。


此外,\textit{威廉姆斯}\cite{williams1992simple}表明,当团队中的单元互连形成多层神经网络时,使用 REINFORCE 的伯努利逻辑单元团队会提升平均奖励梯度。
在这种情况下,奖励信号被广播到网络中的所有单元,尽管奖励可能仅取决于网络输出单元的集体行动。
这意味着伯努利逻辑强化单元的多层团队像通过广泛使用的误差反向传播方法训练的多层网络一样学习,但在这种情况下,反向传播过程被广播的奖励信号取代。
在实践中,误差反向传播方法要快得多,但强化学习团队方法作为一种神经机制更合理,特别是考虑到我们正在了解第~\ref{sec:ac_rules}~节中讨论的奖励调制\textit{脉冲时序的可塑性}。


团队成员自主探索只是团队探索最简单的方式;
如果团队成员能够相互沟通,以便他们能够协调行动,专注于集体行动空间的特定部分,那么更复杂的方法是可能的。
还有比偶然\textit{资格迹}更复杂的机制来解决结构性信用分配,当可能的集体行动集以某种方式受到限制时,这在团队问题中更容易。
一个极端的例子是赢者通吃的安排(例如,大脑横向抑制的结果),它将集体行动限制为只有一个或少数团队成员做出贡献的行动。
在这种情况下,获胜者会因由此产生的奖励或惩罚而受到赞扬或指责。


合作游戏(或团队问题)和非合作游戏问题的学习细节超出了本书的范围。
本章末尾的参考文献和历史评论部分引用了相关出版物的精选内容,包括对集体强化学习对神经科学的影响的研究的广泛参考。


\section{大脑中基于模型的方法}

事实证明,强化学习在无模型算法和基于模型算法之间的区别对于思考动物学习和决策过程非常有用。
第~\ref{sec:habitual_behavior}~节讨论了这种区别如何与习惯性和目标导向的动物行为之间的区别相一致。
上面讨论的关于大脑如何实现行动者/\textit{评论家}算法的假设仅与动物的习惯行为模式相关,因为基本的行动者/\textit{评论家}方法是无模型的。
哪些神经机制负责产生目标导向的行为,它们如何与那些潜在的习惯行为相互作用?


研究这些行为模式所涉及的大脑结构问题的一种方法是灭活老鼠大脑的某个区域,然后观察老鼠在结果贬值实验中的行为(第~\ref{sec:habitual_behavior}~节)。
此类实验的结果表明,上述的\textit{行动者-评论家}假设将\textit{行动者}置于背侧纹状体方面过于简单。
背侧纹状体的一部分——\textit{背外侧纹状体}失活会损害习惯学习,导致动物更多地依赖目标导向的过程。
另一方面,\textit{背内侧纹状体}失活会损害目标导向过程,要求动物更多地依赖习惯学习。
这些结果支持这样的观点:啮齿类动物的\textit{背外侧纹状体}更多地参与无模型过程,而它们的\textit{背内侧纹状体}更多地参与基于模型的过程。
使用功能神经影像学对人类受试者和非人类灵长类动物进行类似实验的研究结果支持这样的观点,即灵长类动物大脑中的类似结构不同地涉及习惯性和目标导向的行为模式。


其他研究确定了与人脑前额叶皮层基于模型的过程相关的活动,额叶皮层的最前部涉及执行功能,包括计划和决策。
特别涉及的是\textit{眶额皮层},即\textit{前额叶皮层}紧邻眼睛上方的部分。
人类的功能神经成像以及猴子单个神经元活动的记录揭示了\textit{眶额皮层}中与生物显著刺激的主观奖励值相关的强烈活动,以及与行动结果预期奖励相关的活动。
虽然并非没有争议,但这些结果表明\textit{眶额皮层}在目标导向选择中发挥了重要作用。
这对于动物环境模型的奖励部分可能至关重要。


涉及基于模型的行为的另一个结构是海马体,它是记忆和空间导航的关键结构。
大鼠的海马体在大鼠以目标导向的方式穿越迷宫的能力中发挥着关键作用,这导致托尔曼得出动物在选择行动时使用模型或认知图的想法(第~\ref{sec:cognitive_maps}~节)。
海马体也可能是我们人类想象新体验能力的关键组成部分\cite{hassabis2007deconstructing,olafsdottir2015hippocampal}。


最直接地将海马体与计划联系在一起的发现——这是一个在做出决策时寻求环境模型所需的过程——来自于对海马体中神经元的活动进行解码的实验,以确定海马体活动在每时每刻的基础上代表了哪一部分空间。
当老鼠在迷宫中的某个选择点停下来时,海马体中的空间表示会沿着动物从该点开始可能采取的路径向前(而不是向后)扫过\cite{johnson2007neural}。
此外,这些扫描所代表的空间轨迹与大鼠随后的导航行为密切对应\cite{pfeiffer2013hippocampal}。
这些结果表明,海马体对动物环境模型的状态转换部分至关重要,它是一个系统的一部分,该系统使用该模型模拟未来可能的状态序列,以评估可能的行动过程的后果:一种规划形式。


上述结果丰富了有关目标导向或基于模型的学习和决策的神经机制的大量文献,但许多问题仍未得到解答。
例如,像\textit{背外侧纹状体}和\textit{背内侧纹状体}这样结构相似的领域如何成为像无模型算法和基于模型算法一样不同的学习和行为模式的重要组成部分?
单独的结构是否负责(我们所说的)环境模型的转换和奖励组件?
所有计划是否都是在决策时通过模拟未来可能的行动方案来进行的,正如海马体中的前向扫荡活动所暗示的那样?
换句话说,所有的计划都类似于推出算法(第~\ref{sec:collective_rl}~节)吗?
或者模型有时会在后台进行细化或重新计算价值信息,如 Dyna 架构(第~\ref{sec:reward_signals}~节)所示?
大脑如何在习惯和目标导向系统的使用之间进行仲裁?
事实上,这些系统的神经基质之间是否存在明显的分离?


证据并未对最后一个问题给出肯定的答案。
Doll\cite{doll2012ubiquity}总结了这种情况,写道“在大脑处理奖励信息的任何地方,基于模型的影响或多或少都无处不在”,即使在被认为对无模型至关重要的区域也是如此。
这包括多巴胺信号本身,除了被认为是无模型过程基础的奖励预测误差之外,它还可以表现出基于模型的信息的影响。


以强化学习的无模型和基于模型的区别为基础的持续神经科学研究有可能加深我们对大脑中习惯性和目标导向过程的理解。
更好地掌握这些神经机制可能会导致算法以计算强化学习中尚未探索的方式结合无模型和基于模型的方法。


\section{成瘾}


了解药物滥用的神经基础是神经科学的一个高度优先目标,有可能为这一严重的公共卫生问题产生新的治疗方法。
一种观点认为,对毒品的渴望是同样的动机和学习过程的结果,这些动机和学习过程引导我们寻求满足我们生物需求的自然奖励体验。
成瘾物质通过强烈强化,有效地吸收了我们学习和决策的自然机制。
这是合理的,因为许多(尽管不是全部)滥用药物直接或间接增加了纹状体中多巴胺神经元轴突末端周围区域的多巴胺水平,纹状体是一种与正常奖励学习密切相关的大脑结构(第~\ref{sec:neural_ac}~节)。
但与毒瘾相关的自我毁灭行为并不是正常学习的特征。
当奖励是成瘾药物的结果时,多巴胺介导的学习有何不同?
成瘾是否是对物质的正常学习的结果,而这些物质在我们的进化史上基本上是不可用的,因此进化无法选择对抗它们的破坏性影响?
或者成瘾物质是否会以某种方式干扰正常的多巴胺介导的学习?


多巴胺神经元活动的奖励预测误差假设及其与\textit{时间差分}学习的联系是 Redish\cite{redish2004addiction} 的模型的基础,该模型描述了成瘾的一些(但肯定不是全部)特征。
该模型基于这样的观察:服用可卡因和其他一些成瘾药物会导致多巴胺短暂增加。
在模型中,假设这种多巴胺激增会增加\textit{时间差分}误差 ,其方式无法通过价值函数的变化来抵消。
换句话说,虽然减少到正常奖励由先行事件预测的程度(第~\ref{sec:td_dopamine}~节),但成瘾刺激的贡献不会随着奖励信号的预测而减少:药物奖励不能被预测。
该模型通过防止奖励信号因成瘾药物而变为负值来实现这一点,从而消除了与药物管理相关的状态的\textit{时间差分}学习的纠错功能。
结果是,值 这些状态的数量无限增加,使得导致这些状态的行动优先于所有其他状态。


成瘾行为比雷迪什模型的结果要复杂得多,但该模型的主要思想可能只是谜题的一部分。
或者模型可能会产生误导。
多巴胺似乎并未在所有形式的成瘾中发挥关键作用,而且并非每个人都同样容易出现成瘾行为。
此外,该模型不包括伴随长期吸毒而发生的许多回路和大脑区域的变化,例如,重复使用导致药物效果减弱的变化。
成瘾也可能涉及基于模型的过程。
尽管如此,雷迪什的模型仍然说明了如何利用强化学习理论来理解重大健康问题。
以类似的方式,强化学习理论对计算精神病学新领域的发展产生了影响,该领域旨在通过数学和计算方法提高对精神障碍的理解。


\section{总结}

大脑奖励系统涉及的神经通路非常复杂且尚未完全了解,但旨在了解这些通路及其在行为中作用的神经科学研究正在迅速进展。
这项研究揭示了大脑的奖励系统与本书中提出的强化学习理论之间惊人的对应关系。


多巴胺神经元活动的奖励预测误差假说是由科学家提出的,他们认识到\textit{时间差分误差}的行为与产生多巴胺的神经元活动之间存在惊人的相似之处,多巴胺是哺乳动物中与奖励相关的学习和行为所必需的神经递质。
神经科学家\textit{舒尔茨}实验室于 20 世纪 80 年代末和 90 年代进行的实验表明,只有当动物没有预料到这些事件时,多巴胺神经元才会以大量的活动爆发(称为阶段性响应)对奖励事件做出响应,这表明多巴胺神经元正在发出奖励信号预测误差而不是奖励本身。
此外,这些实验表明,当动物学会根据先前的感觉线索预测奖励事件时,多巴胺神经元的阶段性活动会转向较早的预测线索,同时减少到较晚的预测线索。
这与强化学习智能体学习预测奖励时\textit{时间差分}误差的后备效应类似。


其他实验结果有力地证实,多巴胺神经元的阶段性活动是学习的强化信号,通过产生多巴胺的神经元的大量分支轴突到达大脑的多个区域。
这些结果与我们对奖励信号 Rt 和强化信号(即我们提出的大多数算法中的\textit{时间差分}误差 t)之间的区别是一致的。
多巴胺神经元的阶段性响应是强化信号,而不是奖励信号。


一个突出的假设是,大脑实现了类似\textit{行动者}/评论家算法的东西。
大脑中的两个结构(纹状体的背侧和腹侧部分)在基于奖励的学习中都发挥着关键作用,可能分别像\textit{行动者}和\textit{评论家}一样发挥作用。
\textit{时间差分}误差对于\textit{行动者}和\textit{评论家}来说都是强化信号,这与多巴胺神经元轴突同时针对纹状体的背侧和腹侧细分这一事实相吻合;
多巴胺似乎对于调节这两种结构的突触可塑性至关重要;
并且诸如多巴胺的神经调节剂对靶结构的影响取决于靶结构的性质而不仅仅取决于神经调节剂的性质。


行动者和\textit{评论家}可以通过由类似神经元的单元组成的人工神经网络来实现,该网络具有基于第 13.5 节中描述的策略梯度行动者/\textit{评论家}方法的学习规则。
这些网络中的每个连接就像大脑中神经元之间的突触,学习规则对应于控制突触效率如何随着突触前和突触后神经元的活动而变化的规则,以及与来自神经元的输入相对应的神经调节输入。
多巴胺神经元。
在此设置中,每个突触都有自己的\textit{资格迹},记录涉及该突触的过去活动。
行动者和\textit{评论家}学习规则之间的唯一区别在于它们使用不同类型的\textit{资格迹}:\textit{评论家}单元的轨迹是非偶然的,因为它们不涉及\textit{评论家}单元的输出,而行动者单元的轨迹是偶然的,因为另外 对于执行者单元的输入,它们取决于执行者单元的输出。
在大脑中\textit{行动者}评价系统的假设实现中,这些学习规则分别对应于控制皮层纹状体突触可塑性的规则,这些突触将信号从皮层传递到背侧和腹侧纹状体细分中的主要神经元,突触还接收来自 多巴胺神经元。


\textit{行动者-评论家}网络中的行动者单元的学习规则与奖励调节的尖峰时间依赖性可塑性密切对应。
在\textit{脉冲时序的可塑性}中,突触前和突触后活动的相对时间决定了突触变化的方向。
在奖励调节\textit{脉冲时序的可塑性}中,突触的变化还取决于神经调节剂(例如多巴胺)在满足\textit{脉冲时序的可塑性}条件后可持续长达 10 秒的时间窗口内到达。
越来越多的证据表明,奖励调节的\textit{脉冲时序的可塑性}发生在皮层纹状体突触处,其中行动者的学习发生在行动者-\textit{评论家}系统的假设神经实现中,这增加了一些动物的大脑中存在类似\textit{行动者-评论家}系统的假设的合理性。


突触资格的思想和行动者学习规则的基本特征源自\textit{克洛普夫}的“享乐神经元”假设\cite{klopf1972brain,klopf1982hedonistic}。
他推测个体神经元通过调整效率来寻求获得奖励并避免惩罚。
根据其动作电位的奖励或惩罚后果来控制突触的行为,神经元的活动可以影响其后来的输入,因为神经元嵌入在许多反馈回路中,其中一些反馈回路位于动物的神经系统和身体内,另一些则通过动物的外部。
\textit{克洛普夫}关于资格的想法是,如果突触参与了神经元环,则它们会被暂时标记为有资格进行修改(使其成为\textit{资格迹}的偶然形式),如果增强信号到达,突触的有效性就会被修改。
我们以细菌的趋化行为作为单细胞指导其运动以寻找某些分子并避开其他分子的例子。


多巴胺系统的一个显著特征是释放多巴胺的纤维广泛投射到大脑的多个部位。
尽管很可能只有某些多巴胺神经元群体广播相同的强化信号,但如果该信号到达参与\textit{行动者}型学习的许多神经元的突触,那么这种情况可以建模为团队问题。
在此类问题中,强化学习智能体集合中的每个智能体都会收到相同的强化信号,其中该信号取决于该集合或团队的所有成员的活动。
如果每个团队成员都使用足够强大的学习算法,即使团队成员彼此不直接沟通,团队也可以集体学习,以通过全球广播的强化信号评估来提高整个团队的绩效。
这与大脑中多巴胺信号的广泛分散相一致,并为广泛使用的用于训练多层网络的误差反向传播方法提供了一种神经学上合理的替代方案。


% 基于模型 -- 目标导向?
\textit{无模型}的强化学习和\textit{基于模型}的强化学习之间的区别正在帮助神经科学家研究\textit{习惯性}和\textit{目标导向}的学习和决策的神经基础。
迄今为止的研究表明,某些大脑区域比另一种过程更多地参与一种类型的过程,但情况仍不清楚,因为无模型和基于模型的过程在大脑中似乎没有被整齐地分开。
许多问题仍然没有答案。
也许最有趣的是有证据表明,海马体这种传统上与空间导航和记忆相关的结构,似乎参与\textit{模拟未来可能的行动过程},作为动物决策过程的一部分。
这表明它是使用环境模型进行规划的系统的一部分。


强化学习理论也影响着对药物滥用背后的神经过程的思考。
毒瘾某些特征的模型基于奖励预测误差假设。
它提出,可卡因等成瘾兴奋剂会破坏\textit{时间差分}学习的稳定性,从而导致与吸毒相关的行为价值无限增长。
这远不是一个完整的成瘾模型,但它说明了如何从计算角度提出可以通过进一步研究进行测试的理论。
计算精神病学的新领域同样侧重于使用计算模型(其中一些模型源自强化学习)来更好地理解精神障碍。



本章仅触及强化学习的神经科学与计算机科学与工程中的强化学习的发展如何相互影响的表面。
强化学习算法的大多数功能都将其设计归因于纯粹的计算考虑,但有些功能受到有关神经学习机制的假设的影响。
值得注意的是,随着有关大脑奖励过程的实验数据不断积累,强化学习算法的许多纯粹由计算驱动的特征被证明与神经科学数据是一致的。
计算强化学习的其他特征,例如\textit{资格迹}和强化学习智能体团队在全球广播强化信号的影响下学习集体行动的能力,也可能会变成平行的实验数据,因为神经科学家继续解开这个问题。
基于奖励的动物学习和行为的神经基础。











\nocite{*} 
\printbibliography


\end{document}

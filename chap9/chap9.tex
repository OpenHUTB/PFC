\chapter{人类前额叶皮层:从指令和想像中产生目标}
这本书提出了关于灵长类动物前额叶皮层基本功能的方案。

\section{介绍}
本章的目的是:检验我们关于PF皮层基本功能的建议是否可以解释当人们执行复杂认知任务时在那里发生的成像激活。人类PF皮层中的许多激活似乎反映了对背景、目标和结果层次结构的阐述,从背景中生成目标,使用事件和抽象规则来选择目标,或目标的前瞻性编码。 其中一些激活发生在可能出现在类人猿或人类进化过程中的区域中,我们试图通过重新表示这一概念来解释这些。现有PF区域的扩展以及新区域的可能出现,导致人脑的大小和形状发生变化,连接方式也发生变化。我们提出了一个观点:现代人和猴子的前额叶皮层执行从他们最后的共同祖先继承的共同功能:它以减少错误的方式产生目标。 在一个关键的进化过程中,人类前额叶皮层进一步减少了错误,因为人们可以从指令或模仿中学习,并且因为人们可以在采取任何行动之前进行心理试错行为。 因此——至少有时——人类可以完全避免错误。
\section{目的}

\section{定义和术语}


\section{指纹}

\subsection{损伤和激活}

\subsection{损伤和活动}

\subsection{活动和激活}




\subsection{结论}


